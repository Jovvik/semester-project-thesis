\chapter{Methodology}

\section{Datasets and models}

% \begin{itemize}
%     \item MNIST, FashionMNIST and (maybe?) show the synthetic datasets
%     \item Describe the model(s) used
% \end{itemize}

For evaulation we have used the MNIST~\cite{deng2012mnist} and FashionMNIST~\cite{xiao2017fashionmnistnovelimagedataset} datasets
with the predefined training and testing splits.
We have also used synthetic 2D and 3D binary classification datasets for more easily interpretable experiments.
They are shown in Figures~\ref{fig:synthetic-datasets-2d} and \ref{fig:synthetic-datasets-3d}.
with each dataset consisting of $10000$ points, uniformly sampled from $[ - 1, 1]^2$
and split into training and testing sets with a ratio of $0.8:0.2$.

\todo{Maybe drop 3D}

\todo{Describe model for 2D/3D if we need that}

For MNIST and FashionMNIST, we have used:
\begin{enumerate}
    \item A simple feed-forward neural network with layer sizes
    $784, 512, 128, 16, 10$, ReLU activations and a softmax output layer.
    % This network is later denoted as ``FC''.
    \item A convolutional neural network with a modification of the architecture given in
    \href{https://github.com/pytorch/examples/tree/main/mnist}{https://github.com/pytorch/examples/tree/main/mnist}.
    The modifications are:
    \begin{enumerate}
        \item Addition of the \verb|scale| parameter, which allows us to change the neural network width.
        It is a multiplicative factor for the number of channels in the convolutional layers and number of neurons in the fully connected layers.
        \item Addition of the \verb|extra_cnn| parameter, which adds convolutional layers with $64 \cdot \verb|scale|$ channels, $3 \times 3$ kernel size,
        1 pixel padding and stride, as well as a ReLU activation.
        \item Additon of the \verb|extra_linear| parameter, which similarly adds a fully connected layer with $128 \cdot \verb|scale|$ neurons,
        as well as a ReLU activation.
    \end{enumerate}
\end{enumerate}

\begin{figure}
    \centering
    \begin{subfigure}{0.45\textwidth}
        \resizebox{\textwidth}{!}{
            \begin{tikzpicture}
                \begin{axis}[
                    enlargelimits=false,
                ]
                \addplot+[
                    only marks,
                    scatter,
                    mark size=2pt,
                    scatter src=explicit]
                table[meta=class]
                {data/2d/manyholes.dat};
                \end{axis}
            \end{tikzpicture}
        }
        \caption{Many holes dataset}
    \end{subfigure}
    \begin{subfigure}{0.45\textwidth}
        \resizebox{\textwidth}{!}{
            \begin{tikzpicture}
                \begin{axis}[
                    enlargelimits=false,
                ]
                \addplot+[
                    only marks,
                    scatter,
                    mark size=2pt,
                    scatter src=explicit]
                table[meta=class]
                {data/2d/nested.dat};
                \end{axis}
            \end{tikzpicture}
        }
        \caption{Nested annuli dataset}
    \end{subfigure}
    \caption{Synthetic 2D datasets}
    \label{fig:synthetic-datasets-2d}
\end{figure}

\section{Metrics and evaluation}

\begin{itemize}
    \item Persistence diagrams
    \item Wasserstein distance
    \item Total bar length
\end{itemize}

\begin{definition}
    A \emph{persistence diagram} is a multiset of points in $\R^2$.
    \todo{finish}
\end{definition}

\begin{definition}
    The \emph{Wasserstein distance} between two persistence diagrams $D_1$ and $D_2$ is defined as
    \todo{finish}
\end{definition}

\begin{definition}
    The \emph{total bar length} of a persistence diagram $D$ is defined as
    \[\sum_{(b, d) \in D} d - b\]
\end{definition}

\section{Simplicial complex}

\begin{itemize}
    \item Labeled Vietoris-Rips complex
    \item Usage of \texttt{ripser++} and \texttt{giotto-ph}
    \item Generalization to multiclass classification
    \item Circumcircle filtering (show why it's needed in 2D, but not in high-dim)
    \item Dowker complex
\end{itemize}