\chapter{Results and discussion}

\section{Sampling stability on MNIST}

Show that with the default parameters, the persistence diagrams change too much under sampling.
Show it's stable under N=10.

\section{Synthetic 2D and 3D data experiments}

\begin{itemize}
    \item Describe the data (or do it in the methodology?)
    \item Show what isn't captured correctly
    \item Show that CC filtering or the Dowker complex can help
\end{itemize}

\section{Results on MNIST}

\begin{itemize}
    % \item Show why CC filtering is not needed for MNIST
    \item Accuracy of the model correlates with the Wasserstein distance and the total bar length
    \item ``Elbow'' behavior --- first epochs have high changes in topological metrics, then it's flatter.
    Probably easier to show for Wasserstein distance, as the total bar length may be both high and low.
    Alternatively, show the difference between total bar length and GT total bar length.
    \item Underfitting and overfitting leads to worse topological metrics
    \item More accurate models have better topological metrics
    \item Size of the model doesn't impact the topological metrics (as long as the accuracy is the same for them)
\end{itemize}

\section{Results on FashionMNIST}

Hopefully, just show it's the same.

\section{Results on multiclass classification}

Show it's the same (?)

\section{Comparison of CNN and MLP}

If there's anything interesting
