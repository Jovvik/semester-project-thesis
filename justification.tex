\chapter{Synthetic 2D data experiments}
\label{chap:synthetic}

This chapter evaluates our topological analysis methods on synthetic 2D datasets
where we can visually verify our results and understand the challenges that
arise. We first demonstrate that the LVR complex can produce spurious
topological features by allowing simplices to cross decision boundary multiple
times. We then propose and evaluate two potential solutions to this limitation:
circumcircle filtering and the Dowker complex.

\section{Initial results}

\begin{figure}
    \centering
    \begin{subfigure}{0.49\textwidth}
        \resizebox{\textwidth}{!}{
            %% Creator: Matplotlib, PGF backend
%%
%% To include the figure in your LaTeX document, write
%%   \input{<filename>.pgf}
%%
%% Make sure the required packages are loaded in your preamble
%%   \usepackage{pgf}
%%
%% Also ensure that all the required font packages are loaded; for instance,
%% the lmodern package is sometimes necessary when using math font.
%%   \usepackage{lmodern}
%%
%% Figures using additional raster images can only be included by \input if
%% they are in the same directory as the main LaTeX file. For loading figures
%% from other directories you can use the `import` package
%%   \usepackage{import}
%%
%% and then include the figures with
%%   \import{<path to file>}{<filename>.pgf}
%%
%% Matplotlib used the following preamble
%%   \def\mathdefault#1{#1}
%%   \everymath=\expandafter{\the\everymath\displaystyle}
%%   
%%   \ifdefined\pdftexversion\else  % non-pdftex case.
%%     \usepackage{fontspec}
%%     \setmainfont{DejaVuSerif.ttf}[Path=\detokenize{/home/snek/repos/homology-decision-bondaries-clean/venv/lib/python3.9/site-packages/matplotlib/mpl-data/fonts/ttf/}]
%%     \setsansfont{DejaVuSans.ttf}[Path=\detokenize{/home/snek/repos/homology-decision-bondaries-clean/venv/lib/python3.9/site-packages/matplotlib/mpl-data/fonts/ttf/}]
%%     \setmonofont{DejaVuSansMono.ttf}[Path=\detokenize{/home/snek/repos/homology-decision-bondaries-clean/venv/lib/python3.9/site-packages/matplotlib/mpl-data/fonts/ttf/}]
%%   \fi
%%   \makeatletter\@ifpackageloaded{underscore}{}{\usepackage[strings]{underscore}}\makeatother
%%
\begingroup%
\makeatletter%
\begin{pgfpicture}%
\pgfpathrectangle{\pgfpointorigin}{\pgfqpoint{4.331782in}{4.144486in}}%
\pgfusepath{use as bounding box, clip}%
\begin{pgfscope}%
\pgfsetbuttcap%
\pgfsetmiterjoin%
\definecolor{currentfill}{rgb}{1.000000,1.000000,1.000000}%
\pgfsetfillcolor{currentfill}%
\pgfsetlinewidth{0.000000pt}%
\definecolor{currentstroke}{rgb}{1.000000,1.000000,1.000000}%
\pgfsetstrokecolor{currentstroke}%
\pgfsetdash{}{0pt}%
\pgfpathmoveto{\pgfqpoint{0.000000in}{-0.000000in}}%
\pgfpathlineto{\pgfqpoint{4.331782in}{-0.000000in}}%
\pgfpathlineto{\pgfqpoint{4.331782in}{4.144486in}}%
\pgfpathlineto{\pgfqpoint{0.000000in}{4.144486in}}%
\pgfpathlineto{\pgfqpoint{0.000000in}{-0.000000in}}%
\pgfpathclose%
\pgfusepath{fill}%
\end{pgfscope}%
\begin{pgfscope}%
\pgfsetbuttcap%
\pgfsetmiterjoin%
\definecolor{currentfill}{rgb}{1.000000,1.000000,1.000000}%
\pgfsetfillcolor{currentfill}%
\pgfsetlinewidth{0.000000pt}%
\definecolor{currentstroke}{rgb}{0.000000,0.000000,0.000000}%
\pgfsetstrokecolor{currentstroke}%
\pgfsetstrokeopacity{0.000000}%
\pgfsetdash{}{0pt}%
\pgfpathmoveto{\pgfqpoint{0.800049in}{0.448486in}}%
\pgfpathlineto{\pgfqpoint{4.331782in}{0.448486in}}%
\pgfpathlineto{\pgfqpoint{4.331782in}{4.144486in}}%
\pgfpathlineto{\pgfqpoint{0.800049in}{4.144486in}}%
\pgfpathlineto{\pgfqpoint{0.800049in}{0.448486in}}%
\pgfpathclose%
\pgfusepath{fill}%
\end{pgfscope}%
\begin{pgfscope}%
\pgfpathrectangle{\pgfqpoint{0.800049in}{0.448486in}}{\pgfqpoint{3.531733in}{3.696000in}}%
\pgfusepath{clip}%
\pgfsetbuttcap%
\pgfsetmiterjoin%
\definecolor{currentfill}{rgb}{0.827451,0.827451,0.827451}%
\pgfsetfillcolor{currentfill}%
\pgfsetlinewidth{1.003750pt}%
\definecolor{currentstroke}{rgb}{0.827451,0.827451,0.827451}%
\pgfsetstrokecolor{currentstroke}%
\pgfsetdash{}{0pt}%
\pgfpathmoveto{\pgfqpoint{0.800049in}{0.448486in}}%
\pgfpathlineto{\pgfqpoint{4.331782in}{0.448486in}}%
\pgfpathlineto{\pgfqpoint{4.331782in}{3.980219in}}%
\pgfpathlineto{\pgfqpoint{0.800049in}{0.448486in}}%
\pgfpathclose%
\pgfusepath{stroke,fill}%
\end{pgfscope}%
\begin{pgfscope}%
\pgfpathrectangle{\pgfqpoint{0.800049in}{0.448486in}}{\pgfqpoint{3.531733in}{3.696000in}}%
\pgfusepath{clip}%
\pgfsetbuttcap%
\pgfsetroundjoin%
\definecolor{currentfill}{rgb}{0.894118,0.101961,0.109804}%
\pgfsetfillcolor{currentfill}%
\pgfsetfillopacity{0.600000}%
\pgfsetlinewidth{1.003750pt}%
\definecolor{currentstroke}{rgb}{0.894118,0.101961,0.109804}%
\pgfsetstrokecolor{currentstroke}%
\pgfsetstrokeopacity{0.600000}%
\pgfsetdash{}{0pt}%
\pgfpathmoveto{\pgfqpoint{4.249649in}{4.020686in}}%
\pgfpathcurveto{\pgfqpoint{4.260699in}{4.020686in}}{\pgfqpoint{4.271298in}{4.025076in}}{\pgfqpoint{4.279111in}{4.032890in}}%
\pgfpathcurveto{\pgfqpoint{4.286925in}{4.040703in}}{\pgfqpoint{4.291315in}{4.051303in}}{\pgfqpoint{4.291315in}{4.062353in}}%
\pgfpathcurveto{\pgfqpoint{4.291315in}{4.073403in}}{\pgfqpoint{4.286925in}{4.084002in}}{\pgfqpoint{4.279111in}{4.091815in}}%
\pgfpathcurveto{\pgfqpoint{4.271298in}{4.099629in}}{\pgfqpoint{4.260699in}{4.104019in}}{\pgfqpoint{4.249649in}{4.104019in}}%
\pgfpathcurveto{\pgfqpoint{4.238598in}{4.104019in}}{\pgfqpoint{4.227999in}{4.099629in}}{\pgfqpoint{4.220186in}{4.091815in}}%
\pgfpathcurveto{\pgfqpoint{4.212372in}{4.084002in}}{\pgfqpoint{4.207982in}{4.073403in}}{\pgfqpoint{4.207982in}{4.062353in}}%
\pgfpathcurveto{\pgfqpoint{4.207982in}{4.051303in}}{\pgfqpoint{4.212372in}{4.040703in}}{\pgfqpoint{4.220186in}{4.032890in}}%
\pgfpathcurveto{\pgfqpoint{4.227999in}{4.025076in}}{\pgfqpoint{4.238598in}{4.020686in}}{\pgfqpoint{4.249649in}{4.020686in}}%
\pgfpathlineto{\pgfqpoint{4.249649in}{4.020686in}}%
\pgfpathclose%
\pgfusepath{stroke,fill}%
\end{pgfscope}%
\begin{pgfscope}%
\pgfpathrectangle{\pgfqpoint{0.800049in}{0.448486in}}{\pgfqpoint{3.531733in}{3.696000in}}%
\pgfusepath{clip}%
\pgfsetbuttcap%
\pgfsetroundjoin%
\definecolor{currentfill}{rgb}{0.894118,0.101961,0.109804}%
\pgfsetfillcolor{currentfill}%
\pgfsetfillopacity{0.600000}%
\pgfsetlinewidth{1.003750pt}%
\definecolor{currentstroke}{rgb}{0.894118,0.101961,0.109804}%
\pgfsetstrokecolor{currentstroke}%
\pgfsetstrokeopacity{0.600000}%
\pgfsetdash{}{0pt}%
\pgfpathmoveto{\pgfqpoint{3.730912in}{4.020686in}}%
\pgfpathcurveto{\pgfqpoint{3.741962in}{4.020686in}}{\pgfqpoint{3.752561in}{4.025076in}}{\pgfqpoint{3.760375in}{4.032890in}}%
\pgfpathcurveto{\pgfqpoint{3.768188in}{4.040703in}}{\pgfqpoint{3.772578in}{4.051303in}}{\pgfqpoint{3.772578in}{4.062353in}}%
\pgfpathcurveto{\pgfqpoint{3.772578in}{4.073403in}}{\pgfqpoint{3.768188in}{4.084002in}}{\pgfqpoint{3.760375in}{4.091815in}}%
\pgfpathcurveto{\pgfqpoint{3.752561in}{4.099629in}}{\pgfqpoint{3.741962in}{4.104019in}}{\pgfqpoint{3.730912in}{4.104019in}}%
\pgfpathcurveto{\pgfqpoint{3.719862in}{4.104019in}}{\pgfqpoint{3.709263in}{4.099629in}}{\pgfqpoint{3.701449in}{4.091815in}}%
\pgfpathcurveto{\pgfqpoint{3.693635in}{4.084002in}}{\pgfqpoint{3.689245in}{4.073403in}}{\pgfqpoint{3.689245in}{4.062353in}}%
\pgfpathcurveto{\pgfqpoint{3.689245in}{4.051303in}}{\pgfqpoint{3.693635in}{4.040703in}}{\pgfqpoint{3.701449in}{4.032890in}}%
\pgfpathcurveto{\pgfqpoint{3.709263in}{4.025076in}}{\pgfqpoint{3.719862in}{4.020686in}}{\pgfqpoint{3.730912in}{4.020686in}}%
\pgfpathlineto{\pgfqpoint{3.730912in}{4.020686in}}%
\pgfpathclose%
\pgfusepath{stroke,fill}%
\end{pgfscope}%
\begin{pgfscope}%
\pgfpathrectangle{\pgfqpoint{0.800049in}{0.448486in}}{\pgfqpoint{3.531733in}{3.696000in}}%
\pgfusepath{clip}%
\pgfsetbuttcap%
\pgfsetroundjoin%
\definecolor{currentfill}{rgb}{0.894118,0.101961,0.109804}%
\pgfsetfillcolor{currentfill}%
\pgfsetfillopacity{0.600000}%
\pgfsetlinewidth{1.003750pt}%
\definecolor{currentstroke}{rgb}{0.894118,0.101961,0.109804}%
\pgfsetstrokecolor{currentstroke}%
\pgfsetstrokeopacity{0.600000}%
\pgfsetdash{}{0pt}%
\pgfpathmoveto{\pgfqpoint{3.212175in}{4.020686in}}%
\pgfpathcurveto{\pgfqpoint{3.223225in}{4.020686in}}{\pgfqpoint{3.233824in}{4.025076in}}{\pgfqpoint{3.241638in}{4.032890in}}%
\pgfpathcurveto{\pgfqpoint{3.249451in}{4.040703in}}{\pgfqpoint{3.253842in}{4.051303in}}{\pgfqpoint{3.253842in}{4.062353in}}%
\pgfpathcurveto{\pgfqpoint{3.253842in}{4.073403in}}{\pgfqpoint{3.249451in}{4.084002in}}{\pgfqpoint{3.241638in}{4.091815in}}%
\pgfpathcurveto{\pgfqpoint{3.233824in}{4.099629in}}{\pgfqpoint{3.223225in}{4.104019in}}{\pgfqpoint{3.212175in}{4.104019in}}%
\pgfpathcurveto{\pgfqpoint{3.201125in}{4.104019in}}{\pgfqpoint{3.190526in}{4.099629in}}{\pgfqpoint{3.182712in}{4.091815in}}%
\pgfpathcurveto{\pgfqpoint{3.174899in}{4.084002in}}{\pgfqpoint{3.170508in}{4.073403in}}{\pgfqpoint{3.170508in}{4.062353in}}%
\pgfpathcurveto{\pgfqpoint{3.170508in}{4.051303in}}{\pgfqpoint{3.174899in}{4.040703in}}{\pgfqpoint{3.182712in}{4.032890in}}%
\pgfpathcurveto{\pgfqpoint{3.190526in}{4.025076in}}{\pgfqpoint{3.201125in}{4.020686in}}{\pgfqpoint{3.212175in}{4.020686in}}%
\pgfpathlineto{\pgfqpoint{3.212175in}{4.020686in}}%
\pgfpathclose%
\pgfusepath{stroke,fill}%
\end{pgfscope}%
\begin{pgfscope}%
\pgfpathrectangle{\pgfqpoint{0.800049in}{0.448486in}}{\pgfqpoint{3.531733in}{3.696000in}}%
\pgfusepath{clip}%
\pgfsetbuttcap%
\pgfsetroundjoin%
\definecolor{currentfill}{rgb}{0.894118,0.101961,0.109804}%
\pgfsetfillcolor{currentfill}%
\pgfsetfillopacity{0.600000}%
\pgfsetlinewidth{1.003750pt}%
\definecolor{currentstroke}{rgb}{0.894118,0.101961,0.109804}%
\pgfsetstrokecolor{currentstroke}%
\pgfsetstrokeopacity{0.600000}%
\pgfsetdash{}{0pt}%
\pgfpathmoveto{\pgfqpoint{3.212175in}{3.337682in}}%
\pgfpathcurveto{\pgfqpoint{3.223225in}{3.337682in}}{\pgfqpoint{3.233824in}{3.342073in}}{\pgfqpoint{3.241638in}{3.349886in}}%
\pgfpathcurveto{\pgfqpoint{3.249451in}{3.357700in}}{\pgfqpoint{3.253842in}{3.368299in}}{\pgfqpoint{3.253842in}{3.379349in}}%
\pgfpathcurveto{\pgfqpoint{3.253842in}{3.390399in}}{\pgfqpoint{3.249451in}{3.400998in}}{\pgfqpoint{3.241638in}{3.408812in}}%
\pgfpathcurveto{\pgfqpoint{3.233824in}{3.416626in}}{\pgfqpoint{3.223225in}{3.421016in}}{\pgfqpoint{3.212175in}{3.421016in}}%
\pgfpathcurveto{\pgfqpoint{3.201125in}{3.421016in}}{\pgfqpoint{3.190526in}{3.416626in}}{\pgfqpoint{3.182712in}{3.408812in}}%
\pgfpathcurveto{\pgfqpoint{3.174899in}{3.400998in}}{\pgfqpoint{3.170508in}{3.390399in}}{\pgfqpoint{3.170508in}{3.379349in}}%
\pgfpathcurveto{\pgfqpoint{3.170508in}{3.368299in}}{\pgfqpoint{3.174899in}{3.357700in}}{\pgfqpoint{3.182712in}{3.349886in}}%
\pgfpathcurveto{\pgfqpoint{3.190526in}{3.342073in}}{\pgfqpoint{3.201125in}{3.337682in}}{\pgfqpoint{3.212175in}{3.337682in}}%
\pgfpathlineto{\pgfqpoint{3.212175in}{3.337682in}}%
\pgfpathclose%
\pgfusepath{stroke,fill}%
\end{pgfscope}%
\begin{pgfscope}%
\pgfpathrectangle{\pgfqpoint{0.800049in}{0.448486in}}{\pgfqpoint{3.531733in}{3.696000in}}%
\pgfusepath{clip}%
\pgfsetbuttcap%
\pgfsetroundjoin%
\definecolor{currentfill}{rgb}{0.894118,0.101961,0.109804}%
\pgfsetfillcolor{currentfill}%
\pgfsetfillopacity{0.600000}%
\pgfsetlinewidth{1.003750pt}%
\definecolor{currentstroke}{rgb}{0.894118,0.101961,0.109804}%
\pgfsetstrokecolor{currentstroke}%
\pgfsetstrokeopacity{0.600000}%
\pgfsetdash{}{0pt}%
\pgfpathmoveto{\pgfqpoint{3.212175in}{3.337682in}}%
\pgfpathcurveto{\pgfqpoint{3.223225in}{3.337682in}}{\pgfqpoint{3.233824in}{3.342073in}}{\pgfqpoint{3.241638in}{3.349886in}}%
\pgfpathcurveto{\pgfqpoint{3.249451in}{3.357700in}}{\pgfqpoint{3.253842in}{3.368299in}}{\pgfqpoint{3.253842in}{3.379349in}}%
\pgfpathcurveto{\pgfqpoint{3.253842in}{3.390399in}}{\pgfqpoint{3.249451in}{3.400998in}}{\pgfqpoint{3.241638in}{3.408812in}}%
\pgfpathcurveto{\pgfqpoint{3.233824in}{3.416626in}}{\pgfqpoint{3.223225in}{3.421016in}}{\pgfqpoint{3.212175in}{3.421016in}}%
\pgfpathcurveto{\pgfqpoint{3.201125in}{3.421016in}}{\pgfqpoint{3.190526in}{3.416626in}}{\pgfqpoint{3.182712in}{3.408812in}}%
\pgfpathcurveto{\pgfqpoint{3.174899in}{3.400998in}}{\pgfqpoint{3.170508in}{3.390399in}}{\pgfqpoint{3.170508in}{3.379349in}}%
\pgfpathcurveto{\pgfqpoint{3.170508in}{3.368299in}}{\pgfqpoint{3.174899in}{3.357700in}}{\pgfqpoint{3.182712in}{3.349886in}}%
\pgfpathcurveto{\pgfqpoint{3.190526in}{3.342073in}}{\pgfqpoint{3.201125in}{3.337682in}}{\pgfqpoint{3.212175in}{3.337682in}}%
\pgfpathlineto{\pgfqpoint{3.212175in}{3.337682in}}%
\pgfpathclose%
\pgfusepath{stroke,fill}%
\end{pgfscope}%
\begin{pgfscope}%
\pgfpathrectangle{\pgfqpoint{0.800049in}{0.448486in}}{\pgfqpoint{3.531733in}{3.696000in}}%
\pgfusepath{clip}%
\pgfsetbuttcap%
\pgfsetroundjoin%
\definecolor{currentfill}{rgb}{0.894118,0.101961,0.109804}%
\pgfsetfillcolor{currentfill}%
\pgfsetfillopacity{0.600000}%
\pgfsetlinewidth{1.003750pt}%
\definecolor{currentstroke}{rgb}{0.894118,0.101961,0.109804}%
\pgfsetstrokecolor{currentstroke}%
\pgfsetstrokeopacity{0.600000}%
\pgfsetdash{}{0pt}%
\pgfpathmoveto{\pgfqpoint{3.212175in}{3.337682in}}%
\pgfpathcurveto{\pgfqpoint{3.223225in}{3.337682in}}{\pgfqpoint{3.233824in}{3.342073in}}{\pgfqpoint{3.241638in}{3.349886in}}%
\pgfpathcurveto{\pgfqpoint{3.249451in}{3.357700in}}{\pgfqpoint{3.253842in}{3.368299in}}{\pgfqpoint{3.253842in}{3.379349in}}%
\pgfpathcurveto{\pgfqpoint{3.253842in}{3.390399in}}{\pgfqpoint{3.249451in}{3.400998in}}{\pgfqpoint{3.241638in}{3.408812in}}%
\pgfpathcurveto{\pgfqpoint{3.233824in}{3.416626in}}{\pgfqpoint{3.223225in}{3.421016in}}{\pgfqpoint{3.212175in}{3.421016in}}%
\pgfpathcurveto{\pgfqpoint{3.201125in}{3.421016in}}{\pgfqpoint{3.190526in}{3.416626in}}{\pgfqpoint{3.182712in}{3.408812in}}%
\pgfpathcurveto{\pgfqpoint{3.174899in}{3.400998in}}{\pgfqpoint{3.170508in}{3.390399in}}{\pgfqpoint{3.170508in}{3.379349in}}%
\pgfpathcurveto{\pgfqpoint{3.170508in}{3.368299in}}{\pgfqpoint{3.174899in}{3.357700in}}{\pgfqpoint{3.182712in}{3.349886in}}%
\pgfpathcurveto{\pgfqpoint{3.190526in}{3.342073in}}{\pgfqpoint{3.201125in}{3.337682in}}{\pgfqpoint{3.212175in}{3.337682in}}%
\pgfpathlineto{\pgfqpoint{3.212175in}{3.337682in}}%
\pgfpathclose%
\pgfusepath{stroke,fill}%
\end{pgfscope}%
\begin{pgfscope}%
\pgfpathrectangle{\pgfqpoint{0.800049in}{0.448486in}}{\pgfqpoint{3.531733in}{3.696000in}}%
\pgfusepath{clip}%
\pgfsetbuttcap%
\pgfsetroundjoin%
\definecolor{currentfill}{rgb}{0.894118,0.101961,0.109804}%
\pgfsetfillcolor{currentfill}%
\pgfsetfillopacity{0.600000}%
\pgfsetlinewidth{1.003750pt}%
\definecolor{currentstroke}{rgb}{0.894118,0.101961,0.109804}%
\pgfsetstrokecolor{currentstroke}%
\pgfsetstrokeopacity{0.600000}%
\pgfsetdash{}{0pt}%
\pgfpathmoveto{\pgfqpoint{3.212175in}{3.337682in}}%
\pgfpathcurveto{\pgfqpoint{3.223225in}{3.337682in}}{\pgfqpoint{3.233824in}{3.342073in}}{\pgfqpoint{3.241638in}{3.349886in}}%
\pgfpathcurveto{\pgfqpoint{3.249451in}{3.357700in}}{\pgfqpoint{3.253842in}{3.368299in}}{\pgfqpoint{3.253842in}{3.379349in}}%
\pgfpathcurveto{\pgfqpoint{3.253842in}{3.390399in}}{\pgfqpoint{3.249451in}{3.400998in}}{\pgfqpoint{3.241638in}{3.408812in}}%
\pgfpathcurveto{\pgfqpoint{3.233824in}{3.416626in}}{\pgfqpoint{3.223225in}{3.421016in}}{\pgfqpoint{3.212175in}{3.421016in}}%
\pgfpathcurveto{\pgfqpoint{3.201125in}{3.421016in}}{\pgfqpoint{3.190526in}{3.416626in}}{\pgfqpoint{3.182712in}{3.408812in}}%
\pgfpathcurveto{\pgfqpoint{3.174899in}{3.400998in}}{\pgfqpoint{3.170508in}{3.390399in}}{\pgfqpoint{3.170508in}{3.379349in}}%
\pgfpathcurveto{\pgfqpoint{3.170508in}{3.368299in}}{\pgfqpoint{3.174899in}{3.357700in}}{\pgfqpoint{3.182712in}{3.349886in}}%
\pgfpathcurveto{\pgfqpoint{3.190526in}{3.342073in}}{\pgfqpoint{3.201125in}{3.337682in}}{\pgfqpoint{3.212175in}{3.337682in}}%
\pgfpathlineto{\pgfqpoint{3.212175in}{3.337682in}}%
\pgfpathclose%
\pgfusepath{stroke,fill}%
\end{pgfscope}%
\begin{pgfscope}%
\pgfpathrectangle{\pgfqpoint{0.800049in}{0.448486in}}{\pgfqpoint{3.531733in}{3.696000in}}%
\pgfusepath{clip}%
\pgfsetbuttcap%
\pgfsetroundjoin%
\definecolor{currentfill}{rgb}{0.894118,0.101961,0.109804}%
\pgfsetfillcolor{currentfill}%
\pgfsetfillopacity{0.600000}%
\pgfsetlinewidth{1.003750pt}%
\definecolor{currentstroke}{rgb}{0.894118,0.101961,0.109804}%
\pgfsetstrokecolor{currentstroke}%
\pgfsetstrokeopacity{0.600000}%
\pgfsetdash{}{0pt}%
\pgfpathmoveto{\pgfqpoint{2.693438in}{4.020686in}}%
\pgfpathcurveto{\pgfqpoint{2.704488in}{4.020686in}}{\pgfqpoint{2.715087in}{4.025076in}}{\pgfqpoint{2.722901in}{4.032890in}}%
\pgfpathcurveto{\pgfqpoint{2.730714in}{4.040703in}}{\pgfqpoint{2.735105in}{4.051303in}}{\pgfqpoint{2.735105in}{4.062353in}}%
\pgfpathcurveto{\pgfqpoint{2.735105in}{4.073403in}}{\pgfqpoint{2.730714in}{4.084002in}}{\pgfqpoint{2.722901in}{4.091815in}}%
\pgfpathcurveto{\pgfqpoint{2.715087in}{4.099629in}}{\pgfqpoint{2.704488in}{4.104019in}}{\pgfqpoint{2.693438in}{4.104019in}}%
\pgfpathcurveto{\pgfqpoint{2.682388in}{4.104019in}}{\pgfqpoint{2.671789in}{4.099629in}}{\pgfqpoint{2.663975in}{4.091815in}}%
\pgfpathcurveto{\pgfqpoint{2.656162in}{4.084002in}}{\pgfqpoint{2.651771in}{4.073403in}}{\pgfqpoint{2.651771in}{4.062353in}}%
\pgfpathcurveto{\pgfqpoint{2.651771in}{4.051303in}}{\pgfqpoint{2.656162in}{4.040703in}}{\pgfqpoint{2.663975in}{4.032890in}}%
\pgfpathcurveto{\pgfqpoint{2.671789in}{4.025076in}}{\pgfqpoint{2.682388in}{4.020686in}}{\pgfqpoint{2.693438in}{4.020686in}}%
\pgfpathlineto{\pgfqpoint{2.693438in}{4.020686in}}%
\pgfpathclose%
\pgfusepath{stroke,fill}%
\end{pgfscope}%
\begin{pgfscope}%
\pgfpathrectangle{\pgfqpoint{0.800049in}{0.448486in}}{\pgfqpoint{3.531733in}{3.696000in}}%
\pgfusepath{clip}%
\pgfsetbuttcap%
\pgfsetroundjoin%
\definecolor{currentfill}{rgb}{0.894118,0.101961,0.109804}%
\pgfsetfillcolor{currentfill}%
\pgfsetfillopacity{0.600000}%
\pgfsetlinewidth{1.003750pt}%
\definecolor{currentstroke}{rgb}{0.894118,0.101961,0.109804}%
\pgfsetstrokecolor{currentstroke}%
\pgfsetstrokeopacity{0.600000}%
\pgfsetdash{}{0pt}%
\pgfpathmoveto{\pgfqpoint{2.693438in}{4.020686in}}%
\pgfpathcurveto{\pgfqpoint{2.704488in}{4.020686in}}{\pgfqpoint{2.715087in}{4.025076in}}{\pgfqpoint{2.722901in}{4.032890in}}%
\pgfpathcurveto{\pgfqpoint{2.730714in}{4.040703in}}{\pgfqpoint{2.735105in}{4.051303in}}{\pgfqpoint{2.735105in}{4.062353in}}%
\pgfpathcurveto{\pgfqpoint{2.735105in}{4.073403in}}{\pgfqpoint{2.730714in}{4.084002in}}{\pgfqpoint{2.722901in}{4.091815in}}%
\pgfpathcurveto{\pgfqpoint{2.715087in}{4.099629in}}{\pgfqpoint{2.704488in}{4.104019in}}{\pgfqpoint{2.693438in}{4.104019in}}%
\pgfpathcurveto{\pgfqpoint{2.682388in}{4.104019in}}{\pgfqpoint{2.671789in}{4.099629in}}{\pgfqpoint{2.663975in}{4.091815in}}%
\pgfpathcurveto{\pgfqpoint{2.656162in}{4.084002in}}{\pgfqpoint{2.651771in}{4.073403in}}{\pgfqpoint{2.651771in}{4.062353in}}%
\pgfpathcurveto{\pgfqpoint{2.651771in}{4.051303in}}{\pgfqpoint{2.656162in}{4.040703in}}{\pgfqpoint{2.663975in}{4.032890in}}%
\pgfpathcurveto{\pgfqpoint{2.671789in}{4.025076in}}{\pgfqpoint{2.682388in}{4.020686in}}{\pgfqpoint{2.693438in}{4.020686in}}%
\pgfpathlineto{\pgfqpoint{2.693438in}{4.020686in}}%
\pgfpathclose%
\pgfusepath{stroke,fill}%
\end{pgfscope}%
\begin{pgfscope}%
\pgfpathrectangle{\pgfqpoint{0.800049in}{0.448486in}}{\pgfqpoint{3.531733in}{3.696000in}}%
\pgfusepath{clip}%
\pgfsetbuttcap%
\pgfsetroundjoin%
\definecolor{currentfill}{rgb}{0.894118,0.101961,0.109804}%
\pgfsetfillcolor{currentfill}%
\pgfsetfillopacity{0.600000}%
\pgfsetlinewidth{1.003750pt}%
\definecolor{currentstroke}{rgb}{0.894118,0.101961,0.109804}%
\pgfsetstrokecolor{currentstroke}%
\pgfsetstrokeopacity{0.600000}%
\pgfsetdash{}{0pt}%
\pgfpathmoveto{\pgfqpoint{2.693438in}{4.020686in}}%
\pgfpathcurveto{\pgfqpoint{2.704488in}{4.020686in}}{\pgfqpoint{2.715087in}{4.025076in}}{\pgfqpoint{2.722901in}{4.032890in}}%
\pgfpathcurveto{\pgfqpoint{2.730714in}{4.040703in}}{\pgfqpoint{2.735105in}{4.051303in}}{\pgfqpoint{2.735105in}{4.062353in}}%
\pgfpathcurveto{\pgfqpoint{2.735105in}{4.073403in}}{\pgfqpoint{2.730714in}{4.084002in}}{\pgfqpoint{2.722901in}{4.091815in}}%
\pgfpathcurveto{\pgfqpoint{2.715087in}{4.099629in}}{\pgfqpoint{2.704488in}{4.104019in}}{\pgfqpoint{2.693438in}{4.104019in}}%
\pgfpathcurveto{\pgfqpoint{2.682388in}{4.104019in}}{\pgfqpoint{2.671789in}{4.099629in}}{\pgfqpoint{2.663975in}{4.091815in}}%
\pgfpathcurveto{\pgfqpoint{2.656162in}{4.084002in}}{\pgfqpoint{2.651771in}{4.073403in}}{\pgfqpoint{2.651771in}{4.062353in}}%
\pgfpathcurveto{\pgfqpoint{2.651771in}{4.051303in}}{\pgfqpoint{2.656162in}{4.040703in}}{\pgfqpoint{2.663975in}{4.032890in}}%
\pgfpathcurveto{\pgfqpoint{2.671789in}{4.025076in}}{\pgfqpoint{2.682388in}{4.020686in}}{\pgfqpoint{2.693438in}{4.020686in}}%
\pgfpathlineto{\pgfqpoint{2.693438in}{4.020686in}}%
\pgfpathclose%
\pgfusepath{stroke,fill}%
\end{pgfscope}%
\begin{pgfscope}%
\pgfpathrectangle{\pgfqpoint{0.800049in}{0.448486in}}{\pgfqpoint{3.531733in}{3.696000in}}%
\pgfusepath{clip}%
\pgfsetbuttcap%
\pgfsetroundjoin%
\definecolor{currentfill}{rgb}{0.894118,0.101961,0.109804}%
\pgfsetfillcolor{currentfill}%
\pgfsetfillopacity{0.600000}%
\pgfsetlinewidth{1.003750pt}%
\definecolor{currentstroke}{rgb}{0.894118,0.101961,0.109804}%
\pgfsetstrokecolor{currentstroke}%
\pgfsetstrokeopacity{0.600000}%
\pgfsetdash{}{0pt}%
\pgfpathmoveto{\pgfqpoint{2.693438in}{3.856419in}}%
\pgfpathcurveto{\pgfqpoint{2.704488in}{3.856419in}}{\pgfqpoint{2.715087in}{3.860810in}}{\pgfqpoint{2.722901in}{3.868623in}}%
\pgfpathcurveto{\pgfqpoint{2.730714in}{3.876437in}}{\pgfqpoint{2.735105in}{3.887036in}}{\pgfqpoint{2.735105in}{3.898086in}}%
\pgfpathcurveto{\pgfqpoint{2.735105in}{3.909136in}}{\pgfqpoint{2.730714in}{3.919735in}}{\pgfqpoint{2.722901in}{3.927549in}}%
\pgfpathcurveto{\pgfqpoint{2.715087in}{3.935362in}}{\pgfqpoint{2.704488in}{3.939753in}}{\pgfqpoint{2.693438in}{3.939753in}}%
\pgfpathcurveto{\pgfqpoint{2.682388in}{3.939753in}}{\pgfqpoint{2.671789in}{3.935362in}}{\pgfqpoint{2.663975in}{3.927549in}}%
\pgfpathcurveto{\pgfqpoint{2.656162in}{3.919735in}}{\pgfqpoint{2.651771in}{3.909136in}}{\pgfqpoint{2.651771in}{3.898086in}}%
\pgfpathcurveto{\pgfqpoint{2.651771in}{3.887036in}}{\pgfqpoint{2.656162in}{3.876437in}}{\pgfqpoint{2.663975in}{3.868623in}}%
\pgfpathcurveto{\pgfqpoint{2.671789in}{3.860810in}}{\pgfqpoint{2.682388in}{3.856419in}}{\pgfqpoint{2.693438in}{3.856419in}}%
\pgfpathlineto{\pgfqpoint{2.693438in}{3.856419in}}%
\pgfpathclose%
\pgfusepath{stroke,fill}%
\end{pgfscope}%
\begin{pgfscope}%
\pgfpathrectangle{\pgfqpoint{0.800049in}{0.448486in}}{\pgfqpoint{3.531733in}{3.696000in}}%
\pgfusepath{clip}%
\pgfsetbuttcap%
\pgfsetroundjoin%
\definecolor{currentfill}{rgb}{0.894118,0.101961,0.109804}%
\pgfsetfillcolor{currentfill}%
\pgfsetfillopacity{0.600000}%
\pgfsetlinewidth{1.003750pt}%
\definecolor{currentstroke}{rgb}{0.894118,0.101961,0.109804}%
\pgfsetstrokecolor{currentstroke}%
\pgfsetstrokeopacity{0.600000}%
\pgfsetdash{}{0pt}%
\pgfpathmoveto{\pgfqpoint{2.693438in}{3.856419in}}%
\pgfpathcurveto{\pgfqpoint{2.704488in}{3.856419in}}{\pgfqpoint{2.715087in}{3.860810in}}{\pgfqpoint{2.722901in}{3.868623in}}%
\pgfpathcurveto{\pgfqpoint{2.730714in}{3.876437in}}{\pgfqpoint{2.735105in}{3.887036in}}{\pgfqpoint{2.735105in}{3.898086in}}%
\pgfpathcurveto{\pgfqpoint{2.735105in}{3.909136in}}{\pgfqpoint{2.730714in}{3.919735in}}{\pgfqpoint{2.722901in}{3.927549in}}%
\pgfpathcurveto{\pgfqpoint{2.715087in}{3.935362in}}{\pgfqpoint{2.704488in}{3.939753in}}{\pgfqpoint{2.693438in}{3.939753in}}%
\pgfpathcurveto{\pgfqpoint{2.682388in}{3.939753in}}{\pgfqpoint{2.671789in}{3.935362in}}{\pgfqpoint{2.663975in}{3.927549in}}%
\pgfpathcurveto{\pgfqpoint{2.656162in}{3.919735in}}{\pgfqpoint{2.651771in}{3.909136in}}{\pgfqpoint{2.651771in}{3.898086in}}%
\pgfpathcurveto{\pgfqpoint{2.651771in}{3.887036in}}{\pgfqpoint{2.656162in}{3.876437in}}{\pgfqpoint{2.663975in}{3.868623in}}%
\pgfpathcurveto{\pgfqpoint{2.671789in}{3.860810in}}{\pgfqpoint{2.682388in}{3.856419in}}{\pgfqpoint{2.693438in}{3.856419in}}%
\pgfpathlineto{\pgfqpoint{2.693438in}{3.856419in}}%
\pgfpathclose%
\pgfusepath{stroke,fill}%
\end{pgfscope}%
\begin{pgfscope}%
\pgfpathrectangle{\pgfqpoint{0.800049in}{0.448486in}}{\pgfqpoint{3.531733in}{3.696000in}}%
\pgfusepath{clip}%
\pgfsetbuttcap%
\pgfsetroundjoin%
\definecolor{currentfill}{rgb}{0.894118,0.101961,0.109804}%
\pgfsetfillcolor{currentfill}%
\pgfsetfillopacity{0.600000}%
\pgfsetlinewidth{1.003750pt}%
\definecolor{currentstroke}{rgb}{0.894118,0.101961,0.109804}%
\pgfsetstrokecolor{currentstroke}%
\pgfsetstrokeopacity{0.600000}%
\pgfsetdash{}{0pt}%
\pgfpathmoveto{\pgfqpoint{2.693438in}{3.856419in}}%
\pgfpathcurveto{\pgfqpoint{2.704488in}{3.856419in}}{\pgfqpoint{2.715087in}{3.860810in}}{\pgfqpoint{2.722901in}{3.868623in}}%
\pgfpathcurveto{\pgfqpoint{2.730714in}{3.876437in}}{\pgfqpoint{2.735105in}{3.887036in}}{\pgfqpoint{2.735105in}{3.898086in}}%
\pgfpathcurveto{\pgfqpoint{2.735105in}{3.909136in}}{\pgfqpoint{2.730714in}{3.919735in}}{\pgfqpoint{2.722901in}{3.927549in}}%
\pgfpathcurveto{\pgfqpoint{2.715087in}{3.935362in}}{\pgfqpoint{2.704488in}{3.939753in}}{\pgfqpoint{2.693438in}{3.939753in}}%
\pgfpathcurveto{\pgfqpoint{2.682388in}{3.939753in}}{\pgfqpoint{2.671789in}{3.935362in}}{\pgfqpoint{2.663975in}{3.927549in}}%
\pgfpathcurveto{\pgfqpoint{2.656162in}{3.919735in}}{\pgfqpoint{2.651771in}{3.909136in}}{\pgfqpoint{2.651771in}{3.898086in}}%
\pgfpathcurveto{\pgfqpoint{2.651771in}{3.887036in}}{\pgfqpoint{2.656162in}{3.876437in}}{\pgfqpoint{2.663975in}{3.868623in}}%
\pgfpathcurveto{\pgfqpoint{2.671789in}{3.860810in}}{\pgfqpoint{2.682388in}{3.856419in}}{\pgfqpoint{2.693438in}{3.856419in}}%
\pgfpathlineto{\pgfqpoint{2.693438in}{3.856419in}}%
\pgfpathclose%
\pgfusepath{stroke,fill}%
\end{pgfscope}%
\begin{pgfscope}%
\pgfpathrectangle{\pgfqpoint{0.800049in}{0.448486in}}{\pgfqpoint{3.531733in}{3.696000in}}%
\pgfusepath{clip}%
\pgfsetbuttcap%
\pgfsetroundjoin%
\definecolor{currentfill}{rgb}{0.894118,0.101961,0.109804}%
\pgfsetfillcolor{currentfill}%
\pgfsetfillopacity{0.600000}%
\pgfsetlinewidth{1.003750pt}%
\definecolor{currentstroke}{rgb}{0.894118,0.101961,0.109804}%
\pgfsetstrokecolor{currentstroke}%
\pgfsetstrokeopacity{0.600000}%
\pgfsetdash{}{0pt}%
\pgfpathmoveto{\pgfqpoint{2.693438in}{3.856419in}}%
\pgfpathcurveto{\pgfqpoint{2.704488in}{3.856419in}}{\pgfqpoint{2.715087in}{3.860810in}}{\pgfqpoint{2.722901in}{3.868623in}}%
\pgfpathcurveto{\pgfqpoint{2.730714in}{3.876437in}}{\pgfqpoint{2.735105in}{3.887036in}}{\pgfqpoint{2.735105in}{3.898086in}}%
\pgfpathcurveto{\pgfqpoint{2.735105in}{3.909136in}}{\pgfqpoint{2.730714in}{3.919735in}}{\pgfqpoint{2.722901in}{3.927549in}}%
\pgfpathcurveto{\pgfqpoint{2.715087in}{3.935362in}}{\pgfqpoint{2.704488in}{3.939753in}}{\pgfqpoint{2.693438in}{3.939753in}}%
\pgfpathcurveto{\pgfqpoint{2.682388in}{3.939753in}}{\pgfqpoint{2.671789in}{3.935362in}}{\pgfqpoint{2.663975in}{3.927549in}}%
\pgfpathcurveto{\pgfqpoint{2.656162in}{3.919735in}}{\pgfqpoint{2.651771in}{3.909136in}}{\pgfqpoint{2.651771in}{3.898086in}}%
\pgfpathcurveto{\pgfqpoint{2.651771in}{3.887036in}}{\pgfqpoint{2.656162in}{3.876437in}}{\pgfqpoint{2.663975in}{3.868623in}}%
\pgfpathcurveto{\pgfqpoint{2.671789in}{3.860810in}}{\pgfqpoint{2.682388in}{3.856419in}}{\pgfqpoint{2.693438in}{3.856419in}}%
\pgfpathlineto{\pgfqpoint{2.693438in}{3.856419in}}%
\pgfpathclose%
\pgfusepath{stroke,fill}%
\end{pgfscope}%
\begin{pgfscope}%
\pgfpathrectangle{\pgfqpoint{0.800049in}{0.448486in}}{\pgfqpoint{3.531733in}{3.696000in}}%
\pgfusepath{clip}%
\pgfsetbuttcap%
\pgfsetroundjoin%
\definecolor{currentfill}{rgb}{0.894118,0.101961,0.109804}%
\pgfsetfillcolor{currentfill}%
\pgfsetfillopacity{0.600000}%
\pgfsetlinewidth{1.003750pt}%
\definecolor{currentstroke}{rgb}{0.894118,0.101961,0.109804}%
\pgfsetstrokecolor{currentstroke}%
\pgfsetstrokeopacity{0.600000}%
\pgfsetdash{}{0pt}%
\pgfpathmoveto{\pgfqpoint{2.693438in}{3.337682in}}%
\pgfpathcurveto{\pgfqpoint{2.704488in}{3.337682in}}{\pgfqpoint{2.715087in}{3.342073in}}{\pgfqpoint{2.722901in}{3.349886in}}%
\pgfpathcurveto{\pgfqpoint{2.730714in}{3.357700in}}{\pgfqpoint{2.735105in}{3.368299in}}{\pgfqpoint{2.735105in}{3.379349in}}%
\pgfpathcurveto{\pgfqpoint{2.735105in}{3.390399in}}{\pgfqpoint{2.730714in}{3.400998in}}{\pgfqpoint{2.722901in}{3.408812in}}%
\pgfpathcurveto{\pgfqpoint{2.715087in}{3.416626in}}{\pgfqpoint{2.704488in}{3.421016in}}{\pgfqpoint{2.693438in}{3.421016in}}%
\pgfpathcurveto{\pgfqpoint{2.682388in}{3.421016in}}{\pgfqpoint{2.671789in}{3.416626in}}{\pgfqpoint{2.663975in}{3.408812in}}%
\pgfpathcurveto{\pgfqpoint{2.656162in}{3.400998in}}{\pgfqpoint{2.651771in}{3.390399in}}{\pgfqpoint{2.651771in}{3.379349in}}%
\pgfpathcurveto{\pgfqpoint{2.651771in}{3.368299in}}{\pgfqpoint{2.656162in}{3.357700in}}{\pgfqpoint{2.663975in}{3.349886in}}%
\pgfpathcurveto{\pgfqpoint{2.671789in}{3.342073in}}{\pgfqpoint{2.682388in}{3.337682in}}{\pgfqpoint{2.693438in}{3.337682in}}%
\pgfpathlineto{\pgfqpoint{2.693438in}{3.337682in}}%
\pgfpathclose%
\pgfusepath{stroke,fill}%
\end{pgfscope}%
\begin{pgfscope}%
\pgfpathrectangle{\pgfqpoint{0.800049in}{0.448486in}}{\pgfqpoint{3.531733in}{3.696000in}}%
\pgfusepath{clip}%
\pgfsetbuttcap%
\pgfsetroundjoin%
\definecolor{currentfill}{rgb}{0.894118,0.101961,0.109804}%
\pgfsetfillcolor{currentfill}%
\pgfsetfillopacity{0.600000}%
\pgfsetlinewidth{1.003750pt}%
\definecolor{currentstroke}{rgb}{0.894118,0.101961,0.109804}%
\pgfsetstrokecolor{currentstroke}%
\pgfsetstrokeopacity{0.600000}%
\pgfsetdash{}{0pt}%
\pgfpathmoveto{\pgfqpoint{2.693438in}{3.337682in}}%
\pgfpathcurveto{\pgfqpoint{2.704488in}{3.337682in}}{\pgfqpoint{2.715087in}{3.342073in}}{\pgfqpoint{2.722901in}{3.349886in}}%
\pgfpathcurveto{\pgfqpoint{2.730714in}{3.357700in}}{\pgfqpoint{2.735105in}{3.368299in}}{\pgfqpoint{2.735105in}{3.379349in}}%
\pgfpathcurveto{\pgfqpoint{2.735105in}{3.390399in}}{\pgfqpoint{2.730714in}{3.400998in}}{\pgfqpoint{2.722901in}{3.408812in}}%
\pgfpathcurveto{\pgfqpoint{2.715087in}{3.416626in}}{\pgfqpoint{2.704488in}{3.421016in}}{\pgfqpoint{2.693438in}{3.421016in}}%
\pgfpathcurveto{\pgfqpoint{2.682388in}{3.421016in}}{\pgfqpoint{2.671789in}{3.416626in}}{\pgfqpoint{2.663975in}{3.408812in}}%
\pgfpathcurveto{\pgfqpoint{2.656162in}{3.400998in}}{\pgfqpoint{2.651771in}{3.390399in}}{\pgfqpoint{2.651771in}{3.379349in}}%
\pgfpathcurveto{\pgfqpoint{2.651771in}{3.368299in}}{\pgfqpoint{2.656162in}{3.357700in}}{\pgfqpoint{2.663975in}{3.349886in}}%
\pgfpathcurveto{\pgfqpoint{2.671789in}{3.342073in}}{\pgfqpoint{2.682388in}{3.337682in}}{\pgfqpoint{2.693438in}{3.337682in}}%
\pgfpathlineto{\pgfqpoint{2.693438in}{3.337682in}}%
\pgfpathclose%
\pgfusepath{stroke,fill}%
\end{pgfscope}%
\begin{pgfscope}%
\pgfpathrectangle{\pgfqpoint{0.800049in}{0.448486in}}{\pgfqpoint{3.531733in}{3.696000in}}%
\pgfusepath{clip}%
\pgfsetbuttcap%
\pgfsetroundjoin%
\definecolor{currentfill}{rgb}{0.894118,0.101961,0.109804}%
\pgfsetfillcolor{currentfill}%
\pgfsetfillopacity{0.600000}%
\pgfsetlinewidth{1.003750pt}%
\definecolor{currentstroke}{rgb}{0.894118,0.101961,0.109804}%
\pgfsetstrokecolor{currentstroke}%
\pgfsetstrokeopacity{0.600000}%
\pgfsetdash{}{0pt}%
\pgfpathmoveto{\pgfqpoint{2.693438in}{3.337682in}}%
\pgfpathcurveto{\pgfqpoint{2.704488in}{3.337682in}}{\pgfqpoint{2.715087in}{3.342073in}}{\pgfqpoint{2.722901in}{3.349886in}}%
\pgfpathcurveto{\pgfqpoint{2.730714in}{3.357700in}}{\pgfqpoint{2.735105in}{3.368299in}}{\pgfqpoint{2.735105in}{3.379349in}}%
\pgfpathcurveto{\pgfqpoint{2.735105in}{3.390399in}}{\pgfqpoint{2.730714in}{3.400998in}}{\pgfqpoint{2.722901in}{3.408812in}}%
\pgfpathcurveto{\pgfqpoint{2.715087in}{3.416626in}}{\pgfqpoint{2.704488in}{3.421016in}}{\pgfqpoint{2.693438in}{3.421016in}}%
\pgfpathcurveto{\pgfqpoint{2.682388in}{3.421016in}}{\pgfqpoint{2.671789in}{3.416626in}}{\pgfqpoint{2.663975in}{3.408812in}}%
\pgfpathcurveto{\pgfqpoint{2.656162in}{3.400998in}}{\pgfqpoint{2.651771in}{3.390399in}}{\pgfqpoint{2.651771in}{3.379349in}}%
\pgfpathcurveto{\pgfqpoint{2.651771in}{3.368299in}}{\pgfqpoint{2.656162in}{3.357700in}}{\pgfqpoint{2.663975in}{3.349886in}}%
\pgfpathcurveto{\pgfqpoint{2.671789in}{3.342073in}}{\pgfqpoint{2.682388in}{3.337682in}}{\pgfqpoint{2.693438in}{3.337682in}}%
\pgfpathlineto{\pgfqpoint{2.693438in}{3.337682in}}%
\pgfpathclose%
\pgfusepath{stroke,fill}%
\end{pgfscope}%
\begin{pgfscope}%
\pgfpathrectangle{\pgfqpoint{0.800049in}{0.448486in}}{\pgfqpoint{3.531733in}{3.696000in}}%
\pgfusepath{clip}%
\pgfsetbuttcap%
\pgfsetroundjoin%
\definecolor{currentfill}{rgb}{0.894118,0.101961,0.109804}%
\pgfsetfillcolor{currentfill}%
\pgfsetfillopacity{0.600000}%
\pgfsetlinewidth{1.003750pt}%
\definecolor{currentstroke}{rgb}{0.894118,0.101961,0.109804}%
\pgfsetstrokecolor{currentstroke}%
\pgfsetstrokeopacity{0.600000}%
\pgfsetdash{}{0pt}%
\pgfpathmoveto{\pgfqpoint{2.693438in}{3.337682in}}%
\pgfpathcurveto{\pgfqpoint{2.704488in}{3.337682in}}{\pgfqpoint{2.715087in}{3.342073in}}{\pgfqpoint{2.722901in}{3.349886in}}%
\pgfpathcurveto{\pgfqpoint{2.730714in}{3.357700in}}{\pgfqpoint{2.735105in}{3.368299in}}{\pgfqpoint{2.735105in}{3.379349in}}%
\pgfpathcurveto{\pgfqpoint{2.735105in}{3.390399in}}{\pgfqpoint{2.730714in}{3.400998in}}{\pgfqpoint{2.722901in}{3.408812in}}%
\pgfpathcurveto{\pgfqpoint{2.715087in}{3.416626in}}{\pgfqpoint{2.704488in}{3.421016in}}{\pgfqpoint{2.693438in}{3.421016in}}%
\pgfpathcurveto{\pgfqpoint{2.682388in}{3.421016in}}{\pgfqpoint{2.671789in}{3.416626in}}{\pgfqpoint{2.663975in}{3.408812in}}%
\pgfpathcurveto{\pgfqpoint{2.656162in}{3.400998in}}{\pgfqpoint{2.651771in}{3.390399in}}{\pgfqpoint{2.651771in}{3.379349in}}%
\pgfpathcurveto{\pgfqpoint{2.651771in}{3.368299in}}{\pgfqpoint{2.656162in}{3.357700in}}{\pgfqpoint{2.663975in}{3.349886in}}%
\pgfpathcurveto{\pgfqpoint{2.671789in}{3.342073in}}{\pgfqpoint{2.682388in}{3.337682in}}{\pgfqpoint{2.693438in}{3.337682in}}%
\pgfpathlineto{\pgfqpoint{2.693438in}{3.337682in}}%
\pgfpathclose%
\pgfusepath{stroke,fill}%
\end{pgfscope}%
\begin{pgfscope}%
\pgfpathrectangle{\pgfqpoint{0.800049in}{0.448486in}}{\pgfqpoint{3.531733in}{3.696000in}}%
\pgfusepath{clip}%
\pgfsetbuttcap%
\pgfsetroundjoin%
\definecolor{currentfill}{rgb}{0.894118,0.101961,0.109804}%
\pgfsetfillcolor{currentfill}%
\pgfsetfillopacity{0.600000}%
\pgfsetlinewidth{1.003750pt}%
\definecolor{currentstroke}{rgb}{0.894118,0.101961,0.109804}%
\pgfsetstrokecolor{currentstroke}%
\pgfsetstrokeopacity{0.600000}%
\pgfsetdash{}{0pt}%
\pgfpathmoveto{\pgfqpoint{2.693438in}{3.337682in}}%
\pgfpathcurveto{\pgfqpoint{2.704488in}{3.337682in}}{\pgfqpoint{2.715087in}{3.342073in}}{\pgfqpoint{2.722901in}{3.349886in}}%
\pgfpathcurveto{\pgfqpoint{2.730714in}{3.357700in}}{\pgfqpoint{2.735105in}{3.368299in}}{\pgfqpoint{2.735105in}{3.379349in}}%
\pgfpathcurveto{\pgfqpoint{2.735105in}{3.390399in}}{\pgfqpoint{2.730714in}{3.400998in}}{\pgfqpoint{2.722901in}{3.408812in}}%
\pgfpathcurveto{\pgfqpoint{2.715087in}{3.416626in}}{\pgfqpoint{2.704488in}{3.421016in}}{\pgfqpoint{2.693438in}{3.421016in}}%
\pgfpathcurveto{\pgfqpoint{2.682388in}{3.421016in}}{\pgfqpoint{2.671789in}{3.416626in}}{\pgfqpoint{2.663975in}{3.408812in}}%
\pgfpathcurveto{\pgfqpoint{2.656162in}{3.400998in}}{\pgfqpoint{2.651771in}{3.390399in}}{\pgfqpoint{2.651771in}{3.379349in}}%
\pgfpathcurveto{\pgfqpoint{2.651771in}{3.368299in}}{\pgfqpoint{2.656162in}{3.357700in}}{\pgfqpoint{2.663975in}{3.349886in}}%
\pgfpathcurveto{\pgfqpoint{2.671789in}{3.342073in}}{\pgfqpoint{2.682388in}{3.337682in}}{\pgfqpoint{2.693438in}{3.337682in}}%
\pgfpathlineto{\pgfqpoint{2.693438in}{3.337682in}}%
\pgfpathclose%
\pgfusepath{stroke,fill}%
\end{pgfscope}%
\begin{pgfscope}%
\pgfpathrectangle{\pgfqpoint{0.800049in}{0.448486in}}{\pgfqpoint{3.531733in}{3.696000in}}%
\pgfusepath{clip}%
\pgfsetbuttcap%
\pgfsetroundjoin%
\definecolor{currentfill}{rgb}{0.894118,0.101961,0.109804}%
\pgfsetfillcolor{currentfill}%
\pgfsetfillopacity{0.600000}%
\pgfsetlinewidth{1.003750pt}%
\definecolor{currentstroke}{rgb}{0.894118,0.101961,0.109804}%
\pgfsetstrokecolor{currentstroke}%
\pgfsetstrokeopacity{0.600000}%
\pgfsetdash{}{0pt}%
\pgfpathmoveto{\pgfqpoint{2.693438in}{2.818946in}}%
\pgfpathcurveto{\pgfqpoint{2.704488in}{2.818946in}}{\pgfqpoint{2.715087in}{2.823336in}}{\pgfqpoint{2.722901in}{2.831150in}}%
\pgfpathcurveto{\pgfqpoint{2.730714in}{2.838963in}}{\pgfqpoint{2.735105in}{2.849562in}}{\pgfqpoint{2.735105in}{2.860612in}}%
\pgfpathcurveto{\pgfqpoint{2.735105in}{2.871662in}}{\pgfqpoint{2.730714in}{2.882261in}}{\pgfqpoint{2.722901in}{2.890075in}}%
\pgfpathcurveto{\pgfqpoint{2.715087in}{2.897889in}}{\pgfqpoint{2.704488in}{2.902279in}}{\pgfqpoint{2.693438in}{2.902279in}}%
\pgfpathcurveto{\pgfqpoint{2.682388in}{2.902279in}}{\pgfqpoint{2.671789in}{2.897889in}}{\pgfqpoint{2.663975in}{2.890075in}}%
\pgfpathcurveto{\pgfqpoint{2.656162in}{2.882261in}}{\pgfqpoint{2.651771in}{2.871662in}}{\pgfqpoint{2.651771in}{2.860612in}}%
\pgfpathcurveto{\pgfqpoint{2.651771in}{2.849562in}}{\pgfqpoint{2.656162in}{2.838963in}}{\pgfqpoint{2.663975in}{2.831150in}}%
\pgfpathcurveto{\pgfqpoint{2.671789in}{2.823336in}}{\pgfqpoint{2.682388in}{2.818946in}}{\pgfqpoint{2.693438in}{2.818946in}}%
\pgfpathlineto{\pgfqpoint{2.693438in}{2.818946in}}%
\pgfpathclose%
\pgfusepath{stroke,fill}%
\end{pgfscope}%
\begin{pgfscope}%
\pgfpathrectangle{\pgfqpoint{0.800049in}{0.448486in}}{\pgfqpoint{3.531733in}{3.696000in}}%
\pgfusepath{clip}%
\pgfsetbuttcap%
\pgfsetroundjoin%
\definecolor{currentfill}{rgb}{0.894118,0.101961,0.109804}%
\pgfsetfillcolor{currentfill}%
\pgfsetfillopacity{0.600000}%
\pgfsetlinewidth{1.003750pt}%
\definecolor{currentstroke}{rgb}{0.894118,0.101961,0.109804}%
\pgfsetstrokecolor{currentstroke}%
\pgfsetstrokeopacity{0.600000}%
\pgfsetdash{}{0pt}%
\pgfpathmoveto{\pgfqpoint{2.693438in}{2.818946in}}%
\pgfpathcurveto{\pgfqpoint{2.704488in}{2.818946in}}{\pgfqpoint{2.715087in}{2.823336in}}{\pgfqpoint{2.722901in}{2.831150in}}%
\pgfpathcurveto{\pgfqpoint{2.730714in}{2.838963in}}{\pgfqpoint{2.735105in}{2.849562in}}{\pgfqpoint{2.735105in}{2.860612in}}%
\pgfpathcurveto{\pgfqpoint{2.735105in}{2.871662in}}{\pgfqpoint{2.730714in}{2.882261in}}{\pgfqpoint{2.722901in}{2.890075in}}%
\pgfpathcurveto{\pgfqpoint{2.715087in}{2.897889in}}{\pgfqpoint{2.704488in}{2.902279in}}{\pgfqpoint{2.693438in}{2.902279in}}%
\pgfpathcurveto{\pgfqpoint{2.682388in}{2.902279in}}{\pgfqpoint{2.671789in}{2.897889in}}{\pgfqpoint{2.663975in}{2.890075in}}%
\pgfpathcurveto{\pgfqpoint{2.656162in}{2.882261in}}{\pgfqpoint{2.651771in}{2.871662in}}{\pgfqpoint{2.651771in}{2.860612in}}%
\pgfpathcurveto{\pgfqpoint{2.651771in}{2.849562in}}{\pgfqpoint{2.656162in}{2.838963in}}{\pgfqpoint{2.663975in}{2.831150in}}%
\pgfpathcurveto{\pgfqpoint{2.671789in}{2.823336in}}{\pgfqpoint{2.682388in}{2.818946in}}{\pgfqpoint{2.693438in}{2.818946in}}%
\pgfpathlineto{\pgfqpoint{2.693438in}{2.818946in}}%
\pgfpathclose%
\pgfusepath{stroke,fill}%
\end{pgfscope}%
\begin{pgfscope}%
\pgfpathrectangle{\pgfqpoint{0.800049in}{0.448486in}}{\pgfqpoint{3.531733in}{3.696000in}}%
\pgfusepath{clip}%
\pgfsetbuttcap%
\pgfsetroundjoin%
\definecolor{currentfill}{rgb}{0.894118,0.101961,0.109804}%
\pgfsetfillcolor{currentfill}%
\pgfsetfillopacity{0.600000}%
\pgfsetlinewidth{1.003750pt}%
\definecolor{currentstroke}{rgb}{0.894118,0.101961,0.109804}%
\pgfsetstrokecolor{currentstroke}%
\pgfsetstrokeopacity{0.600000}%
\pgfsetdash{}{0pt}%
\pgfpathmoveto{\pgfqpoint{2.693438in}{2.818946in}}%
\pgfpathcurveto{\pgfqpoint{2.704488in}{2.818946in}}{\pgfqpoint{2.715087in}{2.823336in}}{\pgfqpoint{2.722901in}{2.831150in}}%
\pgfpathcurveto{\pgfqpoint{2.730714in}{2.838963in}}{\pgfqpoint{2.735105in}{2.849562in}}{\pgfqpoint{2.735105in}{2.860612in}}%
\pgfpathcurveto{\pgfqpoint{2.735105in}{2.871662in}}{\pgfqpoint{2.730714in}{2.882261in}}{\pgfqpoint{2.722901in}{2.890075in}}%
\pgfpathcurveto{\pgfqpoint{2.715087in}{2.897889in}}{\pgfqpoint{2.704488in}{2.902279in}}{\pgfqpoint{2.693438in}{2.902279in}}%
\pgfpathcurveto{\pgfqpoint{2.682388in}{2.902279in}}{\pgfqpoint{2.671789in}{2.897889in}}{\pgfqpoint{2.663975in}{2.890075in}}%
\pgfpathcurveto{\pgfqpoint{2.656162in}{2.882261in}}{\pgfqpoint{2.651771in}{2.871662in}}{\pgfqpoint{2.651771in}{2.860612in}}%
\pgfpathcurveto{\pgfqpoint{2.651771in}{2.849562in}}{\pgfqpoint{2.656162in}{2.838963in}}{\pgfqpoint{2.663975in}{2.831150in}}%
\pgfpathcurveto{\pgfqpoint{2.671789in}{2.823336in}}{\pgfqpoint{2.682388in}{2.818946in}}{\pgfqpoint{2.693438in}{2.818946in}}%
\pgfpathlineto{\pgfqpoint{2.693438in}{2.818946in}}%
\pgfpathclose%
\pgfusepath{stroke,fill}%
\end{pgfscope}%
\begin{pgfscope}%
\pgfpathrectangle{\pgfqpoint{0.800049in}{0.448486in}}{\pgfqpoint{3.531733in}{3.696000in}}%
\pgfusepath{clip}%
\pgfsetbuttcap%
\pgfsetroundjoin%
\definecolor{currentfill}{rgb}{0.894118,0.101961,0.109804}%
\pgfsetfillcolor{currentfill}%
\pgfsetfillopacity{0.600000}%
\pgfsetlinewidth{1.003750pt}%
\definecolor{currentstroke}{rgb}{0.894118,0.101961,0.109804}%
\pgfsetstrokecolor{currentstroke}%
\pgfsetstrokeopacity{0.600000}%
\pgfsetdash{}{0pt}%
\pgfpathmoveto{\pgfqpoint{2.693438in}{2.818946in}}%
\pgfpathcurveto{\pgfqpoint{2.704488in}{2.818946in}}{\pgfqpoint{2.715087in}{2.823336in}}{\pgfqpoint{2.722901in}{2.831150in}}%
\pgfpathcurveto{\pgfqpoint{2.730714in}{2.838963in}}{\pgfqpoint{2.735105in}{2.849562in}}{\pgfqpoint{2.735105in}{2.860612in}}%
\pgfpathcurveto{\pgfqpoint{2.735105in}{2.871662in}}{\pgfqpoint{2.730714in}{2.882261in}}{\pgfqpoint{2.722901in}{2.890075in}}%
\pgfpathcurveto{\pgfqpoint{2.715087in}{2.897889in}}{\pgfqpoint{2.704488in}{2.902279in}}{\pgfqpoint{2.693438in}{2.902279in}}%
\pgfpathcurveto{\pgfqpoint{2.682388in}{2.902279in}}{\pgfqpoint{2.671789in}{2.897889in}}{\pgfqpoint{2.663975in}{2.890075in}}%
\pgfpathcurveto{\pgfqpoint{2.656162in}{2.882261in}}{\pgfqpoint{2.651771in}{2.871662in}}{\pgfqpoint{2.651771in}{2.860612in}}%
\pgfpathcurveto{\pgfqpoint{2.651771in}{2.849562in}}{\pgfqpoint{2.656162in}{2.838963in}}{\pgfqpoint{2.663975in}{2.831150in}}%
\pgfpathcurveto{\pgfqpoint{2.671789in}{2.823336in}}{\pgfqpoint{2.682388in}{2.818946in}}{\pgfqpoint{2.693438in}{2.818946in}}%
\pgfpathlineto{\pgfqpoint{2.693438in}{2.818946in}}%
\pgfpathclose%
\pgfusepath{stroke,fill}%
\end{pgfscope}%
\begin{pgfscope}%
\pgfpathrectangle{\pgfqpoint{0.800049in}{0.448486in}}{\pgfqpoint{3.531733in}{3.696000in}}%
\pgfusepath{clip}%
\pgfsetbuttcap%
\pgfsetroundjoin%
\definecolor{currentfill}{rgb}{0.894118,0.101961,0.109804}%
\pgfsetfillcolor{currentfill}%
\pgfsetfillopacity{0.600000}%
\pgfsetlinewidth{1.003750pt}%
\definecolor{currentstroke}{rgb}{0.894118,0.101961,0.109804}%
\pgfsetstrokecolor{currentstroke}%
\pgfsetstrokeopacity{0.600000}%
\pgfsetdash{}{0pt}%
\pgfpathmoveto{\pgfqpoint{2.693438in}{2.818946in}}%
\pgfpathcurveto{\pgfqpoint{2.704488in}{2.818946in}}{\pgfqpoint{2.715087in}{2.823336in}}{\pgfqpoint{2.722901in}{2.831150in}}%
\pgfpathcurveto{\pgfqpoint{2.730714in}{2.838963in}}{\pgfqpoint{2.735105in}{2.849562in}}{\pgfqpoint{2.735105in}{2.860612in}}%
\pgfpathcurveto{\pgfqpoint{2.735105in}{2.871662in}}{\pgfqpoint{2.730714in}{2.882261in}}{\pgfqpoint{2.722901in}{2.890075in}}%
\pgfpathcurveto{\pgfqpoint{2.715087in}{2.897889in}}{\pgfqpoint{2.704488in}{2.902279in}}{\pgfqpoint{2.693438in}{2.902279in}}%
\pgfpathcurveto{\pgfqpoint{2.682388in}{2.902279in}}{\pgfqpoint{2.671789in}{2.897889in}}{\pgfqpoint{2.663975in}{2.890075in}}%
\pgfpathcurveto{\pgfqpoint{2.656162in}{2.882261in}}{\pgfqpoint{2.651771in}{2.871662in}}{\pgfqpoint{2.651771in}{2.860612in}}%
\pgfpathcurveto{\pgfqpoint{2.651771in}{2.849562in}}{\pgfqpoint{2.656162in}{2.838963in}}{\pgfqpoint{2.663975in}{2.831150in}}%
\pgfpathcurveto{\pgfqpoint{2.671789in}{2.823336in}}{\pgfqpoint{2.682388in}{2.818946in}}{\pgfqpoint{2.693438in}{2.818946in}}%
\pgfpathlineto{\pgfqpoint{2.693438in}{2.818946in}}%
\pgfpathclose%
\pgfusepath{stroke,fill}%
\end{pgfscope}%
\begin{pgfscope}%
\pgfpathrectangle{\pgfqpoint{0.800049in}{0.448486in}}{\pgfqpoint{3.531733in}{3.696000in}}%
\pgfusepath{clip}%
\pgfsetbuttcap%
\pgfsetroundjoin%
\definecolor{currentfill}{rgb}{0.894118,0.101961,0.109804}%
\pgfsetfillcolor{currentfill}%
\pgfsetfillopacity{0.600000}%
\pgfsetlinewidth{1.003750pt}%
\definecolor{currentstroke}{rgb}{0.894118,0.101961,0.109804}%
\pgfsetstrokecolor{currentstroke}%
\pgfsetstrokeopacity{0.600000}%
\pgfsetdash{}{0pt}%
\pgfpathmoveto{\pgfqpoint{2.693438in}{2.818946in}}%
\pgfpathcurveto{\pgfqpoint{2.704488in}{2.818946in}}{\pgfqpoint{2.715087in}{2.823336in}}{\pgfqpoint{2.722901in}{2.831150in}}%
\pgfpathcurveto{\pgfqpoint{2.730714in}{2.838963in}}{\pgfqpoint{2.735105in}{2.849562in}}{\pgfqpoint{2.735105in}{2.860612in}}%
\pgfpathcurveto{\pgfqpoint{2.735105in}{2.871662in}}{\pgfqpoint{2.730714in}{2.882261in}}{\pgfqpoint{2.722901in}{2.890075in}}%
\pgfpathcurveto{\pgfqpoint{2.715087in}{2.897889in}}{\pgfqpoint{2.704488in}{2.902279in}}{\pgfqpoint{2.693438in}{2.902279in}}%
\pgfpathcurveto{\pgfqpoint{2.682388in}{2.902279in}}{\pgfqpoint{2.671789in}{2.897889in}}{\pgfqpoint{2.663975in}{2.890075in}}%
\pgfpathcurveto{\pgfqpoint{2.656162in}{2.882261in}}{\pgfqpoint{2.651771in}{2.871662in}}{\pgfqpoint{2.651771in}{2.860612in}}%
\pgfpathcurveto{\pgfqpoint{2.651771in}{2.849562in}}{\pgfqpoint{2.656162in}{2.838963in}}{\pgfqpoint{2.663975in}{2.831150in}}%
\pgfpathcurveto{\pgfqpoint{2.671789in}{2.823336in}}{\pgfqpoint{2.682388in}{2.818946in}}{\pgfqpoint{2.693438in}{2.818946in}}%
\pgfpathlineto{\pgfqpoint{2.693438in}{2.818946in}}%
\pgfpathclose%
\pgfusepath{stroke,fill}%
\end{pgfscope}%
\begin{pgfscope}%
\pgfpathrectangle{\pgfqpoint{0.800049in}{0.448486in}}{\pgfqpoint{3.531733in}{3.696000in}}%
\pgfusepath{clip}%
\pgfsetbuttcap%
\pgfsetroundjoin%
\definecolor{currentfill}{rgb}{0.894118,0.101961,0.109804}%
\pgfsetfillcolor{currentfill}%
\pgfsetfillopacity{0.600000}%
\pgfsetlinewidth{1.003750pt}%
\definecolor{currentstroke}{rgb}{0.894118,0.101961,0.109804}%
\pgfsetstrokecolor{currentstroke}%
\pgfsetstrokeopacity{0.600000}%
\pgfsetdash{}{0pt}%
\pgfpathmoveto{\pgfqpoint{2.693438in}{2.818946in}}%
\pgfpathcurveto{\pgfqpoint{2.704488in}{2.818946in}}{\pgfqpoint{2.715087in}{2.823336in}}{\pgfqpoint{2.722901in}{2.831150in}}%
\pgfpathcurveto{\pgfqpoint{2.730714in}{2.838963in}}{\pgfqpoint{2.735105in}{2.849562in}}{\pgfqpoint{2.735105in}{2.860612in}}%
\pgfpathcurveto{\pgfqpoint{2.735105in}{2.871662in}}{\pgfqpoint{2.730714in}{2.882261in}}{\pgfqpoint{2.722901in}{2.890075in}}%
\pgfpathcurveto{\pgfqpoint{2.715087in}{2.897889in}}{\pgfqpoint{2.704488in}{2.902279in}}{\pgfqpoint{2.693438in}{2.902279in}}%
\pgfpathcurveto{\pgfqpoint{2.682388in}{2.902279in}}{\pgfqpoint{2.671789in}{2.897889in}}{\pgfqpoint{2.663975in}{2.890075in}}%
\pgfpathcurveto{\pgfqpoint{2.656162in}{2.882261in}}{\pgfqpoint{2.651771in}{2.871662in}}{\pgfqpoint{2.651771in}{2.860612in}}%
\pgfpathcurveto{\pgfqpoint{2.651771in}{2.849562in}}{\pgfqpoint{2.656162in}{2.838963in}}{\pgfqpoint{2.663975in}{2.831150in}}%
\pgfpathcurveto{\pgfqpoint{2.671789in}{2.823336in}}{\pgfqpoint{2.682388in}{2.818946in}}{\pgfqpoint{2.693438in}{2.818946in}}%
\pgfpathlineto{\pgfqpoint{2.693438in}{2.818946in}}%
\pgfpathclose%
\pgfusepath{stroke,fill}%
\end{pgfscope}%
\begin{pgfscope}%
\pgfpathrectangle{\pgfqpoint{0.800049in}{0.448486in}}{\pgfqpoint{3.531733in}{3.696000in}}%
\pgfusepath{clip}%
\pgfsetbuttcap%
\pgfsetroundjoin%
\definecolor{currentfill}{rgb}{0.894118,0.101961,0.109804}%
\pgfsetfillcolor{currentfill}%
\pgfsetfillopacity{0.600000}%
\pgfsetlinewidth{1.003750pt}%
\definecolor{currentstroke}{rgb}{0.894118,0.101961,0.109804}%
\pgfsetstrokecolor{currentstroke}%
\pgfsetstrokeopacity{0.600000}%
\pgfsetdash{}{0pt}%
\pgfpathmoveto{\pgfqpoint{2.174701in}{4.020686in}}%
\pgfpathcurveto{\pgfqpoint{2.185751in}{4.020686in}}{\pgfqpoint{2.196350in}{4.025076in}}{\pgfqpoint{2.204164in}{4.032890in}}%
\pgfpathcurveto{\pgfqpoint{2.211978in}{4.040703in}}{\pgfqpoint{2.216368in}{4.051303in}}{\pgfqpoint{2.216368in}{4.062353in}}%
\pgfpathcurveto{\pgfqpoint{2.216368in}{4.073403in}}{\pgfqpoint{2.211978in}{4.084002in}}{\pgfqpoint{2.204164in}{4.091815in}}%
\pgfpathcurveto{\pgfqpoint{2.196350in}{4.099629in}}{\pgfqpoint{2.185751in}{4.104019in}}{\pgfqpoint{2.174701in}{4.104019in}}%
\pgfpathcurveto{\pgfqpoint{2.163651in}{4.104019in}}{\pgfqpoint{2.153052in}{4.099629in}}{\pgfqpoint{2.145238in}{4.091815in}}%
\pgfpathcurveto{\pgfqpoint{2.137425in}{4.084002in}}{\pgfqpoint{2.133035in}{4.073403in}}{\pgfqpoint{2.133035in}{4.062353in}}%
\pgfpathcurveto{\pgfqpoint{2.133035in}{4.051303in}}{\pgfqpoint{2.137425in}{4.040703in}}{\pgfqpoint{2.145238in}{4.032890in}}%
\pgfpathcurveto{\pgfqpoint{2.153052in}{4.025076in}}{\pgfqpoint{2.163651in}{4.020686in}}{\pgfqpoint{2.174701in}{4.020686in}}%
\pgfpathlineto{\pgfqpoint{2.174701in}{4.020686in}}%
\pgfpathclose%
\pgfusepath{stroke,fill}%
\end{pgfscope}%
\begin{pgfscope}%
\pgfpathrectangle{\pgfqpoint{0.800049in}{0.448486in}}{\pgfqpoint{3.531733in}{3.696000in}}%
\pgfusepath{clip}%
\pgfsetbuttcap%
\pgfsetroundjoin%
\definecolor{currentfill}{rgb}{0.894118,0.101961,0.109804}%
\pgfsetfillcolor{currentfill}%
\pgfsetfillopacity{0.600000}%
\pgfsetlinewidth{1.003750pt}%
\definecolor{currentstroke}{rgb}{0.894118,0.101961,0.109804}%
\pgfsetstrokecolor{currentstroke}%
\pgfsetstrokeopacity{0.600000}%
\pgfsetdash{}{0pt}%
\pgfpathmoveto{\pgfqpoint{2.174701in}{4.020686in}}%
\pgfpathcurveto{\pgfqpoint{2.185751in}{4.020686in}}{\pgfqpoint{2.196350in}{4.025076in}}{\pgfqpoint{2.204164in}{4.032890in}}%
\pgfpathcurveto{\pgfqpoint{2.211978in}{4.040703in}}{\pgfqpoint{2.216368in}{4.051303in}}{\pgfqpoint{2.216368in}{4.062353in}}%
\pgfpathcurveto{\pgfqpoint{2.216368in}{4.073403in}}{\pgfqpoint{2.211978in}{4.084002in}}{\pgfqpoint{2.204164in}{4.091815in}}%
\pgfpathcurveto{\pgfqpoint{2.196350in}{4.099629in}}{\pgfqpoint{2.185751in}{4.104019in}}{\pgfqpoint{2.174701in}{4.104019in}}%
\pgfpathcurveto{\pgfqpoint{2.163651in}{4.104019in}}{\pgfqpoint{2.153052in}{4.099629in}}{\pgfqpoint{2.145238in}{4.091815in}}%
\pgfpathcurveto{\pgfqpoint{2.137425in}{4.084002in}}{\pgfqpoint{2.133035in}{4.073403in}}{\pgfqpoint{2.133035in}{4.062353in}}%
\pgfpathcurveto{\pgfqpoint{2.133035in}{4.051303in}}{\pgfqpoint{2.137425in}{4.040703in}}{\pgfqpoint{2.145238in}{4.032890in}}%
\pgfpathcurveto{\pgfqpoint{2.153052in}{4.025076in}}{\pgfqpoint{2.163651in}{4.020686in}}{\pgfqpoint{2.174701in}{4.020686in}}%
\pgfpathlineto{\pgfqpoint{2.174701in}{4.020686in}}%
\pgfpathclose%
\pgfusepath{stroke,fill}%
\end{pgfscope}%
\begin{pgfscope}%
\pgfpathrectangle{\pgfqpoint{0.800049in}{0.448486in}}{\pgfqpoint{3.531733in}{3.696000in}}%
\pgfusepath{clip}%
\pgfsetbuttcap%
\pgfsetroundjoin%
\definecolor{currentfill}{rgb}{0.894118,0.101961,0.109804}%
\pgfsetfillcolor{currentfill}%
\pgfsetfillopacity{0.600000}%
\pgfsetlinewidth{1.003750pt}%
\definecolor{currentstroke}{rgb}{0.894118,0.101961,0.109804}%
\pgfsetstrokecolor{currentstroke}%
\pgfsetstrokeopacity{0.600000}%
\pgfsetdash{}{0pt}%
\pgfpathmoveto{\pgfqpoint{2.174701in}{4.020686in}}%
\pgfpathcurveto{\pgfqpoint{2.185751in}{4.020686in}}{\pgfqpoint{2.196350in}{4.025076in}}{\pgfqpoint{2.204164in}{4.032890in}}%
\pgfpathcurveto{\pgfqpoint{2.211978in}{4.040703in}}{\pgfqpoint{2.216368in}{4.051303in}}{\pgfqpoint{2.216368in}{4.062353in}}%
\pgfpathcurveto{\pgfqpoint{2.216368in}{4.073403in}}{\pgfqpoint{2.211978in}{4.084002in}}{\pgfqpoint{2.204164in}{4.091815in}}%
\pgfpathcurveto{\pgfqpoint{2.196350in}{4.099629in}}{\pgfqpoint{2.185751in}{4.104019in}}{\pgfqpoint{2.174701in}{4.104019in}}%
\pgfpathcurveto{\pgfqpoint{2.163651in}{4.104019in}}{\pgfqpoint{2.153052in}{4.099629in}}{\pgfqpoint{2.145238in}{4.091815in}}%
\pgfpathcurveto{\pgfqpoint{2.137425in}{4.084002in}}{\pgfqpoint{2.133035in}{4.073403in}}{\pgfqpoint{2.133035in}{4.062353in}}%
\pgfpathcurveto{\pgfqpoint{2.133035in}{4.051303in}}{\pgfqpoint{2.137425in}{4.040703in}}{\pgfqpoint{2.145238in}{4.032890in}}%
\pgfpathcurveto{\pgfqpoint{2.153052in}{4.025076in}}{\pgfqpoint{2.163651in}{4.020686in}}{\pgfqpoint{2.174701in}{4.020686in}}%
\pgfpathlineto{\pgfqpoint{2.174701in}{4.020686in}}%
\pgfpathclose%
\pgfusepath{stroke,fill}%
\end{pgfscope}%
\begin{pgfscope}%
\pgfpathrectangle{\pgfqpoint{0.800049in}{0.448486in}}{\pgfqpoint{3.531733in}{3.696000in}}%
\pgfusepath{clip}%
\pgfsetbuttcap%
\pgfsetroundjoin%
\definecolor{currentfill}{rgb}{0.894118,0.101961,0.109804}%
\pgfsetfillcolor{currentfill}%
\pgfsetfillopacity{0.600000}%
\pgfsetlinewidth{1.003750pt}%
\definecolor{currentstroke}{rgb}{0.894118,0.101961,0.109804}%
\pgfsetstrokecolor{currentstroke}%
\pgfsetstrokeopacity{0.600000}%
\pgfsetdash{}{0pt}%
\pgfpathmoveto{\pgfqpoint{2.174701in}{3.856419in}}%
\pgfpathcurveto{\pgfqpoint{2.185751in}{3.856419in}}{\pgfqpoint{2.196350in}{3.860810in}}{\pgfqpoint{2.204164in}{3.868623in}}%
\pgfpathcurveto{\pgfqpoint{2.211978in}{3.876437in}}{\pgfqpoint{2.216368in}{3.887036in}}{\pgfqpoint{2.216368in}{3.898086in}}%
\pgfpathcurveto{\pgfqpoint{2.216368in}{3.909136in}}{\pgfqpoint{2.211978in}{3.919735in}}{\pgfqpoint{2.204164in}{3.927549in}}%
\pgfpathcurveto{\pgfqpoint{2.196350in}{3.935362in}}{\pgfqpoint{2.185751in}{3.939753in}}{\pgfqpoint{2.174701in}{3.939753in}}%
\pgfpathcurveto{\pgfqpoint{2.163651in}{3.939753in}}{\pgfqpoint{2.153052in}{3.935362in}}{\pgfqpoint{2.145238in}{3.927549in}}%
\pgfpathcurveto{\pgfqpoint{2.137425in}{3.919735in}}{\pgfqpoint{2.133035in}{3.909136in}}{\pgfqpoint{2.133035in}{3.898086in}}%
\pgfpathcurveto{\pgfqpoint{2.133035in}{3.887036in}}{\pgfqpoint{2.137425in}{3.876437in}}{\pgfqpoint{2.145238in}{3.868623in}}%
\pgfpathcurveto{\pgfqpoint{2.153052in}{3.860810in}}{\pgfqpoint{2.163651in}{3.856419in}}{\pgfqpoint{2.174701in}{3.856419in}}%
\pgfpathlineto{\pgfqpoint{2.174701in}{3.856419in}}%
\pgfpathclose%
\pgfusepath{stroke,fill}%
\end{pgfscope}%
\begin{pgfscope}%
\pgfpathrectangle{\pgfqpoint{0.800049in}{0.448486in}}{\pgfqpoint{3.531733in}{3.696000in}}%
\pgfusepath{clip}%
\pgfsetbuttcap%
\pgfsetroundjoin%
\definecolor{currentfill}{rgb}{0.894118,0.101961,0.109804}%
\pgfsetfillcolor{currentfill}%
\pgfsetfillopacity{0.600000}%
\pgfsetlinewidth{1.003750pt}%
\definecolor{currentstroke}{rgb}{0.894118,0.101961,0.109804}%
\pgfsetstrokecolor{currentstroke}%
\pgfsetstrokeopacity{0.600000}%
\pgfsetdash{}{0pt}%
\pgfpathmoveto{\pgfqpoint{2.174701in}{3.856419in}}%
\pgfpathcurveto{\pgfqpoint{2.185751in}{3.856419in}}{\pgfqpoint{2.196350in}{3.860810in}}{\pgfqpoint{2.204164in}{3.868623in}}%
\pgfpathcurveto{\pgfqpoint{2.211978in}{3.876437in}}{\pgfqpoint{2.216368in}{3.887036in}}{\pgfqpoint{2.216368in}{3.898086in}}%
\pgfpathcurveto{\pgfqpoint{2.216368in}{3.909136in}}{\pgfqpoint{2.211978in}{3.919735in}}{\pgfqpoint{2.204164in}{3.927549in}}%
\pgfpathcurveto{\pgfqpoint{2.196350in}{3.935362in}}{\pgfqpoint{2.185751in}{3.939753in}}{\pgfqpoint{2.174701in}{3.939753in}}%
\pgfpathcurveto{\pgfqpoint{2.163651in}{3.939753in}}{\pgfqpoint{2.153052in}{3.935362in}}{\pgfqpoint{2.145238in}{3.927549in}}%
\pgfpathcurveto{\pgfqpoint{2.137425in}{3.919735in}}{\pgfqpoint{2.133035in}{3.909136in}}{\pgfqpoint{2.133035in}{3.898086in}}%
\pgfpathcurveto{\pgfqpoint{2.133035in}{3.887036in}}{\pgfqpoint{2.137425in}{3.876437in}}{\pgfqpoint{2.145238in}{3.868623in}}%
\pgfpathcurveto{\pgfqpoint{2.153052in}{3.860810in}}{\pgfqpoint{2.163651in}{3.856419in}}{\pgfqpoint{2.174701in}{3.856419in}}%
\pgfpathlineto{\pgfqpoint{2.174701in}{3.856419in}}%
\pgfpathclose%
\pgfusepath{stroke,fill}%
\end{pgfscope}%
\begin{pgfscope}%
\pgfpathrectangle{\pgfqpoint{0.800049in}{0.448486in}}{\pgfqpoint{3.531733in}{3.696000in}}%
\pgfusepath{clip}%
\pgfsetbuttcap%
\pgfsetroundjoin%
\definecolor{currentfill}{rgb}{0.894118,0.101961,0.109804}%
\pgfsetfillcolor{currentfill}%
\pgfsetfillopacity{0.600000}%
\pgfsetlinewidth{1.003750pt}%
\definecolor{currentstroke}{rgb}{0.894118,0.101961,0.109804}%
\pgfsetstrokecolor{currentstroke}%
\pgfsetstrokeopacity{0.600000}%
\pgfsetdash{}{0pt}%
\pgfpathmoveto{\pgfqpoint{2.174701in}{3.856419in}}%
\pgfpathcurveto{\pgfqpoint{2.185751in}{3.856419in}}{\pgfqpoint{2.196350in}{3.860810in}}{\pgfqpoint{2.204164in}{3.868623in}}%
\pgfpathcurveto{\pgfqpoint{2.211978in}{3.876437in}}{\pgfqpoint{2.216368in}{3.887036in}}{\pgfqpoint{2.216368in}{3.898086in}}%
\pgfpathcurveto{\pgfqpoint{2.216368in}{3.909136in}}{\pgfqpoint{2.211978in}{3.919735in}}{\pgfqpoint{2.204164in}{3.927549in}}%
\pgfpathcurveto{\pgfqpoint{2.196350in}{3.935362in}}{\pgfqpoint{2.185751in}{3.939753in}}{\pgfqpoint{2.174701in}{3.939753in}}%
\pgfpathcurveto{\pgfqpoint{2.163651in}{3.939753in}}{\pgfqpoint{2.153052in}{3.935362in}}{\pgfqpoint{2.145238in}{3.927549in}}%
\pgfpathcurveto{\pgfqpoint{2.137425in}{3.919735in}}{\pgfqpoint{2.133035in}{3.909136in}}{\pgfqpoint{2.133035in}{3.898086in}}%
\pgfpathcurveto{\pgfqpoint{2.133035in}{3.887036in}}{\pgfqpoint{2.137425in}{3.876437in}}{\pgfqpoint{2.145238in}{3.868623in}}%
\pgfpathcurveto{\pgfqpoint{2.153052in}{3.860810in}}{\pgfqpoint{2.163651in}{3.856419in}}{\pgfqpoint{2.174701in}{3.856419in}}%
\pgfpathlineto{\pgfqpoint{2.174701in}{3.856419in}}%
\pgfpathclose%
\pgfusepath{stroke,fill}%
\end{pgfscope}%
\begin{pgfscope}%
\pgfpathrectangle{\pgfqpoint{0.800049in}{0.448486in}}{\pgfqpoint{3.531733in}{3.696000in}}%
\pgfusepath{clip}%
\pgfsetbuttcap%
\pgfsetroundjoin%
\definecolor{currentfill}{rgb}{0.894118,0.101961,0.109804}%
\pgfsetfillcolor{currentfill}%
\pgfsetfillopacity{0.600000}%
\pgfsetlinewidth{1.003750pt}%
\definecolor{currentstroke}{rgb}{0.894118,0.101961,0.109804}%
\pgfsetstrokecolor{currentstroke}%
\pgfsetstrokeopacity{0.600000}%
\pgfsetdash{}{0pt}%
\pgfpathmoveto{\pgfqpoint{2.174701in}{3.856419in}}%
\pgfpathcurveto{\pgfqpoint{2.185751in}{3.856419in}}{\pgfqpoint{2.196350in}{3.860810in}}{\pgfqpoint{2.204164in}{3.868623in}}%
\pgfpathcurveto{\pgfqpoint{2.211978in}{3.876437in}}{\pgfqpoint{2.216368in}{3.887036in}}{\pgfqpoint{2.216368in}{3.898086in}}%
\pgfpathcurveto{\pgfqpoint{2.216368in}{3.909136in}}{\pgfqpoint{2.211978in}{3.919735in}}{\pgfqpoint{2.204164in}{3.927549in}}%
\pgfpathcurveto{\pgfqpoint{2.196350in}{3.935362in}}{\pgfqpoint{2.185751in}{3.939753in}}{\pgfqpoint{2.174701in}{3.939753in}}%
\pgfpathcurveto{\pgfqpoint{2.163651in}{3.939753in}}{\pgfqpoint{2.153052in}{3.935362in}}{\pgfqpoint{2.145238in}{3.927549in}}%
\pgfpathcurveto{\pgfqpoint{2.137425in}{3.919735in}}{\pgfqpoint{2.133035in}{3.909136in}}{\pgfqpoint{2.133035in}{3.898086in}}%
\pgfpathcurveto{\pgfqpoint{2.133035in}{3.887036in}}{\pgfqpoint{2.137425in}{3.876437in}}{\pgfqpoint{2.145238in}{3.868623in}}%
\pgfpathcurveto{\pgfqpoint{2.153052in}{3.860810in}}{\pgfqpoint{2.163651in}{3.856419in}}{\pgfqpoint{2.174701in}{3.856419in}}%
\pgfpathlineto{\pgfqpoint{2.174701in}{3.856419in}}%
\pgfpathclose%
\pgfusepath{stroke,fill}%
\end{pgfscope}%
\begin{pgfscope}%
\pgfpathrectangle{\pgfqpoint{0.800049in}{0.448486in}}{\pgfqpoint{3.531733in}{3.696000in}}%
\pgfusepath{clip}%
\pgfsetbuttcap%
\pgfsetroundjoin%
\definecolor{currentfill}{rgb}{0.894118,0.101961,0.109804}%
\pgfsetfillcolor{currentfill}%
\pgfsetfillopacity{0.600000}%
\pgfsetlinewidth{1.003750pt}%
\definecolor{currentstroke}{rgb}{0.894118,0.101961,0.109804}%
\pgfsetstrokecolor{currentstroke}%
\pgfsetstrokeopacity{0.600000}%
\pgfsetdash{}{0pt}%
\pgfpathmoveto{\pgfqpoint{2.174701in}{3.856419in}}%
\pgfpathcurveto{\pgfqpoint{2.185751in}{3.856419in}}{\pgfqpoint{2.196350in}{3.860810in}}{\pgfqpoint{2.204164in}{3.868623in}}%
\pgfpathcurveto{\pgfqpoint{2.211978in}{3.876437in}}{\pgfqpoint{2.216368in}{3.887036in}}{\pgfqpoint{2.216368in}{3.898086in}}%
\pgfpathcurveto{\pgfqpoint{2.216368in}{3.909136in}}{\pgfqpoint{2.211978in}{3.919735in}}{\pgfqpoint{2.204164in}{3.927549in}}%
\pgfpathcurveto{\pgfqpoint{2.196350in}{3.935362in}}{\pgfqpoint{2.185751in}{3.939753in}}{\pgfqpoint{2.174701in}{3.939753in}}%
\pgfpathcurveto{\pgfqpoint{2.163651in}{3.939753in}}{\pgfqpoint{2.153052in}{3.935362in}}{\pgfqpoint{2.145238in}{3.927549in}}%
\pgfpathcurveto{\pgfqpoint{2.137425in}{3.919735in}}{\pgfqpoint{2.133035in}{3.909136in}}{\pgfqpoint{2.133035in}{3.898086in}}%
\pgfpathcurveto{\pgfqpoint{2.133035in}{3.887036in}}{\pgfqpoint{2.137425in}{3.876437in}}{\pgfqpoint{2.145238in}{3.868623in}}%
\pgfpathcurveto{\pgfqpoint{2.153052in}{3.860810in}}{\pgfqpoint{2.163651in}{3.856419in}}{\pgfqpoint{2.174701in}{3.856419in}}%
\pgfpathlineto{\pgfqpoint{2.174701in}{3.856419in}}%
\pgfpathclose%
\pgfusepath{stroke,fill}%
\end{pgfscope}%
\begin{pgfscope}%
\pgfpathrectangle{\pgfqpoint{0.800049in}{0.448486in}}{\pgfqpoint{3.531733in}{3.696000in}}%
\pgfusepath{clip}%
\pgfsetbuttcap%
\pgfsetroundjoin%
\definecolor{currentfill}{rgb}{0.894118,0.101961,0.109804}%
\pgfsetfillcolor{currentfill}%
\pgfsetfillopacity{0.600000}%
\pgfsetlinewidth{1.003750pt}%
\definecolor{currentstroke}{rgb}{0.894118,0.101961,0.109804}%
\pgfsetstrokecolor{currentstroke}%
\pgfsetstrokeopacity{0.600000}%
\pgfsetdash{}{0pt}%
\pgfpathmoveto{\pgfqpoint{2.174701in}{3.856419in}}%
\pgfpathcurveto{\pgfqpoint{2.185751in}{3.856419in}}{\pgfqpoint{2.196350in}{3.860810in}}{\pgfqpoint{2.204164in}{3.868623in}}%
\pgfpathcurveto{\pgfqpoint{2.211978in}{3.876437in}}{\pgfqpoint{2.216368in}{3.887036in}}{\pgfqpoint{2.216368in}{3.898086in}}%
\pgfpathcurveto{\pgfqpoint{2.216368in}{3.909136in}}{\pgfqpoint{2.211978in}{3.919735in}}{\pgfqpoint{2.204164in}{3.927549in}}%
\pgfpathcurveto{\pgfqpoint{2.196350in}{3.935362in}}{\pgfqpoint{2.185751in}{3.939753in}}{\pgfqpoint{2.174701in}{3.939753in}}%
\pgfpathcurveto{\pgfqpoint{2.163651in}{3.939753in}}{\pgfqpoint{2.153052in}{3.935362in}}{\pgfqpoint{2.145238in}{3.927549in}}%
\pgfpathcurveto{\pgfqpoint{2.137425in}{3.919735in}}{\pgfqpoint{2.133035in}{3.909136in}}{\pgfqpoint{2.133035in}{3.898086in}}%
\pgfpathcurveto{\pgfqpoint{2.133035in}{3.887036in}}{\pgfqpoint{2.137425in}{3.876437in}}{\pgfqpoint{2.145238in}{3.868623in}}%
\pgfpathcurveto{\pgfqpoint{2.153052in}{3.860810in}}{\pgfqpoint{2.163651in}{3.856419in}}{\pgfqpoint{2.174701in}{3.856419in}}%
\pgfpathlineto{\pgfqpoint{2.174701in}{3.856419in}}%
\pgfpathclose%
\pgfusepath{stroke,fill}%
\end{pgfscope}%
\begin{pgfscope}%
\pgfpathrectangle{\pgfqpoint{0.800049in}{0.448486in}}{\pgfqpoint{3.531733in}{3.696000in}}%
\pgfusepath{clip}%
\pgfsetbuttcap%
\pgfsetroundjoin%
\definecolor{currentfill}{rgb}{0.894118,0.101961,0.109804}%
\pgfsetfillcolor{currentfill}%
\pgfsetfillopacity{0.600000}%
\pgfsetlinewidth{1.003750pt}%
\definecolor{currentstroke}{rgb}{0.894118,0.101961,0.109804}%
\pgfsetstrokecolor{currentstroke}%
\pgfsetstrokeopacity{0.600000}%
\pgfsetdash{}{0pt}%
\pgfpathmoveto{\pgfqpoint{2.174701in}{3.337682in}}%
\pgfpathcurveto{\pgfqpoint{2.185751in}{3.337682in}}{\pgfqpoint{2.196350in}{3.342073in}}{\pgfqpoint{2.204164in}{3.349886in}}%
\pgfpathcurveto{\pgfqpoint{2.211978in}{3.357700in}}{\pgfqpoint{2.216368in}{3.368299in}}{\pgfqpoint{2.216368in}{3.379349in}}%
\pgfpathcurveto{\pgfqpoint{2.216368in}{3.390399in}}{\pgfqpoint{2.211978in}{3.400998in}}{\pgfqpoint{2.204164in}{3.408812in}}%
\pgfpathcurveto{\pgfqpoint{2.196350in}{3.416626in}}{\pgfqpoint{2.185751in}{3.421016in}}{\pgfqpoint{2.174701in}{3.421016in}}%
\pgfpathcurveto{\pgfqpoint{2.163651in}{3.421016in}}{\pgfqpoint{2.153052in}{3.416626in}}{\pgfqpoint{2.145238in}{3.408812in}}%
\pgfpathcurveto{\pgfqpoint{2.137425in}{3.400998in}}{\pgfqpoint{2.133035in}{3.390399in}}{\pgfqpoint{2.133035in}{3.379349in}}%
\pgfpathcurveto{\pgfqpoint{2.133035in}{3.368299in}}{\pgfqpoint{2.137425in}{3.357700in}}{\pgfqpoint{2.145238in}{3.349886in}}%
\pgfpathcurveto{\pgfqpoint{2.153052in}{3.342073in}}{\pgfqpoint{2.163651in}{3.337682in}}{\pgfqpoint{2.174701in}{3.337682in}}%
\pgfpathlineto{\pgfqpoint{2.174701in}{3.337682in}}%
\pgfpathclose%
\pgfusepath{stroke,fill}%
\end{pgfscope}%
\begin{pgfscope}%
\pgfpathrectangle{\pgfqpoint{0.800049in}{0.448486in}}{\pgfqpoint{3.531733in}{3.696000in}}%
\pgfusepath{clip}%
\pgfsetbuttcap%
\pgfsetroundjoin%
\definecolor{currentfill}{rgb}{0.894118,0.101961,0.109804}%
\pgfsetfillcolor{currentfill}%
\pgfsetfillopacity{0.600000}%
\pgfsetlinewidth{1.003750pt}%
\definecolor{currentstroke}{rgb}{0.894118,0.101961,0.109804}%
\pgfsetstrokecolor{currentstroke}%
\pgfsetstrokeopacity{0.600000}%
\pgfsetdash{}{0pt}%
\pgfpathmoveto{\pgfqpoint{2.174701in}{3.337682in}}%
\pgfpathcurveto{\pgfqpoint{2.185751in}{3.337682in}}{\pgfqpoint{2.196350in}{3.342073in}}{\pgfqpoint{2.204164in}{3.349886in}}%
\pgfpathcurveto{\pgfqpoint{2.211978in}{3.357700in}}{\pgfqpoint{2.216368in}{3.368299in}}{\pgfqpoint{2.216368in}{3.379349in}}%
\pgfpathcurveto{\pgfqpoint{2.216368in}{3.390399in}}{\pgfqpoint{2.211978in}{3.400998in}}{\pgfqpoint{2.204164in}{3.408812in}}%
\pgfpathcurveto{\pgfqpoint{2.196350in}{3.416626in}}{\pgfqpoint{2.185751in}{3.421016in}}{\pgfqpoint{2.174701in}{3.421016in}}%
\pgfpathcurveto{\pgfqpoint{2.163651in}{3.421016in}}{\pgfqpoint{2.153052in}{3.416626in}}{\pgfqpoint{2.145238in}{3.408812in}}%
\pgfpathcurveto{\pgfqpoint{2.137425in}{3.400998in}}{\pgfqpoint{2.133035in}{3.390399in}}{\pgfqpoint{2.133035in}{3.379349in}}%
\pgfpathcurveto{\pgfqpoint{2.133035in}{3.368299in}}{\pgfqpoint{2.137425in}{3.357700in}}{\pgfqpoint{2.145238in}{3.349886in}}%
\pgfpathcurveto{\pgfqpoint{2.153052in}{3.342073in}}{\pgfqpoint{2.163651in}{3.337682in}}{\pgfqpoint{2.174701in}{3.337682in}}%
\pgfpathlineto{\pgfqpoint{2.174701in}{3.337682in}}%
\pgfpathclose%
\pgfusepath{stroke,fill}%
\end{pgfscope}%
\begin{pgfscope}%
\pgfpathrectangle{\pgfqpoint{0.800049in}{0.448486in}}{\pgfqpoint{3.531733in}{3.696000in}}%
\pgfusepath{clip}%
\pgfsetbuttcap%
\pgfsetroundjoin%
\definecolor{currentfill}{rgb}{0.894118,0.101961,0.109804}%
\pgfsetfillcolor{currentfill}%
\pgfsetfillopacity{0.600000}%
\pgfsetlinewidth{1.003750pt}%
\definecolor{currentstroke}{rgb}{0.894118,0.101961,0.109804}%
\pgfsetstrokecolor{currentstroke}%
\pgfsetstrokeopacity{0.600000}%
\pgfsetdash{}{0pt}%
\pgfpathmoveto{\pgfqpoint{2.174701in}{3.337682in}}%
\pgfpathcurveto{\pgfqpoint{2.185751in}{3.337682in}}{\pgfqpoint{2.196350in}{3.342073in}}{\pgfqpoint{2.204164in}{3.349886in}}%
\pgfpathcurveto{\pgfqpoint{2.211978in}{3.357700in}}{\pgfqpoint{2.216368in}{3.368299in}}{\pgfqpoint{2.216368in}{3.379349in}}%
\pgfpathcurveto{\pgfqpoint{2.216368in}{3.390399in}}{\pgfqpoint{2.211978in}{3.400998in}}{\pgfqpoint{2.204164in}{3.408812in}}%
\pgfpathcurveto{\pgfqpoint{2.196350in}{3.416626in}}{\pgfqpoint{2.185751in}{3.421016in}}{\pgfqpoint{2.174701in}{3.421016in}}%
\pgfpathcurveto{\pgfqpoint{2.163651in}{3.421016in}}{\pgfqpoint{2.153052in}{3.416626in}}{\pgfqpoint{2.145238in}{3.408812in}}%
\pgfpathcurveto{\pgfqpoint{2.137425in}{3.400998in}}{\pgfqpoint{2.133035in}{3.390399in}}{\pgfqpoint{2.133035in}{3.379349in}}%
\pgfpathcurveto{\pgfqpoint{2.133035in}{3.368299in}}{\pgfqpoint{2.137425in}{3.357700in}}{\pgfqpoint{2.145238in}{3.349886in}}%
\pgfpathcurveto{\pgfqpoint{2.153052in}{3.342073in}}{\pgfqpoint{2.163651in}{3.337682in}}{\pgfqpoint{2.174701in}{3.337682in}}%
\pgfpathlineto{\pgfqpoint{2.174701in}{3.337682in}}%
\pgfpathclose%
\pgfusepath{stroke,fill}%
\end{pgfscope}%
\begin{pgfscope}%
\pgfpathrectangle{\pgfqpoint{0.800049in}{0.448486in}}{\pgfqpoint{3.531733in}{3.696000in}}%
\pgfusepath{clip}%
\pgfsetbuttcap%
\pgfsetroundjoin%
\definecolor{currentfill}{rgb}{0.894118,0.101961,0.109804}%
\pgfsetfillcolor{currentfill}%
\pgfsetfillopacity{0.600000}%
\pgfsetlinewidth{1.003750pt}%
\definecolor{currentstroke}{rgb}{0.894118,0.101961,0.109804}%
\pgfsetstrokecolor{currentstroke}%
\pgfsetstrokeopacity{0.600000}%
\pgfsetdash{}{0pt}%
\pgfpathmoveto{\pgfqpoint{2.174701in}{2.818946in}}%
\pgfpathcurveto{\pgfqpoint{2.185751in}{2.818946in}}{\pgfqpoint{2.196350in}{2.823336in}}{\pgfqpoint{2.204164in}{2.831150in}}%
\pgfpathcurveto{\pgfqpoint{2.211978in}{2.838963in}}{\pgfqpoint{2.216368in}{2.849562in}}{\pgfqpoint{2.216368in}{2.860612in}}%
\pgfpathcurveto{\pgfqpoint{2.216368in}{2.871662in}}{\pgfqpoint{2.211978in}{2.882261in}}{\pgfqpoint{2.204164in}{2.890075in}}%
\pgfpathcurveto{\pgfqpoint{2.196350in}{2.897889in}}{\pgfqpoint{2.185751in}{2.902279in}}{\pgfqpoint{2.174701in}{2.902279in}}%
\pgfpathcurveto{\pgfqpoint{2.163651in}{2.902279in}}{\pgfqpoint{2.153052in}{2.897889in}}{\pgfqpoint{2.145238in}{2.890075in}}%
\pgfpathcurveto{\pgfqpoint{2.137425in}{2.882261in}}{\pgfqpoint{2.133035in}{2.871662in}}{\pgfqpoint{2.133035in}{2.860612in}}%
\pgfpathcurveto{\pgfqpoint{2.133035in}{2.849562in}}{\pgfqpoint{2.137425in}{2.838963in}}{\pgfqpoint{2.145238in}{2.831150in}}%
\pgfpathcurveto{\pgfqpoint{2.153052in}{2.823336in}}{\pgfqpoint{2.163651in}{2.818946in}}{\pgfqpoint{2.174701in}{2.818946in}}%
\pgfpathlineto{\pgfqpoint{2.174701in}{2.818946in}}%
\pgfpathclose%
\pgfusepath{stroke,fill}%
\end{pgfscope}%
\begin{pgfscope}%
\pgfpathrectangle{\pgfqpoint{0.800049in}{0.448486in}}{\pgfqpoint{3.531733in}{3.696000in}}%
\pgfusepath{clip}%
\pgfsetbuttcap%
\pgfsetroundjoin%
\definecolor{currentfill}{rgb}{0.894118,0.101961,0.109804}%
\pgfsetfillcolor{currentfill}%
\pgfsetfillopacity{0.600000}%
\pgfsetlinewidth{1.003750pt}%
\definecolor{currentstroke}{rgb}{0.894118,0.101961,0.109804}%
\pgfsetstrokecolor{currentstroke}%
\pgfsetstrokeopacity{0.600000}%
\pgfsetdash{}{0pt}%
\pgfpathmoveto{\pgfqpoint{2.174701in}{2.818946in}}%
\pgfpathcurveto{\pgfqpoint{2.185751in}{2.818946in}}{\pgfqpoint{2.196350in}{2.823336in}}{\pgfqpoint{2.204164in}{2.831150in}}%
\pgfpathcurveto{\pgfqpoint{2.211978in}{2.838963in}}{\pgfqpoint{2.216368in}{2.849562in}}{\pgfqpoint{2.216368in}{2.860612in}}%
\pgfpathcurveto{\pgfqpoint{2.216368in}{2.871662in}}{\pgfqpoint{2.211978in}{2.882261in}}{\pgfqpoint{2.204164in}{2.890075in}}%
\pgfpathcurveto{\pgfqpoint{2.196350in}{2.897889in}}{\pgfqpoint{2.185751in}{2.902279in}}{\pgfqpoint{2.174701in}{2.902279in}}%
\pgfpathcurveto{\pgfqpoint{2.163651in}{2.902279in}}{\pgfqpoint{2.153052in}{2.897889in}}{\pgfqpoint{2.145238in}{2.890075in}}%
\pgfpathcurveto{\pgfqpoint{2.137425in}{2.882261in}}{\pgfqpoint{2.133035in}{2.871662in}}{\pgfqpoint{2.133035in}{2.860612in}}%
\pgfpathcurveto{\pgfqpoint{2.133035in}{2.849562in}}{\pgfqpoint{2.137425in}{2.838963in}}{\pgfqpoint{2.145238in}{2.831150in}}%
\pgfpathcurveto{\pgfqpoint{2.153052in}{2.823336in}}{\pgfqpoint{2.163651in}{2.818946in}}{\pgfqpoint{2.174701in}{2.818946in}}%
\pgfpathlineto{\pgfqpoint{2.174701in}{2.818946in}}%
\pgfpathclose%
\pgfusepath{stroke,fill}%
\end{pgfscope}%
\begin{pgfscope}%
\pgfpathrectangle{\pgfqpoint{0.800049in}{0.448486in}}{\pgfqpoint{3.531733in}{3.696000in}}%
\pgfusepath{clip}%
\pgfsetbuttcap%
\pgfsetroundjoin%
\definecolor{currentfill}{rgb}{0.894118,0.101961,0.109804}%
\pgfsetfillcolor{currentfill}%
\pgfsetfillopacity{0.600000}%
\pgfsetlinewidth{1.003750pt}%
\definecolor{currentstroke}{rgb}{0.894118,0.101961,0.109804}%
\pgfsetstrokecolor{currentstroke}%
\pgfsetstrokeopacity{0.600000}%
\pgfsetdash{}{0pt}%
\pgfpathmoveto{\pgfqpoint{2.174701in}{2.818946in}}%
\pgfpathcurveto{\pgfqpoint{2.185751in}{2.818946in}}{\pgfqpoint{2.196350in}{2.823336in}}{\pgfqpoint{2.204164in}{2.831150in}}%
\pgfpathcurveto{\pgfqpoint{2.211978in}{2.838963in}}{\pgfqpoint{2.216368in}{2.849562in}}{\pgfqpoint{2.216368in}{2.860612in}}%
\pgfpathcurveto{\pgfqpoint{2.216368in}{2.871662in}}{\pgfqpoint{2.211978in}{2.882261in}}{\pgfqpoint{2.204164in}{2.890075in}}%
\pgfpathcurveto{\pgfqpoint{2.196350in}{2.897889in}}{\pgfqpoint{2.185751in}{2.902279in}}{\pgfqpoint{2.174701in}{2.902279in}}%
\pgfpathcurveto{\pgfqpoint{2.163651in}{2.902279in}}{\pgfqpoint{2.153052in}{2.897889in}}{\pgfqpoint{2.145238in}{2.890075in}}%
\pgfpathcurveto{\pgfqpoint{2.137425in}{2.882261in}}{\pgfqpoint{2.133035in}{2.871662in}}{\pgfqpoint{2.133035in}{2.860612in}}%
\pgfpathcurveto{\pgfqpoint{2.133035in}{2.849562in}}{\pgfqpoint{2.137425in}{2.838963in}}{\pgfqpoint{2.145238in}{2.831150in}}%
\pgfpathcurveto{\pgfqpoint{2.153052in}{2.823336in}}{\pgfqpoint{2.163651in}{2.818946in}}{\pgfqpoint{2.174701in}{2.818946in}}%
\pgfpathlineto{\pgfqpoint{2.174701in}{2.818946in}}%
\pgfpathclose%
\pgfusepath{stroke,fill}%
\end{pgfscope}%
\begin{pgfscope}%
\pgfpathrectangle{\pgfqpoint{0.800049in}{0.448486in}}{\pgfqpoint{3.531733in}{3.696000in}}%
\pgfusepath{clip}%
\pgfsetbuttcap%
\pgfsetroundjoin%
\definecolor{currentfill}{rgb}{0.894118,0.101961,0.109804}%
\pgfsetfillcolor{currentfill}%
\pgfsetfillopacity{0.600000}%
\pgfsetlinewidth{1.003750pt}%
\definecolor{currentstroke}{rgb}{0.894118,0.101961,0.109804}%
\pgfsetstrokecolor{currentstroke}%
\pgfsetstrokeopacity{0.600000}%
\pgfsetdash{}{0pt}%
\pgfpathmoveto{\pgfqpoint{2.174701in}{2.818946in}}%
\pgfpathcurveto{\pgfqpoint{2.185751in}{2.818946in}}{\pgfqpoint{2.196350in}{2.823336in}}{\pgfqpoint{2.204164in}{2.831150in}}%
\pgfpathcurveto{\pgfqpoint{2.211978in}{2.838963in}}{\pgfqpoint{2.216368in}{2.849562in}}{\pgfqpoint{2.216368in}{2.860612in}}%
\pgfpathcurveto{\pgfqpoint{2.216368in}{2.871662in}}{\pgfqpoint{2.211978in}{2.882261in}}{\pgfqpoint{2.204164in}{2.890075in}}%
\pgfpathcurveto{\pgfqpoint{2.196350in}{2.897889in}}{\pgfqpoint{2.185751in}{2.902279in}}{\pgfqpoint{2.174701in}{2.902279in}}%
\pgfpathcurveto{\pgfqpoint{2.163651in}{2.902279in}}{\pgfqpoint{2.153052in}{2.897889in}}{\pgfqpoint{2.145238in}{2.890075in}}%
\pgfpathcurveto{\pgfqpoint{2.137425in}{2.882261in}}{\pgfqpoint{2.133035in}{2.871662in}}{\pgfqpoint{2.133035in}{2.860612in}}%
\pgfpathcurveto{\pgfqpoint{2.133035in}{2.849562in}}{\pgfqpoint{2.137425in}{2.838963in}}{\pgfqpoint{2.145238in}{2.831150in}}%
\pgfpathcurveto{\pgfqpoint{2.153052in}{2.823336in}}{\pgfqpoint{2.163651in}{2.818946in}}{\pgfqpoint{2.174701in}{2.818946in}}%
\pgfpathlineto{\pgfqpoint{2.174701in}{2.818946in}}%
\pgfpathclose%
\pgfusepath{stroke,fill}%
\end{pgfscope}%
\begin{pgfscope}%
\pgfpathrectangle{\pgfqpoint{0.800049in}{0.448486in}}{\pgfqpoint{3.531733in}{3.696000in}}%
\pgfusepath{clip}%
\pgfsetbuttcap%
\pgfsetroundjoin%
\definecolor{currentfill}{rgb}{0.894118,0.101961,0.109804}%
\pgfsetfillcolor{currentfill}%
\pgfsetfillopacity{0.600000}%
\pgfsetlinewidth{1.003750pt}%
\definecolor{currentstroke}{rgb}{0.894118,0.101961,0.109804}%
\pgfsetstrokecolor{currentstroke}%
\pgfsetstrokeopacity{0.600000}%
\pgfsetdash{}{0pt}%
\pgfpathmoveto{\pgfqpoint{2.174701in}{2.818946in}}%
\pgfpathcurveto{\pgfqpoint{2.185751in}{2.818946in}}{\pgfqpoint{2.196350in}{2.823336in}}{\pgfqpoint{2.204164in}{2.831150in}}%
\pgfpathcurveto{\pgfqpoint{2.211978in}{2.838963in}}{\pgfqpoint{2.216368in}{2.849562in}}{\pgfqpoint{2.216368in}{2.860612in}}%
\pgfpathcurveto{\pgfqpoint{2.216368in}{2.871662in}}{\pgfqpoint{2.211978in}{2.882261in}}{\pgfqpoint{2.204164in}{2.890075in}}%
\pgfpathcurveto{\pgfqpoint{2.196350in}{2.897889in}}{\pgfqpoint{2.185751in}{2.902279in}}{\pgfqpoint{2.174701in}{2.902279in}}%
\pgfpathcurveto{\pgfqpoint{2.163651in}{2.902279in}}{\pgfqpoint{2.153052in}{2.897889in}}{\pgfqpoint{2.145238in}{2.890075in}}%
\pgfpathcurveto{\pgfqpoint{2.137425in}{2.882261in}}{\pgfqpoint{2.133035in}{2.871662in}}{\pgfqpoint{2.133035in}{2.860612in}}%
\pgfpathcurveto{\pgfqpoint{2.133035in}{2.849562in}}{\pgfqpoint{2.137425in}{2.838963in}}{\pgfqpoint{2.145238in}{2.831150in}}%
\pgfpathcurveto{\pgfqpoint{2.153052in}{2.823336in}}{\pgfqpoint{2.163651in}{2.818946in}}{\pgfqpoint{2.174701in}{2.818946in}}%
\pgfpathlineto{\pgfqpoint{2.174701in}{2.818946in}}%
\pgfpathclose%
\pgfusepath{stroke,fill}%
\end{pgfscope}%
\begin{pgfscope}%
\pgfpathrectangle{\pgfqpoint{0.800049in}{0.448486in}}{\pgfqpoint{3.531733in}{3.696000in}}%
\pgfusepath{clip}%
\pgfsetbuttcap%
\pgfsetroundjoin%
\definecolor{currentfill}{rgb}{0.894118,0.101961,0.109804}%
\pgfsetfillcolor{currentfill}%
\pgfsetfillopacity{0.600000}%
\pgfsetlinewidth{1.003750pt}%
\definecolor{currentstroke}{rgb}{0.894118,0.101961,0.109804}%
\pgfsetstrokecolor{currentstroke}%
\pgfsetstrokeopacity{0.600000}%
\pgfsetdash{}{0pt}%
\pgfpathmoveto{\pgfqpoint{2.174701in}{2.818946in}}%
\pgfpathcurveto{\pgfqpoint{2.185751in}{2.818946in}}{\pgfqpoint{2.196350in}{2.823336in}}{\pgfqpoint{2.204164in}{2.831150in}}%
\pgfpathcurveto{\pgfqpoint{2.211978in}{2.838963in}}{\pgfqpoint{2.216368in}{2.849562in}}{\pgfqpoint{2.216368in}{2.860612in}}%
\pgfpathcurveto{\pgfqpoint{2.216368in}{2.871662in}}{\pgfqpoint{2.211978in}{2.882261in}}{\pgfqpoint{2.204164in}{2.890075in}}%
\pgfpathcurveto{\pgfqpoint{2.196350in}{2.897889in}}{\pgfqpoint{2.185751in}{2.902279in}}{\pgfqpoint{2.174701in}{2.902279in}}%
\pgfpathcurveto{\pgfqpoint{2.163651in}{2.902279in}}{\pgfqpoint{2.153052in}{2.897889in}}{\pgfqpoint{2.145238in}{2.890075in}}%
\pgfpathcurveto{\pgfqpoint{2.137425in}{2.882261in}}{\pgfqpoint{2.133035in}{2.871662in}}{\pgfqpoint{2.133035in}{2.860612in}}%
\pgfpathcurveto{\pgfqpoint{2.133035in}{2.849562in}}{\pgfqpoint{2.137425in}{2.838963in}}{\pgfqpoint{2.145238in}{2.831150in}}%
\pgfpathcurveto{\pgfqpoint{2.153052in}{2.823336in}}{\pgfqpoint{2.163651in}{2.818946in}}{\pgfqpoint{2.174701in}{2.818946in}}%
\pgfpathlineto{\pgfqpoint{2.174701in}{2.818946in}}%
\pgfpathclose%
\pgfusepath{stroke,fill}%
\end{pgfscope}%
\begin{pgfscope}%
\pgfpathrectangle{\pgfqpoint{0.800049in}{0.448486in}}{\pgfqpoint{3.531733in}{3.696000in}}%
\pgfusepath{clip}%
\pgfsetbuttcap%
\pgfsetroundjoin%
\definecolor{currentfill}{rgb}{0.894118,0.101961,0.109804}%
\pgfsetfillcolor{currentfill}%
\pgfsetfillopacity{0.600000}%
\pgfsetlinewidth{1.003750pt}%
\definecolor{currentstroke}{rgb}{0.894118,0.101961,0.109804}%
\pgfsetstrokecolor{currentstroke}%
\pgfsetstrokeopacity{0.600000}%
\pgfsetdash{}{0pt}%
\pgfpathmoveto{\pgfqpoint{2.174701in}{2.818946in}}%
\pgfpathcurveto{\pgfqpoint{2.185751in}{2.818946in}}{\pgfqpoint{2.196350in}{2.823336in}}{\pgfqpoint{2.204164in}{2.831150in}}%
\pgfpathcurveto{\pgfqpoint{2.211978in}{2.838963in}}{\pgfqpoint{2.216368in}{2.849562in}}{\pgfqpoint{2.216368in}{2.860612in}}%
\pgfpathcurveto{\pgfqpoint{2.216368in}{2.871662in}}{\pgfqpoint{2.211978in}{2.882261in}}{\pgfqpoint{2.204164in}{2.890075in}}%
\pgfpathcurveto{\pgfqpoint{2.196350in}{2.897889in}}{\pgfqpoint{2.185751in}{2.902279in}}{\pgfqpoint{2.174701in}{2.902279in}}%
\pgfpathcurveto{\pgfqpoint{2.163651in}{2.902279in}}{\pgfqpoint{2.153052in}{2.897889in}}{\pgfqpoint{2.145238in}{2.890075in}}%
\pgfpathcurveto{\pgfqpoint{2.137425in}{2.882261in}}{\pgfqpoint{2.133035in}{2.871662in}}{\pgfqpoint{2.133035in}{2.860612in}}%
\pgfpathcurveto{\pgfqpoint{2.133035in}{2.849562in}}{\pgfqpoint{2.137425in}{2.838963in}}{\pgfqpoint{2.145238in}{2.831150in}}%
\pgfpathcurveto{\pgfqpoint{2.153052in}{2.823336in}}{\pgfqpoint{2.163651in}{2.818946in}}{\pgfqpoint{2.174701in}{2.818946in}}%
\pgfpathlineto{\pgfqpoint{2.174701in}{2.818946in}}%
\pgfpathclose%
\pgfusepath{stroke,fill}%
\end{pgfscope}%
\begin{pgfscope}%
\pgfpathrectangle{\pgfqpoint{0.800049in}{0.448486in}}{\pgfqpoint{3.531733in}{3.696000in}}%
\pgfusepath{clip}%
\pgfsetbuttcap%
\pgfsetroundjoin%
\definecolor{currentfill}{rgb}{0.894118,0.101961,0.109804}%
\pgfsetfillcolor{currentfill}%
\pgfsetfillopacity{0.600000}%
\pgfsetlinewidth{1.003750pt}%
\definecolor{currentstroke}{rgb}{0.894118,0.101961,0.109804}%
\pgfsetstrokecolor{currentstroke}%
\pgfsetstrokeopacity{0.600000}%
\pgfsetdash{}{0pt}%
\pgfpathmoveto{\pgfqpoint{2.174701in}{2.300209in}}%
\pgfpathcurveto{\pgfqpoint{2.185751in}{2.300209in}}{\pgfqpoint{2.196350in}{2.304599in}}{\pgfqpoint{2.204164in}{2.312413in}}%
\pgfpathcurveto{\pgfqpoint{2.211978in}{2.320226in}}{\pgfqpoint{2.216368in}{2.330825in}}{\pgfqpoint{2.216368in}{2.341875in}}%
\pgfpathcurveto{\pgfqpoint{2.216368in}{2.352926in}}{\pgfqpoint{2.211978in}{2.363525in}}{\pgfqpoint{2.204164in}{2.371338in}}%
\pgfpathcurveto{\pgfqpoint{2.196350in}{2.379152in}}{\pgfqpoint{2.185751in}{2.383542in}}{\pgfqpoint{2.174701in}{2.383542in}}%
\pgfpathcurveto{\pgfqpoint{2.163651in}{2.383542in}}{\pgfqpoint{2.153052in}{2.379152in}}{\pgfqpoint{2.145238in}{2.371338in}}%
\pgfpathcurveto{\pgfqpoint{2.137425in}{2.363525in}}{\pgfqpoint{2.133035in}{2.352926in}}{\pgfqpoint{2.133035in}{2.341875in}}%
\pgfpathcurveto{\pgfqpoint{2.133035in}{2.330825in}}{\pgfqpoint{2.137425in}{2.320226in}}{\pgfqpoint{2.145238in}{2.312413in}}%
\pgfpathcurveto{\pgfqpoint{2.153052in}{2.304599in}}{\pgfqpoint{2.163651in}{2.300209in}}{\pgfqpoint{2.174701in}{2.300209in}}%
\pgfpathlineto{\pgfqpoint{2.174701in}{2.300209in}}%
\pgfpathclose%
\pgfusepath{stroke,fill}%
\end{pgfscope}%
\begin{pgfscope}%
\pgfpathrectangle{\pgfqpoint{0.800049in}{0.448486in}}{\pgfqpoint{3.531733in}{3.696000in}}%
\pgfusepath{clip}%
\pgfsetbuttcap%
\pgfsetroundjoin%
\definecolor{currentfill}{rgb}{0.894118,0.101961,0.109804}%
\pgfsetfillcolor{currentfill}%
\pgfsetfillopacity{0.600000}%
\pgfsetlinewidth{1.003750pt}%
\definecolor{currentstroke}{rgb}{0.894118,0.101961,0.109804}%
\pgfsetstrokecolor{currentstroke}%
\pgfsetstrokeopacity{0.600000}%
\pgfsetdash{}{0pt}%
\pgfpathmoveto{\pgfqpoint{2.174701in}{2.300209in}}%
\pgfpathcurveto{\pgfqpoint{2.185751in}{2.300209in}}{\pgfqpoint{2.196350in}{2.304599in}}{\pgfqpoint{2.204164in}{2.312413in}}%
\pgfpathcurveto{\pgfqpoint{2.211978in}{2.320226in}}{\pgfqpoint{2.216368in}{2.330825in}}{\pgfqpoint{2.216368in}{2.341875in}}%
\pgfpathcurveto{\pgfqpoint{2.216368in}{2.352926in}}{\pgfqpoint{2.211978in}{2.363525in}}{\pgfqpoint{2.204164in}{2.371338in}}%
\pgfpathcurveto{\pgfqpoint{2.196350in}{2.379152in}}{\pgfqpoint{2.185751in}{2.383542in}}{\pgfqpoint{2.174701in}{2.383542in}}%
\pgfpathcurveto{\pgfqpoint{2.163651in}{2.383542in}}{\pgfqpoint{2.153052in}{2.379152in}}{\pgfqpoint{2.145238in}{2.371338in}}%
\pgfpathcurveto{\pgfqpoint{2.137425in}{2.363525in}}{\pgfqpoint{2.133035in}{2.352926in}}{\pgfqpoint{2.133035in}{2.341875in}}%
\pgfpathcurveto{\pgfqpoint{2.133035in}{2.330825in}}{\pgfqpoint{2.137425in}{2.320226in}}{\pgfqpoint{2.145238in}{2.312413in}}%
\pgfpathcurveto{\pgfqpoint{2.153052in}{2.304599in}}{\pgfqpoint{2.163651in}{2.300209in}}{\pgfqpoint{2.174701in}{2.300209in}}%
\pgfpathlineto{\pgfqpoint{2.174701in}{2.300209in}}%
\pgfpathclose%
\pgfusepath{stroke,fill}%
\end{pgfscope}%
\begin{pgfscope}%
\pgfpathrectangle{\pgfqpoint{0.800049in}{0.448486in}}{\pgfqpoint{3.531733in}{3.696000in}}%
\pgfusepath{clip}%
\pgfsetbuttcap%
\pgfsetroundjoin%
\definecolor{currentfill}{rgb}{0.894118,0.101961,0.109804}%
\pgfsetfillcolor{currentfill}%
\pgfsetfillopacity{0.600000}%
\pgfsetlinewidth{1.003750pt}%
\definecolor{currentstroke}{rgb}{0.894118,0.101961,0.109804}%
\pgfsetstrokecolor{currentstroke}%
\pgfsetstrokeopacity{0.600000}%
\pgfsetdash{}{0pt}%
\pgfpathmoveto{\pgfqpoint{2.174701in}{2.300209in}}%
\pgfpathcurveto{\pgfqpoint{2.185751in}{2.300209in}}{\pgfqpoint{2.196350in}{2.304599in}}{\pgfqpoint{2.204164in}{2.312413in}}%
\pgfpathcurveto{\pgfqpoint{2.211978in}{2.320226in}}{\pgfqpoint{2.216368in}{2.330825in}}{\pgfqpoint{2.216368in}{2.341875in}}%
\pgfpathcurveto{\pgfqpoint{2.216368in}{2.352926in}}{\pgfqpoint{2.211978in}{2.363525in}}{\pgfqpoint{2.204164in}{2.371338in}}%
\pgfpathcurveto{\pgfqpoint{2.196350in}{2.379152in}}{\pgfqpoint{2.185751in}{2.383542in}}{\pgfqpoint{2.174701in}{2.383542in}}%
\pgfpathcurveto{\pgfqpoint{2.163651in}{2.383542in}}{\pgfqpoint{2.153052in}{2.379152in}}{\pgfqpoint{2.145238in}{2.371338in}}%
\pgfpathcurveto{\pgfqpoint{2.137425in}{2.363525in}}{\pgfqpoint{2.133035in}{2.352926in}}{\pgfqpoint{2.133035in}{2.341875in}}%
\pgfpathcurveto{\pgfqpoint{2.133035in}{2.330825in}}{\pgfqpoint{2.137425in}{2.320226in}}{\pgfqpoint{2.145238in}{2.312413in}}%
\pgfpathcurveto{\pgfqpoint{2.153052in}{2.304599in}}{\pgfqpoint{2.163651in}{2.300209in}}{\pgfqpoint{2.174701in}{2.300209in}}%
\pgfpathlineto{\pgfqpoint{2.174701in}{2.300209in}}%
\pgfpathclose%
\pgfusepath{stroke,fill}%
\end{pgfscope}%
\begin{pgfscope}%
\pgfpathrectangle{\pgfqpoint{0.800049in}{0.448486in}}{\pgfqpoint{3.531733in}{3.696000in}}%
\pgfusepath{clip}%
\pgfsetbuttcap%
\pgfsetroundjoin%
\definecolor{currentfill}{rgb}{0.894118,0.101961,0.109804}%
\pgfsetfillcolor{currentfill}%
\pgfsetfillopacity{0.600000}%
\pgfsetlinewidth{1.003750pt}%
\definecolor{currentstroke}{rgb}{0.894118,0.101961,0.109804}%
\pgfsetstrokecolor{currentstroke}%
\pgfsetstrokeopacity{0.600000}%
\pgfsetdash{}{0pt}%
\pgfpathmoveto{\pgfqpoint{2.174701in}{2.300209in}}%
\pgfpathcurveto{\pgfqpoint{2.185751in}{2.300209in}}{\pgfqpoint{2.196350in}{2.304599in}}{\pgfqpoint{2.204164in}{2.312413in}}%
\pgfpathcurveto{\pgfqpoint{2.211978in}{2.320226in}}{\pgfqpoint{2.216368in}{2.330825in}}{\pgfqpoint{2.216368in}{2.341875in}}%
\pgfpathcurveto{\pgfqpoint{2.216368in}{2.352926in}}{\pgfqpoint{2.211978in}{2.363525in}}{\pgfqpoint{2.204164in}{2.371338in}}%
\pgfpathcurveto{\pgfqpoint{2.196350in}{2.379152in}}{\pgfqpoint{2.185751in}{2.383542in}}{\pgfqpoint{2.174701in}{2.383542in}}%
\pgfpathcurveto{\pgfqpoint{2.163651in}{2.383542in}}{\pgfqpoint{2.153052in}{2.379152in}}{\pgfqpoint{2.145238in}{2.371338in}}%
\pgfpathcurveto{\pgfqpoint{2.137425in}{2.363525in}}{\pgfqpoint{2.133035in}{2.352926in}}{\pgfqpoint{2.133035in}{2.341875in}}%
\pgfpathcurveto{\pgfqpoint{2.133035in}{2.330825in}}{\pgfqpoint{2.137425in}{2.320226in}}{\pgfqpoint{2.145238in}{2.312413in}}%
\pgfpathcurveto{\pgfqpoint{2.153052in}{2.304599in}}{\pgfqpoint{2.163651in}{2.300209in}}{\pgfqpoint{2.174701in}{2.300209in}}%
\pgfpathlineto{\pgfqpoint{2.174701in}{2.300209in}}%
\pgfpathclose%
\pgfusepath{stroke,fill}%
\end{pgfscope}%
\begin{pgfscope}%
\pgfpathrectangle{\pgfqpoint{0.800049in}{0.448486in}}{\pgfqpoint{3.531733in}{3.696000in}}%
\pgfusepath{clip}%
\pgfsetbuttcap%
\pgfsetroundjoin%
\definecolor{currentfill}{rgb}{0.894118,0.101961,0.109804}%
\pgfsetfillcolor{currentfill}%
\pgfsetfillopacity{0.600000}%
\pgfsetlinewidth{1.003750pt}%
\definecolor{currentstroke}{rgb}{0.894118,0.101961,0.109804}%
\pgfsetstrokecolor{currentstroke}%
\pgfsetstrokeopacity{0.600000}%
\pgfsetdash{}{0pt}%
\pgfpathmoveto{\pgfqpoint{2.174701in}{2.300209in}}%
\pgfpathcurveto{\pgfqpoint{2.185751in}{2.300209in}}{\pgfqpoint{2.196350in}{2.304599in}}{\pgfqpoint{2.204164in}{2.312413in}}%
\pgfpathcurveto{\pgfqpoint{2.211978in}{2.320226in}}{\pgfqpoint{2.216368in}{2.330825in}}{\pgfqpoint{2.216368in}{2.341875in}}%
\pgfpathcurveto{\pgfqpoint{2.216368in}{2.352926in}}{\pgfqpoint{2.211978in}{2.363525in}}{\pgfqpoint{2.204164in}{2.371338in}}%
\pgfpathcurveto{\pgfqpoint{2.196350in}{2.379152in}}{\pgfqpoint{2.185751in}{2.383542in}}{\pgfqpoint{2.174701in}{2.383542in}}%
\pgfpathcurveto{\pgfqpoint{2.163651in}{2.383542in}}{\pgfqpoint{2.153052in}{2.379152in}}{\pgfqpoint{2.145238in}{2.371338in}}%
\pgfpathcurveto{\pgfqpoint{2.137425in}{2.363525in}}{\pgfqpoint{2.133035in}{2.352926in}}{\pgfqpoint{2.133035in}{2.341875in}}%
\pgfpathcurveto{\pgfqpoint{2.133035in}{2.330825in}}{\pgfqpoint{2.137425in}{2.320226in}}{\pgfqpoint{2.145238in}{2.312413in}}%
\pgfpathcurveto{\pgfqpoint{2.153052in}{2.304599in}}{\pgfqpoint{2.163651in}{2.300209in}}{\pgfqpoint{2.174701in}{2.300209in}}%
\pgfpathlineto{\pgfqpoint{2.174701in}{2.300209in}}%
\pgfpathclose%
\pgfusepath{stroke,fill}%
\end{pgfscope}%
\begin{pgfscope}%
\pgfpathrectangle{\pgfqpoint{0.800049in}{0.448486in}}{\pgfqpoint{3.531733in}{3.696000in}}%
\pgfusepath{clip}%
\pgfsetbuttcap%
\pgfsetroundjoin%
\definecolor{currentfill}{rgb}{0.894118,0.101961,0.109804}%
\pgfsetfillcolor{currentfill}%
\pgfsetfillopacity{0.600000}%
\pgfsetlinewidth{1.003750pt}%
\definecolor{currentstroke}{rgb}{0.894118,0.101961,0.109804}%
\pgfsetstrokecolor{currentstroke}%
\pgfsetstrokeopacity{0.600000}%
\pgfsetdash{}{0pt}%
\pgfpathmoveto{\pgfqpoint{1.655964in}{4.020686in}}%
\pgfpathcurveto{\pgfqpoint{1.667015in}{4.020686in}}{\pgfqpoint{1.677614in}{4.025076in}}{\pgfqpoint{1.685427in}{4.032890in}}%
\pgfpathcurveto{\pgfqpoint{1.693241in}{4.040703in}}{\pgfqpoint{1.697631in}{4.051303in}}{\pgfqpoint{1.697631in}{4.062353in}}%
\pgfpathcurveto{\pgfqpoint{1.697631in}{4.073403in}}{\pgfqpoint{1.693241in}{4.084002in}}{\pgfqpoint{1.685427in}{4.091815in}}%
\pgfpathcurveto{\pgfqpoint{1.677614in}{4.099629in}}{\pgfqpoint{1.667015in}{4.104019in}}{\pgfqpoint{1.655964in}{4.104019in}}%
\pgfpathcurveto{\pgfqpoint{1.644914in}{4.104019in}}{\pgfqpoint{1.634315in}{4.099629in}}{\pgfqpoint{1.626502in}{4.091815in}}%
\pgfpathcurveto{\pgfqpoint{1.618688in}{4.084002in}}{\pgfqpoint{1.614298in}{4.073403in}}{\pgfqpoint{1.614298in}{4.062353in}}%
\pgfpathcurveto{\pgfqpoint{1.614298in}{4.051303in}}{\pgfqpoint{1.618688in}{4.040703in}}{\pgfqpoint{1.626502in}{4.032890in}}%
\pgfpathcurveto{\pgfqpoint{1.634315in}{4.025076in}}{\pgfqpoint{1.644914in}{4.020686in}}{\pgfqpoint{1.655964in}{4.020686in}}%
\pgfpathlineto{\pgfqpoint{1.655964in}{4.020686in}}%
\pgfpathclose%
\pgfusepath{stroke,fill}%
\end{pgfscope}%
\begin{pgfscope}%
\pgfpathrectangle{\pgfqpoint{0.800049in}{0.448486in}}{\pgfqpoint{3.531733in}{3.696000in}}%
\pgfusepath{clip}%
\pgfsetbuttcap%
\pgfsetroundjoin%
\definecolor{currentfill}{rgb}{0.894118,0.101961,0.109804}%
\pgfsetfillcolor{currentfill}%
\pgfsetfillopacity{0.600000}%
\pgfsetlinewidth{1.003750pt}%
\definecolor{currentstroke}{rgb}{0.894118,0.101961,0.109804}%
\pgfsetstrokecolor{currentstroke}%
\pgfsetstrokeopacity{0.600000}%
\pgfsetdash{}{0pt}%
\pgfpathmoveto{\pgfqpoint{1.655964in}{4.020686in}}%
\pgfpathcurveto{\pgfqpoint{1.667015in}{4.020686in}}{\pgfqpoint{1.677614in}{4.025076in}}{\pgfqpoint{1.685427in}{4.032890in}}%
\pgfpathcurveto{\pgfqpoint{1.693241in}{4.040703in}}{\pgfqpoint{1.697631in}{4.051303in}}{\pgfqpoint{1.697631in}{4.062353in}}%
\pgfpathcurveto{\pgfqpoint{1.697631in}{4.073403in}}{\pgfqpoint{1.693241in}{4.084002in}}{\pgfqpoint{1.685427in}{4.091815in}}%
\pgfpathcurveto{\pgfqpoint{1.677614in}{4.099629in}}{\pgfqpoint{1.667015in}{4.104019in}}{\pgfqpoint{1.655964in}{4.104019in}}%
\pgfpathcurveto{\pgfqpoint{1.644914in}{4.104019in}}{\pgfqpoint{1.634315in}{4.099629in}}{\pgfqpoint{1.626502in}{4.091815in}}%
\pgfpathcurveto{\pgfqpoint{1.618688in}{4.084002in}}{\pgfqpoint{1.614298in}{4.073403in}}{\pgfqpoint{1.614298in}{4.062353in}}%
\pgfpathcurveto{\pgfqpoint{1.614298in}{4.051303in}}{\pgfqpoint{1.618688in}{4.040703in}}{\pgfqpoint{1.626502in}{4.032890in}}%
\pgfpathcurveto{\pgfqpoint{1.634315in}{4.025076in}}{\pgfqpoint{1.644914in}{4.020686in}}{\pgfqpoint{1.655964in}{4.020686in}}%
\pgfpathlineto{\pgfqpoint{1.655964in}{4.020686in}}%
\pgfpathclose%
\pgfusepath{stroke,fill}%
\end{pgfscope}%
\begin{pgfscope}%
\pgfpathrectangle{\pgfqpoint{0.800049in}{0.448486in}}{\pgfqpoint{3.531733in}{3.696000in}}%
\pgfusepath{clip}%
\pgfsetbuttcap%
\pgfsetroundjoin%
\definecolor{currentfill}{rgb}{0.894118,0.101961,0.109804}%
\pgfsetfillcolor{currentfill}%
\pgfsetfillopacity{0.600000}%
\pgfsetlinewidth{1.003750pt}%
\definecolor{currentstroke}{rgb}{0.894118,0.101961,0.109804}%
\pgfsetstrokecolor{currentstroke}%
\pgfsetstrokeopacity{0.600000}%
\pgfsetdash{}{0pt}%
\pgfpathmoveto{\pgfqpoint{1.655964in}{4.020686in}}%
\pgfpathcurveto{\pgfqpoint{1.667015in}{4.020686in}}{\pgfqpoint{1.677614in}{4.025076in}}{\pgfqpoint{1.685427in}{4.032890in}}%
\pgfpathcurveto{\pgfqpoint{1.693241in}{4.040703in}}{\pgfqpoint{1.697631in}{4.051303in}}{\pgfqpoint{1.697631in}{4.062353in}}%
\pgfpathcurveto{\pgfqpoint{1.697631in}{4.073403in}}{\pgfqpoint{1.693241in}{4.084002in}}{\pgfqpoint{1.685427in}{4.091815in}}%
\pgfpathcurveto{\pgfqpoint{1.677614in}{4.099629in}}{\pgfqpoint{1.667015in}{4.104019in}}{\pgfqpoint{1.655964in}{4.104019in}}%
\pgfpathcurveto{\pgfqpoint{1.644914in}{4.104019in}}{\pgfqpoint{1.634315in}{4.099629in}}{\pgfqpoint{1.626502in}{4.091815in}}%
\pgfpathcurveto{\pgfqpoint{1.618688in}{4.084002in}}{\pgfqpoint{1.614298in}{4.073403in}}{\pgfqpoint{1.614298in}{4.062353in}}%
\pgfpathcurveto{\pgfqpoint{1.614298in}{4.051303in}}{\pgfqpoint{1.618688in}{4.040703in}}{\pgfqpoint{1.626502in}{4.032890in}}%
\pgfpathcurveto{\pgfqpoint{1.634315in}{4.025076in}}{\pgfqpoint{1.644914in}{4.020686in}}{\pgfqpoint{1.655964in}{4.020686in}}%
\pgfpathlineto{\pgfqpoint{1.655964in}{4.020686in}}%
\pgfpathclose%
\pgfusepath{stroke,fill}%
\end{pgfscope}%
\begin{pgfscope}%
\pgfpathrectangle{\pgfqpoint{0.800049in}{0.448486in}}{\pgfqpoint{3.531733in}{3.696000in}}%
\pgfusepath{clip}%
\pgfsetbuttcap%
\pgfsetroundjoin%
\definecolor{currentfill}{rgb}{0.894118,0.101961,0.109804}%
\pgfsetfillcolor{currentfill}%
\pgfsetfillopacity{0.600000}%
\pgfsetlinewidth{1.003750pt}%
\definecolor{currentstroke}{rgb}{0.894118,0.101961,0.109804}%
\pgfsetstrokecolor{currentstroke}%
\pgfsetstrokeopacity{0.600000}%
\pgfsetdash{}{0pt}%
\pgfpathmoveto{\pgfqpoint{1.655964in}{4.020686in}}%
\pgfpathcurveto{\pgfqpoint{1.667015in}{4.020686in}}{\pgfqpoint{1.677614in}{4.025076in}}{\pgfqpoint{1.685427in}{4.032890in}}%
\pgfpathcurveto{\pgfqpoint{1.693241in}{4.040703in}}{\pgfqpoint{1.697631in}{4.051303in}}{\pgfqpoint{1.697631in}{4.062353in}}%
\pgfpathcurveto{\pgfqpoint{1.697631in}{4.073403in}}{\pgfqpoint{1.693241in}{4.084002in}}{\pgfqpoint{1.685427in}{4.091815in}}%
\pgfpathcurveto{\pgfqpoint{1.677614in}{4.099629in}}{\pgfqpoint{1.667015in}{4.104019in}}{\pgfqpoint{1.655964in}{4.104019in}}%
\pgfpathcurveto{\pgfqpoint{1.644914in}{4.104019in}}{\pgfqpoint{1.634315in}{4.099629in}}{\pgfqpoint{1.626502in}{4.091815in}}%
\pgfpathcurveto{\pgfqpoint{1.618688in}{4.084002in}}{\pgfqpoint{1.614298in}{4.073403in}}{\pgfqpoint{1.614298in}{4.062353in}}%
\pgfpathcurveto{\pgfqpoint{1.614298in}{4.051303in}}{\pgfqpoint{1.618688in}{4.040703in}}{\pgfqpoint{1.626502in}{4.032890in}}%
\pgfpathcurveto{\pgfqpoint{1.634315in}{4.025076in}}{\pgfqpoint{1.644914in}{4.020686in}}{\pgfqpoint{1.655964in}{4.020686in}}%
\pgfpathlineto{\pgfqpoint{1.655964in}{4.020686in}}%
\pgfpathclose%
\pgfusepath{stroke,fill}%
\end{pgfscope}%
\begin{pgfscope}%
\pgfpathrectangle{\pgfqpoint{0.800049in}{0.448486in}}{\pgfqpoint{3.531733in}{3.696000in}}%
\pgfusepath{clip}%
\pgfsetbuttcap%
\pgfsetroundjoin%
\definecolor{currentfill}{rgb}{0.894118,0.101961,0.109804}%
\pgfsetfillcolor{currentfill}%
\pgfsetfillopacity{0.600000}%
\pgfsetlinewidth{1.003750pt}%
\definecolor{currentstroke}{rgb}{0.894118,0.101961,0.109804}%
\pgfsetstrokecolor{currentstroke}%
\pgfsetstrokeopacity{0.600000}%
\pgfsetdash{}{0pt}%
\pgfpathmoveto{\pgfqpoint{1.655964in}{3.856419in}}%
\pgfpathcurveto{\pgfqpoint{1.667015in}{3.856419in}}{\pgfqpoint{1.677614in}{3.860810in}}{\pgfqpoint{1.685427in}{3.868623in}}%
\pgfpathcurveto{\pgfqpoint{1.693241in}{3.876437in}}{\pgfqpoint{1.697631in}{3.887036in}}{\pgfqpoint{1.697631in}{3.898086in}}%
\pgfpathcurveto{\pgfqpoint{1.697631in}{3.909136in}}{\pgfqpoint{1.693241in}{3.919735in}}{\pgfqpoint{1.685427in}{3.927549in}}%
\pgfpathcurveto{\pgfqpoint{1.677614in}{3.935362in}}{\pgfqpoint{1.667015in}{3.939753in}}{\pgfqpoint{1.655964in}{3.939753in}}%
\pgfpathcurveto{\pgfqpoint{1.644914in}{3.939753in}}{\pgfqpoint{1.634315in}{3.935362in}}{\pgfqpoint{1.626502in}{3.927549in}}%
\pgfpathcurveto{\pgfqpoint{1.618688in}{3.919735in}}{\pgfqpoint{1.614298in}{3.909136in}}{\pgfqpoint{1.614298in}{3.898086in}}%
\pgfpathcurveto{\pgfqpoint{1.614298in}{3.887036in}}{\pgfqpoint{1.618688in}{3.876437in}}{\pgfqpoint{1.626502in}{3.868623in}}%
\pgfpathcurveto{\pgfqpoint{1.634315in}{3.860810in}}{\pgfqpoint{1.644914in}{3.856419in}}{\pgfqpoint{1.655964in}{3.856419in}}%
\pgfpathlineto{\pgfqpoint{1.655964in}{3.856419in}}%
\pgfpathclose%
\pgfusepath{stroke,fill}%
\end{pgfscope}%
\begin{pgfscope}%
\pgfpathrectangle{\pgfqpoint{0.800049in}{0.448486in}}{\pgfqpoint{3.531733in}{3.696000in}}%
\pgfusepath{clip}%
\pgfsetbuttcap%
\pgfsetroundjoin%
\definecolor{currentfill}{rgb}{0.894118,0.101961,0.109804}%
\pgfsetfillcolor{currentfill}%
\pgfsetfillopacity{0.600000}%
\pgfsetlinewidth{1.003750pt}%
\definecolor{currentstroke}{rgb}{0.894118,0.101961,0.109804}%
\pgfsetstrokecolor{currentstroke}%
\pgfsetstrokeopacity{0.600000}%
\pgfsetdash{}{0pt}%
\pgfpathmoveto{\pgfqpoint{1.655964in}{3.856419in}}%
\pgfpathcurveto{\pgfqpoint{1.667015in}{3.856419in}}{\pgfqpoint{1.677614in}{3.860810in}}{\pgfqpoint{1.685427in}{3.868623in}}%
\pgfpathcurveto{\pgfqpoint{1.693241in}{3.876437in}}{\pgfqpoint{1.697631in}{3.887036in}}{\pgfqpoint{1.697631in}{3.898086in}}%
\pgfpathcurveto{\pgfqpoint{1.697631in}{3.909136in}}{\pgfqpoint{1.693241in}{3.919735in}}{\pgfqpoint{1.685427in}{3.927549in}}%
\pgfpathcurveto{\pgfqpoint{1.677614in}{3.935362in}}{\pgfqpoint{1.667015in}{3.939753in}}{\pgfqpoint{1.655964in}{3.939753in}}%
\pgfpathcurveto{\pgfqpoint{1.644914in}{3.939753in}}{\pgfqpoint{1.634315in}{3.935362in}}{\pgfqpoint{1.626502in}{3.927549in}}%
\pgfpathcurveto{\pgfqpoint{1.618688in}{3.919735in}}{\pgfqpoint{1.614298in}{3.909136in}}{\pgfqpoint{1.614298in}{3.898086in}}%
\pgfpathcurveto{\pgfqpoint{1.614298in}{3.887036in}}{\pgfqpoint{1.618688in}{3.876437in}}{\pgfqpoint{1.626502in}{3.868623in}}%
\pgfpathcurveto{\pgfqpoint{1.634315in}{3.860810in}}{\pgfqpoint{1.644914in}{3.856419in}}{\pgfqpoint{1.655964in}{3.856419in}}%
\pgfpathlineto{\pgfqpoint{1.655964in}{3.856419in}}%
\pgfpathclose%
\pgfusepath{stroke,fill}%
\end{pgfscope}%
\begin{pgfscope}%
\pgfpathrectangle{\pgfqpoint{0.800049in}{0.448486in}}{\pgfqpoint{3.531733in}{3.696000in}}%
\pgfusepath{clip}%
\pgfsetbuttcap%
\pgfsetroundjoin%
\definecolor{currentfill}{rgb}{0.894118,0.101961,0.109804}%
\pgfsetfillcolor{currentfill}%
\pgfsetfillopacity{0.600000}%
\pgfsetlinewidth{1.003750pt}%
\definecolor{currentstroke}{rgb}{0.894118,0.101961,0.109804}%
\pgfsetstrokecolor{currentstroke}%
\pgfsetstrokeopacity{0.600000}%
\pgfsetdash{}{0pt}%
\pgfpathmoveto{\pgfqpoint{1.655964in}{3.856419in}}%
\pgfpathcurveto{\pgfqpoint{1.667015in}{3.856419in}}{\pgfqpoint{1.677614in}{3.860810in}}{\pgfqpoint{1.685427in}{3.868623in}}%
\pgfpathcurveto{\pgfqpoint{1.693241in}{3.876437in}}{\pgfqpoint{1.697631in}{3.887036in}}{\pgfqpoint{1.697631in}{3.898086in}}%
\pgfpathcurveto{\pgfqpoint{1.697631in}{3.909136in}}{\pgfqpoint{1.693241in}{3.919735in}}{\pgfqpoint{1.685427in}{3.927549in}}%
\pgfpathcurveto{\pgfqpoint{1.677614in}{3.935362in}}{\pgfqpoint{1.667015in}{3.939753in}}{\pgfqpoint{1.655964in}{3.939753in}}%
\pgfpathcurveto{\pgfqpoint{1.644914in}{3.939753in}}{\pgfqpoint{1.634315in}{3.935362in}}{\pgfqpoint{1.626502in}{3.927549in}}%
\pgfpathcurveto{\pgfqpoint{1.618688in}{3.919735in}}{\pgfqpoint{1.614298in}{3.909136in}}{\pgfqpoint{1.614298in}{3.898086in}}%
\pgfpathcurveto{\pgfqpoint{1.614298in}{3.887036in}}{\pgfqpoint{1.618688in}{3.876437in}}{\pgfqpoint{1.626502in}{3.868623in}}%
\pgfpathcurveto{\pgfqpoint{1.634315in}{3.860810in}}{\pgfqpoint{1.644914in}{3.856419in}}{\pgfqpoint{1.655964in}{3.856419in}}%
\pgfpathlineto{\pgfqpoint{1.655964in}{3.856419in}}%
\pgfpathclose%
\pgfusepath{stroke,fill}%
\end{pgfscope}%
\begin{pgfscope}%
\pgfpathrectangle{\pgfqpoint{0.800049in}{0.448486in}}{\pgfqpoint{3.531733in}{3.696000in}}%
\pgfusepath{clip}%
\pgfsetbuttcap%
\pgfsetroundjoin%
\definecolor{currentfill}{rgb}{0.894118,0.101961,0.109804}%
\pgfsetfillcolor{currentfill}%
\pgfsetfillopacity{0.600000}%
\pgfsetlinewidth{1.003750pt}%
\definecolor{currentstroke}{rgb}{0.894118,0.101961,0.109804}%
\pgfsetstrokecolor{currentstroke}%
\pgfsetstrokeopacity{0.600000}%
\pgfsetdash{}{0pt}%
\pgfpathmoveto{\pgfqpoint{1.655964in}{3.337682in}}%
\pgfpathcurveto{\pgfqpoint{1.667015in}{3.337682in}}{\pgfqpoint{1.677614in}{3.342073in}}{\pgfqpoint{1.685427in}{3.349886in}}%
\pgfpathcurveto{\pgfqpoint{1.693241in}{3.357700in}}{\pgfqpoint{1.697631in}{3.368299in}}{\pgfqpoint{1.697631in}{3.379349in}}%
\pgfpathcurveto{\pgfqpoint{1.697631in}{3.390399in}}{\pgfqpoint{1.693241in}{3.400998in}}{\pgfqpoint{1.685427in}{3.408812in}}%
\pgfpathcurveto{\pgfqpoint{1.677614in}{3.416626in}}{\pgfqpoint{1.667015in}{3.421016in}}{\pgfqpoint{1.655964in}{3.421016in}}%
\pgfpathcurveto{\pgfqpoint{1.644914in}{3.421016in}}{\pgfqpoint{1.634315in}{3.416626in}}{\pgfqpoint{1.626502in}{3.408812in}}%
\pgfpathcurveto{\pgfqpoint{1.618688in}{3.400998in}}{\pgfqpoint{1.614298in}{3.390399in}}{\pgfqpoint{1.614298in}{3.379349in}}%
\pgfpathcurveto{\pgfqpoint{1.614298in}{3.368299in}}{\pgfqpoint{1.618688in}{3.357700in}}{\pgfqpoint{1.626502in}{3.349886in}}%
\pgfpathcurveto{\pgfqpoint{1.634315in}{3.342073in}}{\pgfqpoint{1.644914in}{3.337682in}}{\pgfqpoint{1.655964in}{3.337682in}}%
\pgfpathlineto{\pgfqpoint{1.655964in}{3.337682in}}%
\pgfpathclose%
\pgfusepath{stroke,fill}%
\end{pgfscope}%
\begin{pgfscope}%
\pgfpathrectangle{\pgfqpoint{0.800049in}{0.448486in}}{\pgfqpoint{3.531733in}{3.696000in}}%
\pgfusepath{clip}%
\pgfsetbuttcap%
\pgfsetroundjoin%
\definecolor{currentfill}{rgb}{0.894118,0.101961,0.109804}%
\pgfsetfillcolor{currentfill}%
\pgfsetfillopacity{0.600000}%
\pgfsetlinewidth{1.003750pt}%
\definecolor{currentstroke}{rgb}{0.894118,0.101961,0.109804}%
\pgfsetstrokecolor{currentstroke}%
\pgfsetstrokeopacity{0.600000}%
\pgfsetdash{}{0pt}%
\pgfpathmoveto{\pgfqpoint{1.655964in}{3.337682in}}%
\pgfpathcurveto{\pgfqpoint{1.667015in}{3.337682in}}{\pgfqpoint{1.677614in}{3.342073in}}{\pgfqpoint{1.685427in}{3.349886in}}%
\pgfpathcurveto{\pgfqpoint{1.693241in}{3.357700in}}{\pgfqpoint{1.697631in}{3.368299in}}{\pgfqpoint{1.697631in}{3.379349in}}%
\pgfpathcurveto{\pgfqpoint{1.697631in}{3.390399in}}{\pgfqpoint{1.693241in}{3.400998in}}{\pgfqpoint{1.685427in}{3.408812in}}%
\pgfpathcurveto{\pgfqpoint{1.677614in}{3.416626in}}{\pgfqpoint{1.667015in}{3.421016in}}{\pgfqpoint{1.655964in}{3.421016in}}%
\pgfpathcurveto{\pgfqpoint{1.644914in}{3.421016in}}{\pgfqpoint{1.634315in}{3.416626in}}{\pgfqpoint{1.626502in}{3.408812in}}%
\pgfpathcurveto{\pgfqpoint{1.618688in}{3.400998in}}{\pgfqpoint{1.614298in}{3.390399in}}{\pgfqpoint{1.614298in}{3.379349in}}%
\pgfpathcurveto{\pgfqpoint{1.614298in}{3.368299in}}{\pgfqpoint{1.618688in}{3.357700in}}{\pgfqpoint{1.626502in}{3.349886in}}%
\pgfpathcurveto{\pgfqpoint{1.634315in}{3.342073in}}{\pgfqpoint{1.644914in}{3.337682in}}{\pgfqpoint{1.655964in}{3.337682in}}%
\pgfpathlineto{\pgfqpoint{1.655964in}{3.337682in}}%
\pgfpathclose%
\pgfusepath{stroke,fill}%
\end{pgfscope}%
\begin{pgfscope}%
\pgfpathrectangle{\pgfqpoint{0.800049in}{0.448486in}}{\pgfqpoint{3.531733in}{3.696000in}}%
\pgfusepath{clip}%
\pgfsetbuttcap%
\pgfsetroundjoin%
\definecolor{currentfill}{rgb}{0.894118,0.101961,0.109804}%
\pgfsetfillcolor{currentfill}%
\pgfsetfillopacity{0.600000}%
\pgfsetlinewidth{1.003750pt}%
\definecolor{currentstroke}{rgb}{0.894118,0.101961,0.109804}%
\pgfsetstrokecolor{currentstroke}%
\pgfsetstrokeopacity{0.600000}%
\pgfsetdash{}{0pt}%
\pgfpathmoveto{\pgfqpoint{1.655964in}{2.818946in}}%
\pgfpathcurveto{\pgfqpoint{1.667015in}{2.818946in}}{\pgfqpoint{1.677614in}{2.823336in}}{\pgfqpoint{1.685427in}{2.831150in}}%
\pgfpathcurveto{\pgfqpoint{1.693241in}{2.838963in}}{\pgfqpoint{1.697631in}{2.849562in}}{\pgfqpoint{1.697631in}{2.860612in}}%
\pgfpathcurveto{\pgfqpoint{1.697631in}{2.871662in}}{\pgfqpoint{1.693241in}{2.882261in}}{\pgfqpoint{1.685427in}{2.890075in}}%
\pgfpathcurveto{\pgfqpoint{1.677614in}{2.897889in}}{\pgfqpoint{1.667015in}{2.902279in}}{\pgfqpoint{1.655964in}{2.902279in}}%
\pgfpathcurveto{\pgfqpoint{1.644914in}{2.902279in}}{\pgfqpoint{1.634315in}{2.897889in}}{\pgfqpoint{1.626502in}{2.890075in}}%
\pgfpathcurveto{\pgfqpoint{1.618688in}{2.882261in}}{\pgfqpoint{1.614298in}{2.871662in}}{\pgfqpoint{1.614298in}{2.860612in}}%
\pgfpathcurveto{\pgfqpoint{1.614298in}{2.849562in}}{\pgfqpoint{1.618688in}{2.838963in}}{\pgfqpoint{1.626502in}{2.831150in}}%
\pgfpathcurveto{\pgfqpoint{1.634315in}{2.823336in}}{\pgfqpoint{1.644914in}{2.818946in}}{\pgfqpoint{1.655964in}{2.818946in}}%
\pgfpathlineto{\pgfqpoint{1.655964in}{2.818946in}}%
\pgfpathclose%
\pgfusepath{stroke,fill}%
\end{pgfscope}%
\begin{pgfscope}%
\pgfpathrectangle{\pgfqpoint{0.800049in}{0.448486in}}{\pgfqpoint{3.531733in}{3.696000in}}%
\pgfusepath{clip}%
\pgfsetbuttcap%
\pgfsetroundjoin%
\definecolor{currentfill}{rgb}{0.894118,0.101961,0.109804}%
\pgfsetfillcolor{currentfill}%
\pgfsetfillopacity{0.600000}%
\pgfsetlinewidth{1.003750pt}%
\definecolor{currentstroke}{rgb}{0.894118,0.101961,0.109804}%
\pgfsetstrokecolor{currentstroke}%
\pgfsetstrokeopacity{0.600000}%
\pgfsetdash{}{0pt}%
\pgfpathmoveto{\pgfqpoint{1.655964in}{2.300209in}}%
\pgfpathcurveto{\pgfqpoint{1.667015in}{2.300209in}}{\pgfqpoint{1.677614in}{2.304599in}}{\pgfqpoint{1.685427in}{2.312413in}}%
\pgfpathcurveto{\pgfqpoint{1.693241in}{2.320226in}}{\pgfqpoint{1.697631in}{2.330825in}}{\pgfqpoint{1.697631in}{2.341875in}}%
\pgfpathcurveto{\pgfqpoint{1.697631in}{2.352926in}}{\pgfqpoint{1.693241in}{2.363525in}}{\pgfqpoint{1.685427in}{2.371338in}}%
\pgfpathcurveto{\pgfqpoint{1.677614in}{2.379152in}}{\pgfqpoint{1.667015in}{2.383542in}}{\pgfqpoint{1.655964in}{2.383542in}}%
\pgfpathcurveto{\pgfqpoint{1.644914in}{2.383542in}}{\pgfqpoint{1.634315in}{2.379152in}}{\pgfqpoint{1.626502in}{2.371338in}}%
\pgfpathcurveto{\pgfqpoint{1.618688in}{2.363525in}}{\pgfqpoint{1.614298in}{2.352926in}}{\pgfqpoint{1.614298in}{2.341875in}}%
\pgfpathcurveto{\pgfqpoint{1.614298in}{2.330825in}}{\pgfqpoint{1.618688in}{2.320226in}}{\pgfqpoint{1.626502in}{2.312413in}}%
\pgfpathcurveto{\pgfqpoint{1.634315in}{2.304599in}}{\pgfqpoint{1.644914in}{2.300209in}}{\pgfqpoint{1.655964in}{2.300209in}}%
\pgfpathlineto{\pgfqpoint{1.655964in}{2.300209in}}%
\pgfpathclose%
\pgfusepath{stroke,fill}%
\end{pgfscope}%
\begin{pgfscope}%
\pgfpathrectangle{\pgfqpoint{0.800049in}{0.448486in}}{\pgfqpoint{3.531733in}{3.696000in}}%
\pgfusepath{clip}%
\pgfsetbuttcap%
\pgfsetroundjoin%
\definecolor{currentfill}{rgb}{0.894118,0.101961,0.109804}%
\pgfsetfillcolor{currentfill}%
\pgfsetfillopacity{0.600000}%
\pgfsetlinewidth{1.003750pt}%
\definecolor{currentstroke}{rgb}{0.894118,0.101961,0.109804}%
\pgfsetstrokecolor{currentstroke}%
\pgfsetstrokeopacity{0.600000}%
\pgfsetdash{}{0pt}%
\pgfpathmoveto{\pgfqpoint{1.655964in}{1.781472in}}%
\pgfpathcurveto{\pgfqpoint{1.667015in}{1.781472in}}{\pgfqpoint{1.677614in}{1.785862in}}{\pgfqpoint{1.685427in}{1.793676in}}%
\pgfpathcurveto{\pgfqpoint{1.693241in}{1.801489in}}{\pgfqpoint{1.697631in}{1.812088in}}{\pgfqpoint{1.697631in}{1.823139in}}%
\pgfpathcurveto{\pgfqpoint{1.697631in}{1.834189in}}{\pgfqpoint{1.693241in}{1.844788in}}{\pgfqpoint{1.685427in}{1.852601in}}%
\pgfpathcurveto{\pgfqpoint{1.677614in}{1.860415in}}{\pgfqpoint{1.667015in}{1.864805in}}{\pgfqpoint{1.655964in}{1.864805in}}%
\pgfpathcurveto{\pgfqpoint{1.644914in}{1.864805in}}{\pgfqpoint{1.634315in}{1.860415in}}{\pgfqpoint{1.626502in}{1.852601in}}%
\pgfpathcurveto{\pgfqpoint{1.618688in}{1.844788in}}{\pgfqpoint{1.614298in}{1.834189in}}{\pgfqpoint{1.614298in}{1.823139in}}%
\pgfpathcurveto{\pgfqpoint{1.614298in}{1.812088in}}{\pgfqpoint{1.618688in}{1.801489in}}{\pgfqpoint{1.626502in}{1.793676in}}%
\pgfpathcurveto{\pgfqpoint{1.634315in}{1.785862in}}{\pgfqpoint{1.644914in}{1.781472in}}{\pgfqpoint{1.655964in}{1.781472in}}%
\pgfpathlineto{\pgfqpoint{1.655964in}{1.781472in}}%
\pgfpathclose%
\pgfusepath{stroke,fill}%
\end{pgfscope}%
\begin{pgfscope}%
\pgfpathrectangle{\pgfqpoint{0.800049in}{0.448486in}}{\pgfqpoint{3.531733in}{3.696000in}}%
\pgfusepath{clip}%
\pgfsetbuttcap%
\pgfsetroundjoin%
\definecolor{currentfill}{rgb}{0.894118,0.101961,0.109804}%
\pgfsetfillcolor{currentfill}%
\pgfsetfillopacity{0.600000}%
\pgfsetlinewidth{1.003750pt}%
\definecolor{currentstroke}{rgb}{0.894118,0.101961,0.109804}%
\pgfsetstrokecolor{currentstroke}%
\pgfsetstrokeopacity{0.600000}%
\pgfsetdash{}{0pt}%
\pgfpathmoveto{\pgfqpoint{0.964315in}{4.020686in}}%
\pgfpathcurveto{\pgfqpoint{0.975365in}{4.020686in}}{\pgfqpoint{0.985964in}{4.025076in}}{\pgfqpoint{0.993778in}{4.032890in}}%
\pgfpathcurveto{\pgfqpoint{1.001592in}{4.040703in}}{\pgfqpoint{1.005982in}{4.051303in}}{\pgfqpoint{1.005982in}{4.062353in}}%
\pgfpathcurveto{\pgfqpoint{1.005982in}{4.073403in}}{\pgfqpoint{1.001592in}{4.084002in}}{\pgfqpoint{0.993778in}{4.091815in}}%
\pgfpathcurveto{\pgfqpoint{0.985964in}{4.099629in}}{\pgfqpoint{0.975365in}{4.104019in}}{\pgfqpoint{0.964315in}{4.104019in}}%
\pgfpathcurveto{\pgfqpoint{0.953265in}{4.104019in}}{\pgfqpoint{0.942666in}{4.099629in}}{\pgfqpoint{0.934852in}{4.091815in}}%
\pgfpathcurveto{\pgfqpoint{0.927039in}{4.084002in}}{\pgfqpoint{0.922649in}{4.073403in}}{\pgfqpoint{0.922649in}{4.062353in}}%
\pgfpathcurveto{\pgfqpoint{0.922649in}{4.051303in}}{\pgfqpoint{0.927039in}{4.040703in}}{\pgfqpoint{0.934852in}{4.032890in}}%
\pgfpathcurveto{\pgfqpoint{0.942666in}{4.025076in}}{\pgfqpoint{0.953265in}{4.020686in}}{\pgfqpoint{0.964315in}{4.020686in}}%
\pgfpathlineto{\pgfqpoint{0.964315in}{4.020686in}}%
\pgfpathclose%
\pgfusepath{stroke,fill}%
\end{pgfscope}%
\begin{pgfscope}%
\pgfsetbuttcap%
\pgfsetroundjoin%
\definecolor{currentfill}{rgb}{0.000000,0.000000,0.000000}%
\pgfsetfillcolor{currentfill}%
\pgfsetlinewidth{0.803000pt}%
\definecolor{currentstroke}{rgb}{0.000000,0.000000,0.000000}%
\pgfsetstrokecolor{currentstroke}%
\pgfsetdash{}{0pt}%
\pgfsys@defobject{currentmarker}{\pgfqpoint{0.000000in}{-0.048611in}}{\pgfqpoint{0.000000in}{0.000000in}}{%
\pgfpathmoveto{\pgfqpoint{0.000000in}{0.000000in}}%
\pgfpathlineto{\pgfqpoint{0.000000in}{-0.048611in}}%
\pgfusepath{stroke,fill}%
}%
\begin{pgfscope}%
\pgfsys@transformshift{0.964315in}{0.448486in}%
\pgfsys@useobject{currentmarker}{}%
\end{pgfscope}%
\end{pgfscope}%
\begin{pgfscope}%
\definecolor{textcolor}{rgb}{0.000000,0.000000,0.000000}%
\pgfsetstrokecolor{textcolor}%
\pgfsetfillcolor{textcolor}%
\pgftext[x=0.964315in,y=0.351264in,,top]{\color{textcolor}{\rmfamily\fontsize{10.000000}{12.000000}\selectfont\catcode`\^=\active\def^{\ifmmode\sp\else\^{}\fi}\catcode`\%=\active\def%{\%}$\mathdefault{0}$}}%
\end{pgfscope}%
\begin{pgfscope}%
\pgfsetbuttcap%
\pgfsetroundjoin%
\definecolor{currentfill}{rgb}{0.000000,0.000000,0.000000}%
\pgfsetfillcolor{currentfill}%
\pgfsetlinewidth{0.803000pt}%
\definecolor{currentstroke}{rgb}{0.000000,0.000000,0.000000}%
\pgfsetstrokecolor{currentstroke}%
\pgfsetdash{}{0pt}%
\pgfsys@defobject{currentmarker}{\pgfqpoint{0.000000in}{-0.048611in}}{\pgfqpoint{0.000000in}{0.000000in}}{%
\pgfpathmoveto{\pgfqpoint{0.000000in}{0.000000in}}%
\pgfpathlineto{\pgfqpoint{0.000000in}{-0.048611in}}%
\pgfusepath{stroke,fill}%
}%
\begin{pgfscope}%
\pgfsys@transformshift{1.828877in}{0.448486in}%
\pgfsys@useobject{currentmarker}{}%
\end{pgfscope}%
\end{pgfscope}%
\begin{pgfscope}%
\definecolor{textcolor}{rgb}{0.000000,0.000000,0.000000}%
\pgfsetstrokecolor{textcolor}%
\pgfsetfillcolor{textcolor}%
\pgftext[x=1.828877in,y=0.351264in,,top]{\color{textcolor}{\rmfamily\fontsize{10.000000}{12.000000}\selectfont\catcode`\^=\active\def^{\ifmmode\sp\else\^{}\fi}\catcode`\%=\active\def%{\%}$\mathdefault{5}$}}%
\end{pgfscope}%
\begin{pgfscope}%
\pgfsetbuttcap%
\pgfsetroundjoin%
\definecolor{currentfill}{rgb}{0.000000,0.000000,0.000000}%
\pgfsetfillcolor{currentfill}%
\pgfsetlinewidth{0.803000pt}%
\definecolor{currentstroke}{rgb}{0.000000,0.000000,0.000000}%
\pgfsetstrokecolor{currentstroke}%
\pgfsetdash{}{0pt}%
\pgfsys@defobject{currentmarker}{\pgfqpoint{0.000000in}{-0.048611in}}{\pgfqpoint{0.000000in}{0.000000in}}{%
\pgfpathmoveto{\pgfqpoint{0.000000in}{0.000000in}}%
\pgfpathlineto{\pgfqpoint{0.000000in}{-0.048611in}}%
\pgfusepath{stroke,fill}%
}%
\begin{pgfscope}%
\pgfsys@transformshift{2.693438in}{0.448486in}%
\pgfsys@useobject{currentmarker}{}%
\end{pgfscope}%
\end{pgfscope}%
\begin{pgfscope}%
\definecolor{textcolor}{rgb}{0.000000,0.000000,0.000000}%
\pgfsetstrokecolor{textcolor}%
\pgfsetfillcolor{textcolor}%
\pgftext[x=2.693438in,y=0.351264in,,top]{\color{textcolor}{\rmfamily\fontsize{10.000000}{12.000000}\selectfont\catcode`\^=\active\def^{\ifmmode\sp\else\^{}\fi}\catcode`\%=\active\def%{\%}$\mathdefault{10}$}}%
\end{pgfscope}%
\begin{pgfscope}%
\pgfsetbuttcap%
\pgfsetroundjoin%
\definecolor{currentfill}{rgb}{0.000000,0.000000,0.000000}%
\pgfsetfillcolor{currentfill}%
\pgfsetlinewidth{0.803000pt}%
\definecolor{currentstroke}{rgb}{0.000000,0.000000,0.000000}%
\pgfsetstrokecolor{currentstroke}%
\pgfsetdash{}{0pt}%
\pgfsys@defobject{currentmarker}{\pgfqpoint{0.000000in}{-0.048611in}}{\pgfqpoint{0.000000in}{0.000000in}}{%
\pgfpathmoveto{\pgfqpoint{0.000000in}{0.000000in}}%
\pgfpathlineto{\pgfqpoint{0.000000in}{-0.048611in}}%
\pgfusepath{stroke,fill}%
}%
\begin{pgfscope}%
\pgfsys@transformshift{3.557999in}{0.448486in}%
\pgfsys@useobject{currentmarker}{}%
\end{pgfscope}%
\end{pgfscope}%
\begin{pgfscope}%
\definecolor{textcolor}{rgb}{0.000000,0.000000,0.000000}%
\pgfsetstrokecolor{textcolor}%
\pgfsetfillcolor{textcolor}%
\pgftext[x=3.557999in,y=0.351264in,,top]{\color{textcolor}{\rmfamily\fontsize{10.000000}{12.000000}\selectfont\catcode`\^=\active\def^{\ifmmode\sp\else\^{}\fi}\catcode`\%=\active\def%{\%}$\mathdefault{15}$}}%
\end{pgfscope}%
\begin{pgfscope}%
\definecolor{textcolor}{rgb}{0.000000,0.000000,0.000000}%
\pgfsetstrokecolor{textcolor}%
\pgfsetfillcolor{textcolor}%
\pgftext[x=2.565915in,y=0.161295in,,top]{\color{textcolor}{\rmfamily\fontsize{12.000000}{14.400000}\selectfont\catcode`\^=\active\def^{\ifmmode\sp\else\^{}\fi}\catcode`\%=\active\def%{\%}Birth}}%
\end{pgfscope}%
\begin{pgfscope}%
\pgfsetbuttcap%
\pgfsetroundjoin%
\definecolor{currentfill}{rgb}{0.000000,0.000000,0.000000}%
\pgfsetfillcolor{currentfill}%
\pgfsetlinewidth{0.803000pt}%
\definecolor{currentstroke}{rgb}{0.000000,0.000000,0.000000}%
\pgfsetstrokecolor{currentstroke}%
\pgfsetdash{}{0pt}%
\pgfsys@defobject{currentmarker}{\pgfqpoint{-0.048611in}{0.000000in}}{\pgfqpoint{-0.000000in}{0.000000in}}{%
\pgfpathmoveto{\pgfqpoint{-0.000000in}{0.000000in}}%
\pgfpathlineto{\pgfqpoint{-0.048611in}{0.000000in}}%
\pgfusepath{stroke,fill}%
}%
\begin{pgfscope}%
\pgfsys@transformshift{0.800049in}{0.612753in}%
\pgfsys@useobject{currentmarker}{}%
\end{pgfscope}%
\end{pgfscope}%
\begin{pgfscope}%
\definecolor{textcolor}{rgb}{0.000000,0.000000,0.000000}%
\pgfsetstrokecolor{textcolor}%
\pgfsetfillcolor{textcolor}%
\pgftext[x=0.305216in, y=0.559991in, left, base]{\color{textcolor}{\rmfamily\fontsize{10.000000}{12.000000}\selectfont\catcode`\^=\active\def^{\ifmmode\sp\else\^{}\fi}\catcode`\%=\active\def%{\%}0.000}}%
\end{pgfscope}%
\begin{pgfscope}%
\pgfsetbuttcap%
\pgfsetroundjoin%
\definecolor{currentfill}{rgb}{0.000000,0.000000,0.000000}%
\pgfsetfillcolor{currentfill}%
\pgfsetlinewidth{0.803000pt}%
\definecolor{currentstroke}{rgb}{0.000000,0.000000,0.000000}%
\pgfsetstrokecolor{currentstroke}%
\pgfsetdash{}{0pt}%
\pgfsys@defobject{currentmarker}{\pgfqpoint{-0.048611in}{0.000000in}}{\pgfqpoint{-0.000000in}{0.000000in}}{%
\pgfpathmoveto{\pgfqpoint{-0.000000in}{0.000000in}}%
\pgfpathlineto{\pgfqpoint{-0.048611in}{0.000000in}}%
\pgfusepath{stroke,fill}%
}%
\begin{pgfscope}%
\pgfsys@transformshift{0.800049in}{1.477314in}%
\pgfsys@useobject{currentmarker}{}%
\end{pgfscope}%
\end{pgfscope}%
\begin{pgfscope}%
\definecolor{textcolor}{rgb}{0.000000,0.000000,0.000000}%
\pgfsetstrokecolor{textcolor}%
\pgfsetfillcolor{textcolor}%
\pgftext[x=0.305216in, y=1.424553in, left, base]{\color{textcolor}{\rmfamily\fontsize{10.000000}{12.000000}\selectfont\catcode`\^=\active\def^{\ifmmode\sp\else\^{}\fi}\catcode`\%=\active\def%{\%}5.000}}%
\end{pgfscope}%
\begin{pgfscope}%
\pgfsetbuttcap%
\pgfsetroundjoin%
\definecolor{currentfill}{rgb}{0.000000,0.000000,0.000000}%
\pgfsetfillcolor{currentfill}%
\pgfsetlinewidth{0.803000pt}%
\definecolor{currentstroke}{rgb}{0.000000,0.000000,0.000000}%
\pgfsetstrokecolor{currentstroke}%
\pgfsetdash{}{0pt}%
\pgfsys@defobject{currentmarker}{\pgfqpoint{-0.048611in}{0.000000in}}{\pgfqpoint{-0.000000in}{0.000000in}}{%
\pgfpathmoveto{\pgfqpoint{-0.000000in}{0.000000in}}%
\pgfpathlineto{\pgfqpoint{-0.048611in}{0.000000in}}%
\pgfusepath{stroke,fill}%
}%
\begin{pgfscope}%
\pgfsys@transformshift{0.800049in}{2.341875in}%
\pgfsys@useobject{currentmarker}{}%
\end{pgfscope}%
\end{pgfscope}%
\begin{pgfscope}%
\definecolor{textcolor}{rgb}{0.000000,0.000000,0.000000}%
\pgfsetstrokecolor{textcolor}%
\pgfsetfillcolor{textcolor}%
\pgftext[x=0.216851in, y=2.289114in, left, base]{\color{textcolor}{\rmfamily\fontsize{10.000000}{12.000000}\selectfont\catcode`\^=\active\def^{\ifmmode\sp\else\^{}\fi}\catcode`\%=\active\def%{\%}10.000}}%
\end{pgfscope}%
\begin{pgfscope}%
\pgfsetbuttcap%
\pgfsetroundjoin%
\definecolor{currentfill}{rgb}{0.000000,0.000000,0.000000}%
\pgfsetfillcolor{currentfill}%
\pgfsetlinewidth{0.803000pt}%
\definecolor{currentstroke}{rgb}{0.000000,0.000000,0.000000}%
\pgfsetstrokecolor{currentstroke}%
\pgfsetdash{}{0pt}%
\pgfsys@defobject{currentmarker}{\pgfqpoint{-0.048611in}{0.000000in}}{\pgfqpoint{-0.000000in}{0.000000in}}{%
\pgfpathmoveto{\pgfqpoint{-0.000000in}{0.000000in}}%
\pgfpathlineto{\pgfqpoint{-0.048611in}{0.000000in}}%
\pgfusepath{stroke,fill}%
}%
\begin{pgfscope}%
\pgfsys@transformshift{0.800049in}{3.206437in}%
\pgfsys@useobject{currentmarker}{}%
\end{pgfscope}%
\end{pgfscope}%
\begin{pgfscope}%
\definecolor{textcolor}{rgb}{0.000000,0.000000,0.000000}%
\pgfsetstrokecolor{textcolor}%
\pgfsetfillcolor{textcolor}%
\pgftext[x=0.216851in, y=3.153675in, left, base]{\color{textcolor}{\rmfamily\fontsize{10.000000}{12.000000}\selectfont\catcode`\^=\active\def^{\ifmmode\sp\else\^{}\fi}\catcode`\%=\active\def%{\%}15.000}}%
\end{pgfscope}%
\begin{pgfscope}%
\pgfsetbuttcap%
\pgfsetroundjoin%
\definecolor{currentfill}{rgb}{0.000000,0.000000,0.000000}%
\pgfsetfillcolor{currentfill}%
\pgfsetlinewidth{0.803000pt}%
\definecolor{currentstroke}{rgb}{0.000000,0.000000,0.000000}%
\pgfsetstrokecolor{currentstroke}%
\pgfsetdash{}{0pt}%
\pgfsys@defobject{currentmarker}{\pgfqpoint{-0.048611in}{0.000000in}}{\pgfqpoint{-0.000000in}{0.000000in}}{%
\pgfpathmoveto{\pgfqpoint{-0.000000in}{0.000000in}}%
\pgfpathlineto{\pgfqpoint{-0.048611in}{0.000000in}}%
\pgfusepath{stroke,fill}%
}%
\begin{pgfscope}%
\pgfsys@transformshift{0.800049in}{4.062353in}%
\pgfsys@useobject{currentmarker}{}%
\end{pgfscope}%
\end{pgfscope}%
\begin{pgfscope}%
\definecolor{textcolor}{rgb}{0.000000,0.000000,0.000000}%
\pgfsetstrokecolor{textcolor}%
\pgfsetfillcolor{textcolor}%
\pgftext[x=0.455912in, y=4.009591in, left, base]{\color{textcolor}{\rmfamily\fontsize{10.000000}{12.000000}\selectfont\catcode`\^=\active\def^{\ifmmode\sp\else\^{}\fi}\catcode`\%=\active\def%{\%}$+\infty$}}%
\end{pgfscope}%
\begin{pgfscope}%
\definecolor{textcolor}{rgb}{0.000000,0.000000,0.000000}%
\pgfsetstrokecolor{textcolor}%
\pgfsetfillcolor{textcolor}%
\pgftext[x=0.161295in,y=2.296486in,,bottom,rotate=90.000000]{\color{textcolor}{\rmfamily\fontsize{12.000000}{14.400000}\selectfont\catcode`\^=\active\def^{\ifmmode\sp\else\^{}\fi}\catcode`\%=\active\def%{\%}Death}}%
\end{pgfscope}%
\begin{pgfscope}%
\pgfpathrectangle{\pgfqpoint{0.800049in}{0.448486in}}{\pgfqpoint{3.531733in}{3.696000in}}%
\pgfusepath{clip}%
\pgfsetrectcap%
\pgfsetroundjoin%
\pgfsetlinewidth{1.003750pt}%
\definecolor{currentstroke}{rgb}{0.000000,0.000000,0.000000}%
\pgfsetstrokecolor{currentstroke}%
\pgfsetdash{}{0pt}%
\pgfpathmoveto{\pgfqpoint{0.800049in}{0.448486in}}%
\pgfpathlineto{\pgfqpoint{4.331782in}{3.980219in}}%
\pgfusepath{stroke}%
\end{pgfscope}%
\begin{pgfscope}%
\pgfpathrectangle{\pgfqpoint{0.800049in}{0.448486in}}{\pgfqpoint{3.531733in}{3.696000in}}%
\pgfusepath{clip}%
\pgfsetrectcap%
\pgfsetroundjoin%
\pgfsetlinewidth{1.003750pt}%
\definecolor{currentstroke}{rgb}{0.000000,0.000000,0.000000}%
\pgfsetstrokecolor{currentstroke}%
\pgfsetstrokeopacity{0.600000}%
\pgfsetdash{}{0pt}%
\pgfpathmoveto{\pgfqpoint{0.800049in}{4.062353in}}%
\pgfpathlineto{\pgfqpoint{4.331782in}{4.062353in}}%
\pgfusepath{stroke}%
\end{pgfscope}%
\begin{pgfscope}%
\pgfsetrectcap%
\pgfsetmiterjoin%
\pgfsetlinewidth{0.803000pt}%
\definecolor{currentstroke}{rgb}{0.000000,0.000000,0.000000}%
\pgfsetstrokecolor{currentstroke}%
\pgfsetdash{}{0pt}%
\pgfpathmoveto{\pgfqpoint{0.800049in}{0.448486in}}%
\pgfpathlineto{\pgfqpoint{0.800049in}{4.144486in}}%
\pgfusepath{stroke}%
\end{pgfscope}%
\begin{pgfscope}%
\pgfsetrectcap%
\pgfsetmiterjoin%
\pgfsetlinewidth{0.803000pt}%
\definecolor{currentstroke}{rgb}{0.000000,0.000000,0.000000}%
\pgfsetstrokecolor{currentstroke}%
\pgfsetdash{}{0pt}%
\pgfpathmoveto{\pgfqpoint{4.331782in}{0.448486in}}%
\pgfpathlineto{\pgfqpoint{4.331782in}{4.144486in}}%
\pgfusepath{stroke}%
\end{pgfscope}%
\begin{pgfscope}%
\pgfsetrectcap%
\pgfsetmiterjoin%
\pgfsetlinewidth{0.803000pt}%
\definecolor{currentstroke}{rgb}{0.000000,0.000000,0.000000}%
\pgfsetstrokecolor{currentstroke}%
\pgfsetdash{}{0pt}%
\pgfpathmoveto{\pgfqpoint{0.800049in}{0.448486in}}%
\pgfpathlineto{\pgfqpoint{4.331782in}{0.448486in}}%
\pgfusepath{stroke}%
\end{pgfscope}%
\begin{pgfscope}%
\pgfsetrectcap%
\pgfsetmiterjoin%
\pgfsetlinewidth{0.803000pt}%
\definecolor{currentstroke}{rgb}{0.000000,0.000000,0.000000}%
\pgfsetstrokecolor{currentstroke}%
\pgfsetdash{}{0pt}%
\pgfpathmoveto{\pgfqpoint{0.800049in}{4.144486in}}%
\pgfpathlineto{\pgfqpoint{4.331782in}{4.144486in}}%
\pgfusepath{stroke}%
\end{pgfscope}%
\end{pgfpicture}%
\makeatother%
\endgroup%

        }
        \caption{Without circumcircle filtering}
        \label{fig:nested_ph_no_cc}
    \end{subfigure}
    \begin{subfigure}{0.49\textwidth}
        \resizebox{\textwidth}{!}{
            %% Creator: Matplotlib, PGF backend
%%
%% To include the figure in your LaTeX document, write
%%   \input{<filename>.pgf}
%%
%% Make sure the required packages are loaded in your preamble
%%   \usepackage{pgf}
%%
%% Also ensure that all the required font packages are loaded; for instance,
%% the lmodern package is sometimes necessary when using math font.
%%   \usepackage{lmodern}
%%
%% Figures using additional raster images can only be included by \input if
%% they are in the same directory as the main LaTeX file. For loading figures
%% from other directories you can use the `import` package
%%   \usepackage{import}
%%
%% and then include the figures with
%%   \import{<path to file>}{<filename>.pgf}
%%
%% Matplotlib used the following preamble
%%   \def\mathdefault#1{#1}
%%   \everymath=\expandafter{\the\everymath\displaystyle}
%%   
%%   \ifdefined\pdftexversion\else  % non-pdftex case.
%%     \usepackage{fontspec}
%%     \setmainfont{DejaVuSerif.ttf}[Path=\detokenize{/home/snek/repos/homology-decision-bondaries-clean/venv/lib/python3.9/site-packages/matplotlib/mpl-data/fonts/ttf/}]
%%     \setsansfont{DejaVuSans.ttf}[Path=\detokenize{/home/snek/repos/homology-decision-bondaries-clean/venv/lib/python3.9/site-packages/matplotlib/mpl-data/fonts/ttf/}]
%%     \setmonofont{DejaVuSansMono.ttf}[Path=\detokenize{/home/snek/repos/homology-decision-bondaries-clean/venv/lib/python3.9/site-packages/matplotlib/mpl-data/fonts/ttf/}]
%%   \fi
%%   \makeatletter\@ifpackageloaded{underscore}{}{\usepackage[strings]{underscore}}\makeatother
%%
\begingroup%
\makeatletter%
\begin{pgfpicture}%
\pgfpathrectangle{\pgfpointorigin}{\pgfqpoint{4.331782in}{4.144486in}}%
\pgfusepath{use as bounding box, clip}%
\begin{pgfscope}%
\pgfsetbuttcap%
\pgfsetmiterjoin%
\definecolor{currentfill}{rgb}{1.000000,1.000000,1.000000}%
\pgfsetfillcolor{currentfill}%
\pgfsetlinewidth{0.000000pt}%
\definecolor{currentstroke}{rgb}{1.000000,1.000000,1.000000}%
\pgfsetstrokecolor{currentstroke}%
\pgfsetdash{}{0pt}%
\pgfpathmoveto{\pgfqpoint{0.000000in}{-0.000000in}}%
\pgfpathlineto{\pgfqpoint{4.331782in}{-0.000000in}}%
\pgfpathlineto{\pgfqpoint{4.331782in}{4.144486in}}%
\pgfpathlineto{\pgfqpoint{0.000000in}{4.144486in}}%
\pgfpathlineto{\pgfqpoint{0.000000in}{-0.000000in}}%
\pgfpathclose%
\pgfusepath{fill}%
\end{pgfscope}%
\begin{pgfscope}%
\pgfsetbuttcap%
\pgfsetmiterjoin%
\definecolor{currentfill}{rgb}{1.000000,1.000000,1.000000}%
\pgfsetfillcolor{currentfill}%
\pgfsetlinewidth{0.000000pt}%
\definecolor{currentstroke}{rgb}{0.000000,0.000000,0.000000}%
\pgfsetstrokecolor{currentstroke}%
\pgfsetstrokeopacity{0.000000}%
\pgfsetdash{}{0pt}%
\pgfpathmoveto{\pgfqpoint{0.800049in}{0.448486in}}%
\pgfpathlineto{\pgfqpoint{4.331782in}{0.448486in}}%
\pgfpathlineto{\pgfqpoint{4.331782in}{4.144486in}}%
\pgfpathlineto{\pgfqpoint{0.800049in}{4.144486in}}%
\pgfpathlineto{\pgfqpoint{0.800049in}{0.448486in}}%
\pgfpathclose%
\pgfusepath{fill}%
\end{pgfscope}%
\begin{pgfscope}%
\pgfpathrectangle{\pgfqpoint{0.800049in}{0.448486in}}{\pgfqpoint{3.531733in}{3.696000in}}%
\pgfusepath{clip}%
\pgfsetbuttcap%
\pgfsetmiterjoin%
\definecolor{currentfill}{rgb}{0.827451,0.827451,0.827451}%
\pgfsetfillcolor{currentfill}%
\pgfsetlinewidth{1.003750pt}%
\definecolor{currentstroke}{rgb}{0.827451,0.827451,0.827451}%
\pgfsetstrokecolor{currentstroke}%
\pgfsetdash{}{0pt}%
\pgfpathmoveto{\pgfqpoint{0.800049in}{0.448486in}}%
\pgfpathlineto{\pgfqpoint{4.331782in}{0.448486in}}%
\pgfpathlineto{\pgfqpoint{4.331782in}{3.980219in}}%
\pgfpathlineto{\pgfqpoint{0.800049in}{0.448486in}}%
\pgfpathclose%
\pgfusepath{stroke,fill}%
\end{pgfscope}%
\begin{pgfscope}%
\pgfpathrectangle{\pgfqpoint{0.800049in}{0.448486in}}{\pgfqpoint{3.531733in}{3.696000in}}%
\pgfusepath{clip}%
\pgfsetbuttcap%
\pgfsetroundjoin%
\definecolor{currentfill}{rgb}{0.894118,0.101961,0.109804}%
\pgfsetfillcolor{currentfill}%
\pgfsetfillopacity{0.600000}%
\pgfsetlinewidth{1.003750pt}%
\definecolor{currentstroke}{rgb}{0.894118,0.101961,0.109804}%
\pgfsetstrokecolor{currentstroke}%
\pgfsetstrokeopacity{0.600000}%
\pgfsetdash{}{0pt}%
\pgfpathmoveto{\pgfqpoint{4.249649in}{4.020686in}}%
\pgfpathcurveto{\pgfqpoint{4.260699in}{4.020686in}}{\pgfqpoint{4.271298in}{4.025076in}}{\pgfqpoint{4.279111in}{4.032890in}}%
\pgfpathcurveto{\pgfqpoint{4.286925in}{4.040703in}}{\pgfqpoint{4.291315in}{4.051303in}}{\pgfqpoint{4.291315in}{4.062353in}}%
\pgfpathcurveto{\pgfqpoint{4.291315in}{4.073403in}}{\pgfqpoint{4.286925in}{4.084002in}}{\pgfqpoint{4.279111in}{4.091815in}}%
\pgfpathcurveto{\pgfqpoint{4.271298in}{4.099629in}}{\pgfqpoint{4.260699in}{4.104019in}}{\pgfqpoint{4.249649in}{4.104019in}}%
\pgfpathcurveto{\pgfqpoint{4.238598in}{4.104019in}}{\pgfqpoint{4.227999in}{4.099629in}}{\pgfqpoint{4.220186in}{4.091815in}}%
\pgfpathcurveto{\pgfqpoint{4.212372in}{4.084002in}}{\pgfqpoint{4.207982in}{4.073403in}}{\pgfqpoint{4.207982in}{4.062353in}}%
\pgfpathcurveto{\pgfqpoint{4.207982in}{4.051303in}}{\pgfqpoint{4.212372in}{4.040703in}}{\pgfqpoint{4.220186in}{4.032890in}}%
\pgfpathcurveto{\pgfqpoint{4.227999in}{4.025076in}}{\pgfqpoint{4.238598in}{4.020686in}}{\pgfqpoint{4.249649in}{4.020686in}}%
\pgfpathlineto{\pgfqpoint{4.249649in}{4.020686in}}%
\pgfpathclose%
\pgfusepath{stroke,fill}%
\end{pgfscope}%
\begin{pgfscope}%
\pgfpathrectangle{\pgfqpoint{0.800049in}{0.448486in}}{\pgfqpoint{3.531733in}{3.696000in}}%
\pgfusepath{clip}%
\pgfsetbuttcap%
\pgfsetroundjoin%
\definecolor{currentfill}{rgb}{0.894118,0.101961,0.109804}%
\pgfsetfillcolor{currentfill}%
\pgfsetfillopacity{0.600000}%
\pgfsetlinewidth{1.003750pt}%
\definecolor{currentstroke}{rgb}{0.894118,0.101961,0.109804}%
\pgfsetstrokecolor{currentstroke}%
\pgfsetstrokeopacity{0.600000}%
\pgfsetdash{}{0pt}%
\pgfpathmoveto{\pgfqpoint{4.249649in}{4.020686in}}%
\pgfpathcurveto{\pgfqpoint{4.260699in}{4.020686in}}{\pgfqpoint{4.271298in}{4.025076in}}{\pgfqpoint{4.279111in}{4.032890in}}%
\pgfpathcurveto{\pgfqpoint{4.286925in}{4.040703in}}{\pgfqpoint{4.291315in}{4.051303in}}{\pgfqpoint{4.291315in}{4.062353in}}%
\pgfpathcurveto{\pgfqpoint{4.291315in}{4.073403in}}{\pgfqpoint{4.286925in}{4.084002in}}{\pgfqpoint{4.279111in}{4.091815in}}%
\pgfpathcurveto{\pgfqpoint{4.271298in}{4.099629in}}{\pgfqpoint{4.260699in}{4.104019in}}{\pgfqpoint{4.249649in}{4.104019in}}%
\pgfpathcurveto{\pgfqpoint{4.238598in}{4.104019in}}{\pgfqpoint{4.227999in}{4.099629in}}{\pgfqpoint{4.220186in}{4.091815in}}%
\pgfpathcurveto{\pgfqpoint{4.212372in}{4.084002in}}{\pgfqpoint{4.207982in}{4.073403in}}{\pgfqpoint{4.207982in}{4.062353in}}%
\pgfpathcurveto{\pgfqpoint{4.207982in}{4.051303in}}{\pgfqpoint{4.212372in}{4.040703in}}{\pgfqpoint{4.220186in}{4.032890in}}%
\pgfpathcurveto{\pgfqpoint{4.227999in}{4.025076in}}{\pgfqpoint{4.238598in}{4.020686in}}{\pgfqpoint{4.249649in}{4.020686in}}%
\pgfpathlineto{\pgfqpoint{4.249649in}{4.020686in}}%
\pgfpathclose%
\pgfusepath{stroke,fill}%
\end{pgfscope}%
\begin{pgfscope}%
\pgfpathrectangle{\pgfqpoint{0.800049in}{0.448486in}}{\pgfqpoint{3.531733in}{3.696000in}}%
\pgfusepath{clip}%
\pgfsetbuttcap%
\pgfsetroundjoin%
\definecolor{currentfill}{rgb}{0.894118,0.101961,0.109804}%
\pgfsetfillcolor{currentfill}%
\pgfsetfillopacity{0.600000}%
\pgfsetlinewidth{1.003750pt}%
\definecolor{currentstroke}{rgb}{0.894118,0.101961,0.109804}%
\pgfsetstrokecolor{currentstroke}%
\pgfsetstrokeopacity{0.600000}%
\pgfsetdash{}{0pt}%
\pgfpathmoveto{\pgfqpoint{3.428315in}{4.020686in}}%
\pgfpathcurveto{\pgfqpoint{3.439365in}{4.020686in}}{\pgfqpoint{3.449964in}{4.025076in}}{\pgfqpoint{3.457778in}{4.032890in}}%
\pgfpathcurveto{\pgfqpoint{3.465592in}{4.040703in}}{\pgfqpoint{3.469982in}{4.051303in}}{\pgfqpoint{3.469982in}{4.062353in}}%
\pgfpathcurveto{\pgfqpoint{3.469982in}{4.073403in}}{\pgfqpoint{3.465592in}{4.084002in}}{\pgfqpoint{3.457778in}{4.091815in}}%
\pgfpathcurveto{\pgfqpoint{3.449964in}{4.099629in}}{\pgfqpoint{3.439365in}{4.104019in}}{\pgfqpoint{3.428315in}{4.104019in}}%
\pgfpathcurveto{\pgfqpoint{3.417265in}{4.104019in}}{\pgfqpoint{3.406666in}{4.099629in}}{\pgfqpoint{3.398852in}{4.091815in}}%
\pgfpathcurveto{\pgfqpoint{3.391039in}{4.084002in}}{\pgfqpoint{3.386649in}{4.073403in}}{\pgfqpoint{3.386649in}{4.062353in}}%
\pgfpathcurveto{\pgfqpoint{3.386649in}{4.051303in}}{\pgfqpoint{3.391039in}{4.040703in}}{\pgfqpoint{3.398852in}{4.032890in}}%
\pgfpathcurveto{\pgfqpoint{3.406666in}{4.025076in}}{\pgfqpoint{3.417265in}{4.020686in}}{\pgfqpoint{3.428315in}{4.020686in}}%
\pgfpathlineto{\pgfqpoint{3.428315in}{4.020686in}}%
\pgfpathclose%
\pgfusepath{stroke,fill}%
\end{pgfscope}%
\begin{pgfscope}%
\pgfpathrectangle{\pgfqpoint{0.800049in}{0.448486in}}{\pgfqpoint{3.531733in}{3.696000in}}%
\pgfusepath{clip}%
\pgfsetbuttcap%
\pgfsetroundjoin%
\definecolor{currentfill}{rgb}{0.894118,0.101961,0.109804}%
\pgfsetfillcolor{currentfill}%
\pgfsetfillopacity{0.600000}%
\pgfsetlinewidth{1.003750pt}%
\definecolor{currentstroke}{rgb}{0.894118,0.101961,0.109804}%
\pgfsetstrokecolor{currentstroke}%
\pgfsetstrokeopacity{0.600000}%
\pgfsetdash{}{0pt}%
\pgfpathmoveto{\pgfqpoint{3.428315in}{4.020686in}}%
\pgfpathcurveto{\pgfqpoint{3.439365in}{4.020686in}}{\pgfqpoint{3.449964in}{4.025076in}}{\pgfqpoint{3.457778in}{4.032890in}}%
\pgfpathcurveto{\pgfqpoint{3.465592in}{4.040703in}}{\pgfqpoint{3.469982in}{4.051303in}}{\pgfqpoint{3.469982in}{4.062353in}}%
\pgfpathcurveto{\pgfqpoint{3.469982in}{4.073403in}}{\pgfqpoint{3.465592in}{4.084002in}}{\pgfqpoint{3.457778in}{4.091815in}}%
\pgfpathcurveto{\pgfqpoint{3.449964in}{4.099629in}}{\pgfqpoint{3.439365in}{4.104019in}}{\pgfqpoint{3.428315in}{4.104019in}}%
\pgfpathcurveto{\pgfqpoint{3.417265in}{4.104019in}}{\pgfqpoint{3.406666in}{4.099629in}}{\pgfqpoint{3.398852in}{4.091815in}}%
\pgfpathcurveto{\pgfqpoint{3.391039in}{4.084002in}}{\pgfqpoint{3.386649in}{4.073403in}}{\pgfqpoint{3.386649in}{4.062353in}}%
\pgfpathcurveto{\pgfqpoint{3.386649in}{4.051303in}}{\pgfqpoint{3.391039in}{4.040703in}}{\pgfqpoint{3.398852in}{4.032890in}}%
\pgfpathcurveto{\pgfqpoint{3.406666in}{4.025076in}}{\pgfqpoint{3.417265in}{4.020686in}}{\pgfqpoint{3.428315in}{4.020686in}}%
\pgfpathlineto{\pgfqpoint{3.428315in}{4.020686in}}%
\pgfpathclose%
\pgfusepath{stroke,fill}%
\end{pgfscope}%
\begin{pgfscope}%
\pgfpathrectangle{\pgfqpoint{0.800049in}{0.448486in}}{\pgfqpoint{3.531733in}{3.696000in}}%
\pgfusepath{clip}%
\pgfsetbuttcap%
\pgfsetroundjoin%
\definecolor{currentfill}{rgb}{0.894118,0.101961,0.109804}%
\pgfsetfillcolor{currentfill}%
\pgfsetfillopacity{0.600000}%
\pgfsetlinewidth{1.003750pt}%
\definecolor{currentstroke}{rgb}{0.894118,0.101961,0.109804}%
\pgfsetstrokecolor{currentstroke}%
\pgfsetstrokeopacity{0.600000}%
\pgfsetdash{}{0pt}%
\pgfpathmoveto{\pgfqpoint{2.606982in}{4.020686in}}%
\pgfpathcurveto{\pgfqpoint{2.618032in}{4.020686in}}{\pgfqpoint{2.628631in}{4.025076in}}{\pgfqpoint{2.636445in}{4.032890in}}%
\pgfpathcurveto{\pgfqpoint{2.644258in}{4.040703in}}{\pgfqpoint{2.648649in}{4.051303in}}{\pgfqpoint{2.648649in}{4.062353in}}%
\pgfpathcurveto{\pgfqpoint{2.648649in}{4.073403in}}{\pgfqpoint{2.644258in}{4.084002in}}{\pgfqpoint{2.636445in}{4.091815in}}%
\pgfpathcurveto{\pgfqpoint{2.628631in}{4.099629in}}{\pgfqpoint{2.618032in}{4.104019in}}{\pgfqpoint{2.606982in}{4.104019in}}%
\pgfpathcurveto{\pgfqpoint{2.595932in}{4.104019in}}{\pgfqpoint{2.585333in}{4.099629in}}{\pgfqpoint{2.577519in}{4.091815in}}%
\pgfpathcurveto{\pgfqpoint{2.569706in}{4.084002in}}{\pgfqpoint{2.565315in}{4.073403in}}{\pgfqpoint{2.565315in}{4.062353in}}%
\pgfpathcurveto{\pgfqpoint{2.565315in}{4.051303in}}{\pgfqpoint{2.569706in}{4.040703in}}{\pgfqpoint{2.577519in}{4.032890in}}%
\pgfpathcurveto{\pgfqpoint{2.585333in}{4.025076in}}{\pgfqpoint{2.595932in}{4.020686in}}{\pgfqpoint{2.606982in}{4.020686in}}%
\pgfpathlineto{\pgfqpoint{2.606982in}{4.020686in}}%
\pgfpathclose%
\pgfusepath{stroke,fill}%
\end{pgfscope}%
\begin{pgfscope}%
\pgfpathrectangle{\pgfqpoint{0.800049in}{0.448486in}}{\pgfqpoint{3.531733in}{3.696000in}}%
\pgfusepath{clip}%
\pgfsetbuttcap%
\pgfsetroundjoin%
\definecolor{currentfill}{rgb}{0.894118,0.101961,0.109804}%
\pgfsetfillcolor{currentfill}%
\pgfsetfillopacity{0.600000}%
\pgfsetlinewidth{1.003750pt}%
\definecolor{currentstroke}{rgb}{0.894118,0.101961,0.109804}%
\pgfsetstrokecolor{currentstroke}%
\pgfsetstrokeopacity{0.600000}%
\pgfsetdash{}{0pt}%
\pgfpathmoveto{\pgfqpoint{2.606982in}{4.020686in}}%
\pgfpathcurveto{\pgfqpoint{2.618032in}{4.020686in}}{\pgfqpoint{2.628631in}{4.025076in}}{\pgfqpoint{2.636445in}{4.032890in}}%
\pgfpathcurveto{\pgfqpoint{2.644258in}{4.040703in}}{\pgfqpoint{2.648649in}{4.051303in}}{\pgfqpoint{2.648649in}{4.062353in}}%
\pgfpathcurveto{\pgfqpoint{2.648649in}{4.073403in}}{\pgfqpoint{2.644258in}{4.084002in}}{\pgfqpoint{2.636445in}{4.091815in}}%
\pgfpathcurveto{\pgfqpoint{2.628631in}{4.099629in}}{\pgfqpoint{2.618032in}{4.104019in}}{\pgfqpoint{2.606982in}{4.104019in}}%
\pgfpathcurveto{\pgfqpoint{2.595932in}{4.104019in}}{\pgfqpoint{2.585333in}{4.099629in}}{\pgfqpoint{2.577519in}{4.091815in}}%
\pgfpathcurveto{\pgfqpoint{2.569706in}{4.084002in}}{\pgfqpoint{2.565315in}{4.073403in}}{\pgfqpoint{2.565315in}{4.062353in}}%
\pgfpathcurveto{\pgfqpoint{2.565315in}{4.051303in}}{\pgfqpoint{2.569706in}{4.040703in}}{\pgfqpoint{2.577519in}{4.032890in}}%
\pgfpathcurveto{\pgfqpoint{2.585333in}{4.025076in}}{\pgfqpoint{2.595932in}{4.020686in}}{\pgfqpoint{2.606982in}{4.020686in}}%
\pgfpathlineto{\pgfqpoint{2.606982in}{4.020686in}}%
\pgfpathclose%
\pgfusepath{stroke,fill}%
\end{pgfscope}%
\begin{pgfscope}%
\pgfpathrectangle{\pgfqpoint{0.800049in}{0.448486in}}{\pgfqpoint{3.531733in}{3.696000in}}%
\pgfusepath{clip}%
\pgfsetbuttcap%
\pgfsetroundjoin%
\definecolor{currentfill}{rgb}{0.894118,0.101961,0.109804}%
\pgfsetfillcolor{currentfill}%
\pgfsetfillopacity{0.600000}%
\pgfsetlinewidth{1.003750pt}%
\definecolor{currentstroke}{rgb}{0.894118,0.101961,0.109804}%
\pgfsetstrokecolor{currentstroke}%
\pgfsetstrokeopacity{0.600000}%
\pgfsetdash{}{0pt}%
\pgfpathmoveto{\pgfqpoint{2.606982in}{4.020686in}}%
\pgfpathcurveto{\pgfqpoint{2.618032in}{4.020686in}}{\pgfqpoint{2.628631in}{4.025076in}}{\pgfqpoint{2.636445in}{4.032890in}}%
\pgfpathcurveto{\pgfqpoint{2.644258in}{4.040703in}}{\pgfqpoint{2.648649in}{4.051303in}}{\pgfqpoint{2.648649in}{4.062353in}}%
\pgfpathcurveto{\pgfqpoint{2.648649in}{4.073403in}}{\pgfqpoint{2.644258in}{4.084002in}}{\pgfqpoint{2.636445in}{4.091815in}}%
\pgfpathcurveto{\pgfqpoint{2.628631in}{4.099629in}}{\pgfqpoint{2.618032in}{4.104019in}}{\pgfqpoint{2.606982in}{4.104019in}}%
\pgfpathcurveto{\pgfqpoint{2.595932in}{4.104019in}}{\pgfqpoint{2.585333in}{4.099629in}}{\pgfqpoint{2.577519in}{4.091815in}}%
\pgfpathcurveto{\pgfqpoint{2.569706in}{4.084002in}}{\pgfqpoint{2.565315in}{4.073403in}}{\pgfqpoint{2.565315in}{4.062353in}}%
\pgfpathcurveto{\pgfqpoint{2.565315in}{4.051303in}}{\pgfqpoint{2.569706in}{4.040703in}}{\pgfqpoint{2.577519in}{4.032890in}}%
\pgfpathcurveto{\pgfqpoint{2.585333in}{4.025076in}}{\pgfqpoint{2.595932in}{4.020686in}}{\pgfqpoint{2.606982in}{4.020686in}}%
\pgfpathlineto{\pgfqpoint{2.606982in}{4.020686in}}%
\pgfpathclose%
\pgfusepath{stroke,fill}%
\end{pgfscope}%
\begin{pgfscope}%
\pgfpathrectangle{\pgfqpoint{0.800049in}{0.448486in}}{\pgfqpoint{3.531733in}{3.696000in}}%
\pgfusepath{clip}%
\pgfsetbuttcap%
\pgfsetroundjoin%
\definecolor{currentfill}{rgb}{0.894118,0.101961,0.109804}%
\pgfsetfillcolor{currentfill}%
\pgfsetfillopacity{0.600000}%
\pgfsetlinewidth{1.003750pt}%
\definecolor{currentstroke}{rgb}{0.894118,0.101961,0.109804}%
\pgfsetstrokecolor{currentstroke}%
\pgfsetstrokeopacity{0.600000}%
\pgfsetdash{}{0pt}%
\pgfpathmoveto{\pgfqpoint{1.785649in}{4.020686in}}%
\pgfpathcurveto{\pgfqpoint{1.796699in}{4.020686in}}{\pgfqpoint{1.807298in}{4.025076in}}{\pgfqpoint{1.815111in}{4.032890in}}%
\pgfpathcurveto{\pgfqpoint{1.822925in}{4.040703in}}{\pgfqpoint{1.827315in}{4.051303in}}{\pgfqpoint{1.827315in}{4.062353in}}%
\pgfpathcurveto{\pgfqpoint{1.827315in}{4.073403in}}{\pgfqpoint{1.822925in}{4.084002in}}{\pgfqpoint{1.815111in}{4.091815in}}%
\pgfpathcurveto{\pgfqpoint{1.807298in}{4.099629in}}{\pgfqpoint{1.796699in}{4.104019in}}{\pgfqpoint{1.785649in}{4.104019in}}%
\pgfpathcurveto{\pgfqpoint{1.774598in}{4.104019in}}{\pgfqpoint{1.763999in}{4.099629in}}{\pgfqpoint{1.756186in}{4.091815in}}%
\pgfpathcurveto{\pgfqpoint{1.748372in}{4.084002in}}{\pgfqpoint{1.743982in}{4.073403in}}{\pgfqpoint{1.743982in}{4.062353in}}%
\pgfpathcurveto{\pgfqpoint{1.743982in}{4.051303in}}{\pgfqpoint{1.748372in}{4.040703in}}{\pgfqpoint{1.756186in}{4.032890in}}%
\pgfpathcurveto{\pgfqpoint{1.763999in}{4.025076in}}{\pgfqpoint{1.774598in}{4.020686in}}{\pgfqpoint{1.785649in}{4.020686in}}%
\pgfpathlineto{\pgfqpoint{1.785649in}{4.020686in}}%
\pgfpathclose%
\pgfusepath{stroke,fill}%
\end{pgfscope}%
\begin{pgfscope}%
\pgfpathrectangle{\pgfqpoint{0.800049in}{0.448486in}}{\pgfqpoint{3.531733in}{3.696000in}}%
\pgfusepath{clip}%
\pgfsetbuttcap%
\pgfsetroundjoin%
\definecolor{currentfill}{rgb}{0.894118,0.101961,0.109804}%
\pgfsetfillcolor{currentfill}%
\pgfsetfillopacity{0.600000}%
\pgfsetlinewidth{1.003750pt}%
\definecolor{currentstroke}{rgb}{0.894118,0.101961,0.109804}%
\pgfsetstrokecolor{currentstroke}%
\pgfsetstrokeopacity{0.600000}%
\pgfsetdash{}{0pt}%
\pgfpathmoveto{\pgfqpoint{1.785649in}{4.020686in}}%
\pgfpathcurveto{\pgfqpoint{1.796699in}{4.020686in}}{\pgfqpoint{1.807298in}{4.025076in}}{\pgfqpoint{1.815111in}{4.032890in}}%
\pgfpathcurveto{\pgfqpoint{1.822925in}{4.040703in}}{\pgfqpoint{1.827315in}{4.051303in}}{\pgfqpoint{1.827315in}{4.062353in}}%
\pgfpathcurveto{\pgfqpoint{1.827315in}{4.073403in}}{\pgfqpoint{1.822925in}{4.084002in}}{\pgfqpoint{1.815111in}{4.091815in}}%
\pgfpathcurveto{\pgfqpoint{1.807298in}{4.099629in}}{\pgfqpoint{1.796699in}{4.104019in}}{\pgfqpoint{1.785649in}{4.104019in}}%
\pgfpathcurveto{\pgfqpoint{1.774598in}{4.104019in}}{\pgfqpoint{1.763999in}{4.099629in}}{\pgfqpoint{1.756186in}{4.091815in}}%
\pgfpathcurveto{\pgfqpoint{1.748372in}{4.084002in}}{\pgfqpoint{1.743982in}{4.073403in}}{\pgfqpoint{1.743982in}{4.062353in}}%
\pgfpathcurveto{\pgfqpoint{1.743982in}{4.051303in}}{\pgfqpoint{1.748372in}{4.040703in}}{\pgfqpoint{1.756186in}{4.032890in}}%
\pgfpathcurveto{\pgfqpoint{1.763999in}{4.025076in}}{\pgfqpoint{1.774598in}{4.020686in}}{\pgfqpoint{1.785649in}{4.020686in}}%
\pgfpathlineto{\pgfqpoint{1.785649in}{4.020686in}}%
\pgfpathclose%
\pgfusepath{stroke,fill}%
\end{pgfscope}%
\begin{pgfscope}%
\pgfpathrectangle{\pgfqpoint{0.800049in}{0.448486in}}{\pgfqpoint{3.531733in}{3.696000in}}%
\pgfusepath{clip}%
\pgfsetbuttcap%
\pgfsetroundjoin%
\definecolor{currentfill}{rgb}{0.894118,0.101961,0.109804}%
\pgfsetfillcolor{currentfill}%
\pgfsetfillopacity{0.600000}%
\pgfsetlinewidth{1.003750pt}%
\definecolor{currentstroke}{rgb}{0.894118,0.101961,0.109804}%
\pgfsetstrokecolor{currentstroke}%
\pgfsetstrokeopacity{0.600000}%
\pgfsetdash{}{0pt}%
\pgfpathmoveto{\pgfqpoint{1.785649in}{4.020686in}}%
\pgfpathcurveto{\pgfqpoint{1.796699in}{4.020686in}}{\pgfqpoint{1.807298in}{4.025076in}}{\pgfqpoint{1.815111in}{4.032890in}}%
\pgfpathcurveto{\pgfqpoint{1.822925in}{4.040703in}}{\pgfqpoint{1.827315in}{4.051303in}}{\pgfqpoint{1.827315in}{4.062353in}}%
\pgfpathcurveto{\pgfqpoint{1.827315in}{4.073403in}}{\pgfqpoint{1.822925in}{4.084002in}}{\pgfqpoint{1.815111in}{4.091815in}}%
\pgfpathcurveto{\pgfqpoint{1.807298in}{4.099629in}}{\pgfqpoint{1.796699in}{4.104019in}}{\pgfqpoint{1.785649in}{4.104019in}}%
\pgfpathcurveto{\pgfqpoint{1.774598in}{4.104019in}}{\pgfqpoint{1.763999in}{4.099629in}}{\pgfqpoint{1.756186in}{4.091815in}}%
\pgfpathcurveto{\pgfqpoint{1.748372in}{4.084002in}}{\pgfqpoint{1.743982in}{4.073403in}}{\pgfqpoint{1.743982in}{4.062353in}}%
\pgfpathcurveto{\pgfqpoint{1.743982in}{4.051303in}}{\pgfqpoint{1.748372in}{4.040703in}}{\pgfqpoint{1.756186in}{4.032890in}}%
\pgfpathcurveto{\pgfqpoint{1.763999in}{4.025076in}}{\pgfqpoint{1.774598in}{4.020686in}}{\pgfqpoint{1.785649in}{4.020686in}}%
\pgfpathlineto{\pgfqpoint{1.785649in}{4.020686in}}%
\pgfpathclose%
\pgfusepath{stroke,fill}%
\end{pgfscope}%
\begin{pgfscope}%
\pgfpathrectangle{\pgfqpoint{0.800049in}{0.448486in}}{\pgfqpoint{3.531733in}{3.696000in}}%
\pgfusepath{clip}%
\pgfsetbuttcap%
\pgfsetroundjoin%
\definecolor{currentfill}{rgb}{0.894118,0.101961,0.109804}%
\pgfsetfillcolor{currentfill}%
\pgfsetfillopacity{0.600000}%
\pgfsetlinewidth{1.003750pt}%
\definecolor{currentstroke}{rgb}{0.894118,0.101961,0.109804}%
\pgfsetstrokecolor{currentstroke}%
\pgfsetstrokeopacity{0.600000}%
\pgfsetdash{}{0pt}%
\pgfpathmoveto{\pgfqpoint{1.785649in}{4.020686in}}%
\pgfpathcurveto{\pgfqpoint{1.796699in}{4.020686in}}{\pgfqpoint{1.807298in}{4.025076in}}{\pgfqpoint{1.815111in}{4.032890in}}%
\pgfpathcurveto{\pgfqpoint{1.822925in}{4.040703in}}{\pgfqpoint{1.827315in}{4.051303in}}{\pgfqpoint{1.827315in}{4.062353in}}%
\pgfpathcurveto{\pgfqpoint{1.827315in}{4.073403in}}{\pgfqpoint{1.822925in}{4.084002in}}{\pgfqpoint{1.815111in}{4.091815in}}%
\pgfpathcurveto{\pgfqpoint{1.807298in}{4.099629in}}{\pgfqpoint{1.796699in}{4.104019in}}{\pgfqpoint{1.785649in}{4.104019in}}%
\pgfpathcurveto{\pgfqpoint{1.774598in}{4.104019in}}{\pgfqpoint{1.763999in}{4.099629in}}{\pgfqpoint{1.756186in}{4.091815in}}%
\pgfpathcurveto{\pgfqpoint{1.748372in}{4.084002in}}{\pgfqpoint{1.743982in}{4.073403in}}{\pgfqpoint{1.743982in}{4.062353in}}%
\pgfpathcurveto{\pgfqpoint{1.743982in}{4.051303in}}{\pgfqpoint{1.748372in}{4.040703in}}{\pgfqpoint{1.756186in}{4.032890in}}%
\pgfpathcurveto{\pgfqpoint{1.763999in}{4.025076in}}{\pgfqpoint{1.774598in}{4.020686in}}{\pgfqpoint{1.785649in}{4.020686in}}%
\pgfpathlineto{\pgfqpoint{1.785649in}{4.020686in}}%
\pgfpathclose%
\pgfusepath{stroke,fill}%
\end{pgfscope}%
\begin{pgfscope}%
\pgfpathrectangle{\pgfqpoint{0.800049in}{0.448486in}}{\pgfqpoint{3.531733in}{3.696000in}}%
\pgfusepath{clip}%
\pgfsetbuttcap%
\pgfsetroundjoin%
\definecolor{currentfill}{rgb}{0.894118,0.101961,0.109804}%
\pgfsetfillcolor{currentfill}%
\pgfsetfillopacity{0.600000}%
\pgfsetlinewidth{1.003750pt}%
\definecolor{currentstroke}{rgb}{0.894118,0.101961,0.109804}%
\pgfsetstrokecolor{currentstroke}%
\pgfsetstrokeopacity{0.600000}%
\pgfsetdash{}{0pt}%
\pgfpathmoveto{\pgfqpoint{1.785649in}{4.020686in}}%
\pgfpathcurveto{\pgfqpoint{1.796699in}{4.020686in}}{\pgfqpoint{1.807298in}{4.025076in}}{\pgfqpoint{1.815111in}{4.032890in}}%
\pgfpathcurveto{\pgfqpoint{1.822925in}{4.040703in}}{\pgfqpoint{1.827315in}{4.051303in}}{\pgfqpoint{1.827315in}{4.062353in}}%
\pgfpathcurveto{\pgfqpoint{1.827315in}{4.073403in}}{\pgfqpoint{1.822925in}{4.084002in}}{\pgfqpoint{1.815111in}{4.091815in}}%
\pgfpathcurveto{\pgfqpoint{1.807298in}{4.099629in}}{\pgfqpoint{1.796699in}{4.104019in}}{\pgfqpoint{1.785649in}{4.104019in}}%
\pgfpathcurveto{\pgfqpoint{1.774598in}{4.104019in}}{\pgfqpoint{1.763999in}{4.099629in}}{\pgfqpoint{1.756186in}{4.091815in}}%
\pgfpathcurveto{\pgfqpoint{1.748372in}{4.084002in}}{\pgfqpoint{1.743982in}{4.073403in}}{\pgfqpoint{1.743982in}{4.062353in}}%
\pgfpathcurveto{\pgfqpoint{1.743982in}{4.051303in}}{\pgfqpoint{1.748372in}{4.040703in}}{\pgfqpoint{1.756186in}{4.032890in}}%
\pgfpathcurveto{\pgfqpoint{1.763999in}{4.025076in}}{\pgfqpoint{1.774598in}{4.020686in}}{\pgfqpoint{1.785649in}{4.020686in}}%
\pgfpathlineto{\pgfqpoint{1.785649in}{4.020686in}}%
\pgfpathclose%
\pgfusepath{stroke,fill}%
\end{pgfscope}%
\begin{pgfscope}%
\pgfpathrectangle{\pgfqpoint{0.800049in}{0.448486in}}{\pgfqpoint{3.531733in}{3.696000in}}%
\pgfusepath{clip}%
\pgfsetbuttcap%
\pgfsetroundjoin%
\definecolor{currentfill}{rgb}{0.894118,0.101961,0.109804}%
\pgfsetfillcolor{currentfill}%
\pgfsetfillopacity{0.600000}%
\pgfsetlinewidth{1.003750pt}%
\definecolor{currentstroke}{rgb}{0.894118,0.101961,0.109804}%
\pgfsetstrokecolor{currentstroke}%
\pgfsetstrokeopacity{0.600000}%
\pgfsetdash{}{0pt}%
\pgfpathmoveto{\pgfqpoint{0.964315in}{4.020686in}}%
\pgfpathcurveto{\pgfqpoint{0.975365in}{4.020686in}}{\pgfqpoint{0.985964in}{4.025076in}}{\pgfqpoint{0.993778in}{4.032890in}}%
\pgfpathcurveto{\pgfqpoint{1.001592in}{4.040703in}}{\pgfqpoint{1.005982in}{4.051303in}}{\pgfqpoint{1.005982in}{4.062353in}}%
\pgfpathcurveto{\pgfqpoint{1.005982in}{4.073403in}}{\pgfqpoint{1.001592in}{4.084002in}}{\pgfqpoint{0.993778in}{4.091815in}}%
\pgfpathcurveto{\pgfqpoint{0.985964in}{4.099629in}}{\pgfqpoint{0.975365in}{4.104019in}}{\pgfqpoint{0.964315in}{4.104019in}}%
\pgfpathcurveto{\pgfqpoint{0.953265in}{4.104019in}}{\pgfqpoint{0.942666in}{4.099629in}}{\pgfqpoint{0.934852in}{4.091815in}}%
\pgfpathcurveto{\pgfqpoint{0.927039in}{4.084002in}}{\pgfqpoint{0.922649in}{4.073403in}}{\pgfqpoint{0.922649in}{4.062353in}}%
\pgfpathcurveto{\pgfqpoint{0.922649in}{4.051303in}}{\pgfqpoint{0.927039in}{4.040703in}}{\pgfqpoint{0.934852in}{4.032890in}}%
\pgfpathcurveto{\pgfqpoint{0.942666in}{4.025076in}}{\pgfqpoint{0.953265in}{4.020686in}}{\pgfqpoint{0.964315in}{4.020686in}}%
\pgfpathlineto{\pgfqpoint{0.964315in}{4.020686in}}%
\pgfpathclose%
\pgfusepath{stroke,fill}%
\end{pgfscope}%
\begin{pgfscope}%
\pgfpathrectangle{\pgfqpoint{0.800049in}{0.448486in}}{\pgfqpoint{3.531733in}{3.696000in}}%
\pgfusepath{clip}%
\pgfsetbuttcap%
\pgfsetroundjoin%
\definecolor{currentfill}{rgb}{0.894118,0.101961,0.109804}%
\pgfsetfillcolor{currentfill}%
\pgfsetfillopacity{0.600000}%
\pgfsetlinewidth{1.003750pt}%
\definecolor{currentstroke}{rgb}{0.894118,0.101961,0.109804}%
\pgfsetstrokecolor{currentstroke}%
\pgfsetstrokeopacity{0.600000}%
\pgfsetdash{}{0pt}%
\pgfpathmoveto{\pgfqpoint{0.964315in}{4.020686in}}%
\pgfpathcurveto{\pgfqpoint{0.975365in}{4.020686in}}{\pgfqpoint{0.985964in}{4.025076in}}{\pgfqpoint{0.993778in}{4.032890in}}%
\pgfpathcurveto{\pgfqpoint{1.001592in}{4.040703in}}{\pgfqpoint{1.005982in}{4.051303in}}{\pgfqpoint{1.005982in}{4.062353in}}%
\pgfpathcurveto{\pgfqpoint{1.005982in}{4.073403in}}{\pgfqpoint{1.001592in}{4.084002in}}{\pgfqpoint{0.993778in}{4.091815in}}%
\pgfpathcurveto{\pgfqpoint{0.985964in}{4.099629in}}{\pgfqpoint{0.975365in}{4.104019in}}{\pgfqpoint{0.964315in}{4.104019in}}%
\pgfpathcurveto{\pgfqpoint{0.953265in}{4.104019in}}{\pgfqpoint{0.942666in}{4.099629in}}{\pgfqpoint{0.934852in}{4.091815in}}%
\pgfpathcurveto{\pgfqpoint{0.927039in}{4.084002in}}{\pgfqpoint{0.922649in}{4.073403in}}{\pgfqpoint{0.922649in}{4.062353in}}%
\pgfpathcurveto{\pgfqpoint{0.922649in}{4.051303in}}{\pgfqpoint{0.927039in}{4.040703in}}{\pgfqpoint{0.934852in}{4.032890in}}%
\pgfpathcurveto{\pgfqpoint{0.942666in}{4.025076in}}{\pgfqpoint{0.953265in}{4.020686in}}{\pgfqpoint{0.964315in}{4.020686in}}%
\pgfpathlineto{\pgfqpoint{0.964315in}{4.020686in}}%
\pgfpathclose%
\pgfusepath{stroke,fill}%
\end{pgfscope}%
\begin{pgfscope}%
\pgfsetbuttcap%
\pgfsetroundjoin%
\definecolor{currentfill}{rgb}{0.000000,0.000000,0.000000}%
\pgfsetfillcolor{currentfill}%
\pgfsetlinewidth{0.803000pt}%
\definecolor{currentstroke}{rgb}{0.000000,0.000000,0.000000}%
\pgfsetstrokecolor{currentstroke}%
\pgfsetdash{}{0pt}%
\pgfsys@defobject{currentmarker}{\pgfqpoint{0.000000in}{-0.048611in}}{\pgfqpoint{0.000000in}{0.000000in}}{%
\pgfpathmoveto{\pgfqpoint{0.000000in}{0.000000in}}%
\pgfpathlineto{\pgfqpoint{0.000000in}{-0.048611in}}%
\pgfusepath{stroke,fill}%
}%
\begin{pgfscope}%
\pgfsys@transformshift{1.238093in}{0.448486in}%
\pgfsys@useobject{currentmarker}{}%
\end{pgfscope}%
\end{pgfscope}%
\begin{pgfscope}%
\definecolor{textcolor}{rgb}{0.000000,0.000000,0.000000}%
\pgfsetstrokecolor{textcolor}%
\pgfsetfillcolor{textcolor}%
\pgftext[x=1.238093in,y=0.351264in,,top]{\color{textcolor}{\rmfamily\fontsize{10.000000}{12.000000}\selectfont\catcode`\^=\active\def^{\ifmmode\sp\else\^{}\fi}\catcode`\%=\active\def%{\%}$\mathdefault{8}$}}%
\end{pgfscope}%
\begin{pgfscope}%
\pgfsetbuttcap%
\pgfsetroundjoin%
\definecolor{currentfill}{rgb}{0.000000,0.000000,0.000000}%
\pgfsetfillcolor{currentfill}%
\pgfsetlinewidth{0.803000pt}%
\definecolor{currentstroke}{rgb}{0.000000,0.000000,0.000000}%
\pgfsetstrokecolor{currentstroke}%
\pgfsetdash{}{0pt}%
\pgfsys@defobject{currentmarker}{\pgfqpoint{0.000000in}{-0.048611in}}{\pgfqpoint{0.000000in}{0.000000in}}{%
\pgfpathmoveto{\pgfqpoint{0.000000in}{0.000000in}}%
\pgfpathlineto{\pgfqpoint{0.000000in}{-0.048611in}}%
\pgfusepath{stroke,fill}%
}%
\begin{pgfscope}%
\pgfsys@transformshift{1.785649in}{0.448486in}%
\pgfsys@useobject{currentmarker}{}%
\end{pgfscope}%
\end{pgfscope}%
\begin{pgfscope}%
\definecolor{textcolor}{rgb}{0.000000,0.000000,0.000000}%
\pgfsetstrokecolor{textcolor}%
\pgfsetfillcolor{textcolor}%
\pgftext[x=1.785649in,y=0.351264in,,top]{\color{textcolor}{\rmfamily\fontsize{10.000000}{12.000000}\selectfont\catcode`\^=\active\def^{\ifmmode\sp\else\^{}\fi}\catcode`\%=\active\def%{\%}$\mathdefault{10}$}}%
\end{pgfscope}%
\begin{pgfscope}%
\pgfsetbuttcap%
\pgfsetroundjoin%
\definecolor{currentfill}{rgb}{0.000000,0.000000,0.000000}%
\pgfsetfillcolor{currentfill}%
\pgfsetlinewidth{0.803000pt}%
\definecolor{currentstroke}{rgb}{0.000000,0.000000,0.000000}%
\pgfsetstrokecolor{currentstroke}%
\pgfsetdash{}{0pt}%
\pgfsys@defobject{currentmarker}{\pgfqpoint{0.000000in}{-0.048611in}}{\pgfqpoint{0.000000in}{0.000000in}}{%
\pgfpathmoveto{\pgfqpoint{0.000000in}{0.000000in}}%
\pgfpathlineto{\pgfqpoint{0.000000in}{-0.048611in}}%
\pgfusepath{stroke,fill}%
}%
\begin{pgfscope}%
\pgfsys@transformshift{2.333204in}{0.448486in}%
\pgfsys@useobject{currentmarker}{}%
\end{pgfscope}%
\end{pgfscope}%
\begin{pgfscope}%
\definecolor{textcolor}{rgb}{0.000000,0.000000,0.000000}%
\pgfsetstrokecolor{textcolor}%
\pgfsetfillcolor{textcolor}%
\pgftext[x=2.333204in,y=0.351264in,,top]{\color{textcolor}{\rmfamily\fontsize{10.000000}{12.000000}\selectfont\catcode`\^=\active\def^{\ifmmode\sp\else\^{}\fi}\catcode`\%=\active\def%{\%}$\mathdefault{12}$}}%
\end{pgfscope}%
\begin{pgfscope}%
\pgfsetbuttcap%
\pgfsetroundjoin%
\definecolor{currentfill}{rgb}{0.000000,0.000000,0.000000}%
\pgfsetfillcolor{currentfill}%
\pgfsetlinewidth{0.803000pt}%
\definecolor{currentstroke}{rgb}{0.000000,0.000000,0.000000}%
\pgfsetstrokecolor{currentstroke}%
\pgfsetdash{}{0pt}%
\pgfsys@defobject{currentmarker}{\pgfqpoint{0.000000in}{-0.048611in}}{\pgfqpoint{0.000000in}{0.000000in}}{%
\pgfpathmoveto{\pgfqpoint{0.000000in}{0.000000in}}%
\pgfpathlineto{\pgfqpoint{0.000000in}{-0.048611in}}%
\pgfusepath{stroke,fill}%
}%
\begin{pgfscope}%
\pgfsys@transformshift{2.880760in}{0.448486in}%
\pgfsys@useobject{currentmarker}{}%
\end{pgfscope}%
\end{pgfscope}%
\begin{pgfscope}%
\definecolor{textcolor}{rgb}{0.000000,0.000000,0.000000}%
\pgfsetstrokecolor{textcolor}%
\pgfsetfillcolor{textcolor}%
\pgftext[x=2.880760in,y=0.351264in,,top]{\color{textcolor}{\rmfamily\fontsize{10.000000}{12.000000}\selectfont\catcode`\^=\active\def^{\ifmmode\sp\else\^{}\fi}\catcode`\%=\active\def%{\%}$\mathdefault{14}$}}%
\end{pgfscope}%
\begin{pgfscope}%
\pgfsetbuttcap%
\pgfsetroundjoin%
\definecolor{currentfill}{rgb}{0.000000,0.000000,0.000000}%
\pgfsetfillcolor{currentfill}%
\pgfsetlinewidth{0.803000pt}%
\definecolor{currentstroke}{rgb}{0.000000,0.000000,0.000000}%
\pgfsetstrokecolor{currentstroke}%
\pgfsetdash{}{0pt}%
\pgfsys@defobject{currentmarker}{\pgfqpoint{0.000000in}{-0.048611in}}{\pgfqpoint{0.000000in}{0.000000in}}{%
\pgfpathmoveto{\pgfqpoint{0.000000in}{0.000000in}}%
\pgfpathlineto{\pgfqpoint{0.000000in}{-0.048611in}}%
\pgfusepath{stroke,fill}%
}%
\begin{pgfscope}%
\pgfsys@transformshift{3.428315in}{0.448486in}%
\pgfsys@useobject{currentmarker}{}%
\end{pgfscope}%
\end{pgfscope}%
\begin{pgfscope}%
\definecolor{textcolor}{rgb}{0.000000,0.000000,0.000000}%
\pgfsetstrokecolor{textcolor}%
\pgfsetfillcolor{textcolor}%
\pgftext[x=3.428315in,y=0.351264in,,top]{\color{textcolor}{\rmfamily\fontsize{10.000000}{12.000000}\selectfont\catcode`\^=\active\def^{\ifmmode\sp\else\^{}\fi}\catcode`\%=\active\def%{\%}$\mathdefault{16}$}}%
\end{pgfscope}%
\begin{pgfscope}%
\pgfsetbuttcap%
\pgfsetroundjoin%
\definecolor{currentfill}{rgb}{0.000000,0.000000,0.000000}%
\pgfsetfillcolor{currentfill}%
\pgfsetlinewidth{0.803000pt}%
\definecolor{currentstroke}{rgb}{0.000000,0.000000,0.000000}%
\pgfsetstrokecolor{currentstroke}%
\pgfsetdash{}{0pt}%
\pgfsys@defobject{currentmarker}{\pgfqpoint{0.000000in}{-0.048611in}}{\pgfqpoint{0.000000in}{0.000000in}}{%
\pgfpathmoveto{\pgfqpoint{0.000000in}{0.000000in}}%
\pgfpathlineto{\pgfqpoint{0.000000in}{-0.048611in}}%
\pgfusepath{stroke,fill}%
}%
\begin{pgfscope}%
\pgfsys@transformshift{3.975871in}{0.448486in}%
\pgfsys@useobject{currentmarker}{}%
\end{pgfscope}%
\end{pgfscope}%
\begin{pgfscope}%
\definecolor{textcolor}{rgb}{0.000000,0.000000,0.000000}%
\pgfsetstrokecolor{textcolor}%
\pgfsetfillcolor{textcolor}%
\pgftext[x=3.975871in,y=0.351264in,,top]{\color{textcolor}{\rmfamily\fontsize{10.000000}{12.000000}\selectfont\catcode`\^=\active\def^{\ifmmode\sp\else\^{}\fi}\catcode`\%=\active\def%{\%}$\mathdefault{18}$}}%
\end{pgfscope}%
\begin{pgfscope}%
\definecolor{textcolor}{rgb}{0.000000,0.000000,0.000000}%
\pgfsetstrokecolor{textcolor}%
\pgfsetfillcolor{textcolor}%
\pgftext[x=2.565915in,y=0.161295in,,top]{\color{textcolor}{\rmfamily\fontsize{12.000000}{14.400000}\selectfont\catcode`\^=\active\def^{\ifmmode\sp\else\^{}\fi}\catcode`\%=\active\def%{\%}Birth}}%
\end{pgfscope}%
\begin{pgfscope}%
\pgfsetbuttcap%
\pgfsetroundjoin%
\definecolor{currentfill}{rgb}{0.000000,0.000000,0.000000}%
\pgfsetfillcolor{currentfill}%
\pgfsetlinewidth{0.803000pt}%
\definecolor{currentstroke}{rgb}{0.000000,0.000000,0.000000}%
\pgfsetstrokecolor{currentstroke}%
\pgfsetdash{}{0pt}%
\pgfsys@defobject{currentmarker}{\pgfqpoint{-0.048611in}{0.000000in}}{\pgfqpoint{-0.000000in}{0.000000in}}{%
\pgfpathmoveto{\pgfqpoint{-0.000000in}{0.000000in}}%
\pgfpathlineto{\pgfqpoint{-0.048611in}{0.000000in}}%
\pgfusepath{stroke,fill}%
}%
\begin{pgfscope}%
\pgfsys@transformshift{0.800049in}{0.886530in}%
\pgfsys@useobject{currentmarker}{}%
\end{pgfscope}%
\end{pgfscope}%
\begin{pgfscope}%
\definecolor{textcolor}{rgb}{0.000000,0.000000,0.000000}%
\pgfsetstrokecolor{textcolor}%
\pgfsetfillcolor{textcolor}%
\pgftext[x=0.305216in, y=0.833769in, left, base]{\color{textcolor}{\rmfamily\fontsize{10.000000}{12.000000}\selectfont\catcode`\^=\active\def^{\ifmmode\sp\else\^{}\fi}\catcode`\%=\active\def%{\%}8.000}}%
\end{pgfscope}%
\begin{pgfscope}%
\pgfsetbuttcap%
\pgfsetroundjoin%
\definecolor{currentfill}{rgb}{0.000000,0.000000,0.000000}%
\pgfsetfillcolor{currentfill}%
\pgfsetlinewidth{0.803000pt}%
\definecolor{currentstroke}{rgb}{0.000000,0.000000,0.000000}%
\pgfsetstrokecolor{currentstroke}%
\pgfsetdash{}{0pt}%
\pgfsys@defobject{currentmarker}{\pgfqpoint{-0.048611in}{0.000000in}}{\pgfqpoint{-0.000000in}{0.000000in}}{%
\pgfpathmoveto{\pgfqpoint{-0.000000in}{0.000000in}}%
\pgfpathlineto{\pgfqpoint{-0.048611in}{0.000000in}}%
\pgfusepath{stroke,fill}%
}%
\begin{pgfscope}%
\pgfsys@transformshift{0.800049in}{1.434086in}%
\pgfsys@useobject{currentmarker}{}%
\end{pgfscope}%
\end{pgfscope}%
\begin{pgfscope}%
\definecolor{textcolor}{rgb}{0.000000,0.000000,0.000000}%
\pgfsetstrokecolor{textcolor}%
\pgfsetfillcolor{textcolor}%
\pgftext[x=0.216851in, y=1.381324in, left, base]{\color{textcolor}{\rmfamily\fontsize{10.000000}{12.000000}\selectfont\catcode`\^=\active\def^{\ifmmode\sp\else\^{}\fi}\catcode`\%=\active\def%{\%}10.000}}%
\end{pgfscope}%
\begin{pgfscope}%
\pgfsetbuttcap%
\pgfsetroundjoin%
\definecolor{currentfill}{rgb}{0.000000,0.000000,0.000000}%
\pgfsetfillcolor{currentfill}%
\pgfsetlinewidth{0.803000pt}%
\definecolor{currentstroke}{rgb}{0.000000,0.000000,0.000000}%
\pgfsetstrokecolor{currentstroke}%
\pgfsetdash{}{0pt}%
\pgfsys@defobject{currentmarker}{\pgfqpoint{-0.048611in}{0.000000in}}{\pgfqpoint{-0.000000in}{0.000000in}}{%
\pgfpathmoveto{\pgfqpoint{-0.000000in}{0.000000in}}%
\pgfpathlineto{\pgfqpoint{-0.048611in}{0.000000in}}%
\pgfusepath{stroke,fill}%
}%
\begin{pgfscope}%
\pgfsys@transformshift{0.800049in}{1.981642in}%
\pgfsys@useobject{currentmarker}{}%
\end{pgfscope}%
\end{pgfscope}%
\begin{pgfscope}%
\definecolor{textcolor}{rgb}{0.000000,0.000000,0.000000}%
\pgfsetstrokecolor{textcolor}%
\pgfsetfillcolor{textcolor}%
\pgftext[x=0.216851in, y=1.928880in, left, base]{\color{textcolor}{\rmfamily\fontsize{10.000000}{12.000000}\selectfont\catcode`\^=\active\def^{\ifmmode\sp\else\^{}\fi}\catcode`\%=\active\def%{\%}12.000}}%
\end{pgfscope}%
\begin{pgfscope}%
\pgfsetbuttcap%
\pgfsetroundjoin%
\definecolor{currentfill}{rgb}{0.000000,0.000000,0.000000}%
\pgfsetfillcolor{currentfill}%
\pgfsetlinewidth{0.803000pt}%
\definecolor{currentstroke}{rgb}{0.000000,0.000000,0.000000}%
\pgfsetstrokecolor{currentstroke}%
\pgfsetdash{}{0pt}%
\pgfsys@defobject{currentmarker}{\pgfqpoint{-0.048611in}{0.000000in}}{\pgfqpoint{-0.000000in}{0.000000in}}{%
\pgfpathmoveto{\pgfqpoint{-0.000000in}{0.000000in}}%
\pgfpathlineto{\pgfqpoint{-0.048611in}{0.000000in}}%
\pgfusepath{stroke,fill}%
}%
\begin{pgfscope}%
\pgfsys@transformshift{0.800049in}{2.529197in}%
\pgfsys@useobject{currentmarker}{}%
\end{pgfscope}%
\end{pgfscope}%
\begin{pgfscope}%
\definecolor{textcolor}{rgb}{0.000000,0.000000,0.000000}%
\pgfsetstrokecolor{textcolor}%
\pgfsetfillcolor{textcolor}%
\pgftext[x=0.216851in, y=2.476436in, left, base]{\color{textcolor}{\rmfamily\fontsize{10.000000}{12.000000}\selectfont\catcode`\^=\active\def^{\ifmmode\sp\else\^{}\fi}\catcode`\%=\active\def%{\%}14.000}}%
\end{pgfscope}%
\begin{pgfscope}%
\pgfsetbuttcap%
\pgfsetroundjoin%
\definecolor{currentfill}{rgb}{0.000000,0.000000,0.000000}%
\pgfsetfillcolor{currentfill}%
\pgfsetlinewidth{0.803000pt}%
\definecolor{currentstroke}{rgb}{0.000000,0.000000,0.000000}%
\pgfsetstrokecolor{currentstroke}%
\pgfsetdash{}{0pt}%
\pgfsys@defobject{currentmarker}{\pgfqpoint{-0.048611in}{0.000000in}}{\pgfqpoint{-0.000000in}{0.000000in}}{%
\pgfpathmoveto{\pgfqpoint{-0.000000in}{0.000000in}}%
\pgfpathlineto{\pgfqpoint{-0.048611in}{0.000000in}}%
\pgfusepath{stroke,fill}%
}%
\begin{pgfscope}%
\pgfsys@transformshift{0.800049in}{3.076753in}%
\pgfsys@useobject{currentmarker}{}%
\end{pgfscope}%
\end{pgfscope}%
\begin{pgfscope}%
\definecolor{textcolor}{rgb}{0.000000,0.000000,0.000000}%
\pgfsetstrokecolor{textcolor}%
\pgfsetfillcolor{textcolor}%
\pgftext[x=0.216851in, y=3.023991in, left, base]{\color{textcolor}{\rmfamily\fontsize{10.000000}{12.000000}\selectfont\catcode`\^=\active\def^{\ifmmode\sp\else\^{}\fi}\catcode`\%=\active\def%{\%}16.000}}%
\end{pgfscope}%
\begin{pgfscope}%
\pgfsetbuttcap%
\pgfsetroundjoin%
\definecolor{currentfill}{rgb}{0.000000,0.000000,0.000000}%
\pgfsetfillcolor{currentfill}%
\pgfsetlinewidth{0.803000pt}%
\definecolor{currentstroke}{rgb}{0.000000,0.000000,0.000000}%
\pgfsetstrokecolor{currentstroke}%
\pgfsetdash{}{0pt}%
\pgfsys@defobject{currentmarker}{\pgfqpoint{-0.048611in}{0.000000in}}{\pgfqpoint{-0.000000in}{0.000000in}}{%
\pgfpathmoveto{\pgfqpoint{-0.000000in}{0.000000in}}%
\pgfpathlineto{\pgfqpoint{-0.048611in}{0.000000in}}%
\pgfusepath{stroke,fill}%
}%
\begin{pgfscope}%
\pgfsys@transformshift{0.800049in}{3.624308in}%
\pgfsys@useobject{currentmarker}{}%
\end{pgfscope}%
\end{pgfscope}%
\begin{pgfscope}%
\definecolor{textcolor}{rgb}{0.000000,0.000000,0.000000}%
\pgfsetstrokecolor{textcolor}%
\pgfsetfillcolor{textcolor}%
\pgftext[x=0.216851in, y=3.571547in, left, base]{\color{textcolor}{\rmfamily\fontsize{10.000000}{12.000000}\selectfont\catcode`\^=\active\def^{\ifmmode\sp\else\^{}\fi}\catcode`\%=\active\def%{\%}18.000}}%
\end{pgfscope}%
\begin{pgfscope}%
\pgfsetbuttcap%
\pgfsetroundjoin%
\definecolor{currentfill}{rgb}{0.000000,0.000000,0.000000}%
\pgfsetfillcolor{currentfill}%
\pgfsetlinewidth{0.803000pt}%
\definecolor{currentstroke}{rgb}{0.000000,0.000000,0.000000}%
\pgfsetstrokecolor{currentstroke}%
\pgfsetdash{}{0pt}%
\pgfsys@defobject{currentmarker}{\pgfqpoint{-0.048611in}{0.000000in}}{\pgfqpoint{-0.000000in}{0.000000in}}{%
\pgfpathmoveto{\pgfqpoint{-0.000000in}{0.000000in}}%
\pgfpathlineto{\pgfqpoint{-0.048611in}{0.000000in}}%
\pgfusepath{stroke,fill}%
}%
\begin{pgfscope}%
\pgfsys@transformshift{0.800049in}{4.062353in}%
\pgfsys@useobject{currentmarker}{}%
\end{pgfscope}%
\end{pgfscope}%
\begin{pgfscope}%
\definecolor{textcolor}{rgb}{0.000000,0.000000,0.000000}%
\pgfsetstrokecolor{textcolor}%
\pgfsetfillcolor{textcolor}%
\pgftext[x=0.455912in, y=4.009591in, left, base]{\color{textcolor}{\rmfamily\fontsize{10.000000}{12.000000}\selectfont\catcode`\^=\active\def^{\ifmmode\sp\else\^{}\fi}\catcode`\%=\active\def%{\%}$+\infty$}}%
\end{pgfscope}%
\begin{pgfscope}%
\definecolor{textcolor}{rgb}{0.000000,0.000000,0.000000}%
\pgfsetstrokecolor{textcolor}%
\pgfsetfillcolor{textcolor}%
\pgftext[x=0.161295in,y=2.296486in,,bottom,rotate=90.000000]{\color{textcolor}{\rmfamily\fontsize{12.000000}{14.400000}\selectfont\catcode`\^=\active\def^{\ifmmode\sp\else\^{}\fi}\catcode`\%=\active\def%{\%}Death}}%
\end{pgfscope}%
\begin{pgfscope}%
\pgfpathrectangle{\pgfqpoint{0.800049in}{0.448486in}}{\pgfqpoint{3.531733in}{3.696000in}}%
\pgfusepath{clip}%
\pgfsetrectcap%
\pgfsetroundjoin%
\pgfsetlinewidth{1.003750pt}%
\definecolor{currentstroke}{rgb}{0.000000,0.000000,0.000000}%
\pgfsetstrokecolor{currentstroke}%
\pgfsetdash{}{0pt}%
\pgfpathmoveto{\pgfqpoint{0.800049in}{0.448486in}}%
\pgfpathlineto{\pgfqpoint{4.331782in}{3.980219in}}%
\pgfusepath{stroke}%
\end{pgfscope}%
\begin{pgfscope}%
\pgfpathrectangle{\pgfqpoint{0.800049in}{0.448486in}}{\pgfqpoint{3.531733in}{3.696000in}}%
\pgfusepath{clip}%
\pgfsetrectcap%
\pgfsetroundjoin%
\pgfsetlinewidth{1.003750pt}%
\definecolor{currentstroke}{rgb}{0.000000,0.000000,0.000000}%
\pgfsetstrokecolor{currentstroke}%
\pgfsetstrokeopacity{0.600000}%
\pgfsetdash{}{0pt}%
\pgfpathmoveto{\pgfqpoint{0.800049in}{4.062353in}}%
\pgfpathlineto{\pgfqpoint{4.331782in}{4.062353in}}%
\pgfusepath{stroke}%
\end{pgfscope}%
\begin{pgfscope}%
\pgfsetrectcap%
\pgfsetmiterjoin%
\pgfsetlinewidth{0.803000pt}%
\definecolor{currentstroke}{rgb}{0.000000,0.000000,0.000000}%
\pgfsetstrokecolor{currentstroke}%
\pgfsetdash{}{0pt}%
\pgfpathmoveto{\pgfqpoint{0.800049in}{0.448486in}}%
\pgfpathlineto{\pgfqpoint{0.800049in}{4.144486in}}%
\pgfusepath{stroke}%
\end{pgfscope}%
\begin{pgfscope}%
\pgfsetrectcap%
\pgfsetmiterjoin%
\pgfsetlinewidth{0.803000pt}%
\definecolor{currentstroke}{rgb}{0.000000,0.000000,0.000000}%
\pgfsetstrokecolor{currentstroke}%
\pgfsetdash{}{0pt}%
\pgfpathmoveto{\pgfqpoint{4.331782in}{0.448486in}}%
\pgfpathlineto{\pgfqpoint{4.331782in}{4.144486in}}%
\pgfusepath{stroke}%
\end{pgfscope}%
\begin{pgfscope}%
\pgfsetrectcap%
\pgfsetmiterjoin%
\pgfsetlinewidth{0.803000pt}%
\definecolor{currentstroke}{rgb}{0.000000,0.000000,0.000000}%
\pgfsetstrokecolor{currentstroke}%
\pgfsetdash{}{0pt}%
\pgfpathmoveto{\pgfqpoint{0.800049in}{0.448486in}}%
\pgfpathlineto{\pgfqpoint{4.331782in}{0.448486in}}%
\pgfusepath{stroke}%
\end{pgfscope}%
\begin{pgfscope}%
\pgfsetrectcap%
\pgfsetmiterjoin%
\pgfsetlinewidth{0.803000pt}%
\definecolor{currentstroke}{rgb}{0.000000,0.000000,0.000000}%
\pgfsetstrokecolor{currentstroke}%
\pgfsetdash{}{0pt}%
\pgfpathmoveto{\pgfqpoint{0.800049in}{4.144486in}}%
\pgfpathlineto{\pgfqpoint{4.331782in}{4.144486in}}%
\pgfusepath{stroke}%
\end{pgfscope}%
\end{pgfpicture}%
\makeatother%
\endgroup%

        }
        \caption{With circumcircle filtering}
        \label{fig:nested_ph_cc}
    \end{subfigure}
    \caption{Persistence diagrams of the LVR complex on the nested dataset with and without circumcircle filtering.}
    \label{fig:nested_ph}
\end{figure}


To assess how well the LVR complex captures the topology of the decision boundary,
we have created two synthetic datasets. Because the points in these datasets are
distributed uniformly, we used the LVR complex rather than the LS-LVR complex.

On the nested dataset, we would expect to see five 1-dimensional homology
classes that persist indefinitely. However, as shown in
Figure~\ref{fig:nested_ph_no_cc}, the persistence diagram reveals more such
classes than expected, and some of them do not persist until infinity.
By examining the 2-skeleton of the complex in Figure~\ref{fig:nested_complex_no_cc},
we see that the LVR complex includes edges and triangles that cross the decision
boundary multiple times. This is a problem, because this splits a true homology class
into several classes, some of which may not persist until infinity.

\section{Circumcircle filtering}

\begin{figure}
    \centering
    \begin{subfigure}{0.32\textwidth}
        \begin{tikzpicture}[scale=0.5]
            \coordinate (A) at (0,0);
            \coordinate (B) at (5,2);
            \coordinate (C) at (3.5,-1.5); 

            % circle center
            \coordinate (M) at ($(A)!0.5!(B)$);

            % triangle
            \draw[thick] (A) -- (B) -- (C) -- cycle;
            \draw ($(C)!0.1!(B)$) -- ($(C) + (-0.2,0.5)$) -- ($(C)!0.1!(A)$);

            % circle
            \pgfmathsetmacro{\radius}{sqrt((5)^2 + (2)^2)/2 * 0.5}
            \draw[fill=blue!50, fill opacity=0.3] (M) circle (\radius);

            \node[left]  at (A) {$A$};
            \node[right] at (B) {$B$};
            \node[below] at (C) {$C$};
        \end{tikzpicture}
        \caption{Small filtering region at $\theta = 1.3$}
    \end{subfigure}
    \begin{subfigure}{0.32\textwidth}
        \begin{tikzpicture}[scale=0.5]
            \coordinate (A) at (0,0);
            \coordinate (B) at (5,2);
            \coordinate (C) at (3.5,-1.5); 

            % circle center
            \coordinate (M) at ($(A)!0.5!(B)$);

            % triangle
            \draw[thick] (A) -- (B) -- (C) -- cycle;
            \draw ($(C)!0.1!(B)$) -- ($(C) + (-0.2,0.5)$) -- ($(C)!0.1!(A)$);

            % circle
            \pgfmathsetmacro{\radius}{sqrt((5)^2 + (2)^2)/2}
            \draw[fill=blue!50, fill opacity=0.3] (M) circle (\radius);

            \node[left]  at (A) {$A$};
            \node[right] at (B) {$B$};
            \node[below] at (C) {$C$};
        \end{tikzpicture}
        \caption{Medium-sized filtering region at $\theta = 1.0$}
    \end{subfigure}
    \begin{subfigure}{0.32\textwidth}
        \begin{tikzpicture}[scale=0.5]
            \coordinate (A) at (0,0);
            \coordinate (B) at (5,2);
            \coordinate (C) at (3.5,-1.5); 

            % circle center
            \coordinate (M) at ($(A)!0.5!(B)$);

            % triangle
            \draw[thick] (A) -- (B) -- (C) -- cycle;
            \draw ($(C)!0.1!(B)$) -- ($(C) + (-0.2,0.5)$) -- ($(C)!0.1!(A)$);

            % circle
            \pgfmathsetmacro{\radius}{sqrt((5)^2 + (2)^2)/2 * 1.5}
            \draw[fill=blue!50, fill opacity=0.3] (M) circle (\radius);

            \node[left]  at (A) {$A$};
            \node[right] at (B) {$B$};
            \node[below] at (C) {$C$};
        \end{tikzpicture}
        \caption{Large filtering region at $\theta = 0.7$}
    \end{subfigure}
    \caption{Circumcircle filtering with different values of $\theta$. The light blue circle 
    shows the region where any point's presence would cause edge $AB$ to be removed. A larger 
    value of $\theta$ leads to a smaller filtering region, while a smaller $\theta$ increases the 
    region and thus leads to more aggressive filtering.}
    \label{fig:cc}
\end{figure}

To address this drawback, we propose \emph{circumcircle filtering} (CC).
This method removes an edge between vertices \(A\) and \(B\) (and all higher simplices containing this edge) from the simplicial complex
if there exists a vertex \(C\) such that $\norm{A - B}^2 > (\norm{A - C}^2 + \norm{B - C}^2) \theta$,
where \(\theta \in [0, 2]\) is a parameter.
Intuitively, this approach aims to make a simplicial complex more similar
to the Alpha complex~\cite{edelsbrunner2010computational} by removing some (but not all) non-Delaunay edges.
The set of edges that should be removed with CC can be computed in
$\mathcal{O}(n^2 \cdot d)$ time and $\mathcal{O}(n^2)$ memory by computing all
pairwise distances of points. These computational costs are neglible compared
to building the LVR complex and computing its persistent homology.

The CC method can be visualized by considering the plane with the points $A, B$ and $C$ in Figure~\ref{fig:cc}.
At $\theta = 1$, an edge $AB$ is removed if $\norm{A - B}^2 > \norm{A - C}^2 + \norm{B - C}^2$,
i.e.\ if $C$ lies inside a circle which has $AB$ as its diameter,
hence the name of \emph{circumcircle filtering}.
Increasing or decreasing the value of $\theta$ decreases or respectively increases the size of the circle, keeping the center the same.
At $\theta = 0$, all edges are removed, because $\norm{A - B}^2 > 0$ always holds,
and at $\theta = 2$, no edges are removed, because
\begin{align}
    2 \cdot (\norm{A - C}^2 + \norm{B - C}^2)
    & \geq  \norm{A - C}^2 + \norm{B - C}^2 + 2\norm{A - B}\norm{B - C} \\
    & = (\norm{A - C} + \norm{B - C})^2,
\end{align}
and
\begin{equation}
    \norm{A - B}^2 \leq (\norm{A - C} + \norm{B - C})^2
\end{equation}
by the triangle inequality.

For the 2D datasets, we set $\theta = 1.4$, as with lower values almost no
simplices were being filtered, and higher values filtered out too many simplices
and the complex became very disconnected.
\begin{figure}
    \centering
    \begin{subfigure}{0.49\textwidth}
        \includegraphics[width=\textwidth]{plots/nested_complex_9_no_cc.jpg}
        \caption{Without circumcircle filtering.}
        \label{fig:nested_complex_no_cc}
    \end{subfigure}
    \begin{subfigure}{0.49\textwidth}
        \includegraphics[width=\textwidth]{plots/nested_complex_9_cc.jpg}
        \caption{With circumcircle filtering}
        \label{fig:nested_complex_cc}
    \end{subfigure}
    \caption{A visualisation of the 2-skeleton of the LVR complex on the nested dataset with and without circumcircle filtering. Not all 2-simplices are shown for clarity.}
    \label{fig:nested_complex}
\end{figure}


\begin{figure}
    \centering
    \begin{subfigure}{0.49\textwidth}
        \includegraphics[width=\textwidth]{plots/manyholes_complex_16_no_cc.jpg}
        \caption{Without circumcircle filtering.}
        \label{fig:manyholes_complex_no_cc}
    \end{subfigure}
    \begin{subfigure}{0.49\textwidth}
        \includegraphics[width=\textwidth]{plots/manyholes_complex_16_cc.jpg}
        \caption{With circumcircle filtering}
        \label{fig:manyholes_complex_cc}
    \end{subfigure}
    \caption{A visualisation of the 2-skeleton of the LVR complex on the manyholes dataset with and without circumcircle filtering. Not all 2-simplices are shown for clarity.}
    \label{fig:manyholes_complex}
\end{figure}

In Figure~\ref{fig:nested_ph_cc}, we see that applying CC lowers the number of
homology classes and removes all classes that do not persist until infinity.
Examining the 2-skeleton of the complex in Figure~\ref{fig:nested_complex_cc},
we see that almost all simplices that cross the decision boundary multiple times
are removed.  We can also see that CC has an added benefit of reducing the
number of simplices in the complex, which can help reduce the computational cost
of constructing the complex and computing its persistence diagram.

In Figure~\ref{fig:manyholes_complex} and Figure~\ref{fig:manyholes_ph}, we
observe a similar effect on the manyholes dataset, where without CC the complex
contains spurious holes, such as the one around the
point $(0.5, - 0.1)$. Applying CC removes these spurious holes.
A visualisation of how LVR changes as the filtration parameter is varied
with and without CC can be found in Appendix~\ref{app:lvr}.

\begin{figure}
    \centering
    \begin{subfigure}{0.49\textwidth}
        \resizebox{\textwidth}{!}{
            %% Creator: Matplotlib, PGF backend
%%
%% To include the figure in your LaTeX document, write
%%   \input{<filename>.pgf}
%%
%% Make sure the required packages are loaded in your preamble
%%   \usepackage{pgf}
%%
%% Also ensure that all the required font packages are loaded; for instance,
%% the lmodern package is sometimes necessary when using math font.
%%   \usepackage{lmodern}
%%
%% Figures using additional raster images can only be included by \input if
%% they are in the same directory as the main LaTeX file. For loading figures
%% from other directories you can use the `import` package
%%   \usepackage{import}
%%
%% and then include the figures with
%%   \import{<path to file>}{<filename>.pgf}
%%
%% Matplotlib used the following preamble
%%   \def\mathdefault#1{#1}
%%   \everymath=\expandafter{\the\everymath\displaystyle}
%%   
%%   \ifdefined\pdftexversion\else  % non-pdftex case.
%%     \usepackage{fontspec}
%%     \setmainfont{DejaVuSerif.ttf}[Path=\detokenize{/home/snek/repos/homology-decision-bondaries-clean/venv/lib/python3.9/site-packages/matplotlib/mpl-data/fonts/ttf/}]
%%     \setsansfont{DejaVuSans.ttf}[Path=\detokenize{/home/snek/repos/homology-decision-bondaries-clean/venv/lib/python3.9/site-packages/matplotlib/mpl-data/fonts/ttf/}]
%%     \setmonofont{DejaVuSansMono.ttf}[Path=\detokenize{/home/snek/repos/homology-decision-bondaries-clean/venv/lib/python3.9/site-packages/matplotlib/mpl-data/fonts/ttf/}]
%%   \fi
%%   \makeatletter\@ifpackageloaded{underscore}{}{\usepackage[strings]{underscore}}\makeatother
%%
\begingroup%
\makeatletter%
\begin{pgfpicture}%
\pgfpathrectangle{\pgfpointorigin}{\pgfqpoint{4.331782in}{4.144486in}}%
\pgfusepath{use as bounding box, clip}%
\begin{pgfscope}%
\pgfsetbuttcap%
\pgfsetmiterjoin%
\definecolor{currentfill}{rgb}{1.000000,1.000000,1.000000}%
\pgfsetfillcolor{currentfill}%
\pgfsetlinewidth{0.000000pt}%
\definecolor{currentstroke}{rgb}{1.000000,1.000000,1.000000}%
\pgfsetstrokecolor{currentstroke}%
\pgfsetdash{}{0pt}%
\pgfpathmoveto{\pgfqpoint{0.000000in}{-0.000000in}}%
\pgfpathlineto{\pgfqpoint{4.331782in}{-0.000000in}}%
\pgfpathlineto{\pgfqpoint{4.331782in}{4.144486in}}%
\pgfpathlineto{\pgfqpoint{0.000000in}{4.144486in}}%
\pgfpathlineto{\pgfqpoint{0.000000in}{-0.000000in}}%
\pgfpathclose%
\pgfusepath{fill}%
\end{pgfscope}%
\begin{pgfscope}%
\pgfsetbuttcap%
\pgfsetmiterjoin%
\definecolor{currentfill}{rgb}{1.000000,1.000000,1.000000}%
\pgfsetfillcolor{currentfill}%
\pgfsetlinewidth{0.000000pt}%
\definecolor{currentstroke}{rgb}{0.000000,0.000000,0.000000}%
\pgfsetstrokecolor{currentstroke}%
\pgfsetstrokeopacity{0.000000}%
\pgfsetdash{}{0pt}%
\pgfpathmoveto{\pgfqpoint{0.800049in}{0.448486in}}%
\pgfpathlineto{\pgfqpoint{4.331782in}{0.448486in}}%
\pgfpathlineto{\pgfqpoint{4.331782in}{4.144486in}}%
\pgfpathlineto{\pgfqpoint{0.800049in}{4.144486in}}%
\pgfpathlineto{\pgfqpoint{0.800049in}{0.448486in}}%
\pgfpathclose%
\pgfusepath{fill}%
\end{pgfscope}%
\begin{pgfscope}%
\pgfpathrectangle{\pgfqpoint{0.800049in}{0.448486in}}{\pgfqpoint{3.531733in}{3.696000in}}%
\pgfusepath{clip}%
\pgfsetbuttcap%
\pgfsetmiterjoin%
\definecolor{currentfill}{rgb}{0.827451,0.827451,0.827451}%
\pgfsetfillcolor{currentfill}%
\pgfsetlinewidth{1.003750pt}%
\definecolor{currentstroke}{rgb}{0.827451,0.827451,0.827451}%
\pgfsetstrokecolor{currentstroke}%
\pgfsetdash{}{0pt}%
\pgfpathmoveto{\pgfqpoint{0.800049in}{0.448486in}}%
\pgfpathlineto{\pgfqpoint{4.331782in}{0.448486in}}%
\pgfpathlineto{\pgfqpoint{4.331782in}{3.980219in}}%
\pgfpathlineto{\pgfqpoint{0.800049in}{0.448486in}}%
\pgfpathclose%
\pgfusepath{stroke,fill}%
\end{pgfscope}%
\begin{pgfscope}%
\pgfpathrectangle{\pgfqpoint{0.800049in}{0.448486in}}{\pgfqpoint{3.531733in}{3.696000in}}%
\pgfusepath{clip}%
\pgfsetbuttcap%
\pgfsetroundjoin%
\definecolor{currentfill}{rgb}{0.894118,0.101961,0.109804}%
\pgfsetfillcolor{currentfill}%
\pgfsetfillopacity{0.600000}%
\pgfsetlinewidth{1.003750pt}%
\definecolor{currentstroke}{rgb}{0.894118,0.101961,0.109804}%
\pgfsetstrokecolor{currentstroke}%
\pgfsetstrokeopacity{0.600000}%
\pgfsetdash{}{0pt}%
\pgfpathmoveto{\pgfqpoint{4.249649in}{4.020686in}}%
\pgfpathcurveto{\pgfqpoint{4.260699in}{4.020686in}}{\pgfqpoint{4.271298in}{4.025076in}}{\pgfqpoint{4.279111in}{4.032890in}}%
\pgfpathcurveto{\pgfqpoint{4.286925in}{4.040703in}}{\pgfqpoint{4.291315in}{4.051303in}}{\pgfqpoint{4.291315in}{4.062353in}}%
\pgfpathcurveto{\pgfqpoint{4.291315in}{4.073403in}}{\pgfqpoint{4.286925in}{4.084002in}}{\pgfqpoint{4.279111in}{4.091815in}}%
\pgfpathcurveto{\pgfqpoint{4.271298in}{4.099629in}}{\pgfqpoint{4.260699in}{4.104019in}}{\pgfqpoint{4.249649in}{4.104019in}}%
\pgfpathcurveto{\pgfqpoint{4.238598in}{4.104019in}}{\pgfqpoint{4.227999in}{4.099629in}}{\pgfqpoint{4.220186in}{4.091815in}}%
\pgfpathcurveto{\pgfqpoint{4.212372in}{4.084002in}}{\pgfqpoint{4.207982in}{4.073403in}}{\pgfqpoint{4.207982in}{4.062353in}}%
\pgfpathcurveto{\pgfqpoint{4.207982in}{4.051303in}}{\pgfqpoint{4.212372in}{4.040703in}}{\pgfqpoint{4.220186in}{4.032890in}}%
\pgfpathcurveto{\pgfqpoint{4.227999in}{4.025076in}}{\pgfqpoint{4.238598in}{4.020686in}}{\pgfqpoint{4.249649in}{4.020686in}}%
\pgfpathlineto{\pgfqpoint{4.249649in}{4.020686in}}%
\pgfpathclose%
\pgfusepath{stroke,fill}%
\end{pgfscope}%
\begin{pgfscope}%
\pgfpathrectangle{\pgfqpoint{0.800049in}{0.448486in}}{\pgfqpoint{3.531733in}{3.696000in}}%
\pgfusepath{clip}%
\pgfsetbuttcap%
\pgfsetroundjoin%
\definecolor{currentfill}{rgb}{0.894118,0.101961,0.109804}%
\pgfsetfillcolor{currentfill}%
\pgfsetfillopacity{0.600000}%
\pgfsetlinewidth{1.003750pt}%
\definecolor{currentstroke}{rgb}{0.894118,0.101961,0.109804}%
\pgfsetstrokecolor{currentstroke}%
\pgfsetstrokeopacity{0.600000}%
\pgfsetdash{}{0pt}%
\pgfpathmoveto{\pgfqpoint{4.249649in}{4.020686in}}%
\pgfpathcurveto{\pgfqpoint{4.260699in}{4.020686in}}{\pgfqpoint{4.271298in}{4.025076in}}{\pgfqpoint{4.279111in}{4.032890in}}%
\pgfpathcurveto{\pgfqpoint{4.286925in}{4.040703in}}{\pgfqpoint{4.291315in}{4.051303in}}{\pgfqpoint{4.291315in}{4.062353in}}%
\pgfpathcurveto{\pgfqpoint{4.291315in}{4.073403in}}{\pgfqpoint{4.286925in}{4.084002in}}{\pgfqpoint{4.279111in}{4.091815in}}%
\pgfpathcurveto{\pgfqpoint{4.271298in}{4.099629in}}{\pgfqpoint{4.260699in}{4.104019in}}{\pgfqpoint{4.249649in}{4.104019in}}%
\pgfpathcurveto{\pgfqpoint{4.238598in}{4.104019in}}{\pgfqpoint{4.227999in}{4.099629in}}{\pgfqpoint{4.220186in}{4.091815in}}%
\pgfpathcurveto{\pgfqpoint{4.212372in}{4.084002in}}{\pgfqpoint{4.207982in}{4.073403in}}{\pgfqpoint{4.207982in}{4.062353in}}%
\pgfpathcurveto{\pgfqpoint{4.207982in}{4.051303in}}{\pgfqpoint{4.212372in}{4.040703in}}{\pgfqpoint{4.220186in}{4.032890in}}%
\pgfpathcurveto{\pgfqpoint{4.227999in}{4.025076in}}{\pgfqpoint{4.238598in}{4.020686in}}{\pgfqpoint{4.249649in}{4.020686in}}%
\pgfpathlineto{\pgfqpoint{4.249649in}{4.020686in}}%
\pgfpathclose%
\pgfusepath{stroke,fill}%
\end{pgfscope}%
\begin{pgfscope}%
\pgfpathrectangle{\pgfqpoint{0.800049in}{0.448486in}}{\pgfqpoint{3.531733in}{3.696000in}}%
\pgfusepath{clip}%
\pgfsetbuttcap%
\pgfsetroundjoin%
\definecolor{currentfill}{rgb}{0.894118,0.101961,0.109804}%
\pgfsetfillcolor{currentfill}%
\pgfsetfillopacity{0.600000}%
\pgfsetlinewidth{1.003750pt}%
\definecolor{currentstroke}{rgb}{0.894118,0.101961,0.109804}%
\pgfsetstrokecolor{currentstroke}%
\pgfsetstrokeopacity{0.600000}%
\pgfsetdash{}{0pt}%
\pgfpathmoveto{\pgfqpoint{3.491495in}{4.020686in}}%
\pgfpathcurveto{\pgfqpoint{3.502545in}{4.020686in}}{\pgfqpoint{3.513144in}{4.025076in}}{\pgfqpoint{3.520958in}{4.032890in}}%
\pgfpathcurveto{\pgfqpoint{3.528771in}{4.040703in}}{\pgfqpoint{3.533161in}{4.051303in}}{\pgfqpoint{3.533161in}{4.062353in}}%
\pgfpathcurveto{\pgfqpoint{3.533161in}{4.073403in}}{\pgfqpoint{3.528771in}{4.084002in}}{\pgfqpoint{3.520958in}{4.091815in}}%
\pgfpathcurveto{\pgfqpoint{3.513144in}{4.099629in}}{\pgfqpoint{3.502545in}{4.104019in}}{\pgfqpoint{3.491495in}{4.104019in}}%
\pgfpathcurveto{\pgfqpoint{3.480445in}{4.104019in}}{\pgfqpoint{3.469846in}{4.099629in}}{\pgfqpoint{3.462032in}{4.091815in}}%
\pgfpathcurveto{\pgfqpoint{3.454218in}{4.084002in}}{\pgfqpoint{3.449828in}{4.073403in}}{\pgfqpoint{3.449828in}{4.062353in}}%
\pgfpathcurveto{\pgfqpoint{3.449828in}{4.051303in}}{\pgfqpoint{3.454218in}{4.040703in}}{\pgfqpoint{3.462032in}{4.032890in}}%
\pgfpathcurveto{\pgfqpoint{3.469846in}{4.025076in}}{\pgfqpoint{3.480445in}{4.020686in}}{\pgfqpoint{3.491495in}{4.020686in}}%
\pgfpathlineto{\pgfqpoint{3.491495in}{4.020686in}}%
\pgfpathclose%
\pgfusepath{stroke,fill}%
\end{pgfscope}%
\begin{pgfscope}%
\pgfpathrectangle{\pgfqpoint{0.800049in}{0.448486in}}{\pgfqpoint{3.531733in}{3.696000in}}%
\pgfusepath{clip}%
\pgfsetbuttcap%
\pgfsetroundjoin%
\definecolor{currentfill}{rgb}{0.894118,0.101961,0.109804}%
\pgfsetfillcolor{currentfill}%
\pgfsetfillopacity{0.600000}%
\pgfsetlinewidth{1.003750pt}%
\definecolor{currentstroke}{rgb}{0.894118,0.101961,0.109804}%
\pgfsetstrokecolor{currentstroke}%
\pgfsetstrokeopacity{0.600000}%
\pgfsetdash{}{0pt}%
\pgfpathmoveto{\pgfqpoint{3.491495in}{4.020686in}}%
\pgfpathcurveto{\pgfqpoint{3.502545in}{4.020686in}}{\pgfqpoint{3.513144in}{4.025076in}}{\pgfqpoint{3.520958in}{4.032890in}}%
\pgfpathcurveto{\pgfqpoint{3.528771in}{4.040703in}}{\pgfqpoint{3.533161in}{4.051303in}}{\pgfqpoint{3.533161in}{4.062353in}}%
\pgfpathcurveto{\pgfqpoint{3.533161in}{4.073403in}}{\pgfqpoint{3.528771in}{4.084002in}}{\pgfqpoint{3.520958in}{4.091815in}}%
\pgfpathcurveto{\pgfqpoint{3.513144in}{4.099629in}}{\pgfqpoint{3.502545in}{4.104019in}}{\pgfqpoint{3.491495in}{4.104019in}}%
\pgfpathcurveto{\pgfqpoint{3.480445in}{4.104019in}}{\pgfqpoint{3.469846in}{4.099629in}}{\pgfqpoint{3.462032in}{4.091815in}}%
\pgfpathcurveto{\pgfqpoint{3.454218in}{4.084002in}}{\pgfqpoint{3.449828in}{4.073403in}}{\pgfqpoint{3.449828in}{4.062353in}}%
\pgfpathcurveto{\pgfqpoint{3.449828in}{4.051303in}}{\pgfqpoint{3.454218in}{4.040703in}}{\pgfqpoint{3.462032in}{4.032890in}}%
\pgfpathcurveto{\pgfqpoint{3.469846in}{4.025076in}}{\pgfqpoint{3.480445in}{4.020686in}}{\pgfqpoint{3.491495in}{4.020686in}}%
\pgfpathlineto{\pgfqpoint{3.491495in}{4.020686in}}%
\pgfpathclose%
\pgfusepath{stroke,fill}%
\end{pgfscope}%
\begin{pgfscope}%
\pgfpathrectangle{\pgfqpoint{0.800049in}{0.448486in}}{\pgfqpoint{3.531733in}{3.696000in}}%
\pgfusepath{clip}%
\pgfsetbuttcap%
\pgfsetroundjoin%
\definecolor{currentfill}{rgb}{0.894118,0.101961,0.109804}%
\pgfsetfillcolor{currentfill}%
\pgfsetfillopacity{0.600000}%
\pgfsetlinewidth{1.003750pt}%
\definecolor{currentstroke}{rgb}{0.894118,0.101961,0.109804}%
\pgfsetstrokecolor{currentstroke}%
\pgfsetstrokeopacity{0.600000}%
\pgfsetdash{}{0pt}%
\pgfpathmoveto{\pgfqpoint{2.733341in}{4.020686in}}%
\pgfpathcurveto{\pgfqpoint{2.744391in}{4.020686in}}{\pgfqpoint{2.754990in}{4.025076in}}{\pgfqpoint{2.762804in}{4.032890in}}%
\pgfpathcurveto{\pgfqpoint{2.770617in}{4.040703in}}{\pgfqpoint{2.775008in}{4.051303in}}{\pgfqpoint{2.775008in}{4.062353in}}%
\pgfpathcurveto{\pgfqpoint{2.775008in}{4.073403in}}{\pgfqpoint{2.770617in}{4.084002in}}{\pgfqpoint{2.762804in}{4.091815in}}%
\pgfpathcurveto{\pgfqpoint{2.754990in}{4.099629in}}{\pgfqpoint{2.744391in}{4.104019in}}{\pgfqpoint{2.733341in}{4.104019in}}%
\pgfpathcurveto{\pgfqpoint{2.722291in}{4.104019in}}{\pgfqpoint{2.711692in}{4.099629in}}{\pgfqpoint{2.703878in}{4.091815in}}%
\pgfpathcurveto{\pgfqpoint{2.696065in}{4.084002in}}{\pgfqpoint{2.691674in}{4.073403in}}{\pgfqpoint{2.691674in}{4.062353in}}%
\pgfpathcurveto{\pgfqpoint{2.691674in}{4.051303in}}{\pgfqpoint{2.696065in}{4.040703in}}{\pgfqpoint{2.703878in}{4.032890in}}%
\pgfpathcurveto{\pgfqpoint{2.711692in}{4.025076in}}{\pgfqpoint{2.722291in}{4.020686in}}{\pgfqpoint{2.733341in}{4.020686in}}%
\pgfpathlineto{\pgfqpoint{2.733341in}{4.020686in}}%
\pgfpathclose%
\pgfusepath{stroke,fill}%
\end{pgfscope}%
\begin{pgfscope}%
\pgfpathrectangle{\pgfqpoint{0.800049in}{0.448486in}}{\pgfqpoint{3.531733in}{3.696000in}}%
\pgfusepath{clip}%
\pgfsetbuttcap%
\pgfsetroundjoin%
\definecolor{currentfill}{rgb}{0.894118,0.101961,0.109804}%
\pgfsetfillcolor{currentfill}%
\pgfsetfillopacity{0.600000}%
\pgfsetlinewidth{1.003750pt}%
\definecolor{currentstroke}{rgb}{0.894118,0.101961,0.109804}%
\pgfsetstrokecolor{currentstroke}%
\pgfsetstrokeopacity{0.600000}%
\pgfsetdash{}{0pt}%
\pgfpathmoveto{\pgfqpoint{1.975187in}{4.020686in}}%
\pgfpathcurveto{\pgfqpoint{1.986237in}{4.020686in}}{\pgfqpoint{1.996836in}{4.025076in}}{\pgfqpoint{2.004650in}{4.032890in}}%
\pgfpathcurveto{\pgfqpoint{2.012463in}{4.040703in}}{\pgfqpoint{2.016854in}{4.051303in}}{\pgfqpoint{2.016854in}{4.062353in}}%
\pgfpathcurveto{\pgfqpoint{2.016854in}{4.073403in}}{\pgfqpoint{2.012463in}{4.084002in}}{\pgfqpoint{2.004650in}{4.091815in}}%
\pgfpathcurveto{\pgfqpoint{1.996836in}{4.099629in}}{\pgfqpoint{1.986237in}{4.104019in}}{\pgfqpoint{1.975187in}{4.104019in}}%
\pgfpathcurveto{\pgfqpoint{1.964137in}{4.104019in}}{\pgfqpoint{1.953538in}{4.099629in}}{\pgfqpoint{1.945724in}{4.091815in}}%
\pgfpathcurveto{\pgfqpoint{1.937911in}{4.084002in}}{\pgfqpoint{1.933520in}{4.073403in}}{\pgfqpoint{1.933520in}{4.062353in}}%
\pgfpathcurveto{\pgfqpoint{1.933520in}{4.051303in}}{\pgfqpoint{1.937911in}{4.040703in}}{\pgfqpoint{1.945724in}{4.032890in}}%
\pgfpathcurveto{\pgfqpoint{1.953538in}{4.025076in}}{\pgfqpoint{1.964137in}{4.020686in}}{\pgfqpoint{1.975187in}{4.020686in}}%
\pgfpathlineto{\pgfqpoint{1.975187in}{4.020686in}}%
\pgfpathclose%
\pgfusepath{stroke,fill}%
\end{pgfscope}%
\begin{pgfscope}%
\pgfpathrectangle{\pgfqpoint{0.800049in}{0.448486in}}{\pgfqpoint{3.531733in}{3.696000in}}%
\pgfusepath{clip}%
\pgfsetbuttcap%
\pgfsetroundjoin%
\definecolor{currentfill}{rgb}{0.894118,0.101961,0.109804}%
\pgfsetfillcolor{currentfill}%
\pgfsetfillopacity{0.600000}%
\pgfsetlinewidth{1.003750pt}%
\definecolor{currentstroke}{rgb}{0.894118,0.101961,0.109804}%
\pgfsetstrokecolor{currentstroke}%
\pgfsetstrokeopacity{0.600000}%
\pgfsetdash{}{0pt}%
\pgfpathmoveto{\pgfqpoint{1.975187in}{4.020686in}}%
\pgfpathcurveto{\pgfqpoint{1.986237in}{4.020686in}}{\pgfqpoint{1.996836in}{4.025076in}}{\pgfqpoint{2.004650in}{4.032890in}}%
\pgfpathcurveto{\pgfqpoint{2.012463in}{4.040703in}}{\pgfqpoint{2.016854in}{4.051303in}}{\pgfqpoint{2.016854in}{4.062353in}}%
\pgfpathcurveto{\pgfqpoint{2.016854in}{4.073403in}}{\pgfqpoint{2.012463in}{4.084002in}}{\pgfqpoint{2.004650in}{4.091815in}}%
\pgfpathcurveto{\pgfqpoint{1.996836in}{4.099629in}}{\pgfqpoint{1.986237in}{4.104019in}}{\pgfqpoint{1.975187in}{4.104019in}}%
\pgfpathcurveto{\pgfqpoint{1.964137in}{4.104019in}}{\pgfqpoint{1.953538in}{4.099629in}}{\pgfqpoint{1.945724in}{4.091815in}}%
\pgfpathcurveto{\pgfqpoint{1.937911in}{4.084002in}}{\pgfqpoint{1.933520in}{4.073403in}}{\pgfqpoint{1.933520in}{4.062353in}}%
\pgfpathcurveto{\pgfqpoint{1.933520in}{4.051303in}}{\pgfqpoint{1.937911in}{4.040703in}}{\pgfqpoint{1.945724in}{4.032890in}}%
\pgfpathcurveto{\pgfqpoint{1.953538in}{4.025076in}}{\pgfqpoint{1.964137in}{4.020686in}}{\pgfqpoint{1.975187in}{4.020686in}}%
\pgfpathlineto{\pgfqpoint{1.975187in}{4.020686in}}%
\pgfpathclose%
\pgfusepath{stroke,fill}%
\end{pgfscope}%
\begin{pgfscope}%
\pgfpathrectangle{\pgfqpoint{0.800049in}{0.448486in}}{\pgfqpoint{3.531733in}{3.696000in}}%
\pgfusepath{clip}%
\pgfsetbuttcap%
\pgfsetroundjoin%
\definecolor{currentfill}{rgb}{0.894118,0.101961,0.109804}%
\pgfsetfillcolor{currentfill}%
\pgfsetfillopacity{0.600000}%
\pgfsetlinewidth{1.003750pt}%
\definecolor{currentstroke}{rgb}{0.894118,0.101961,0.109804}%
\pgfsetstrokecolor{currentstroke}%
\pgfsetstrokeopacity{0.600000}%
\pgfsetdash{}{0pt}%
\pgfpathmoveto{\pgfqpoint{1.975187in}{4.020686in}}%
\pgfpathcurveto{\pgfqpoint{1.986237in}{4.020686in}}{\pgfqpoint{1.996836in}{4.025076in}}{\pgfqpoint{2.004650in}{4.032890in}}%
\pgfpathcurveto{\pgfqpoint{2.012463in}{4.040703in}}{\pgfqpoint{2.016854in}{4.051303in}}{\pgfqpoint{2.016854in}{4.062353in}}%
\pgfpathcurveto{\pgfqpoint{2.016854in}{4.073403in}}{\pgfqpoint{2.012463in}{4.084002in}}{\pgfqpoint{2.004650in}{4.091815in}}%
\pgfpathcurveto{\pgfqpoint{1.996836in}{4.099629in}}{\pgfqpoint{1.986237in}{4.104019in}}{\pgfqpoint{1.975187in}{4.104019in}}%
\pgfpathcurveto{\pgfqpoint{1.964137in}{4.104019in}}{\pgfqpoint{1.953538in}{4.099629in}}{\pgfqpoint{1.945724in}{4.091815in}}%
\pgfpathcurveto{\pgfqpoint{1.937911in}{4.084002in}}{\pgfqpoint{1.933520in}{4.073403in}}{\pgfqpoint{1.933520in}{4.062353in}}%
\pgfpathcurveto{\pgfqpoint{1.933520in}{4.051303in}}{\pgfqpoint{1.937911in}{4.040703in}}{\pgfqpoint{1.945724in}{4.032890in}}%
\pgfpathcurveto{\pgfqpoint{1.953538in}{4.025076in}}{\pgfqpoint{1.964137in}{4.020686in}}{\pgfqpoint{1.975187in}{4.020686in}}%
\pgfpathlineto{\pgfqpoint{1.975187in}{4.020686in}}%
\pgfpathclose%
\pgfusepath{stroke,fill}%
\end{pgfscope}%
\begin{pgfscope}%
\pgfpathrectangle{\pgfqpoint{0.800049in}{0.448486in}}{\pgfqpoint{3.531733in}{3.696000in}}%
\pgfusepath{clip}%
\pgfsetbuttcap%
\pgfsetroundjoin%
\definecolor{currentfill}{rgb}{0.894118,0.101961,0.109804}%
\pgfsetfillcolor{currentfill}%
\pgfsetfillopacity{0.600000}%
\pgfsetlinewidth{1.003750pt}%
\definecolor{currentstroke}{rgb}{0.894118,0.101961,0.109804}%
\pgfsetstrokecolor{currentstroke}%
\pgfsetstrokeopacity{0.600000}%
\pgfsetdash{}{0pt}%
\pgfpathmoveto{\pgfqpoint{1.975187in}{4.020686in}}%
\pgfpathcurveto{\pgfqpoint{1.986237in}{4.020686in}}{\pgfqpoint{1.996836in}{4.025076in}}{\pgfqpoint{2.004650in}{4.032890in}}%
\pgfpathcurveto{\pgfqpoint{2.012463in}{4.040703in}}{\pgfqpoint{2.016854in}{4.051303in}}{\pgfqpoint{2.016854in}{4.062353in}}%
\pgfpathcurveto{\pgfqpoint{2.016854in}{4.073403in}}{\pgfqpoint{2.012463in}{4.084002in}}{\pgfqpoint{2.004650in}{4.091815in}}%
\pgfpathcurveto{\pgfqpoint{1.996836in}{4.099629in}}{\pgfqpoint{1.986237in}{4.104019in}}{\pgfqpoint{1.975187in}{4.104019in}}%
\pgfpathcurveto{\pgfqpoint{1.964137in}{4.104019in}}{\pgfqpoint{1.953538in}{4.099629in}}{\pgfqpoint{1.945724in}{4.091815in}}%
\pgfpathcurveto{\pgfqpoint{1.937911in}{4.084002in}}{\pgfqpoint{1.933520in}{4.073403in}}{\pgfqpoint{1.933520in}{4.062353in}}%
\pgfpathcurveto{\pgfqpoint{1.933520in}{4.051303in}}{\pgfqpoint{1.937911in}{4.040703in}}{\pgfqpoint{1.945724in}{4.032890in}}%
\pgfpathcurveto{\pgfqpoint{1.953538in}{4.025076in}}{\pgfqpoint{1.964137in}{4.020686in}}{\pgfqpoint{1.975187in}{4.020686in}}%
\pgfpathlineto{\pgfqpoint{1.975187in}{4.020686in}}%
\pgfpathclose%
\pgfusepath{stroke,fill}%
\end{pgfscope}%
\begin{pgfscope}%
\pgfpathrectangle{\pgfqpoint{0.800049in}{0.448486in}}{\pgfqpoint{3.531733in}{3.696000in}}%
\pgfusepath{clip}%
\pgfsetbuttcap%
\pgfsetroundjoin%
\definecolor{currentfill}{rgb}{0.894118,0.101961,0.109804}%
\pgfsetfillcolor{currentfill}%
\pgfsetfillopacity{0.600000}%
\pgfsetlinewidth{1.003750pt}%
\definecolor{currentstroke}{rgb}{0.894118,0.101961,0.109804}%
\pgfsetstrokecolor{currentstroke}%
\pgfsetstrokeopacity{0.600000}%
\pgfsetdash{}{0pt}%
\pgfpathmoveto{\pgfqpoint{1.975187in}{4.020686in}}%
\pgfpathcurveto{\pgfqpoint{1.986237in}{4.020686in}}{\pgfqpoint{1.996836in}{4.025076in}}{\pgfqpoint{2.004650in}{4.032890in}}%
\pgfpathcurveto{\pgfqpoint{2.012463in}{4.040703in}}{\pgfqpoint{2.016854in}{4.051303in}}{\pgfqpoint{2.016854in}{4.062353in}}%
\pgfpathcurveto{\pgfqpoint{2.016854in}{4.073403in}}{\pgfqpoint{2.012463in}{4.084002in}}{\pgfqpoint{2.004650in}{4.091815in}}%
\pgfpathcurveto{\pgfqpoint{1.996836in}{4.099629in}}{\pgfqpoint{1.986237in}{4.104019in}}{\pgfqpoint{1.975187in}{4.104019in}}%
\pgfpathcurveto{\pgfqpoint{1.964137in}{4.104019in}}{\pgfqpoint{1.953538in}{4.099629in}}{\pgfqpoint{1.945724in}{4.091815in}}%
\pgfpathcurveto{\pgfqpoint{1.937911in}{4.084002in}}{\pgfqpoint{1.933520in}{4.073403in}}{\pgfqpoint{1.933520in}{4.062353in}}%
\pgfpathcurveto{\pgfqpoint{1.933520in}{4.051303in}}{\pgfqpoint{1.937911in}{4.040703in}}{\pgfqpoint{1.945724in}{4.032890in}}%
\pgfpathcurveto{\pgfqpoint{1.953538in}{4.025076in}}{\pgfqpoint{1.964137in}{4.020686in}}{\pgfqpoint{1.975187in}{4.020686in}}%
\pgfpathlineto{\pgfqpoint{1.975187in}{4.020686in}}%
\pgfpathclose%
\pgfusepath{stroke,fill}%
\end{pgfscope}%
\begin{pgfscope}%
\pgfpathrectangle{\pgfqpoint{0.800049in}{0.448486in}}{\pgfqpoint{3.531733in}{3.696000in}}%
\pgfusepath{clip}%
\pgfsetbuttcap%
\pgfsetroundjoin%
\definecolor{currentfill}{rgb}{0.894118,0.101961,0.109804}%
\pgfsetfillcolor{currentfill}%
\pgfsetfillopacity{0.600000}%
\pgfsetlinewidth{1.003750pt}%
\definecolor{currentstroke}{rgb}{0.894118,0.101961,0.109804}%
\pgfsetstrokecolor{currentstroke}%
\pgfsetstrokeopacity{0.600000}%
\pgfsetdash{}{0pt}%
\pgfpathmoveto{\pgfqpoint{1.975187in}{2.340112in}}%
\pgfpathcurveto{\pgfqpoint{1.986237in}{2.340112in}}{\pgfqpoint{1.996836in}{2.344502in}}{\pgfqpoint{2.004650in}{2.352315in}}%
\pgfpathcurveto{\pgfqpoint{2.012463in}{2.360129in}}{\pgfqpoint{2.016854in}{2.370728in}}{\pgfqpoint{2.016854in}{2.381778in}}%
\pgfpathcurveto{\pgfqpoint{2.016854in}{2.392828in}}{\pgfqpoint{2.012463in}{2.403427in}}{\pgfqpoint{2.004650in}{2.411241in}}%
\pgfpathcurveto{\pgfqpoint{1.996836in}{2.419055in}}{\pgfqpoint{1.986237in}{2.423445in}}{\pgfqpoint{1.975187in}{2.423445in}}%
\pgfpathcurveto{\pgfqpoint{1.964137in}{2.423445in}}{\pgfqpoint{1.953538in}{2.419055in}}{\pgfqpoint{1.945724in}{2.411241in}}%
\pgfpathcurveto{\pgfqpoint{1.937911in}{2.403427in}}{\pgfqpoint{1.933520in}{2.392828in}}{\pgfqpoint{1.933520in}{2.381778in}}%
\pgfpathcurveto{\pgfqpoint{1.933520in}{2.370728in}}{\pgfqpoint{1.937911in}{2.360129in}}{\pgfqpoint{1.945724in}{2.352315in}}%
\pgfpathcurveto{\pgfqpoint{1.953538in}{2.344502in}}{\pgfqpoint{1.964137in}{2.340112in}}{\pgfqpoint{1.975187in}{2.340112in}}%
\pgfpathlineto{\pgfqpoint{1.975187in}{2.340112in}}%
\pgfpathclose%
\pgfusepath{stroke,fill}%
\end{pgfscope}%
\begin{pgfscope}%
\pgfpathrectangle{\pgfqpoint{0.800049in}{0.448486in}}{\pgfqpoint{3.531733in}{3.696000in}}%
\pgfusepath{clip}%
\pgfsetbuttcap%
\pgfsetroundjoin%
\definecolor{currentfill}{rgb}{0.894118,0.101961,0.109804}%
\pgfsetfillcolor{currentfill}%
\pgfsetfillopacity{0.600000}%
\pgfsetlinewidth{1.003750pt}%
\definecolor{currentstroke}{rgb}{0.894118,0.101961,0.109804}%
\pgfsetstrokecolor{currentstroke}%
\pgfsetstrokeopacity{0.600000}%
\pgfsetdash{}{0pt}%
\pgfpathmoveto{\pgfqpoint{0.964315in}{4.020686in}}%
\pgfpathcurveto{\pgfqpoint{0.975365in}{4.020686in}}{\pgfqpoint{0.985964in}{4.025076in}}{\pgfqpoint{0.993778in}{4.032890in}}%
\pgfpathcurveto{\pgfqpoint{1.001592in}{4.040703in}}{\pgfqpoint{1.005982in}{4.051303in}}{\pgfqpoint{1.005982in}{4.062353in}}%
\pgfpathcurveto{\pgfqpoint{1.005982in}{4.073403in}}{\pgfqpoint{1.001592in}{4.084002in}}{\pgfqpoint{0.993778in}{4.091815in}}%
\pgfpathcurveto{\pgfqpoint{0.985964in}{4.099629in}}{\pgfqpoint{0.975365in}{4.104019in}}{\pgfqpoint{0.964315in}{4.104019in}}%
\pgfpathcurveto{\pgfqpoint{0.953265in}{4.104019in}}{\pgfqpoint{0.942666in}{4.099629in}}{\pgfqpoint{0.934852in}{4.091815in}}%
\pgfpathcurveto{\pgfqpoint{0.927039in}{4.084002in}}{\pgfqpoint{0.922649in}{4.073403in}}{\pgfqpoint{0.922649in}{4.062353in}}%
\pgfpathcurveto{\pgfqpoint{0.922649in}{4.051303in}}{\pgfqpoint{0.927039in}{4.040703in}}{\pgfqpoint{0.934852in}{4.032890in}}%
\pgfpathcurveto{\pgfqpoint{0.942666in}{4.025076in}}{\pgfqpoint{0.953265in}{4.020686in}}{\pgfqpoint{0.964315in}{4.020686in}}%
\pgfpathlineto{\pgfqpoint{0.964315in}{4.020686in}}%
\pgfpathclose%
\pgfusepath{stroke,fill}%
\end{pgfscope}%
\begin{pgfscope}%
\pgfpathrectangle{\pgfqpoint{0.800049in}{0.448486in}}{\pgfqpoint{3.531733in}{3.696000in}}%
\pgfusepath{clip}%
\pgfsetbuttcap%
\pgfsetroundjoin%
\definecolor{currentfill}{rgb}{0.894118,0.101961,0.109804}%
\pgfsetfillcolor{currentfill}%
\pgfsetfillopacity{0.600000}%
\pgfsetlinewidth{1.003750pt}%
\definecolor{currentstroke}{rgb}{0.894118,0.101961,0.109804}%
\pgfsetstrokecolor{currentstroke}%
\pgfsetstrokeopacity{0.600000}%
\pgfsetdash{}{0pt}%
\pgfpathmoveto{\pgfqpoint{0.964315in}{3.098265in}}%
\pgfpathcurveto{\pgfqpoint{0.975365in}{3.098265in}}{\pgfqpoint{0.985964in}{3.102656in}}{\pgfqpoint{0.993778in}{3.110469in}}%
\pgfpathcurveto{\pgfqpoint{1.001592in}{3.118283in}}{\pgfqpoint{1.005982in}{3.128882in}}{\pgfqpoint{1.005982in}{3.139932in}}%
\pgfpathcurveto{\pgfqpoint{1.005982in}{3.150982in}}{\pgfqpoint{1.001592in}{3.161581in}}{\pgfqpoint{0.993778in}{3.169395in}}%
\pgfpathcurveto{\pgfqpoint{0.985964in}{3.177209in}}{\pgfqpoint{0.975365in}{3.181599in}}{\pgfqpoint{0.964315in}{3.181599in}}%
\pgfpathcurveto{\pgfqpoint{0.953265in}{3.181599in}}{\pgfqpoint{0.942666in}{3.177209in}}{\pgfqpoint{0.934852in}{3.169395in}}%
\pgfpathcurveto{\pgfqpoint{0.927039in}{3.161581in}}{\pgfqpoint{0.922649in}{3.150982in}}{\pgfqpoint{0.922649in}{3.139932in}}%
\pgfpathcurveto{\pgfqpoint{0.922649in}{3.128882in}}{\pgfqpoint{0.927039in}{3.118283in}}{\pgfqpoint{0.934852in}{3.110469in}}%
\pgfpathcurveto{\pgfqpoint{0.942666in}{3.102656in}}{\pgfqpoint{0.953265in}{3.098265in}}{\pgfqpoint{0.964315in}{3.098265in}}%
\pgfpathlineto{\pgfqpoint{0.964315in}{3.098265in}}%
\pgfpathclose%
\pgfusepath{stroke,fill}%
\end{pgfscope}%
\begin{pgfscope}%
\pgfpathrectangle{\pgfqpoint{0.800049in}{0.448486in}}{\pgfqpoint{3.531733in}{3.696000in}}%
\pgfusepath{clip}%
\pgfsetbuttcap%
\pgfsetroundjoin%
\definecolor{currentfill}{rgb}{0.894118,0.101961,0.109804}%
\pgfsetfillcolor{currentfill}%
\pgfsetfillopacity{0.600000}%
\pgfsetlinewidth{1.003750pt}%
\definecolor{currentstroke}{rgb}{0.894118,0.101961,0.109804}%
\pgfsetstrokecolor{currentstroke}%
\pgfsetstrokeopacity{0.600000}%
\pgfsetdash{}{0pt}%
\pgfpathmoveto{\pgfqpoint{0.964315in}{2.340112in}}%
\pgfpathcurveto{\pgfqpoint{0.975365in}{2.340112in}}{\pgfqpoint{0.985964in}{2.344502in}}{\pgfqpoint{0.993778in}{2.352315in}}%
\pgfpathcurveto{\pgfqpoint{1.001592in}{2.360129in}}{\pgfqpoint{1.005982in}{2.370728in}}{\pgfqpoint{1.005982in}{2.381778in}}%
\pgfpathcurveto{\pgfqpoint{1.005982in}{2.392828in}}{\pgfqpoint{1.001592in}{2.403427in}}{\pgfqpoint{0.993778in}{2.411241in}}%
\pgfpathcurveto{\pgfqpoint{0.985964in}{2.419055in}}{\pgfqpoint{0.975365in}{2.423445in}}{\pgfqpoint{0.964315in}{2.423445in}}%
\pgfpathcurveto{\pgfqpoint{0.953265in}{2.423445in}}{\pgfqpoint{0.942666in}{2.419055in}}{\pgfqpoint{0.934852in}{2.411241in}}%
\pgfpathcurveto{\pgfqpoint{0.927039in}{2.403427in}}{\pgfqpoint{0.922649in}{2.392828in}}{\pgfqpoint{0.922649in}{2.381778in}}%
\pgfpathcurveto{\pgfqpoint{0.922649in}{2.370728in}}{\pgfqpoint{0.927039in}{2.360129in}}{\pgfqpoint{0.934852in}{2.352315in}}%
\pgfpathcurveto{\pgfqpoint{0.942666in}{2.344502in}}{\pgfqpoint{0.953265in}{2.340112in}}{\pgfqpoint{0.964315in}{2.340112in}}%
\pgfpathlineto{\pgfqpoint{0.964315in}{2.340112in}}%
\pgfpathclose%
\pgfusepath{stroke,fill}%
\end{pgfscope}%
\begin{pgfscope}%
\pgfsetbuttcap%
\pgfsetroundjoin%
\definecolor{currentfill}{rgb}{0.000000,0.000000,0.000000}%
\pgfsetfillcolor{currentfill}%
\pgfsetlinewidth{0.803000pt}%
\definecolor{currentstroke}{rgb}{0.000000,0.000000,0.000000}%
\pgfsetstrokecolor{currentstroke}%
\pgfsetdash{}{0pt}%
\pgfsys@defobject{currentmarker}{\pgfqpoint{0.000000in}{-0.048611in}}{\pgfqpoint{0.000000in}{0.000000in}}{%
\pgfpathmoveto{\pgfqpoint{0.000000in}{0.000000in}}%
\pgfpathlineto{\pgfqpoint{0.000000in}{-0.048611in}}%
\pgfusepath{stroke,fill}%
}%
\begin{pgfscope}%
\pgfsys@transformshift{0.964315in}{0.448486in}%
\pgfsys@useobject{currentmarker}{}%
\end{pgfscope}%
\end{pgfscope}%
\begin{pgfscope}%
\definecolor{textcolor}{rgb}{0.000000,0.000000,0.000000}%
\pgfsetstrokecolor{textcolor}%
\pgfsetfillcolor{textcolor}%
\pgftext[x=0.964315in,y=0.351264in,,top]{\color{textcolor}{\rmfamily\fontsize{10.000000}{12.000000}\selectfont\catcode`\^=\active\def^{\ifmmode\sp\else\^{}\fi}\catcode`\%=\active\def%{\%}$\mathdefault{0}$}}%
\end{pgfscope}%
\begin{pgfscope}%
\pgfsetbuttcap%
\pgfsetroundjoin%
\definecolor{currentfill}{rgb}{0.000000,0.000000,0.000000}%
\pgfsetfillcolor{currentfill}%
\pgfsetlinewidth{0.803000pt}%
\definecolor{currentstroke}{rgb}{0.000000,0.000000,0.000000}%
\pgfsetstrokecolor{currentstroke}%
\pgfsetdash{}{0pt}%
\pgfsys@defobject{currentmarker}{\pgfqpoint{0.000000in}{-0.048611in}}{\pgfqpoint{0.000000in}{0.000000in}}{%
\pgfpathmoveto{\pgfqpoint{0.000000in}{0.000000in}}%
\pgfpathlineto{\pgfqpoint{0.000000in}{-0.048611in}}%
\pgfusepath{stroke,fill}%
}%
\begin{pgfscope}%
\pgfsys@transformshift{1.469751in}{0.448486in}%
\pgfsys@useobject{currentmarker}{}%
\end{pgfscope}%
\end{pgfscope}%
\begin{pgfscope}%
\definecolor{textcolor}{rgb}{0.000000,0.000000,0.000000}%
\pgfsetstrokecolor{textcolor}%
\pgfsetfillcolor{textcolor}%
\pgftext[x=1.469751in,y=0.351264in,,top]{\color{textcolor}{\rmfamily\fontsize{10.000000}{12.000000}\selectfont\catcode`\^=\active\def^{\ifmmode\sp\else\^{}\fi}\catcode`\%=\active\def%{\%}$\mathdefault{2}$}}%
\end{pgfscope}%
\begin{pgfscope}%
\pgfsetbuttcap%
\pgfsetroundjoin%
\definecolor{currentfill}{rgb}{0.000000,0.000000,0.000000}%
\pgfsetfillcolor{currentfill}%
\pgfsetlinewidth{0.803000pt}%
\definecolor{currentstroke}{rgb}{0.000000,0.000000,0.000000}%
\pgfsetstrokecolor{currentstroke}%
\pgfsetdash{}{0pt}%
\pgfsys@defobject{currentmarker}{\pgfqpoint{0.000000in}{-0.048611in}}{\pgfqpoint{0.000000in}{0.000000in}}{%
\pgfpathmoveto{\pgfqpoint{0.000000in}{0.000000in}}%
\pgfpathlineto{\pgfqpoint{0.000000in}{-0.048611in}}%
\pgfusepath{stroke,fill}%
}%
\begin{pgfscope}%
\pgfsys@transformshift{1.975187in}{0.448486in}%
\pgfsys@useobject{currentmarker}{}%
\end{pgfscope}%
\end{pgfscope}%
\begin{pgfscope}%
\definecolor{textcolor}{rgb}{0.000000,0.000000,0.000000}%
\pgfsetstrokecolor{textcolor}%
\pgfsetfillcolor{textcolor}%
\pgftext[x=1.975187in,y=0.351264in,,top]{\color{textcolor}{\rmfamily\fontsize{10.000000}{12.000000}\selectfont\catcode`\^=\active\def^{\ifmmode\sp\else\^{}\fi}\catcode`\%=\active\def%{\%}$\mathdefault{4}$}}%
\end{pgfscope}%
\begin{pgfscope}%
\pgfsetbuttcap%
\pgfsetroundjoin%
\definecolor{currentfill}{rgb}{0.000000,0.000000,0.000000}%
\pgfsetfillcolor{currentfill}%
\pgfsetlinewidth{0.803000pt}%
\definecolor{currentstroke}{rgb}{0.000000,0.000000,0.000000}%
\pgfsetstrokecolor{currentstroke}%
\pgfsetdash{}{0pt}%
\pgfsys@defobject{currentmarker}{\pgfqpoint{0.000000in}{-0.048611in}}{\pgfqpoint{0.000000in}{0.000000in}}{%
\pgfpathmoveto{\pgfqpoint{0.000000in}{0.000000in}}%
\pgfpathlineto{\pgfqpoint{0.000000in}{-0.048611in}}%
\pgfusepath{stroke,fill}%
}%
\begin{pgfscope}%
\pgfsys@transformshift{2.480623in}{0.448486in}%
\pgfsys@useobject{currentmarker}{}%
\end{pgfscope}%
\end{pgfscope}%
\begin{pgfscope}%
\definecolor{textcolor}{rgb}{0.000000,0.000000,0.000000}%
\pgfsetstrokecolor{textcolor}%
\pgfsetfillcolor{textcolor}%
\pgftext[x=2.480623in,y=0.351264in,,top]{\color{textcolor}{\rmfamily\fontsize{10.000000}{12.000000}\selectfont\catcode`\^=\active\def^{\ifmmode\sp\else\^{}\fi}\catcode`\%=\active\def%{\%}$\mathdefault{6}$}}%
\end{pgfscope}%
\begin{pgfscope}%
\pgfsetbuttcap%
\pgfsetroundjoin%
\definecolor{currentfill}{rgb}{0.000000,0.000000,0.000000}%
\pgfsetfillcolor{currentfill}%
\pgfsetlinewidth{0.803000pt}%
\definecolor{currentstroke}{rgb}{0.000000,0.000000,0.000000}%
\pgfsetstrokecolor{currentstroke}%
\pgfsetdash{}{0pt}%
\pgfsys@defobject{currentmarker}{\pgfqpoint{0.000000in}{-0.048611in}}{\pgfqpoint{0.000000in}{0.000000in}}{%
\pgfpathmoveto{\pgfqpoint{0.000000in}{0.000000in}}%
\pgfpathlineto{\pgfqpoint{0.000000in}{-0.048611in}}%
\pgfusepath{stroke,fill}%
}%
\begin{pgfscope}%
\pgfsys@transformshift{2.986059in}{0.448486in}%
\pgfsys@useobject{currentmarker}{}%
\end{pgfscope}%
\end{pgfscope}%
\begin{pgfscope}%
\definecolor{textcolor}{rgb}{0.000000,0.000000,0.000000}%
\pgfsetstrokecolor{textcolor}%
\pgfsetfillcolor{textcolor}%
\pgftext[x=2.986059in,y=0.351264in,,top]{\color{textcolor}{\rmfamily\fontsize{10.000000}{12.000000}\selectfont\catcode`\^=\active\def^{\ifmmode\sp\else\^{}\fi}\catcode`\%=\active\def%{\%}$\mathdefault{8}$}}%
\end{pgfscope}%
\begin{pgfscope}%
\pgfsetbuttcap%
\pgfsetroundjoin%
\definecolor{currentfill}{rgb}{0.000000,0.000000,0.000000}%
\pgfsetfillcolor{currentfill}%
\pgfsetlinewidth{0.803000pt}%
\definecolor{currentstroke}{rgb}{0.000000,0.000000,0.000000}%
\pgfsetstrokecolor{currentstroke}%
\pgfsetdash{}{0pt}%
\pgfsys@defobject{currentmarker}{\pgfqpoint{0.000000in}{-0.048611in}}{\pgfqpoint{0.000000in}{0.000000in}}{%
\pgfpathmoveto{\pgfqpoint{0.000000in}{0.000000in}}%
\pgfpathlineto{\pgfqpoint{0.000000in}{-0.048611in}}%
\pgfusepath{stroke,fill}%
}%
\begin{pgfscope}%
\pgfsys@transformshift{3.491495in}{0.448486in}%
\pgfsys@useobject{currentmarker}{}%
\end{pgfscope}%
\end{pgfscope}%
\begin{pgfscope}%
\definecolor{textcolor}{rgb}{0.000000,0.000000,0.000000}%
\pgfsetstrokecolor{textcolor}%
\pgfsetfillcolor{textcolor}%
\pgftext[x=3.491495in,y=0.351264in,,top]{\color{textcolor}{\rmfamily\fontsize{10.000000}{12.000000}\selectfont\catcode`\^=\active\def^{\ifmmode\sp\else\^{}\fi}\catcode`\%=\active\def%{\%}$\mathdefault{10}$}}%
\end{pgfscope}%
\begin{pgfscope}%
\pgfsetbuttcap%
\pgfsetroundjoin%
\definecolor{currentfill}{rgb}{0.000000,0.000000,0.000000}%
\pgfsetfillcolor{currentfill}%
\pgfsetlinewidth{0.803000pt}%
\definecolor{currentstroke}{rgb}{0.000000,0.000000,0.000000}%
\pgfsetstrokecolor{currentstroke}%
\pgfsetdash{}{0pt}%
\pgfsys@defobject{currentmarker}{\pgfqpoint{0.000000in}{-0.048611in}}{\pgfqpoint{0.000000in}{0.000000in}}{%
\pgfpathmoveto{\pgfqpoint{0.000000in}{0.000000in}}%
\pgfpathlineto{\pgfqpoint{0.000000in}{-0.048611in}}%
\pgfusepath{stroke,fill}%
}%
\begin{pgfscope}%
\pgfsys@transformshift{3.996931in}{0.448486in}%
\pgfsys@useobject{currentmarker}{}%
\end{pgfscope}%
\end{pgfscope}%
\begin{pgfscope}%
\definecolor{textcolor}{rgb}{0.000000,0.000000,0.000000}%
\pgfsetstrokecolor{textcolor}%
\pgfsetfillcolor{textcolor}%
\pgftext[x=3.996931in,y=0.351264in,,top]{\color{textcolor}{\rmfamily\fontsize{10.000000}{12.000000}\selectfont\catcode`\^=\active\def^{\ifmmode\sp\else\^{}\fi}\catcode`\%=\active\def%{\%}$\mathdefault{12}$}}%
\end{pgfscope}%
\begin{pgfscope}%
\definecolor{textcolor}{rgb}{0.000000,0.000000,0.000000}%
\pgfsetstrokecolor{textcolor}%
\pgfsetfillcolor{textcolor}%
\pgftext[x=2.565915in,y=0.161295in,,top]{\color{textcolor}{\rmfamily\fontsize{12.000000}{14.400000}\selectfont\catcode`\^=\active\def^{\ifmmode\sp\else\^{}\fi}\catcode`\%=\active\def%{\%}Birth}}%
\end{pgfscope}%
\begin{pgfscope}%
\pgfsetbuttcap%
\pgfsetroundjoin%
\definecolor{currentfill}{rgb}{0.000000,0.000000,0.000000}%
\pgfsetfillcolor{currentfill}%
\pgfsetlinewidth{0.803000pt}%
\definecolor{currentstroke}{rgb}{0.000000,0.000000,0.000000}%
\pgfsetstrokecolor{currentstroke}%
\pgfsetdash{}{0pt}%
\pgfsys@defobject{currentmarker}{\pgfqpoint{-0.048611in}{0.000000in}}{\pgfqpoint{-0.000000in}{0.000000in}}{%
\pgfpathmoveto{\pgfqpoint{-0.000000in}{0.000000in}}%
\pgfpathlineto{\pgfqpoint{-0.048611in}{0.000000in}}%
\pgfusepath{stroke,fill}%
}%
\begin{pgfscope}%
\pgfsys@transformshift{0.800049in}{0.612753in}%
\pgfsys@useobject{currentmarker}{}%
\end{pgfscope}%
\end{pgfscope}%
\begin{pgfscope}%
\definecolor{textcolor}{rgb}{0.000000,0.000000,0.000000}%
\pgfsetstrokecolor{textcolor}%
\pgfsetfillcolor{textcolor}%
\pgftext[x=0.305216in, y=0.559991in, left, base]{\color{textcolor}{\rmfamily\fontsize{10.000000}{12.000000}\selectfont\catcode`\^=\active\def^{\ifmmode\sp\else\^{}\fi}\catcode`\%=\active\def%{\%}0.000}}%
\end{pgfscope}%
\begin{pgfscope}%
\pgfsetbuttcap%
\pgfsetroundjoin%
\definecolor{currentfill}{rgb}{0.000000,0.000000,0.000000}%
\pgfsetfillcolor{currentfill}%
\pgfsetlinewidth{0.803000pt}%
\definecolor{currentstroke}{rgb}{0.000000,0.000000,0.000000}%
\pgfsetstrokecolor{currentstroke}%
\pgfsetdash{}{0pt}%
\pgfsys@defobject{currentmarker}{\pgfqpoint{-0.048611in}{0.000000in}}{\pgfqpoint{-0.000000in}{0.000000in}}{%
\pgfpathmoveto{\pgfqpoint{-0.000000in}{0.000000in}}%
\pgfpathlineto{\pgfqpoint{-0.048611in}{0.000000in}}%
\pgfusepath{stroke,fill}%
}%
\begin{pgfscope}%
\pgfsys@transformshift{0.800049in}{1.118189in}%
\pgfsys@useobject{currentmarker}{}%
\end{pgfscope}%
\end{pgfscope}%
\begin{pgfscope}%
\definecolor{textcolor}{rgb}{0.000000,0.000000,0.000000}%
\pgfsetstrokecolor{textcolor}%
\pgfsetfillcolor{textcolor}%
\pgftext[x=0.305216in, y=1.065427in, left, base]{\color{textcolor}{\rmfamily\fontsize{10.000000}{12.000000}\selectfont\catcode`\^=\active\def^{\ifmmode\sp\else\^{}\fi}\catcode`\%=\active\def%{\%}2.000}}%
\end{pgfscope}%
\begin{pgfscope}%
\pgfsetbuttcap%
\pgfsetroundjoin%
\definecolor{currentfill}{rgb}{0.000000,0.000000,0.000000}%
\pgfsetfillcolor{currentfill}%
\pgfsetlinewidth{0.803000pt}%
\definecolor{currentstroke}{rgb}{0.000000,0.000000,0.000000}%
\pgfsetstrokecolor{currentstroke}%
\pgfsetdash{}{0pt}%
\pgfsys@defobject{currentmarker}{\pgfqpoint{-0.048611in}{0.000000in}}{\pgfqpoint{-0.000000in}{0.000000in}}{%
\pgfpathmoveto{\pgfqpoint{-0.000000in}{0.000000in}}%
\pgfpathlineto{\pgfqpoint{-0.048611in}{0.000000in}}%
\pgfusepath{stroke,fill}%
}%
\begin{pgfscope}%
\pgfsys@transformshift{0.800049in}{1.623624in}%
\pgfsys@useobject{currentmarker}{}%
\end{pgfscope}%
\end{pgfscope}%
\begin{pgfscope}%
\definecolor{textcolor}{rgb}{0.000000,0.000000,0.000000}%
\pgfsetstrokecolor{textcolor}%
\pgfsetfillcolor{textcolor}%
\pgftext[x=0.305216in, y=1.570863in, left, base]{\color{textcolor}{\rmfamily\fontsize{10.000000}{12.000000}\selectfont\catcode`\^=\active\def^{\ifmmode\sp\else\^{}\fi}\catcode`\%=\active\def%{\%}4.000}}%
\end{pgfscope}%
\begin{pgfscope}%
\pgfsetbuttcap%
\pgfsetroundjoin%
\definecolor{currentfill}{rgb}{0.000000,0.000000,0.000000}%
\pgfsetfillcolor{currentfill}%
\pgfsetlinewidth{0.803000pt}%
\definecolor{currentstroke}{rgb}{0.000000,0.000000,0.000000}%
\pgfsetstrokecolor{currentstroke}%
\pgfsetdash{}{0pt}%
\pgfsys@defobject{currentmarker}{\pgfqpoint{-0.048611in}{0.000000in}}{\pgfqpoint{-0.000000in}{0.000000in}}{%
\pgfpathmoveto{\pgfqpoint{-0.000000in}{0.000000in}}%
\pgfpathlineto{\pgfqpoint{-0.048611in}{0.000000in}}%
\pgfusepath{stroke,fill}%
}%
\begin{pgfscope}%
\pgfsys@transformshift{0.800049in}{2.129060in}%
\pgfsys@useobject{currentmarker}{}%
\end{pgfscope}%
\end{pgfscope}%
\begin{pgfscope}%
\definecolor{textcolor}{rgb}{0.000000,0.000000,0.000000}%
\pgfsetstrokecolor{textcolor}%
\pgfsetfillcolor{textcolor}%
\pgftext[x=0.305216in, y=2.076299in, left, base]{\color{textcolor}{\rmfamily\fontsize{10.000000}{12.000000}\selectfont\catcode`\^=\active\def^{\ifmmode\sp\else\^{}\fi}\catcode`\%=\active\def%{\%}6.000}}%
\end{pgfscope}%
\begin{pgfscope}%
\pgfsetbuttcap%
\pgfsetroundjoin%
\definecolor{currentfill}{rgb}{0.000000,0.000000,0.000000}%
\pgfsetfillcolor{currentfill}%
\pgfsetlinewidth{0.803000pt}%
\definecolor{currentstroke}{rgb}{0.000000,0.000000,0.000000}%
\pgfsetstrokecolor{currentstroke}%
\pgfsetdash{}{0pt}%
\pgfsys@defobject{currentmarker}{\pgfqpoint{-0.048611in}{0.000000in}}{\pgfqpoint{-0.000000in}{0.000000in}}{%
\pgfpathmoveto{\pgfqpoint{-0.000000in}{0.000000in}}%
\pgfpathlineto{\pgfqpoint{-0.048611in}{0.000000in}}%
\pgfusepath{stroke,fill}%
}%
\begin{pgfscope}%
\pgfsys@transformshift{0.800049in}{2.634496in}%
\pgfsys@useobject{currentmarker}{}%
\end{pgfscope}%
\end{pgfscope}%
\begin{pgfscope}%
\definecolor{textcolor}{rgb}{0.000000,0.000000,0.000000}%
\pgfsetstrokecolor{textcolor}%
\pgfsetfillcolor{textcolor}%
\pgftext[x=0.305216in, y=2.581735in, left, base]{\color{textcolor}{\rmfamily\fontsize{10.000000}{12.000000}\selectfont\catcode`\^=\active\def^{\ifmmode\sp\else\^{}\fi}\catcode`\%=\active\def%{\%}8.000}}%
\end{pgfscope}%
\begin{pgfscope}%
\pgfsetbuttcap%
\pgfsetroundjoin%
\definecolor{currentfill}{rgb}{0.000000,0.000000,0.000000}%
\pgfsetfillcolor{currentfill}%
\pgfsetlinewidth{0.803000pt}%
\definecolor{currentstroke}{rgb}{0.000000,0.000000,0.000000}%
\pgfsetstrokecolor{currentstroke}%
\pgfsetdash{}{0pt}%
\pgfsys@defobject{currentmarker}{\pgfqpoint{-0.048611in}{0.000000in}}{\pgfqpoint{-0.000000in}{0.000000in}}{%
\pgfpathmoveto{\pgfqpoint{-0.000000in}{0.000000in}}%
\pgfpathlineto{\pgfqpoint{-0.048611in}{0.000000in}}%
\pgfusepath{stroke,fill}%
}%
\begin{pgfscope}%
\pgfsys@transformshift{0.800049in}{3.139932in}%
\pgfsys@useobject{currentmarker}{}%
\end{pgfscope}%
\end{pgfscope}%
\begin{pgfscope}%
\definecolor{textcolor}{rgb}{0.000000,0.000000,0.000000}%
\pgfsetstrokecolor{textcolor}%
\pgfsetfillcolor{textcolor}%
\pgftext[x=0.216851in, y=3.087171in, left, base]{\color{textcolor}{\rmfamily\fontsize{10.000000}{12.000000}\selectfont\catcode`\^=\active\def^{\ifmmode\sp\else\^{}\fi}\catcode`\%=\active\def%{\%}10.000}}%
\end{pgfscope}%
\begin{pgfscope}%
\pgfsetbuttcap%
\pgfsetroundjoin%
\definecolor{currentfill}{rgb}{0.000000,0.000000,0.000000}%
\pgfsetfillcolor{currentfill}%
\pgfsetlinewidth{0.803000pt}%
\definecolor{currentstroke}{rgb}{0.000000,0.000000,0.000000}%
\pgfsetstrokecolor{currentstroke}%
\pgfsetdash{}{0pt}%
\pgfsys@defobject{currentmarker}{\pgfqpoint{-0.048611in}{0.000000in}}{\pgfqpoint{-0.000000in}{0.000000in}}{%
\pgfpathmoveto{\pgfqpoint{-0.000000in}{0.000000in}}%
\pgfpathlineto{\pgfqpoint{-0.048611in}{0.000000in}}%
\pgfusepath{stroke,fill}%
}%
\begin{pgfscope}%
\pgfsys@transformshift{0.800049in}{3.645368in}%
\pgfsys@useobject{currentmarker}{}%
\end{pgfscope}%
\end{pgfscope}%
\begin{pgfscope}%
\definecolor{textcolor}{rgb}{0.000000,0.000000,0.000000}%
\pgfsetstrokecolor{textcolor}%
\pgfsetfillcolor{textcolor}%
\pgftext[x=0.216851in, y=3.592606in, left, base]{\color{textcolor}{\rmfamily\fontsize{10.000000}{12.000000}\selectfont\catcode`\^=\active\def^{\ifmmode\sp\else\^{}\fi}\catcode`\%=\active\def%{\%}12.000}}%
\end{pgfscope}%
\begin{pgfscope}%
\pgfsetbuttcap%
\pgfsetroundjoin%
\definecolor{currentfill}{rgb}{0.000000,0.000000,0.000000}%
\pgfsetfillcolor{currentfill}%
\pgfsetlinewidth{0.803000pt}%
\definecolor{currentstroke}{rgb}{0.000000,0.000000,0.000000}%
\pgfsetstrokecolor{currentstroke}%
\pgfsetdash{}{0pt}%
\pgfsys@defobject{currentmarker}{\pgfqpoint{-0.048611in}{0.000000in}}{\pgfqpoint{-0.000000in}{0.000000in}}{%
\pgfpathmoveto{\pgfqpoint{-0.000000in}{0.000000in}}%
\pgfpathlineto{\pgfqpoint{-0.048611in}{0.000000in}}%
\pgfusepath{stroke,fill}%
}%
\begin{pgfscope}%
\pgfsys@transformshift{0.800049in}{4.062353in}%
\pgfsys@useobject{currentmarker}{}%
\end{pgfscope}%
\end{pgfscope}%
\begin{pgfscope}%
\definecolor{textcolor}{rgb}{0.000000,0.000000,0.000000}%
\pgfsetstrokecolor{textcolor}%
\pgfsetfillcolor{textcolor}%
\pgftext[x=0.455912in, y=4.009591in, left, base]{\color{textcolor}{\rmfamily\fontsize{10.000000}{12.000000}\selectfont\catcode`\^=\active\def^{\ifmmode\sp\else\^{}\fi}\catcode`\%=\active\def%{\%}$+\infty$}}%
\end{pgfscope}%
\begin{pgfscope}%
\definecolor{textcolor}{rgb}{0.000000,0.000000,0.000000}%
\pgfsetstrokecolor{textcolor}%
\pgfsetfillcolor{textcolor}%
\pgftext[x=0.161295in,y=2.296486in,,bottom,rotate=90.000000]{\color{textcolor}{\rmfamily\fontsize{12.000000}{14.400000}\selectfont\catcode`\^=\active\def^{\ifmmode\sp\else\^{}\fi}\catcode`\%=\active\def%{\%}Death}}%
\end{pgfscope}%
\begin{pgfscope}%
\pgfpathrectangle{\pgfqpoint{0.800049in}{0.448486in}}{\pgfqpoint{3.531733in}{3.696000in}}%
\pgfusepath{clip}%
\pgfsetrectcap%
\pgfsetroundjoin%
\pgfsetlinewidth{1.003750pt}%
\definecolor{currentstroke}{rgb}{0.000000,0.000000,0.000000}%
\pgfsetstrokecolor{currentstroke}%
\pgfsetdash{}{0pt}%
\pgfpathmoveto{\pgfqpoint{0.800049in}{0.448486in}}%
\pgfpathlineto{\pgfqpoint{4.331782in}{3.980219in}}%
\pgfusepath{stroke}%
\end{pgfscope}%
\begin{pgfscope}%
\pgfpathrectangle{\pgfqpoint{0.800049in}{0.448486in}}{\pgfqpoint{3.531733in}{3.696000in}}%
\pgfusepath{clip}%
\pgfsetrectcap%
\pgfsetroundjoin%
\pgfsetlinewidth{1.003750pt}%
\definecolor{currentstroke}{rgb}{0.000000,0.000000,0.000000}%
\pgfsetstrokecolor{currentstroke}%
\pgfsetstrokeopacity{0.600000}%
\pgfsetdash{}{0pt}%
\pgfpathmoveto{\pgfqpoint{0.800049in}{4.062353in}}%
\pgfpathlineto{\pgfqpoint{4.331782in}{4.062353in}}%
\pgfusepath{stroke}%
\end{pgfscope}%
\begin{pgfscope}%
\pgfsetrectcap%
\pgfsetmiterjoin%
\pgfsetlinewidth{0.803000pt}%
\definecolor{currentstroke}{rgb}{0.000000,0.000000,0.000000}%
\pgfsetstrokecolor{currentstroke}%
\pgfsetdash{}{0pt}%
\pgfpathmoveto{\pgfqpoint{0.800049in}{0.448486in}}%
\pgfpathlineto{\pgfqpoint{0.800049in}{4.144486in}}%
\pgfusepath{stroke}%
\end{pgfscope}%
\begin{pgfscope}%
\pgfsetrectcap%
\pgfsetmiterjoin%
\pgfsetlinewidth{0.803000pt}%
\definecolor{currentstroke}{rgb}{0.000000,0.000000,0.000000}%
\pgfsetstrokecolor{currentstroke}%
\pgfsetdash{}{0pt}%
\pgfpathmoveto{\pgfqpoint{4.331782in}{0.448486in}}%
\pgfpathlineto{\pgfqpoint{4.331782in}{4.144486in}}%
\pgfusepath{stroke}%
\end{pgfscope}%
\begin{pgfscope}%
\pgfsetrectcap%
\pgfsetmiterjoin%
\pgfsetlinewidth{0.803000pt}%
\definecolor{currentstroke}{rgb}{0.000000,0.000000,0.000000}%
\pgfsetstrokecolor{currentstroke}%
\pgfsetdash{}{0pt}%
\pgfpathmoveto{\pgfqpoint{0.800049in}{0.448486in}}%
\pgfpathlineto{\pgfqpoint{4.331782in}{0.448486in}}%
\pgfusepath{stroke}%
\end{pgfscope}%
\begin{pgfscope}%
\pgfsetrectcap%
\pgfsetmiterjoin%
\pgfsetlinewidth{0.803000pt}%
\definecolor{currentstroke}{rgb}{0.000000,0.000000,0.000000}%
\pgfsetstrokecolor{currentstroke}%
\pgfsetdash{}{0pt}%
\pgfpathmoveto{\pgfqpoint{0.800049in}{4.144486in}}%
\pgfpathlineto{\pgfqpoint{4.331782in}{4.144486in}}%
\pgfusepath{stroke}%
\end{pgfscope}%
\end{pgfpicture}%
\makeatother%
\endgroup%

        }
        \caption{Without circumcircle filtering}
        \label{fig:manyholes_ph_no_cc}
    \end{subfigure}
    \begin{subfigure}{0.49\textwidth}
        \resizebox{\textwidth}{!}{
            \input{plots/manyholes_ph_cc.pgf}
        }
        \caption{With circumcircle filtering}
        \label{fig:manyholes_ph_cc}
    \end{subfigure}
    \caption{Persistence diagrams of the LVR complex on the manyholes dataset with and without circumcircle filtering.}
    \label{fig:manyholes_ph}
\end{figure}

\section{Dowker complex}
\label{sec:dowker}

Alternatively, the \emph{Dowker complex} can be used to combat the same drawback.
However, computing it na\"ively up to 3-simplices requires \(\mathcal{O}(n^3 \cdot d)\)
runtime and \(\mathcal{O}(n^3)\) memory for \(n\) points in \(\R^d\),
which is prohibitively expensive for the image datasets we consider.
This approach may be more tractable within a more sample-efficient framework,
such as one proposed in~\cite{li2020finding}.
Additionally, the Dowker complex is not applicable to the multiclass setting.

In Figure~\ref{fig:dowker_complex}, we show the Dowker complex on the nested and manyholes datasets.
Due to the computational cost of constructing the Dowker complex, we only use 300 points from each dataset.
We can see that the problem with simplices crossing the decision boundary multiple times is present in the Dowker complex as well.
A visualisation of how the Dowker complex changes as the filtration parameter is varied
can be found in Appendix~\ref{app:dowker}.
The persistence diagrams shown in Figure~\ref{fig:dowker_ph} similarly demonstrate that the Dowker complex does not capture the topology of the decision boundary well.
% \lipsum[1]

\begin{figure}
    \centering
    \begin{subfigure}{0.49\textwidth}
        \includegraphics[width=\textwidth]{plots/dowker_nested_complex.jpg}
        \caption{Nested dataset}
        \label{fig:dowker_manyholes_complex_no_cc}
    \end{subfigure}
    \begin{subfigure}{0.49\textwidth}
        \includegraphics[width=\textwidth]{plots/dowker_manyholes_complex.jpg}
        \caption{Manyholes dataset}
        \label{fig:dowker_manyholes_complex_cc}
    \end{subfigure}
    \caption{A visualisation of the 2-skeleton of the Dowker complex on the nested and manyholes datasets.}
    \label{fig:dowker_complex}
\end{figure}

\begin{figure}
    \centering
    \begin{subfigure}{0.49\textwidth}
        \resizebox{\textwidth}{!}{
            %% Creator: Matplotlib, PGF backend
%%
%% To include the figure in your LaTeX document, write
%%   \input{<filename>.pgf}
%%
%% Make sure the required packages are loaded in your preamble
%%   \usepackage{pgf}
%%
%% Also ensure that all the required font packages are loaded; for instance,
%% the lmodern package is sometimes necessary when using math font.
%%   \usepackage{lmodern}
%%
%% Figures using additional raster images can only be included by \input if
%% they are in the same directory as the main LaTeX file. For loading figures
%% from other directories you can use the `import` package
%%   \usepackage{import}
%%
%% and then include the figures with
%%   \import{<path to file>}{<filename>.pgf}
%%
%% Matplotlib used the following preamble
%%   \def\mathdefault#1{#1}
%%   \everymath=\expandafter{\the\everymath\displaystyle}
%%   
%%   \ifdefined\pdftexversion\else  % non-pdftex case.
%%     \usepackage{fontspec}
%%     \setmainfont{DejaVuSerif.ttf}[Path=\detokenize{/home/snek/repos/homology-decision-bondaries-clean/venv/lib/python3.9/site-packages/matplotlib/mpl-data/fonts/ttf/}]
%%     \setsansfont{DejaVuSans.ttf}[Path=\detokenize{/home/snek/repos/homology-decision-bondaries-clean/venv/lib/python3.9/site-packages/matplotlib/mpl-data/fonts/ttf/}]
%%     \setmonofont{DejaVuSansMono.ttf}[Path=\detokenize{/home/snek/repos/homology-decision-bondaries-clean/venv/lib/python3.9/site-packages/matplotlib/mpl-data/fonts/ttf/}]
%%   \fi
%%   \makeatletter\@ifpackageloaded{underscore}{}{\usepackage[strings]{underscore}}\makeatother
%%
\begingroup%
\makeatletter%
\begin{pgfpicture}%
\pgfpathrectangle{\pgfpointorigin}{\pgfqpoint{6.400000in}{4.800000in}}%
\pgfusepath{use as bounding box, clip}%
\begin{pgfscope}%
\pgfsetbuttcap%
\pgfsetmiterjoin%
\definecolor{currentfill}{rgb}{1.000000,1.000000,1.000000}%
\pgfsetfillcolor{currentfill}%
\pgfsetlinewidth{0.000000pt}%
\definecolor{currentstroke}{rgb}{1.000000,1.000000,1.000000}%
\pgfsetstrokecolor{currentstroke}%
\pgfsetdash{}{0pt}%
\pgfpathmoveto{\pgfqpoint{0.000000in}{0.000000in}}%
\pgfpathlineto{\pgfqpoint{6.400000in}{0.000000in}}%
\pgfpathlineto{\pgfqpoint{6.400000in}{4.800000in}}%
\pgfpathlineto{\pgfqpoint{0.000000in}{4.800000in}}%
\pgfpathlineto{\pgfqpoint{0.000000in}{0.000000in}}%
\pgfpathclose%
\pgfusepath{fill}%
\end{pgfscope}%
\begin{pgfscope}%
\pgfsetbuttcap%
\pgfsetmiterjoin%
\definecolor{currentfill}{rgb}{1.000000,1.000000,1.000000}%
\pgfsetfillcolor{currentfill}%
\pgfsetlinewidth{0.000000pt}%
\definecolor{currentstroke}{rgb}{0.000000,0.000000,0.000000}%
\pgfsetstrokecolor{currentstroke}%
\pgfsetstrokeopacity{0.000000}%
\pgfsetdash{}{0pt}%
\pgfpathmoveto{\pgfqpoint{0.800000in}{0.528000in}}%
\pgfpathlineto{\pgfqpoint{5.760000in}{0.528000in}}%
\pgfpathlineto{\pgfqpoint{5.760000in}{4.224000in}}%
\pgfpathlineto{\pgfqpoint{0.800000in}{4.224000in}}%
\pgfpathlineto{\pgfqpoint{0.800000in}{0.528000in}}%
\pgfpathclose%
\pgfusepath{fill}%
\end{pgfscope}%
\begin{pgfscope}%
\pgfpathrectangle{\pgfqpoint{0.800000in}{0.528000in}}{\pgfqpoint{4.960000in}{3.696000in}}%
\pgfusepath{clip}%
\pgfsetbuttcap%
\pgfsetmiterjoin%
\definecolor{currentfill}{rgb}{0.827451,0.827451,0.827451}%
\pgfsetfillcolor{currentfill}%
\pgfsetlinewidth{1.003750pt}%
\definecolor{currentstroke}{rgb}{0.827451,0.827451,0.827451}%
\pgfsetstrokecolor{currentstroke}%
\pgfsetdash{}{0pt}%
\pgfpathmoveto{\pgfqpoint{0.800000in}{0.528000in}}%
\pgfpathlineto{\pgfqpoint{5.760000in}{0.528000in}}%
\pgfpathlineto{\pgfqpoint{5.760000in}{3.928320in}}%
\pgfpathlineto{\pgfqpoint{0.800000in}{0.528000in}}%
\pgfpathclose%
\pgfusepath{stroke,fill}%
\end{pgfscope}%
\begin{pgfscope}%
\pgfpathrectangle{\pgfqpoint{0.800000in}{0.528000in}}{\pgfqpoint{4.960000in}{3.696000in}}%
\pgfusepath{clip}%
\pgfsetbuttcap%
\pgfsetroundjoin%
\definecolor{currentfill}{rgb}{0.215686,0.494118,0.721569}%
\pgfsetfillcolor{currentfill}%
\pgfsetfillopacity{0.600000}%
\pgfsetlinewidth{1.003750pt}%
\definecolor{currentstroke}{rgb}{0.215686,0.494118,0.721569}%
\pgfsetstrokecolor{currentstroke}%
\pgfsetstrokeopacity{0.600000}%
\pgfsetdash{}{0pt}%
\pgfpathmoveto{\pgfqpoint{3.846861in}{3.738813in}}%
\pgfpathcurveto{\pgfqpoint{3.857911in}{3.738813in}}{\pgfqpoint{3.868510in}{3.743204in}}{\pgfqpoint{3.876323in}{3.751017in}}%
\pgfpathcurveto{\pgfqpoint{3.884137in}{3.758831in}}{\pgfqpoint{3.888527in}{3.769430in}}{\pgfqpoint{3.888527in}{3.780480in}}%
\pgfpathcurveto{\pgfqpoint{3.888527in}{3.791530in}}{\pgfqpoint{3.884137in}{3.802129in}}{\pgfqpoint{3.876323in}{3.809943in}}%
\pgfpathcurveto{\pgfqpoint{3.868510in}{3.817756in}}{\pgfqpoint{3.857911in}{3.822147in}}{\pgfqpoint{3.846861in}{3.822147in}}%
\pgfpathcurveto{\pgfqpoint{3.835810in}{3.822147in}}{\pgfqpoint{3.825211in}{3.817756in}}{\pgfqpoint{3.817398in}{3.809943in}}%
\pgfpathcurveto{\pgfqpoint{3.809584in}{3.802129in}}{\pgfqpoint{3.805194in}{3.791530in}}{\pgfqpoint{3.805194in}{3.780480in}}%
\pgfpathcurveto{\pgfqpoint{3.805194in}{3.769430in}}{\pgfqpoint{3.809584in}{3.758831in}}{\pgfqpoint{3.817398in}{3.751017in}}%
\pgfpathcurveto{\pgfqpoint{3.825211in}{3.743204in}}{\pgfqpoint{3.835810in}{3.738813in}}{\pgfqpoint{3.846861in}{3.738813in}}%
\pgfpathlineto{\pgfqpoint{3.846861in}{3.738813in}}%
\pgfpathclose%
\pgfusepath{stroke,fill}%
\end{pgfscope}%
\begin{pgfscope}%
\pgfpathrectangle{\pgfqpoint{0.800000in}{0.528000in}}{\pgfqpoint{4.960000in}{3.696000in}}%
\pgfusepath{clip}%
\pgfsetbuttcap%
\pgfsetroundjoin%
\definecolor{currentfill}{rgb}{0.215686,0.494118,0.721569}%
\pgfsetfillcolor{currentfill}%
\pgfsetfillopacity{0.600000}%
\pgfsetlinewidth{1.003750pt}%
\definecolor{currentstroke}{rgb}{0.215686,0.494118,0.721569}%
\pgfsetstrokecolor{currentstroke}%
\pgfsetstrokeopacity{0.600000}%
\pgfsetdash{}{0pt}%
\pgfpathmoveto{\pgfqpoint{3.628879in}{3.576425in}}%
\pgfpathcurveto{\pgfqpoint{3.639929in}{3.576425in}}{\pgfqpoint{3.650528in}{3.580815in}}{\pgfqpoint{3.658341in}{3.588629in}}%
\pgfpathcurveto{\pgfqpoint{3.666155in}{3.596443in}}{\pgfqpoint{3.670545in}{3.607042in}}{\pgfqpoint{3.670545in}{3.618092in}}%
\pgfpathcurveto{\pgfqpoint{3.670545in}{3.629142in}}{\pgfqpoint{3.666155in}{3.639741in}}{\pgfqpoint{3.658341in}{3.647555in}}%
\pgfpathcurveto{\pgfqpoint{3.650528in}{3.655368in}}{\pgfqpoint{3.639929in}{3.659759in}}{\pgfqpoint{3.628879in}{3.659759in}}%
\pgfpathcurveto{\pgfqpoint{3.617828in}{3.659759in}}{\pgfqpoint{3.607229in}{3.655368in}}{\pgfqpoint{3.599416in}{3.647555in}}%
\pgfpathcurveto{\pgfqpoint{3.591602in}{3.639741in}}{\pgfqpoint{3.587212in}{3.629142in}}{\pgfqpoint{3.587212in}{3.618092in}}%
\pgfpathcurveto{\pgfqpoint{3.587212in}{3.607042in}}{\pgfqpoint{3.591602in}{3.596443in}}{\pgfqpoint{3.599416in}{3.588629in}}%
\pgfpathcurveto{\pgfqpoint{3.607229in}{3.580815in}}{\pgfqpoint{3.617828in}{3.576425in}}{\pgfqpoint{3.628879in}{3.576425in}}%
\pgfpathlineto{\pgfqpoint{3.628879in}{3.576425in}}%
\pgfpathclose%
\pgfusepath{stroke,fill}%
\end{pgfscope}%
\begin{pgfscope}%
\pgfpathrectangle{\pgfqpoint{0.800000in}{0.528000in}}{\pgfqpoint{4.960000in}{3.696000in}}%
\pgfusepath{clip}%
\pgfsetbuttcap%
\pgfsetroundjoin%
\definecolor{currentfill}{rgb}{0.215686,0.494118,0.721569}%
\pgfsetfillcolor{currentfill}%
\pgfsetfillopacity{0.600000}%
\pgfsetlinewidth{1.003750pt}%
\definecolor{currentstroke}{rgb}{0.215686,0.494118,0.721569}%
\pgfsetstrokecolor{currentstroke}%
\pgfsetstrokeopacity{0.600000}%
\pgfsetdash{}{0pt}%
\pgfpathmoveto{\pgfqpoint{3.467506in}{3.359086in}}%
\pgfpathcurveto{\pgfqpoint{3.478557in}{3.359086in}}{\pgfqpoint{3.489156in}{3.363476in}}{\pgfqpoint{3.496969in}{3.371290in}}%
\pgfpathcurveto{\pgfqpoint{3.504783in}{3.379104in}}{\pgfqpoint{3.509173in}{3.389703in}}{\pgfqpoint{3.509173in}{3.400753in}}%
\pgfpathcurveto{\pgfqpoint{3.509173in}{3.411803in}}{\pgfqpoint{3.504783in}{3.422402in}}{\pgfqpoint{3.496969in}{3.430215in}}%
\pgfpathcurveto{\pgfqpoint{3.489156in}{3.438029in}}{\pgfqpoint{3.478557in}{3.442419in}}{\pgfqpoint{3.467506in}{3.442419in}}%
\pgfpathcurveto{\pgfqpoint{3.456456in}{3.442419in}}{\pgfqpoint{3.445857in}{3.438029in}}{\pgfqpoint{3.438044in}{3.430215in}}%
\pgfpathcurveto{\pgfqpoint{3.430230in}{3.422402in}}{\pgfqpoint{3.425840in}{3.411803in}}{\pgfqpoint{3.425840in}{3.400753in}}%
\pgfpathcurveto{\pgfqpoint{3.425840in}{3.389703in}}{\pgfqpoint{3.430230in}{3.379104in}}{\pgfqpoint{3.438044in}{3.371290in}}%
\pgfpathcurveto{\pgfqpoint{3.445857in}{3.363476in}}{\pgfqpoint{3.456456in}{3.359086in}}{\pgfqpoint{3.467506in}{3.359086in}}%
\pgfpathlineto{\pgfqpoint{3.467506in}{3.359086in}}%
\pgfpathclose%
\pgfusepath{stroke,fill}%
\end{pgfscope}%
\begin{pgfscope}%
\pgfpathrectangle{\pgfqpoint{0.800000in}{0.528000in}}{\pgfqpoint{4.960000in}{3.696000in}}%
\pgfusepath{clip}%
\pgfsetbuttcap%
\pgfsetroundjoin%
\definecolor{currentfill}{rgb}{0.215686,0.494118,0.721569}%
\pgfsetfillcolor{currentfill}%
\pgfsetfillopacity{0.600000}%
\pgfsetlinewidth{1.003750pt}%
\definecolor{currentstroke}{rgb}{0.215686,0.494118,0.721569}%
\pgfsetstrokecolor{currentstroke}%
\pgfsetstrokeopacity{0.600000}%
\pgfsetdash{}{0pt}%
\pgfpathmoveto{\pgfqpoint{4.020357in}{3.702994in}}%
\pgfpathcurveto{\pgfqpoint{4.031407in}{3.702994in}}{\pgfqpoint{4.042006in}{3.707384in}}{\pgfqpoint{4.049819in}{3.715198in}}%
\pgfpathcurveto{\pgfqpoint{4.057633in}{3.723012in}}{\pgfqpoint{4.062023in}{3.733611in}}{\pgfqpoint{4.062023in}{3.744661in}}%
\pgfpathcurveto{\pgfqpoint{4.062023in}{3.755711in}}{\pgfqpoint{4.057633in}{3.766310in}}{\pgfqpoint{4.049819in}{3.774123in}}%
\pgfpathcurveto{\pgfqpoint{4.042006in}{3.781937in}}{\pgfqpoint{4.031407in}{3.786327in}}{\pgfqpoint{4.020357in}{3.786327in}}%
\pgfpathcurveto{\pgfqpoint{4.009306in}{3.786327in}}{\pgfqpoint{3.998707in}{3.781937in}}{\pgfqpoint{3.990894in}{3.774123in}}%
\pgfpathcurveto{\pgfqpoint{3.983080in}{3.766310in}}{\pgfqpoint{3.978690in}{3.755711in}}{\pgfqpoint{3.978690in}{3.744661in}}%
\pgfpathcurveto{\pgfqpoint{3.978690in}{3.733611in}}{\pgfqpoint{3.983080in}{3.723012in}}{\pgfqpoint{3.990894in}{3.715198in}}%
\pgfpathcurveto{\pgfqpoint{3.998707in}{3.707384in}}{\pgfqpoint{4.009306in}{3.702994in}}{\pgfqpoint{4.020357in}{3.702994in}}%
\pgfpathlineto{\pgfqpoint{4.020357in}{3.702994in}}%
\pgfpathclose%
\pgfusepath{stroke,fill}%
\end{pgfscope}%
\begin{pgfscope}%
\pgfpathrectangle{\pgfqpoint{0.800000in}{0.528000in}}{\pgfqpoint{4.960000in}{3.696000in}}%
\pgfusepath{clip}%
\pgfsetbuttcap%
\pgfsetroundjoin%
\definecolor{currentfill}{rgb}{0.215686,0.494118,0.721569}%
\pgfsetfillcolor{currentfill}%
\pgfsetfillopacity{0.600000}%
\pgfsetlinewidth{1.003750pt}%
\definecolor{currentstroke}{rgb}{0.215686,0.494118,0.721569}%
\pgfsetstrokecolor{currentstroke}%
\pgfsetstrokeopacity{0.600000}%
\pgfsetdash{}{0pt}%
\pgfpathmoveto{\pgfqpoint{4.235661in}{3.456930in}}%
\pgfpathcurveto{\pgfqpoint{4.246711in}{3.456930in}}{\pgfqpoint{4.257310in}{3.461321in}}{\pgfqpoint{4.265124in}{3.469134in}}%
\pgfpathcurveto{\pgfqpoint{4.272937in}{3.476948in}}{\pgfqpoint{4.277328in}{3.487547in}}{\pgfqpoint{4.277328in}{3.498597in}}%
\pgfpathcurveto{\pgfqpoint{4.277328in}{3.509647in}}{\pgfqpoint{4.272937in}{3.520246in}}{\pgfqpoint{4.265124in}{3.528060in}}%
\pgfpathcurveto{\pgfqpoint{4.257310in}{3.535873in}}{\pgfqpoint{4.246711in}{3.540264in}}{\pgfqpoint{4.235661in}{3.540264in}}%
\pgfpathcurveto{\pgfqpoint{4.224611in}{3.540264in}}{\pgfqpoint{4.214012in}{3.535873in}}{\pgfqpoint{4.206198in}{3.528060in}}%
\pgfpathcurveto{\pgfqpoint{4.198385in}{3.520246in}}{\pgfqpoint{4.193994in}{3.509647in}}{\pgfqpoint{4.193994in}{3.498597in}}%
\pgfpathcurveto{\pgfqpoint{4.193994in}{3.487547in}}{\pgfqpoint{4.198385in}{3.476948in}}{\pgfqpoint{4.206198in}{3.469134in}}%
\pgfpathcurveto{\pgfqpoint{4.214012in}{3.461321in}}{\pgfqpoint{4.224611in}{3.456930in}}{\pgfqpoint{4.235661in}{3.456930in}}%
\pgfpathlineto{\pgfqpoint{4.235661in}{3.456930in}}%
\pgfpathclose%
\pgfusepath{stroke,fill}%
\end{pgfscope}%
\begin{pgfscope}%
\pgfpathrectangle{\pgfqpoint{0.800000in}{0.528000in}}{\pgfqpoint{4.960000in}{3.696000in}}%
\pgfusepath{clip}%
\pgfsetbuttcap%
\pgfsetroundjoin%
\definecolor{currentfill}{rgb}{0.215686,0.494118,0.721569}%
\pgfsetfillcolor{currentfill}%
\pgfsetfillopacity{0.600000}%
\pgfsetlinewidth{1.003750pt}%
\definecolor{currentstroke}{rgb}{0.215686,0.494118,0.721569}%
\pgfsetstrokecolor{currentstroke}%
\pgfsetstrokeopacity{0.600000}%
\pgfsetdash{}{0pt}%
\pgfpathmoveto{\pgfqpoint{3.950050in}{3.210576in}}%
\pgfpathcurveto{\pgfqpoint{3.961100in}{3.210576in}}{\pgfqpoint{3.971699in}{3.214966in}}{\pgfqpoint{3.979512in}{3.222780in}}%
\pgfpathcurveto{\pgfqpoint{3.987326in}{3.230594in}}{\pgfqpoint{3.991716in}{3.241193in}}{\pgfqpoint{3.991716in}{3.252243in}}%
\pgfpathcurveto{\pgfqpoint{3.991716in}{3.263293in}}{\pgfqpoint{3.987326in}{3.273892in}}{\pgfqpoint{3.979512in}{3.281705in}}%
\pgfpathcurveto{\pgfqpoint{3.971699in}{3.289519in}}{\pgfqpoint{3.961100in}{3.293909in}}{\pgfqpoint{3.950050in}{3.293909in}}%
\pgfpathcurveto{\pgfqpoint{3.938999in}{3.293909in}}{\pgfqpoint{3.928400in}{3.289519in}}{\pgfqpoint{3.920587in}{3.281705in}}%
\pgfpathcurveto{\pgfqpoint{3.912773in}{3.273892in}}{\pgfqpoint{3.908383in}{3.263293in}}{\pgfqpoint{3.908383in}{3.252243in}}%
\pgfpathcurveto{\pgfqpoint{3.908383in}{3.241193in}}{\pgfqpoint{3.912773in}{3.230594in}}{\pgfqpoint{3.920587in}{3.222780in}}%
\pgfpathcurveto{\pgfqpoint{3.928400in}{3.214966in}}{\pgfqpoint{3.938999in}{3.210576in}}{\pgfqpoint{3.950050in}{3.210576in}}%
\pgfpathlineto{\pgfqpoint{3.950050in}{3.210576in}}%
\pgfpathclose%
\pgfusepath{stroke,fill}%
\end{pgfscope}%
\begin{pgfscope}%
\pgfpathrectangle{\pgfqpoint{0.800000in}{0.528000in}}{\pgfqpoint{4.960000in}{3.696000in}}%
\pgfusepath{clip}%
\pgfsetbuttcap%
\pgfsetroundjoin%
\definecolor{currentfill}{rgb}{0.215686,0.494118,0.721569}%
\pgfsetfillcolor{currentfill}%
\pgfsetfillopacity{0.600000}%
\pgfsetlinewidth{1.003750pt}%
\definecolor{currentstroke}{rgb}{0.215686,0.494118,0.721569}%
\pgfsetstrokecolor{currentstroke}%
\pgfsetstrokeopacity{0.600000}%
\pgfsetdash{}{0pt}%
\pgfpathmoveto{\pgfqpoint{3.716637in}{3.006895in}}%
\pgfpathcurveto{\pgfqpoint{3.727687in}{3.006895in}}{\pgfqpoint{3.738286in}{3.011285in}}{\pgfqpoint{3.746100in}{3.019099in}}%
\pgfpathcurveto{\pgfqpoint{3.753913in}{3.026913in}}{\pgfqpoint{3.758303in}{3.037512in}}{\pgfqpoint{3.758303in}{3.048562in}}%
\pgfpathcurveto{\pgfqpoint{3.758303in}{3.059612in}}{\pgfqpoint{3.753913in}{3.070211in}}{\pgfqpoint{3.746100in}{3.078024in}}%
\pgfpathcurveto{\pgfqpoint{3.738286in}{3.085838in}}{\pgfqpoint{3.727687in}{3.090228in}}{\pgfqpoint{3.716637in}{3.090228in}}%
\pgfpathcurveto{\pgfqpoint{3.705587in}{3.090228in}}{\pgfqpoint{3.694988in}{3.085838in}}{\pgfqpoint{3.687174in}{3.078024in}}%
\pgfpathcurveto{\pgfqpoint{3.679360in}{3.070211in}}{\pgfqpoint{3.674970in}{3.059612in}}{\pgfqpoint{3.674970in}{3.048562in}}%
\pgfpathcurveto{\pgfqpoint{3.674970in}{3.037512in}}{\pgfqpoint{3.679360in}{3.026913in}}{\pgfqpoint{3.687174in}{3.019099in}}%
\pgfpathcurveto{\pgfqpoint{3.694988in}{3.011285in}}{\pgfqpoint{3.705587in}{3.006895in}}{\pgfqpoint{3.716637in}{3.006895in}}%
\pgfpathlineto{\pgfqpoint{3.716637in}{3.006895in}}%
\pgfpathclose%
\pgfusepath{stroke,fill}%
\end{pgfscope}%
\begin{pgfscope}%
\pgfpathrectangle{\pgfqpoint{0.800000in}{0.528000in}}{\pgfqpoint{4.960000in}{3.696000in}}%
\pgfusepath{clip}%
\pgfsetbuttcap%
\pgfsetroundjoin%
\definecolor{currentfill}{rgb}{0.215686,0.494118,0.721569}%
\pgfsetfillcolor{currentfill}%
\pgfsetfillopacity{0.600000}%
\pgfsetlinewidth{1.003750pt}%
\definecolor{currentstroke}{rgb}{0.215686,0.494118,0.721569}%
\pgfsetstrokecolor{currentstroke}%
\pgfsetstrokeopacity{0.600000}%
\pgfsetdash{}{0pt}%
\pgfpathmoveto{\pgfqpoint{3.355908in}{2.679355in}}%
\pgfpathcurveto{\pgfqpoint{3.366958in}{2.679355in}}{\pgfqpoint{3.377557in}{2.683746in}}{\pgfqpoint{3.385371in}{2.691559in}}%
\pgfpathcurveto{\pgfqpoint{3.393184in}{2.699373in}}{\pgfqpoint{3.397575in}{2.709972in}}{\pgfqpoint{3.397575in}{2.721022in}}%
\pgfpathcurveto{\pgfqpoint{3.397575in}{2.732072in}}{\pgfqpoint{3.393184in}{2.742671in}}{\pgfqpoint{3.385371in}{2.750485in}}%
\pgfpathcurveto{\pgfqpoint{3.377557in}{2.758299in}}{\pgfqpoint{3.366958in}{2.762689in}}{\pgfqpoint{3.355908in}{2.762689in}}%
\pgfpathcurveto{\pgfqpoint{3.344858in}{2.762689in}}{\pgfqpoint{3.334259in}{2.758299in}}{\pgfqpoint{3.326445in}{2.750485in}}%
\pgfpathcurveto{\pgfqpoint{3.318631in}{2.742671in}}{\pgfqpoint{3.314241in}{2.732072in}}{\pgfqpoint{3.314241in}{2.721022in}}%
\pgfpathcurveto{\pgfqpoint{3.314241in}{2.709972in}}{\pgfqpoint{3.318631in}{2.699373in}}{\pgfqpoint{3.326445in}{2.691559in}}%
\pgfpathcurveto{\pgfqpoint{3.334259in}{2.683746in}}{\pgfqpoint{3.344858in}{2.679355in}}{\pgfqpoint{3.355908in}{2.679355in}}%
\pgfpathlineto{\pgfqpoint{3.355908in}{2.679355in}}%
\pgfpathclose%
\pgfusepath{stroke,fill}%
\end{pgfscope}%
\begin{pgfscope}%
\pgfpathrectangle{\pgfqpoint{0.800000in}{0.528000in}}{\pgfqpoint{4.960000in}{3.696000in}}%
\pgfusepath{clip}%
\pgfsetbuttcap%
\pgfsetroundjoin%
\definecolor{currentfill}{rgb}{0.215686,0.494118,0.721569}%
\pgfsetfillcolor{currentfill}%
\pgfsetfillopacity{0.600000}%
\pgfsetlinewidth{1.003750pt}%
\definecolor{currentstroke}{rgb}{0.215686,0.494118,0.721569}%
\pgfsetstrokecolor{currentstroke}%
\pgfsetstrokeopacity{0.600000}%
\pgfsetdash{}{0pt}%
\pgfpathmoveto{\pgfqpoint{3.263034in}{2.525378in}}%
\pgfpathcurveto{\pgfqpoint{3.274084in}{2.525378in}}{\pgfqpoint{3.284683in}{2.529768in}}{\pgfqpoint{3.292497in}{2.537582in}}%
\pgfpathcurveto{\pgfqpoint{3.300311in}{2.545395in}}{\pgfqpoint{3.304701in}{2.555994in}}{\pgfqpoint{3.304701in}{2.567045in}}%
\pgfpathcurveto{\pgfqpoint{3.304701in}{2.578095in}}{\pgfqpoint{3.300311in}{2.588694in}}{\pgfqpoint{3.292497in}{2.596507in}}%
\pgfpathcurveto{\pgfqpoint{3.284683in}{2.604321in}}{\pgfqpoint{3.274084in}{2.608711in}}{\pgfqpoint{3.263034in}{2.608711in}}%
\pgfpathcurveto{\pgfqpoint{3.251984in}{2.608711in}}{\pgfqpoint{3.241385in}{2.604321in}}{\pgfqpoint{3.233571in}{2.596507in}}%
\pgfpathcurveto{\pgfqpoint{3.225758in}{2.588694in}}{\pgfqpoint{3.221368in}{2.578095in}}{\pgfqpoint{3.221368in}{2.567045in}}%
\pgfpathcurveto{\pgfqpoint{3.221368in}{2.555994in}}{\pgfqpoint{3.225758in}{2.545395in}}{\pgfqpoint{3.233571in}{2.537582in}}%
\pgfpathcurveto{\pgfqpoint{3.241385in}{2.529768in}}{\pgfqpoint{3.251984in}{2.525378in}}{\pgfqpoint{3.263034in}{2.525378in}}%
\pgfpathlineto{\pgfqpoint{3.263034in}{2.525378in}}%
\pgfpathclose%
\pgfusepath{stroke,fill}%
\end{pgfscope}%
\begin{pgfscope}%
\pgfpathrectangle{\pgfqpoint{0.800000in}{0.528000in}}{\pgfqpoint{4.960000in}{3.696000in}}%
\pgfusepath{clip}%
\pgfsetbuttcap%
\pgfsetroundjoin%
\definecolor{currentfill}{rgb}{0.215686,0.494118,0.721569}%
\pgfsetfillcolor{currentfill}%
\pgfsetfillopacity{0.600000}%
\pgfsetlinewidth{1.003750pt}%
\definecolor{currentstroke}{rgb}{0.215686,0.494118,0.721569}%
\pgfsetstrokecolor{currentstroke}%
\pgfsetstrokeopacity{0.600000}%
\pgfsetdash{}{0pt}%
\pgfpathmoveto{\pgfqpoint{3.761576in}{2.837454in}}%
\pgfpathcurveto{\pgfqpoint{3.772626in}{2.837454in}}{\pgfqpoint{3.783225in}{2.841844in}}{\pgfqpoint{3.791039in}{2.849657in}}%
\pgfpathcurveto{\pgfqpoint{3.798852in}{2.857471in}}{\pgfqpoint{3.803243in}{2.868070in}}{\pgfqpoint{3.803243in}{2.879120in}}%
\pgfpathcurveto{\pgfqpoint{3.803243in}{2.890170in}}{\pgfqpoint{3.798852in}{2.900769in}}{\pgfqpoint{3.791039in}{2.908583in}}%
\pgfpathcurveto{\pgfqpoint{3.783225in}{2.916397in}}{\pgfqpoint{3.772626in}{2.920787in}}{\pgfqpoint{3.761576in}{2.920787in}}%
\pgfpathcurveto{\pgfqpoint{3.750526in}{2.920787in}}{\pgfqpoint{3.739927in}{2.916397in}}{\pgfqpoint{3.732113in}{2.908583in}}%
\pgfpathcurveto{\pgfqpoint{3.724300in}{2.900769in}}{\pgfqpoint{3.719909in}{2.890170in}}{\pgfqpoint{3.719909in}{2.879120in}}%
\pgfpathcurveto{\pgfqpoint{3.719909in}{2.868070in}}{\pgfqpoint{3.724300in}{2.857471in}}{\pgfqpoint{3.732113in}{2.849657in}}%
\pgfpathcurveto{\pgfqpoint{3.739927in}{2.841844in}}{\pgfqpoint{3.750526in}{2.837454in}}{\pgfqpoint{3.761576in}{2.837454in}}%
\pgfpathlineto{\pgfqpoint{3.761576in}{2.837454in}}%
\pgfpathclose%
\pgfusepath{stroke,fill}%
\end{pgfscope}%
\begin{pgfscope}%
\pgfpathrectangle{\pgfqpoint{0.800000in}{0.528000in}}{\pgfqpoint{4.960000in}{3.696000in}}%
\pgfusepath{clip}%
\pgfsetbuttcap%
\pgfsetroundjoin%
\definecolor{currentfill}{rgb}{0.215686,0.494118,0.721569}%
\pgfsetfillcolor{currentfill}%
\pgfsetfillopacity{0.600000}%
\pgfsetlinewidth{1.003750pt}%
\definecolor{currentstroke}{rgb}{0.215686,0.494118,0.721569}%
\pgfsetstrokecolor{currentstroke}%
\pgfsetstrokeopacity{0.600000}%
\pgfsetdash{}{0pt}%
\pgfpathmoveto{\pgfqpoint{3.877214in}{2.894625in}}%
\pgfpathcurveto{\pgfqpoint{3.888264in}{2.894625in}}{\pgfqpoint{3.898863in}{2.899016in}}{\pgfqpoint{3.906677in}{2.906829in}}%
\pgfpathcurveto{\pgfqpoint{3.914490in}{2.914643in}}{\pgfqpoint{3.918880in}{2.925242in}}{\pgfqpoint{3.918880in}{2.936292in}}%
\pgfpathcurveto{\pgfqpoint{3.918880in}{2.947342in}}{\pgfqpoint{3.914490in}{2.957941in}}{\pgfqpoint{3.906677in}{2.965755in}}%
\pgfpathcurveto{\pgfqpoint{3.898863in}{2.973569in}}{\pgfqpoint{3.888264in}{2.977959in}}{\pgfqpoint{3.877214in}{2.977959in}}%
\pgfpathcurveto{\pgfqpoint{3.866164in}{2.977959in}}{\pgfqpoint{3.855565in}{2.973569in}}{\pgfqpoint{3.847751in}{2.965755in}}%
\pgfpathcurveto{\pgfqpoint{3.839937in}{2.957941in}}{\pgfqpoint{3.835547in}{2.947342in}}{\pgfqpoint{3.835547in}{2.936292in}}%
\pgfpathcurveto{\pgfqpoint{3.835547in}{2.925242in}}{\pgfqpoint{3.839937in}{2.914643in}}{\pgfqpoint{3.847751in}{2.906829in}}%
\pgfpathcurveto{\pgfqpoint{3.855565in}{2.899016in}}{\pgfqpoint{3.866164in}{2.894625in}}{\pgfqpoint{3.877214in}{2.894625in}}%
\pgfpathlineto{\pgfqpoint{3.877214in}{2.894625in}}%
\pgfpathclose%
\pgfusepath{stroke,fill}%
\end{pgfscope}%
\begin{pgfscope}%
\pgfpathrectangle{\pgfqpoint{0.800000in}{0.528000in}}{\pgfqpoint{4.960000in}{3.696000in}}%
\pgfusepath{clip}%
\pgfsetbuttcap%
\pgfsetroundjoin%
\definecolor{currentfill}{rgb}{0.215686,0.494118,0.721569}%
\pgfsetfillcolor{currentfill}%
\pgfsetfillopacity{0.600000}%
\pgfsetlinewidth{1.003750pt}%
\definecolor{currentstroke}{rgb}{0.215686,0.494118,0.721569}%
\pgfsetstrokecolor{currentstroke}%
\pgfsetstrokeopacity{0.600000}%
\pgfsetdash{}{0pt}%
\pgfpathmoveto{\pgfqpoint{3.410140in}{2.538589in}}%
\pgfpathcurveto{\pgfqpoint{3.421191in}{2.538589in}}{\pgfqpoint{3.431790in}{2.542979in}}{\pgfqpoint{3.439603in}{2.550792in}}%
\pgfpathcurveto{\pgfqpoint{3.447417in}{2.558606in}}{\pgfqpoint{3.451807in}{2.569205in}}{\pgfqpoint{3.451807in}{2.580255in}}%
\pgfpathcurveto{\pgfqpoint{3.451807in}{2.591305in}}{\pgfqpoint{3.447417in}{2.601904in}}{\pgfqpoint{3.439603in}{2.609718in}}%
\pgfpathcurveto{\pgfqpoint{3.431790in}{2.617532in}}{\pgfqpoint{3.421191in}{2.621922in}}{\pgfqpoint{3.410140in}{2.621922in}}%
\pgfpathcurveto{\pgfqpoint{3.399090in}{2.621922in}}{\pgfqpoint{3.388491in}{2.617532in}}{\pgfqpoint{3.380678in}{2.609718in}}%
\pgfpathcurveto{\pgfqpoint{3.372864in}{2.601904in}}{\pgfqpoint{3.368474in}{2.591305in}}{\pgfqpoint{3.368474in}{2.580255in}}%
\pgfpathcurveto{\pgfqpoint{3.368474in}{2.569205in}}{\pgfqpoint{3.372864in}{2.558606in}}{\pgfqpoint{3.380678in}{2.550792in}}%
\pgfpathcurveto{\pgfqpoint{3.388491in}{2.542979in}}{\pgfqpoint{3.399090in}{2.538589in}}{\pgfqpoint{3.410140in}{2.538589in}}%
\pgfpathlineto{\pgfqpoint{3.410140in}{2.538589in}}%
\pgfpathclose%
\pgfusepath{stroke,fill}%
\end{pgfscope}%
\begin{pgfscope}%
\pgfpathrectangle{\pgfqpoint{0.800000in}{0.528000in}}{\pgfqpoint{4.960000in}{3.696000in}}%
\pgfusepath{clip}%
\pgfsetbuttcap%
\pgfsetroundjoin%
\definecolor{currentfill}{rgb}{0.215686,0.494118,0.721569}%
\pgfsetfillcolor{currentfill}%
\pgfsetfillopacity{0.600000}%
\pgfsetlinewidth{1.003750pt}%
\definecolor{currentstroke}{rgb}{0.215686,0.494118,0.721569}%
\pgfsetstrokecolor{currentstroke}%
\pgfsetstrokeopacity{0.600000}%
\pgfsetdash{}{0pt}%
\pgfpathmoveto{\pgfqpoint{3.453622in}{2.560226in}}%
\pgfpathcurveto{\pgfqpoint{3.464672in}{2.560226in}}{\pgfqpoint{3.475271in}{2.564616in}}{\pgfqpoint{3.483085in}{2.572430in}}%
\pgfpathcurveto{\pgfqpoint{3.490898in}{2.580243in}}{\pgfqpoint{3.495289in}{2.590842in}}{\pgfqpoint{3.495289in}{2.601892in}}%
\pgfpathcurveto{\pgfqpoint{3.495289in}{2.612942in}}{\pgfqpoint{3.490898in}{2.623541in}}{\pgfqpoint{3.483085in}{2.631355in}}%
\pgfpathcurveto{\pgfqpoint{3.475271in}{2.639169in}}{\pgfqpoint{3.464672in}{2.643559in}}{\pgfqpoint{3.453622in}{2.643559in}}%
\pgfpathcurveto{\pgfqpoint{3.442572in}{2.643559in}}{\pgfqpoint{3.431973in}{2.639169in}}{\pgfqpoint{3.424159in}{2.631355in}}%
\pgfpathcurveto{\pgfqpoint{3.416346in}{2.623541in}}{\pgfqpoint{3.411955in}{2.612942in}}{\pgfqpoint{3.411955in}{2.601892in}}%
\pgfpathcurveto{\pgfqpoint{3.411955in}{2.590842in}}{\pgfqpoint{3.416346in}{2.580243in}}{\pgfqpoint{3.424159in}{2.572430in}}%
\pgfpathcurveto{\pgfqpoint{3.431973in}{2.564616in}}{\pgfqpoint{3.442572in}{2.560226in}}{\pgfqpoint{3.453622in}{2.560226in}}%
\pgfpathlineto{\pgfqpoint{3.453622in}{2.560226in}}%
\pgfpathclose%
\pgfusepath{stroke,fill}%
\end{pgfscope}%
\begin{pgfscope}%
\pgfpathrectangle{\pgfqpoint{0.800000in}{0.528000in}}{\pgfqpoint{4.960000in}{3.696000in}}%
\pgfusepath{clip}%
\pgfsetbuttcap%
\pgfsetroundjoin%
\definecolor{currentfill}{rgb}{0.215686,0.494118,0.721569}%
\pgfsetfillcolor{currentfill}%
\pgfsetfillopacity{0.600000}%
\pgfsetlinewidth{1.003750pt}%
\definecolor{currentstroke}{rgb}{0.215686,0.494118,0.721569}%
\pgfsetstrokecolor{currentstroke}%
\pgfsetstrokeopacity{0.600000}%
\pgfsetdash{}{0pt}%
\pgfpathmoveto{\pgfqpoint{4.850783in}{3.504203in}}%
\pgfpathcurveto{\pgfqpoint{4.861834in}{3.504203in}}{\pgfqpoint{4.872433in}{3.508594in}}{\pgfqpoint{4.880246in}{3.516407in}}%
\pgfpathcurveto{\pgfqpoint{4.888060in}{3.524221in}}{\pgfqpoint{4.892450in}{3.534820in}}{\pgfqpoint{4.892450in}{3.545870in}}%
\pgfpathcurveto{\pgfqpoint{4.892450in}{3.556920in}}{\pgfqpoint{4.888060in}{3.567519in}}{\pgfqpoint{4.880246in}{3.575333in}}%
\pgfpathcurveto{\pgfqpoint{4.872433in}{3.583146in}}{\pgfqpoint{4.861834in}{3.587537in}}{\pgfqpoint{4.850783in}{3.587537in}}%
\pgfpathcurveto{\pgfqpoint{4.839733in}{3.587537in}}{\pgfqpoint{4.829134in}{3.583146in}}{\pgfqpoint{4.821321in}{3.575333in}}%
\pgfpathcurveto{\pgfqpoint{4.813507in}{3.567519in}}{\pgfqpoint{4.809117in}{3.556920in}}{\pgfqpoint{4.809117in}{3.545870in}}%
\pgfpathcurveto{\pgfqpoint{4.809117in}{3.534820in}}{\pgfqpoint{4.813507in}{3.524221in}}{\pgfqpoint{4.821321in}{3.516407in}}%
\pgfpathcurveto{\pgfqpoint{4.829134in}{3.508594in}}{\pgfqpoint{4.839733in}{3.504203in}}{\pgfqpoint{4.850783in}{3.504203in}}%
\pgfpathlineto{\pgfqpoint{4.850783in}{3.504203in}}%
\pgfpathclose%
\pgfusepath{stroke,fill}%
\end{pgfscope}%
\begin{pgfscope}%
\pgfpathrectangle{\pgfqpoint{0.800000in}{0.528000in}}{\pgfqpoint{4.960000in}{3.696000in}}%
\pgfusepath{clip}%
\pgfsetbuttcap%
\pgfsetroundjoin%
\definecolor{currentfill}{rgb}{0.215686,0.494118,0.721569}%
\pgfsetfillcolor{currentfill}%
\pgfsetfillopacity{0.600000}%
\pgfsetlinewidth{1.003750pt}%
\definecolor{currentstroke}{rgb}{0.215686,0.494118,0.721569}%
\pgfsetstrokecolor{currentstroke}%
\pgfsetstrokeopacity{0.600000}%
\pgfsetdash{}{0pt}%
\pgfpathmoveto{\pgfqpoint{3.415585in}{2.469098in}}%
\pgfpathcurveto{\pgfqpoint{3.426635in}{2.469098in}}{\pgfqpoint{3.437234in}{2.473488in}}{\pgfqpoint{3.445048in}{2.481302in}}%
\pgfpathcurveto{\pgfqpoint{3.452861in}{2.489115in}}{\pgfqpoint{3.457252in}{2.499714in}}{\pgfqpoint{3.457252in}{2.510765in}}%
\pgfpathcurveto{\pgfqpoint{3.457252in}{2.521815in}}{\pgfqpoint{3.452861in}{2.532414in}}{\pgfqpoint{3.445048in}{2.540227in}}%
\pgfpathcurveto{\pgfqpoint{3.437234in}{2.548041in}}{\pgfqpoint{3.426635in}{2.552431in}}{\pgfqpoint{3.415585in}{2.552431in}}%
\pgfpathcurveto{\pgfqpoint{3.404535in}{2.552431in}}{\pgfqpoint{3.393936in}{2.548041in}}{\pgfqpoint{3.386122in}{2.540227in}}%
\pgfpathcurveto{\pgfqpoint{3.378308in}{2.532414in}}{\pgfqpoint{3.373918in}{2.521815in}}{\pgfqpoint{3.373918in}{2.510765in}}%
\pgfpathcurveto{\pgfqpoint{3.373918in}{2.499714in}}{\pgfqpoint{3.378308in}{2.489115in}}{\pgfqpoint{3.386122in}{2.481302in}}%
\pgfpathcurveto{\pgfqpoint{3.393936in}{2.473488in}}{\pgfqpoint{3.404535in}{2.469098in}}{\pgfqpoint{3.415585in}{2.469098in}}%
\pgfpathlineto{\pgfqpoint{3.415585in}{2.469098in}}%
\pgfpathclose%
\pgfusepath{stroke,fill}%
\end{pgfscope}%
\begin{pgfscope}%
\pgfpathrectangle{\pgfqpoint{0.800000in}{0.528000in}}{\pgfqpoint{4.960000in}{3.696000in}}%
\pgfusepath{clip}%
\pgfsetbuttcap%
\pgfsetroundjoin%
\definecolor{currentfill}{rgb}{0.215686,0.494118,0.721569}%
\pgfsetfillcolor{currentfill}%
\pgfsetfillopacity{0.600000}%
\pgfsetlinewidth{1.003750pt}%
\definecolor{currentstroke}{rgb}{0.215686,0.494118,0.721569}%
\pgfsetstrokecolor{currentstroke}%
\pgfsetstrokeopacity{0.600000}%
\pgfsetdash{}{0pt}%
\pgfpathmoveto{\pgfqpoint{4.754593in}{3.354505in}}%
\pgfpathcurveto{\pgfqpoint{4.765643in}{3.354505in}}{\pgfqpoint{4.776242in}{3.358895in}}{\pgfqpoint{4.784055in}{3.366709in}}%
\pgfpathcurveto{\pgfqpoint{4.791869in}{3.374523in}}{\pgfqpoint{4.796259in}{3.385122in}}{\pgfqpoint{4.796259in}{3.396172in}}%
\pgfpathcurveto{\pgfqpoint{4.796259in}{3.407222in}}{\pgfqpoint{4.791869in}{3.417821in}}{\pgfqpoint{4.784055in}{3.425635in}}%
\pgfpathcurveto{\pgfqpoint{4.776242in}{3.433448in}}{\pgfqpoint{4.765643in}{3.437838in}}{\pgfqpoint{4.754593in}{3.437838in}}%
\pgfpathcurveto{\pgfqpoint{4.743542in}{3.437838in}}{\pgfqpoint{4.732943in}{3.433448in}}{\pgfqpoint{4.725130in}{3.425635in}}%
\pgfpathcurveto{\pgfqpoint{4.717316in}{3.417821in}}{\pgfqpoint{4.712926in}{3.407222in}}{\pgfqpoint{4.712926in}{3.396172in}}%
\pgfpathcurveto{\pgfqpoint{4.712926in}{3.385122in}}{\pgfqpoint{4.717316in}{3.374523in}}{\pgfqpoint{4.725130in}{3.366709in}}%
\pgfpathcurveto{\pgfqpoint{4.732943in}{3.358895in}}{\pgfqpoint{4.743542in}{3.354505in}}{\pgfqpoint{4.754593in}{3.354505in}}%
\pgfpathlineto{\pgfqpoint{4.754593in}{3.354505in}}%
\pgfpathclose%
\pgfusepath{stroke,fill}%
\end{pgfscope}%
\begin{pgfscope}%
\pgfpathrectangle{\pgfqpoint{0.800000in}{0.528000in}}{\pgfqpoint{4.960000in}{3.696000in}}%
\pgfusepath{clip}%
\pgfsetbuttcap%
\pgfsetroundjoin%
\definecolor{currentfill}{rgb}{0.215686,0.494118,0.721569}%
\pgfsetfillcolor{currentfill}%
\pgfsetfillopacity{0.600000}%
\pgfsetlinewidth{1.003750pt}%
\definecolor{currentstroke}{rgb}{0.215686,0.494118,0.721569}%
\pgfsetstrokecolor{currentstroke}%
\pgfsetstrokeopacity{0.600000}%
\pgfsetdash{}{0pt}%
\pgfpathmoveto{\pgfqpoint{3.658051in}{2.602262in}}%
\pgfpathcurveto{\pgfqpoint{3.669101in}{2.602262in}}{\pgfqpoint{3.679700in}{2.606652in}}{\pgfqpoint{3.687514in}{2.614466in}}%
\pgfpathcurveto{\pgfqpoint{3.695327in}{2.622279in}}{\pgfqpoint{3.699717in}{2.632878in}}{\pgfqpoint{3.699717in}{2.643929in}}%
\pgfpathcurveto{\pgfqpoint{3.699717in}{2.654979in}}{\pgfqpoint{3.695327in}{2.665578in}}{\pgfqpoint{3.687514in}{2.673391in}}%
\pgfpathcurveto{\pgfqpoint{3.679700in}{2.681205in}}{\pgfqpoint{3.669101in}{2.685595in}}{\pgfqpoint{3.658051in}{2.685595in}}%
\pgfpathcurveto{\pgfqpoint{3.647001in}{2.685595in}}{\pgfqpoint{3.636402in}{2.681205in}}{\pgfqpoint{3.628588in}{2.673391in}}%
\pgfpathcurveto{\pgfqpoint{3.620774in}{2.665578in}}{\pgfqpoint{3.616384in}{2.654979in}}{\pgfqpoint{3.616384in}{2.643929in}}%
\pgfpathcurveto{\pgfqpoint{3.616384in}{2.632878in}}{\pgfqpoint{3.620774in}{2.622279in}}{\pgfqpoint{3.628588in}{2.614466in}}%
\pgfpathcurveto{\pgfqpoint{3.636402in}{2.606652in}}{\pgfqpoint{3.647001in}{2.602262in}}{\pgfqpoint{3.658051in}{2.602262in}}%
\pgfpathlineto{\pgfqpoint{3.658051in}{2.602262in}}%
\pgfpathclose%
\pgfusepath{stroke,fill}%
\end{pgfscope}%
\begin{pgfscope}%
\pgfpathrectangle{\pgfqpoint{0.800000in}{0.528000in}}{\pgfqpoint{4.960000in}{3.696000in}}%
\pgfusepath{clip}%
\pgfsetbuttcap%
\pgfsetroundjoin%
\definecolor{currentfill}{rgb}{0.215686,0.494118,0.721569}%
\pgfsetfillcolor{currentfill}%
\pgfsetfillopacity{0.600000}%
\pgfsetlinewidth{1.003750pt}%
\definecolor{currentstroke}{rgb}{0.215686,0.494118,0.721569}%
\pgfsetstrokecolor{currentstroke}%
\pgfsetstrokeopacity{0.600000}%
\pgfsetdash{}{0pt}%
\pgfpathmoveto{\pgfqpoint{4.068457in}{2.868495in}}%
\pgfpathcurveto{\pgfqpoint{4.079508in}{2.868495in}}{\pgfqpoint{4.090107in}{2.872886in}}{\pgfqpoint{4.097920in}{2.880699in}}%
\pgfpathcurveto{\pgfqpoint{4.105734in}{2.888513in}}{\pgfqpoint{4.110124in}{2.899112in}}{\pgfqpoint{4.110124in}{2.910162in}}%
\pgfpathcurveto{\pgfqpoint{4.110124in}{2.921212in}}{\pgfqpoint{4.105734in}{2.931811in}}{\pgfqpoint{4.097920in}{2.939625in}}%
\pgfpathcurveto{\pgfqpoint{4.090107in}{2.947438in}}{\pgfqpoint{4.079508in}{2.951829in}}{\pgfqpoint{4.068457in}{2.951829in}}%
\pgfpathcurveto{\pgfqpoint{4.057407in}{2.951829in}}{\pgfqpoint{4.046808in}{2.947438in}}{\pgfqpoint{4.038995in}{2.939625in}}%
\pgfpathcurveto{\pgfqpoint{4.031181in}{2.931811in}}{\pgfqpoint{4.026791in}{2.921212in}}{\pgfqpoint{4.026791in}{2.910162in}}%
\pgfpathcurveto{\pgfqpoint{4.026791in}{2.899112in}}{\pgfqpoint{4.031181in}{2.888513in}}{\pgfqpoint{4.038995in}{2.880699in}}%
\pgfpathcurveto{\pgfqpoint{4.046808in}{2.872886in}}{\pgfqpoint{4.057407in}{2.868495in}}{\pgfqpoint{4.068457in}{2.868495in}}%
\pgfpathlineto{\pgfqpoint{4.068457in}{2.868495in}}%
\pgfpathclose%
\pgfusepath{stroke,fill}%
\end{pgfscope}%
\begin{pgfscope}%
\pgfpathrectangle{\pgfqpoint{0.800000in}{0.528000in}}{\pgfqpoint{4.960000in}{3.696000in}}%
\pgfusepath{clip}%
\pgfsetbuttcap%
\pgfsetroundjoin%
\definecolor{currentfill}{rgb}{0.215686,0.494118,0.721569}%
\pgfsetfillcolor{currentfill}%
\pgfsetfillopacity{0.600000}%
\pgfsetlinewidth{1.003750pt}%
\definecolor{currentstroke}{rgb}{0.215686,0.494118,0.721569}%
\pgfsetstrokecolor{currentstroke}%
\pgfsetstrokeopacity{0.600000}%
\pgfsetdash{}{0pt}%
\pgfpathmoveto{\pgfqpoint{4.577438in}{3.211236in}}%
\pgfpathcurveto{\pgfqpoint{4.588488in}{3.211236in}}{\pgfqpoint{4.599087in}{3.215626in}}{\pgfqpoint{4.606901in}{3.223440in}}%
\pgfpathcurveto{\pgfqpoint{4.614715in}{3.231254in}}{\pgfqpoint{4.619105in}{3.241853in}}{\pgfqpoint{4.619105in}{3.252903in}}%
\pgfpathcurveto{\pgfqpoint{4.619105in}{3.263953in}}{\pgfqpoint{4.614715in}{3.274552in}}{\pgfqpoint{4.606901in}{3.282366in}}%
\pgfpathcurveto{\pgfqpoint{4.599087in}{3.290179in}}{\pgfqpoint{4.588488in}{3.294569in}}{\pgfqpoint{4.577438in}{3.294569in}}%
\pgfpathcurveto{\pgfqpoint{4.566388in}{3.294569in}}{\pgfqpoint{4.555789in}{3.290179in}}{\pgfqpoint{4.547976in}{3.282366in}}%
\pgfpathcurveto{\pgfqpoint{4.540162in}{3.274552in}}{\pgfqpoint{4.535772in}{3.263953in}}{\pgfqpoint{4.535772in}{3.252903in}}%
\pgfpathcurveto{\pgfqpoint{4.535772in}{3.241853in}}{\pgfqpoint{4.540162in}{3.231254in}}{\pgfqpoint{4.547976in}{3.223440in}}%
\pgfpathcurveto{\pgfqpoint{4.555789in}{3.215626in}}{\pgfqpoint{4.566388in}{3.211236in}}{\pgfqpoint{4.577438in}{3.211236in}}%
\pgfpathlineto{\pgfqpoint{4.577438in}{3.211236in}}%
\pgfpathclose%
\pgfusepath{stroke,fill}%
\end{pgfscope}%
\begin{pgfscope}%
\pgfpathrectangle{\pgfqpoint{0.800000in}{0.528000in}}{\pgfqpoint{4.960000in}{3.696000in}}%
\pgfusepath{clip}%
\pgfsetbuttcap%
\pgfsetroundjoin%
\definecolor{currentfill}{rgb}{0.215686,0.494118,0.721569}%
\pgfsetfillcolor{currentfill}%
\pgfsetfillopacity{0.600000}%
\pgfsetlinewidth{1.003750pt}%
\definecolor{currentstroke}{rgb}{0.215686,0.494118,0.721569}%
\pgfsetstrokecolor{currentstroke}%
\pgfsetstrokeopacity{0.600000}%
\pgfsetdash{}{0pt}%
\pgfpathmoveto{\pgfqpoint{4.652535in}{3.261746in}}%
\pgfpathcurveto{\pgfqpoint{4.663585in}{3.261746in}}{\pgfqpoint{4.674184in}{3.266136in}}{\pgfqpoint{4.681998in}{3.273950in}}%
\pgfpathcurveto{\pgfqpoint{4.689811in}{3.281764in}}{\pgfqpoint{4.694202in}{3.292363in}}{\pgfqpoint{4.694202in}{3.303413in}}%
\pgfpathcurveto{\pgfqpoint{4.694202in}{3.314463in}}{\pgfqpoint{4.689811in}{3.325062in}}{\pgfqpoint{4.681998in}{3.332876in}}%
\pgfpathcurveto{\pgfqpoint{4.674184in}{3.340689in}}{\pgfqpoint{4.663585in}{3.345079in}}{\pgfqpoint{4.652535in}{3.345079in}}%
\pgfpathcurveto{\pgfqpoint{4.641485in}{3.345079in}}{\pgfqpoint{4.630886in}{3.340689in}}{\pgfqpoint{4.623072in}{3.332876in}}%
\pgfpathcurveto{\pgfqpoint{4.615259in}{3.325062in}}{\pgfqpoint{4.610868in}{3.314463in}}{\pgfqpoint{4.610868in}{3.303413in}}%
\pgfpathcurveto{\pgfqpoint{4.610868in}{3.292363in}}{\pgfqpoint{4.615259in}{3.281764in}}{\pgfqpoint{4.623072in}{3.273950in}}%
\pgfpathcurveto{\pgfqpoint{4.630886in}{3.266136in}}{\pgfqpoint{4.641485in}{3.261746in}}{\pgfqpoint{4.652535in}{3.261746in}}%
\pgfpathlineto{\pgfqpoint{4.652535in}{3.261746in}}%
\pgfpathclose%
\pgfusepath{stroke,fill}%
\end{pgfscope}%
\begin{pgfscope}%
\pgfpathrectangle{\pgfqpoint{0.800000in}{0.528000in}}{\pgfqpoint{4.960000in}{3.696000in}}%
\pgfusepath{clip}%
\pgfsetbuttcap%
\pgfsetroundjoin%
\definecolor{currentfill}{rgb}{0.215686,0.494118,0.721569}%
\pgfsetfillcolor{currentfill}%
\pgfsetfillopacity{0.600000}%
\pgfsetlinewidth{1.003750pt}%
\definecolor{currentstroke}{rgb}{0.215686,0.494118,0.721569}%
\pgfsetstrokecolor{currentstroke}%
\pgfsetstrokeopacity{0.600000}%
\pgfsetdash{}{0pt}%
\pgfpathmoveto{\pgfqpoint{3.458157in}{2.408211in}}%
\pgfpathcurveto{\pgfqpoint{3.469207in}{2.408211in}}{\pgfqpoint{3.479806in}{2.412601in}}{\pgfqpoint{3.487620in}{2.420415in}}%
\pgfpathcurveto{\pgfqpoint{3.495434in}{2.428229in}}{\pgfqpoint{3.499824in}{2.438828in}}{\pgfqpoint{3.499824in}{2.449878in}}%
\pgfpathcurveto{\pgfqpoint{3.499824in}{2.460928in}}{\pgfqpoint{3.495434in}{2.471527in}}{\pgfqpoint{3.487620in}{2.479341in}}%
\pgfpathcurveto{\pgfqpoint{3.479806in}{2.487154in}}{\pgfqpoint{3.469207in}{2.491544in}}{\pgfqpoint{3.458157in}{2.491544in}}%
\pgfpathcurveto{\pgfqpoint{3.447107in}{2.491544in}}{\pgfqpoint{3.436508in}{2.487154in}}{\pgfqpoint{3.428694in}{2.479341in}}%
\pgfpathcurveto{\pgfqpoint{3.420881in}{2.471527in}}{\pgfqpoint{3.416491in}{2.460928in}}{\pgfqpoint{3.416491in}{2.449878in}}%
\pgfpathcurveto{\pgfqpoint{3.416491in}{2.438828in}}{\pgfqpoint{3.420881in}{2.428229in}}{\pgfqpoint{3.428694in}{2.420415in}}%
\pgfpathcurveto{\pgfqpoint{3.436508in}{2.412601in}}{\pgfqpoint{3.447107in}{2.408211in}}{\pgfqpoint{3.458157in}{2.408211in}}%
\pgfpathlineto{\pgfqpoint{3.458157in}{2.408211in}}%
\pgfpathclose%
\pgfusepath{stroke,fill}%
\end{pgfscope}%
\begin{pgfscope}%
\pgfpathrectangle{\pgfqpoint{0.800000in}{0.528000in}}{\pgfqpoint{4.960000in}{3.696000in}}%
\pgfusepath{clip}%
\pgfsetbuttcap%
\pgfsetroundjoin%
\definecolor{currentfill}{rgb}{0.215686,0.494118,0.721569}%
\pgfsetfillcolor{currentfill}%
\pgfsetfillopacity{0.600000}%
\pgfsetlinewidth{1.003750pt}%
\definecolor{currentstroke}{rgb}{0.215686,0.494118,0.721569}%
\pgfsetstrokecolor{currentstroke}%
\pgfsetstrokeopacity{0.600000}%
\pgfsetdash{}{0pt}%
\pgfpathmoveto{\pgfqpoint{3.439747in}{2.379499in}}%
\pgfpathcurveto{\pgfqpoint{3.450798in}{2.379499in}}{\pgfqpoint{3.461397in}{2.383889in}}{\pgfqpoint{3.469210in}{2.391702in}}%
\pgfpathcurveto{\pgfqpoint{3.477024in}{2.399516in}}{\pgfqpoint{3.481414in}{2.410115in}}{\pgfqpoint{3.481414in}{2.421165in}}%
\pgfpathcurveto{\pgfqpoint{3.481414in}{2.432215in}}{\pgfqpoint{3.477024in}{2.442814in}}{\pgfqpoint{3.469210in}{2.450628in}}%
\pgfpathcurveto{\pgfqpoint{3.461397in}{2.458442in}}{\pgfqpoint{3.450798in}{2.462832in}}{\pgfqpoint{3.439747in}{2.462832in}}%
\pgfpathcurveto{\pgfqpoint{3.428697in}{2.462832in}}{\pgfqpoint{3.418098in}{2.458442in}}{\pgfqpoint{3.410285in}{2.450628in}}%
\pgfpathcurveto{\pgfqpoint{3.402471in}{2.442814in}}{\pgfqpoint{3.398081in}{2.432215in}}{\pgfqpoint{3.398081in}{2.421165in}}%
\pgfpathcurveto{\pgfqpoint{3.398081in}{2.410115in}}{\pgfqpoint{3.402471in}{2.399516in}}{\pgfqpoint{3.410285in}{2.391702in}}%
\pgfpathcurveto{\pgfqpoint{3.418098in}{2.383889in}}{\pgfqpoint{3.428697in}{2.379499in}}{\pgfqpoint{3.439747in}{2.379499in}}%
\pgfpathlineto{\pgfqpoint{3.439747in}{2.379499in}}%
\pgfpathclose%
\pgfusepath{stroke,fill}%
\end{pgfscope}%
\begin{pgfscope}%
\pgfpathrectangle{\pgfqpoint{0.800000in}{0.528000in}}{\pgfqpoint{4.960000in}{3.696000in}}%
\pgfusepath{clip}%
\pgfsetbuttcap%
\pgfsetroundjoin%
\definecolor{currentfill}{rgb}{0.215686,0.494118,0.721569}%
\pgfsetfillcolor{currentfill}%
\pgfsetfillopacity{0.600000}%
\pgfsetlinewidth{1.003750pt}%
\definecolor{currentstroke}{rgb}{0.215686,0.494118,0.721569}%
\pgfsetstrokecolor{currentstroke}%
\pgfsetstrokeopacity{0.600000}%
\pgfsetdash{}{0pt}%
\pgfpathmoveto{\pgfqpoint{3.802498in}{2.625044in}}%
\pgfpathcurveto{\pgfqpoint{3.813548in}{2.625044in}}{\pgfqpoint{3.824147in}{2.629435in}}{\pgfqpoint{3.831961in}{2.637248in}}%
\pgfpathcurveto{\pgfqpoint{3.839774in}{2.645062in}}{\pgfqpoint{3.844164in}{2.655661in}}{\pgfqpoint{3.844164in}{2.666711in}}%
\pgfpathcurveto{\pgfqpoint{3.844164in}{2.677761in}}{\pgfqpoint{3.839774in}{2.688360in}}{\pgfqpoint{3.831961in}{2.696174in}}%
\pgfpathcurveto{\pgfqpoint{3.824147in}{2.703987in}}{\pgfqpoint{3.813548in}{2.708378in}}{\pgfqpoint{3.802498in}{2.708378in}}%
\pgfpathcurveto{\pgfqpoint{3.791448in}{2.708378in}}{\pgfqpoint{3.780849in}{2.703987in}}{\pgfqpoint{3.773035in}{2.696174in}}%
\pgfpathcurveto{\pgfqpoint{3.765221in}{2.688360in}}{\pgfqpoint{3.760831in}{2.677761in}}{\pgfqpoint{3.760831in}{2.666711in}}%
\pgfpathcurveto{\pgfqpoint{3.760831in}{2.655661in}}{\pgfqpoint{3.765221in}{2.645062in}}{\pgfqpoint{3.773035in}{2.637248in}}%
\pgfpathcurveto{\pgfqpoint{3.780849in}{2.629435in}}{\pgfqpoint{3.791448in}{2.625044in}}{\pgfqpoint{3.802498in}{2.625044in}}%
\pgfpathlineto{\pgfqpoint{3.802498in}{2.625044in}}%
\pgfpathclose%
\pgfusepath{stroke,fill}%
\end{pgfscope}%
\begin{pgfscope}%
\pgfpathrectangle{\pgfqpoint{0.800000in}{0.528000in}}{\pgfqpoint{4.960000in}{3.696000in}}%
\pgfusepath{clip}%
\pgfsetbuttcap%
\pgfsetroundjoin%
\definecolor{currentfill}{rgb}{0.215686,0.494118,0.721569}%
\pgfsetfillcolor{currentfill}%
\pgfsetfillopacity{0.600000}%
\pgfsetlinewidth{1.003750pt}%
\definecolor{currentstroke}{rgb}{0.215686,0.494118,0.721569}%
\pgfsetstrokecolor{currentstroke}%
\pgfsetstrokeopacity{0.600000}%
\pgfsetdash{}{0pt}%
\pgfpathmoveto{\pgfqpoint{2.759150in}{1.837062in}}%
\pgfpathcurveto{\pgfqpoint{2.770200in}{1.837062in}}{\pgfqpoint{2.780800in}{1.841453in}}{\pgfqpoint{2.788613in}{1.849266in}}%
\pgfpathcurveto{\pgfqpoint{2.796427in}{1.857080in}}{\pgfqpoint{2.800817in}{1.867679in}}{\pgfqpoint{2.800817in}{1.878729in}}%
\pgfpathcurveto{\pgfqpoint{2.800817in}{1.889779in}}{\pgfqpoint{2.796427in}{1.900378in}}{\pgfqpoint{2.788613in}{1.908192in}}%
\pgfpathcurveto{\pgfqpoint{2.780800in}{1.916006in}}{\pgfqpoint{2.770200in}{1.920396in}}{\pgfqpoint{2.759150in}{1.920396in}}%
\pgfpathcurveto{\pgfqpoint{2.748100in}{1.920396in}}{\pgfqpoint{2.737501in}{1.916006in}}{\pgfqpoint{2.729688in}{1.908192in}}%
\pgfpathcurveto{\pgfqpoint{2.721874in}{1.900378in}}{\pgfqpoint{2.717484in}{1.889779in}}{\pgfqpoint{2.717484in}{1.878729in}}%
\pgfpathcurveto{\pgfqpoint{2.717484in}{1.867679in}}{\pgfqpoint{2.721874in}{1.857080in}}{\pgfqpoint{2.729688in}{1.849266in}}%
\pgfpathcurveto{\pgfqpoint{2.737501in}{1.841453in}}{\pgfqpoint{2.748100in}{1.837062in}}{\pgfqpoint{2.759150in}{1.837062in}}%
\pgfpathlineto{\pgfqpoint{2.759150in}{1.837062in}}%
\pgfpathclose%
\pgfusepath{stroke,fill}%
\end{pgfscope}%
\begin{pgfscope}%
\pgfpathrectangle{\pgfqpoint{0.800000in}{0.528000in}}{\pgfqpoint{4.960000in}{3.696000in}}%
\pgfusepath{clip}%
\pgfsetbuttcap%
\pgfsetroundjoin%
\definecolor{currentfill}{rgb}{0.894118,0.101961,0.109804}%
\pgfsetfillcolor{currentfill}%
\pgfsetfillopacity{0.600000}%
\pgfsetlinewidth{1.003750pt}%
\definecolor{currentstroke}{rgb}{0.894118,0.101961,0.109804}%
\pgfsetstrokecolor{currentstroke}%
\pgfsetstrokeopacity{0.600000}%
\pgfsetdash{}{0pt}%
\pgfpathmoveto{\pgfqpoint{1.231304in}{4.034493in}}%
\pgfpathcurveto{\pgfqpoint{1.242354in}{4.034493in}}{\pgfqpoint{1.252954in}{4.038884in}}{\pgfqpoint{1.260767in}{4.046697in}}%
\pgfpathcurveto{\pgfqpoint{1.268581in}{4.054511in}}{\pgfqpoint{1.272971in}{4.065110in}}{\pgfqpoint{1.272971in}{4.076160in}}%
\pgfpathcurveto{\pgfqpoint{1.272971in}{4.087210in}}{\pgfqpoint{1.268581in}{4.097809in}}{\pgfqpoint{1.260767in}{4.105623in}}%
\pgfpathcurveto{\pgfqpoint{1.252954in}{4.113436in}}{\pgfqpoint{1.242354in}{4.117827in}}{\pgfqpoint{1.231304in}{4.117827in}}%
\pgfpathcurveto{\pgfqpoint{1.220254in}{4.117827in}}{\pgfqpoint{1.209655in}{4.113436in}}{\pgfqpoint{1.201842in}{4.105623in}}%
\pgfpathcurveto{\pgfqpoint{1.194028in}{4.097809in}}{\pgfqpoint{1.189638in}{4.087210in}}{\pgfqpoint{1.189638in}{4.076160in}}%
\pgfpathcurveto{\pgfqpoint{1.189638in}{4.065110in}}{\pgfqpoint{1.194028in}{4.054511in}}{\pgfqpoint{1.201842in}{4.046697in}}%
\pgfpathcurveto{\pgfqpoint{1.209655in}{4.038884in}}{\pgfqpoint{1.220254in}{4.034493in}}{\pgfqpoint{1.231304in}{4.034493in}}%
\pgfpathlineto{\pgfqpoint{1.231304in}{4.034493in}}%
\pgfpathclose%
\pgfusepath{stroke,fill}%
\end{pgfscope}%
\begin{pgfscope}%
\pgfpathrectangle{\pgfqpoint{0.800000in}{0.528000in}}{\pgfqpoint{4.960000in}{3.696000in}}%
\pgfusepath{clip}%
\pgfsetbuttcap%
\pgfsetroundjoin%
\definecolor{currentfill}{rgb}{0.894118,0.101961,0.109804}%
\pgfsetfillcolor{currentfill}%
\pgfsetfillopacity{0.600000}%
\pgfsetlinewidth{1.003750pt}%
\definecolor{currentstroke}{rgb}{0.894118,0.101961,0.109804}%
\pgfsetstrokecolor{currentstroke}%
\pgfsetstrokeopacity{0.600000}%
\pgfsetdash{}{0pt}%
\pgfpathmoveto{\pgfqpoint{1.516214in}{2.497340in}}%
\pgfpathcurveto{\pgfqpoint{1.527264in}{2.497340in}}{\pgfqpoint{1.537863in}{2.501730in}}{\pgfqpoint{1.545677in}{2.509544in}}%
\pgfpathcurveto{\pgfqpoint{1.553490in}{2.517358in}}{\pgfqpoint{1.557881in}{2.527957in}}{\pgfqpoint{1.557881in}{2.539007in}}%
\pgfpathcurveto{\pgfqpoint{1.557881in}{2.550057in}}{\pgfqpoint{1.553490in}{2.560656in}}{\pgfqpoint{1.545677in}{2.568470in}}%
\pgfpathcurveto{\pgfqpoint{1.537863in}{2.576283in}}{\pgfqpoint{1.527264in}{2.580673in}}{\pgfqpoint{1.516214in}{2.580673in}}%
\pgfpathcurveto{\pgfqpoint{1.505164in}{2.580673in}}{\pgfqpoint{1.494565in}{2.576283in}}{\pgfqpoint{1.486751in}{2.568470in}}%
\pgfpathcurveto{\pgfqpoint{1.478938in}{2.560656in}}{\pgfqpoint{1.474547in}{2.550057in}}{\pgfqpoint{1.474547in}{2.539007in}}%
\pgfpathcurveto{\pgfqpoint{1.474547in}{2.527957in}}{\pgfqpoint{1.478938in}{2.517358in}}{\pgfqpoint{1.486751in}{2.509544in}}%
\pgfpathcurveto{\pgfqpoint{1.494565in}{2.501730in}}{\pgfqpoint{1.505164in}{2.497340in}}{\pgfqpoint{1.516214in}{2.497340in}}%
\pgfpathlineto{\pgfqpoint{1.516214in}{2.497340in}}%
\pgfpathclose%
\pgfusepath{stroke,fill}%
\end{pgfscope}%
\begin{pgfscope}%
\pgfpathrectangle{\pgfqpoint{0.800000in}{0.528000in}}{\pgfqpoint{4.960000in}{3.696000in}}%
\pgfusepath{clip}%
\pgfsetbuttcap%
\pgfsetroundjoin%
\definecolor{currentfill}{rgb}{0.894118,0.101961,0.109804}%
\pgfsetfillcolor{currentfill}%
\pgfsetfillopacity{0.600000}%
\pgfsetlinewidth{1.003750pt}%
\definecolor{currentstroke}{rgb}{0.894118,0.101961,0.109804}%
\pgfsetstrokecolor{currentstroke}%
\pgfsetstrokeopacity{0.600000}%
\pgfsetdash{}{0pt}%
\pgfpathmoveto{\pgfqpoint{1.472435in}{2.189440in}}%
\pgfpathcurveto{\pgfqpoint{1.483485in}{2.189440in}}{\pgfqpoint{1.494084in}{2.193830in}}{\pgfqpoint{1.501898in}{2.201644in}}%
\pgfpathcurveto{\pgfqpoint{1.509711in}{2.209458in}}{\pgfqpoint{1.514102in}{2.220057in}}{\pgfqpoint{1.514102in}{2.231107in}}%
\pgfpathcurveto{\pgfqpoint{1.514102in}{2.242157in}}{\pgfqpoint{1.509711in}{2.252756in}}{\pgfqpoint{1.501898in}{2.260570in}}%
\pgfpathcurveto{\pgfqpoint{1.494084in}{2.268383in}}{\pgfqpoint{1.483485in}{2.272773in}}{\pgfqpoint{1.472435in}{2.272773in}}%
\pgfpathcurveto{\pgfqpoint{1.461385in}{2.272773in}}{\pgfqpoint{1.450786in}{2.268383in}}{\pgfqpoint{1.442972in}{2.260570in}}%
\pgfpathcurveto{\pgfqpoint{1.435159in}{2.252756in}}{\pgfqpoint{1.430768in}{2.242157in}}{\pgfqpoint{1.430768in}{2.231107in}}%
\pgfpathcurveto{\pgfqpoint{1.430768in}{2.220057in}}{\pgfqpoint{1.435159in}{2.209458in}}{\pgfqpoint{1.442972in}{2.201644in}}%
\pgfpathcurveto{\pgfqpoint{1.450786in}{2.193830in}}{\pgfqpoint{1.461385in}{2.189440in}}{\pgfqpoint{1.472435in}{2.189440in}}%
\pgfpathlineto{\pgfqpoint{1.472435in}{2.189440in}}%
\pgfpathclose%
\pgfusepath{stroke,fill}%
\end{pgfscope}%
\begin{pgfscope}%
\pgfpathrectangle{\pgfqpoint{0.800000in}{0.528000in}}{\pgfqpoint{4.960000in}{3.696000in}}%
\pgfusepath{clip}%
\pgfsetbuttcap%
\pgfsetroundjoin%
\definecolor{currentfill}{rgb}{0.894118,0.101961,0.109804}%
\pgfsetfillcolor{currentfill}%
\pgfsetfillopacity{0.600000}%
\pgfsetlinewidth{1.003750pt}%
\definecolor{currentstroke}{rgb}{0.894118,0.101961,0.109804}%
\pgfsetstrokecolor{currentstroke}%
\pgfsetstrokeopacity{0.600000}%
\pgfsetdash{}{0pt}%
\pgfpathmoveto{\pgfqpoint{1.741135in}{2.244009in}}%
\pgfpathcurveto{\pgfqpoint{1.752185in}{2.244009in}}{\pgfqpoint{1.762784in}{2.248399in}}{\pgfqpoint{1.770598in}{2.256213in}}%
\pgfpathcurveto{\pgfqpoint{1.778411in}{2.264026in}}{\pgfqpoint{1.782801in}{2.274625in}}{\pgfqpoint{1.782801in}{2.285675in}}%
\pgfpathcurveto{\pgfqpoint{1.782801in}{2.296725in}}{\pgfqpoint{1.778411in}{2.307324in}}{\pgfqpoint{1.770598in}{2.315138in}}%
\pgfpathcurveto{\pgfqpoint{1.762784in}{2.322952in}}{\pgfqpoint{1.752185in}{2.327342in}}{\pgfqpoint{1.741135in}{2.327342in}}%
\pgfpathcurveto{\pgfqpoint{1.730085in}{2.327342in}}{\pgfqpoint{1.719486in}{2.322952in}}{\pgfqpoint{1.711672in}{2.315138in}}%
\pgfpathcurveto{\pgfqpoint{1.703858in}{2.307324in}}{\pgfqpoint{1.699468in}{2.296725in}}{\pgfqpoint{1.699468in}{2.285675in}}%
\pgfpathcurveto{\pgfqpoint{1.699468in}{2.274625in}}{\pgfqpoint{1.703858in}{2.264026in}}{\pgfqpoint{1.711672in}{2.256213in}}%
\pgfpathcurveto{\pgfqpoint{1.719486in}{2.248399in}}{\pgfqpoint{1.730085in}{2.244009in}}{\pgfqpoint{1.741135in}{2.244009in}}%
\pgfpathlineto{\pgfqpoint{1.741135in}{2.244009in}}%
\pgfpathclose%
\pgfusepath{stroke,fill}%
\end{pgfscope}%
\begin{pgfscope}%
\pgfpathrectangle{\pgfqpoint{0.800000in}{0.528000in}}{\pgfqpoint{4.960000in}{3.696000in}}%
\pgfusepath{clip}%
\pgfsetbuttcap%
\pgfsetroundjoin%
\definecolor{currentfill}{rgb}{0.894118,0.101961,0.109804}%
\pgfsetfillcolor{currentfill}%
\pgfsetfillopacity{0.600000}%
\pgfsetlinewidth{1.003750pt}%
\definecolor{currentstroke}{rgb}{0.894118,0.101961,0.109804}%
\pgfsetstrokecolor{currentstroke}%
\pgfsetstrokeopacity{0.600000}%
\pgfsetdash{}{0pt}%
\pgfpathmoveto{\pgfqpoint{1.597607in}{1.974308in}}%
\pgfpathcurveto{\pgfqpoint{1.608657in}{1.974308in}}{\pgfqpoint{1.619256in}{1.978698in}}{\pgfqpoint{1.627070in}{1.986511in}}%
\pgfpathcurveto{\pgfqpoint{1.634884in}{1.994325in}}{\pgfqpoint{1.639274in}{2.004924in}}{\pgfqpoint{1.639274in}{2.015974in}}%
\pgfpathcurveto{\pgfqpoint{1.639274in}{2.027024in}}{\pgfqpoint{1.634884in}{2.037623in}}{\pgfqpoint{1.627070in}{2.045437in}}%
\pgfpathcurveto{\pgfqpoint{1.619256in}{2.053251in}}{\pgfqpoint{1.608657in}{2.057641in}}{\pgfqpoint{1.597607in}{2.057641in}}%
\pgfpathcurveto{\pgfqpoint{1.586557in}{2.057641in}}{\pgfqpoint{1.575958in}{2.053251in}}{\pgfqpoint{1.568145in}{2.045437in}}%
\pgfpathcurveto{\pgfqpoint{1.560331in}{2.037623in}}{\pgfqpoint{1.555941in}{2.027024in}}{\pgfqpoint{1.555941in}{2.015974in}}%
\pgfpathcurveto{\pgfqpoint{1.555941in}{2.004924in}}{\pgfqpoint{1.560331in}{1.994325in}}{\pgfqpoint{1.568145in}{1.986511in}}%
\pgfpathcurveto{\pgfqpoint{1.575958in}{1.978698in}}{\pgfqpoint{1.586557in}{1.974308in}}{\pgfqpoint{1.597607in}{1.974308in}}%
\pgfpathlineto{\pgfqpoint{1.597607in}{1.974308in}}%
\pgfpathclose%
\pgfusepath{stroke,fill}%
\end{pgfscope}%
\begin{pgfscope}%
\pgfpathrectangle{\pgfqpoint{0.800000in}{0.528000in}}{\pgfqpoint{4.960000in}{3.696000in}}%
\pgfusepath{clip}%
\pgfsetbuttcap%
\pgfsetroundjoin%
\definecolor{currentfill}{rgb}{0.894118,0.101961,0.109804}%
\pgfsetfillcolor{currentfill}%
\pgfsetfillopacity{0.600000}%
\pgfsetlinewidth{1.003750pt}%
\definecolor{currentstroke}{rgb}{0.894118,0.101961,0.109804}%
\pgfsetstrokecolor{currentstroke}%
\pgfsetstrokeopacity{0.600000}%
\pgfsetdash{}{0pt}%
\pgfpathmoveto{\pgfqpoint{2.137719in}{2.304149in}}%
\pgfpathcurveto{\pgfqpoint{2.148769in}{2.304149in}}{\pgfqpoint{2.159368in}{2.308539in}}{\pgfqpoint{2.167182in}{2.316353in}}%
\pgfpathcurveto{\pgfqpoint{2.174995in}{2.324166in}}{\pgfqpoint{2.179386in}{2.334765in}}{\pgfqpoint{2.179386in}{2.345815in}}%
\pgfpathcurveto{\pgfqpoint{2.179386in}{2.356866in}}{\pgfqpoint{2.174995in}{2.367465in}}{\pgfqpoint{2.167182in}{2.375278in}}%
\pgfpathcurveto{\pgfqpoint{2.159368in}{2.383092in}}{\pgfqpoint{2.148769in}{2.387482in}}{\pgfqpoint{2.137719in}{2.387482in}}%
\pgfpathcurveto{\pgfqpoint{2.126669in}{2.387482in}}{\pgfqpoint{2.116070in}{2.383092in}}{\pgfqpoint{2.108256in}{2.375278in}}%
\pgfpathcurveto{\pgfqpoint{2.100443in}{2.367465in}}{\pgfqpoint{2.096052in}{2.356866in}}{\pgfqpoint{2.096052in}{2.345815in}}%
\pgfpathcurveto{\pgfqpoint{2.096052in}{2.334765in}}{\pgfqpoint{2.100443in}{2.324166in}}{\pgfqpoint{2.108256in}{2.316353in}}%
\pgfpathcurveto{\pgfqpoint{2.116070in}{2.308539in}}{\pgfqpoint{2.126669in}{2.304149in}}{\pgfqpoint{2.137719in}{2.304149in}}%
\pgfpathlineto{\pgfqpoint{2.137719in}{2.304149in}}%
\pgfpathclose%
\pgfusepath{stroke,fill}%
\end{pgfscope}%
\begin{pgfscope}%
\pgfpathrectangle{\pgfqpoint{0.800000in}{0.528000in}}{\pgfqpoint{4.960000in}{3.696000in}}%
\pgfusepath{clip}%
\pgfsetbuttcap%
\pgfsetroundjoin%
\definecolor{currentfill}{rgb}{0.894118,0.101961,0.109804}%
\pgfsetfillcolor{currentfill}%
\pgfsetfillopacity{0.600000}%
\pgfsetlinewidth{1.003750pt}%
\definecolor{currentstroke}{rgb}{0.894118,0.101961,0.109804}%
\pgfsetstrokecolor{currentstroke}%
\pgfsetstrokeopacity{0.600000}%
\pgfsetdash{}{0pt}%
\pgfpathmoveto{\pgfqpoint{2.263540in}{2.383384in}}%
\pgfpathcurveto{\pgfqpoint{2.274590in}{2.383384in}}{\pgfqpoint{2.285189in}{2.387774in}}{\pgfqpoint{2.293003in}{2.395588in}}%
\pgfpathcurveto{\pgfqpoint{2.300816in}{2.403402in}}{\pgfqpoint{2.305206in}{2.414001in}}{\pgfqpoint{2.305206in}{2.425051in}}%
\pgfpathcurveto{\pgfqpoint{2.305206in}{2.436101in}}{\pgfqpoint{2.300816in}{2.446700in}}{\pgfqpoint{2.293003in}{2.454514in}}%
\pgfpathcurveto{\pgfqpoint{2.285189in}{2.462327in}}{\pgfqpoint{2.274590in}{2.466718in}}{\pgfqpoint{2.263540in}{2.466718in}}%
\pgfpathcurveto{\pgfqpoint{2.252490in}{2.466718in}}{\pgfqpoint{2.241891in}{2.462327in}}{\pgfqpoint{2.234077in}{2.454514in}}%
\pgfpathcurveto{\pgfqpoint{2.226263in}{2.446700in}}{\pgfqpoint{2.221873in}{2.436101in}}{\pgfqpoint{2.221873in}{2.425051in}}%
\pgfpathcurveto{\pgfqpoint{2.221873in}{2.414001in}}{\pgfqpoint{2.226263in}{2.403402in}}{\pgfqpoint{2.234077in}{2.395588in}}%
\pgfpathcurveto{\pgfqpoint{2.241891in}{2.387774in}}{\pgfqpoint{2.252490in}{2.383384in}}{\pgfqpoint{2.263540in}{2.383384in}}%
\pgfpathlineto{\pgfqpoint{2.263540in}{2.383384in}}%
\pgfpathclose%
\pgfusepath{stroke,fill}%
\end{pgfscope}%
\begin{pgfscope}%
\pgfpathrectangle{\pgfqpoint{0.800000in}{0.528000in}}{\pgfqpoint{4.960000in}{3.696000in}}%
\pgfusepath{clip}%
\pgfsetbuttcap%
\pgfsetroundjoin%
\definecolor{currentfill}{rgb}{0.894118,0.101961,0.109804}%
\pgfsetfillcolor{currentfill}%
\pgfsetfillopacity{0.600000}%
\pgfsetlinewidth{1.003750pt}%
\definecolor{currentstroke}{rgb}{0.894118,0.101961,0.109804}%
\pgfsetstrokecolor{currentstroke}%
\pgfsetstrokeopacity{0.600000}%
\pgfsetdash{}{0pt}%
\pgfpathmoveto{\pgfqpoint{2.051992in}{2.209569in}}%
\pgfpathcurveto{\pgfqpoint{2.063042in}{2.209569in}}{\pgfqpoint{2.073641in}{2.213960in}}{\pgfqpoint{2.081454in}{2.221773in}}%
\pgfpathcurveto{\pgfqpoint{2.089268in}{2.229587in}}{\pgfqpoint{2.093658in}{2.240186in}}{\pgfqpoint{2.093658in}{2.251236in}}%
\pgfpathcurveto{\pgfqpoint{2.093658in}{2.262286in}}{\pgfqpoint{2.089268in}{2.272885in}}{\pgfqpoint{2.081454in}{2.280699in}}%
\pgfpathcurveto{\pgfqpoint{2.073641in}{2.288512in}}{\pgfqpoint{2.063042in}{2.292903in}}{\pgfqpoint{2.051992in}{2.292903in}}%
\pgfpathcurveto{\pgfqpoint{2.040941in}{2.292903in}}{\pgfqpoint{2.030342in}{2.288512in}}{\pgfqpoint{2.022529in}{2.280699in}}%
\pgfpathcurveto{\pgfqpoint{2.014715in}{2.272885in}}{\pgfqpoint{2.010325in}{2.262286in}}{\pgfqpoint{2.010325in}{2.251236in}}%
\pgfpathcurveto{\pgfqpoint{2.010325in}{2.240186in}}{\pgfqpoint{2.014715in}{2.229587in}}{\pgfqpoint{2.022529in}{2.221773in}}%
\pgfpathcurveto{\pgfqpoint{2.030342in}{2.213960in}}{\pgfqpoint{2.040941in}{2.209569in}}{\pgfqpoint{2.051992in}{2.209569in}}%
\pgfpathlineto{\pgfqpoint{2.051992in}{2.209569in}}%
\pgfpathclose%
\pgfusepath{stroke,fill}%
\end{pgfscope}%
\begin{pgfscope}%
\pgfpathrectangle{\pgfqpoint{0.800000in}{0.528000in}}{\pgfqpoint{4.960000in}{3.696000in}}%
\pgfusepath{clip}%
\pgfsetbuttcap%
\pgfsetroundjoin%
\definecolor{currentfill}{rgb}{0.894118,0.101961,0.109804}%
\pgfsetfillcolor{currentfill}%
\pgfsetfillopacity{0.600000}%
\pgfsetlinewidth{1.003750pt}%
\definecolor{currentstroke}{rgb}{0.894118,0.101961,0.109804}%
\pgfsetstrokecolor{currentstroke}%
\pgfsetstrokeopacity{0.600000}%
\pgfsetdash{}{0pt}%
\pgfpathmoveto{\pgfqpoint{1.734487in}{1.925745in}}%
\pgfpathcurveto{\pgfqpoint{1.745538in}{1.925745in}}{\pgfqpoint{1.756137in}{1.930135in}}{\pgfqpoint{1.763950in}{1.937948in}}%
\pgfpathcurveto{\pgfqpoint{1.771764in}{1.945762in}}{\pgfqpoint{1.776154in}{1.956361in}}{\pgfqpoint{1.776154in}{1.967411in}}%
\pgfpathcurveto{\pgfqpoint{1.776154in}{1.978461in}}{\pgfqpoint{1.771764in}{1.989060in}}{\pgfqpoint{1.763950in}{1.996874in}}%
\pgfpathcurveto{\pgfqpoint{1.756137in}{2.004688in}}{\pgfqpoint{1.745538in}{2.009078in}}{\pgfqpoint{1.734487in}{2.009078in}}%
\pgfpathcurveto{\pgfqpoint{1.723437in}{2.009078in}}{\pgfqpoint{1.712838in}{2.004688in}}{\pgfqpoint{1.705025in}{1.996874in}}%
\pgfpathcurveto{\pgfqpoint{1.697211in}{1.989060in}}{\pgfqpoint{1.692821in}{1.978461in}}{\pgfqpoint{1.692821in}{1.967411in}}%
\pgfpathcurveto{\pgfqpoint{1.692821in}{1.956361in}}{\pgfqpoint{1.697211in}{1.945762in}}{\pgfqpoint{1.705025in}{1.937948in}}%
\pgfpathcurveto{\pgfqpoint{1.712838in}{1.930135in}}{\pgfqpoint{1.723437in}{1.925745in}}{\pgfqpoint{1.734487in}{1.925745in}}%
\pgfpathlineto{\pgfqpoint{1.734487in}{1.925745in}}%
\pgfpathclose%
\pgfusepath{stroke,fill}%
\end{pgfscope}%
\begin{pgfscope}%
\pgfpathrectangle{\pgfqpoint{0.800000in}{0.528000in}}{\pgfqpoint{4.960000in}{3.696000in}}%
\pgfusepath{clip}%
\pgfsetbuttcap%
\pgfsetroundjoin%
\definecolor{currentfill}{rgb}{0.894118,0.101961,0.109804}%
\pgfsetfillcolor{currentfill}%
\pgfsetfillopacity{0.600000}%
\pgfsetlinewidth{1.003750pt}%
\definecolor{currentstroke}{rgb}{0.894118,0.101961,0.109804}%
\pgfsetstrokecolor{currentstroke}%
\pgfsetstrokeopacity{0.600000}%
\pgfsetdash{}{0pt}%
\pgfpathmoveto{\pgfqpoint{2.457687in}{2.385429in}}%
\pgfpathcurveto{\pgfqpoint{2.468737in}{2.385429in}}{\pgfqpoint{2.479336in}{2.389820in}}{\pgfqpoint{2.487150in}{2.397633in}}%
\pgfpathcurveto{\pgfqpoint{2.494963in}{2.405447in}}{\pgfqpoint{2.499354in}{2.416046in}}{\pgfqpoint{2.499354in}{2.427096in}}%
\pgfpathcurveto{\pgfqpoint{2.499354in}{2.438146in}}{\pgfqpoint{2.494963in}{2.448745in}}{\pgfqpoint{2.487150in}{2.456559in}}%
\pgfpathcurveto{\pgfqpoint{2.479336in}{2.464372in}}{\pgfqpoint{2.468737in}{2.468763in}}{\pgfqpoint{2.457687in}{2.468763in}}%
\pgfpathcurveto{\pgfqpoint{2.446637in}{2.468763in}}{\pgfqpoint{2.436038in}{2.464372in}}{\pgfqpoint{2.428224in}{2.456559in}}%
\pgfpathcurveto{\pgfqpoint{2.420410in}{2.448745in}}{\pgfqpoint{2.416020in}{2.438146in}}{\pgfqpoint{2.416020in}{2.427096in}}%
\pgfpathcurveto{\pgfqpoint{2.416020in}{2.416046in}}{\pgfqpoint{2.420410in}{2.405447in}}{\pgfqpoint{2.428224in}{2.397633in}}%
\pgfpathcurveto{\pgfqpoint{2.436038in}{2.389820in}}{\pgfqpoint{2.446637in}{2.385429in}}{\pgfqpoint{2.457687in}{2.385429in}}%
\pgfpathlineto{\pgfqpoint{2.457687in}{2.385429in}}%
\pgfpathclose%
\pgfusepath{stroke,fill}%
\end{pgfscope}%
\begin{pgfscope}%
\pgfpathrectangle{\pgfqpoint{0.800000in}{0.528000in}}{\pgfqpoint{4.960000in}{3.696000in}}%
\pgfusepath{clip}%
\pgfsetbuttcap%
\pgfsetroundjoin%
\definecolor{currentfill}{rgb}{0.894118,0.101961,0.109804}%
\pgfsetfillcolor{currentfill}%
\pgfsetfillopacity{0.600000}%
\pgfsetlinewidth{1.003750pt}%
\definecolor{currentstroke}{rgb}{0.894118,0.101961,0.109804}%
\pgfsetstrokecolor{currentstroke}%
\pgfsetstrokeopacity{0.600000}%
\pgfsetdash{}{0pt}%
\pgfpathmoveto{\pgfqpoint{2.348459in}{2.249708in}}%
\pgfpathcurveto{\pgfqpoint{2.359509in}{2.249708in}}{\pgfqpoint{2.370108in}{2.254099in}}{\pgfqpoint{2.377922in}{2.261912in}}%
\pgfpathcurveto{\pgfqpoint{2.385735in}{2.269726in}}{\pgfqpoint{2.390126in}{2.280325in}}{\pgfqpoint{2.390126in}{2.291375in}}%
\pgfpathcurveto{\pgfqpoint{2.390126in}{2.302425in}}{\pgfqpoint{2.385735in}{2.313024in}}{\pgfqpoint{2.377922in}{2.320838in}}%
\pgfpathcurveto{\pgfqpoint{2.370108in}{2.328651in}}{\pgfqpoint{2.359509in}{2.333042in}}{\pgfqpoint{2.348459in}{2.333042in}}%
\pgfpathcurveto{\pgfqpoint{2.337409in}{2.333042in}}{\pgfqpoint{2.326810in}{2.328651in}}{\pgfqpoint{2.318996in}{2.320838in}}%
\pgfpathcurveto{\pgfqpoint{2.311183in}{2.313024in}}{\pgfqpoint{2.306792in}{2.302425in}}{\pgfqpoint{2.306792in}{2.291375in}}%
\pgfpathcurveto{\pgfqpoint{2.306792in}{2.280325in}}{\pgfqpoint{2.311183in}{2.269726in}}{\pgfqpoint{2.318996in}{2.261912in}}%
\pgfpathcurveto{\pgfqpoint{2.326810in}{2.254099in}}{\pgfqpoint{2.337409in}{2.249708in}}{\pgfqpoint{2.348459in}{2.249708in}}%
\pgfpathlineto{\pgfqpoint{2.348459in}{2.249708in}}%
\pgfpathclose%
\pgfusepath{stroke,fill}%
\end{pgfscope}%
\begin{pgfscope}%
\pgfpathrectangle{\pgfqpoint{0.800000in}{0.528000in}}{\pgfqpoint{4.960000in}{3.696000in}}%
\pgfusepath{clip}%
\pgfsetbuttcap%
\pgfsetroundjoin%
\definecolor{currentfill}{rgb}{0.894118,0.101961,0.109804}%
\pgfsetfillcolor{currentfill}%
\pgfsetfillopacity{0.600000}%
\pgfsetlinewidth{1.003750pt}%
\definecolor{currentstroke}{rgb}{0.894118,0.101961,0.109804}%
\pgfsetstrokecolor{currentstroke}%
\pgfsetstrokeopacity{0.600000}%
\pgfsetdash{}{0pt}%
\pgfpathmoveto{\pgfqpoint{3.127484in}{2.774518in}}%
\pgfpathcurveto{\pgfqpoint{3.138534in}{2.774518in}}{\pgfqpoint{3.149133in}{2.778908in}}{\pgfqpoint{3.156947in}{2.786721in}}%
\pgfpathcurveto{\pgfqpoint{3.164761in}{2.794535in}}{\pgfqpoint{3.169151in}{2.805134in}}{\pgfqpoint{3.169151in}{2.816184in}}%
\pgfpathcurveto{\pgfqpoint{3.169151in}{2.827234in}}{\pgfqpoint{3.164761in}{2.837833in}}{\pgfqpoint{3.156947in}{2.845647in}}%
\pgfpathcurveto{\pgfqpoint{3.149133in}{2.853461in}}{\pgfqpoint{3.138534in}{2.857851in}}{\pgfqpoint{3.127484in}{2.857851in}}%
\pgfpathcurveto{\pgfqpoint{3.116434in}{2.857851in}}{\pgfqpoint{3.105835in}{2.853461in}}{\pgfqpoint{3.098021in}{2.845647in}}%
\pgfpathcurveto{\pgfqpoint{3.090208in}{2.837833in}}{\pgfqpoint{3.085818in}{2.827234in}}{\pgfqpoint{3.085818in}{2.816184in}}%
\pgfpathcurveto{\pgfqpoint{3.085818in}{2.805134in}}{\pgfqpoint{3.090208in}{2.794535in}}{\pgfqpoint{3.098021in}{2.786721in}}%
\pgfpathcurveto{\pgfqpoint{3.105835in}{2.778908in}}{\pgfqpoint{3.116434in}{2.774518in}}{\pgfqpoint{3.127484in}{2.774518in}}%
\pgfpathlineto{\pgfqpoint{3.127484in}{2.774518in}}%
\pgfpathclose%
\pgfusepath{stroke,fill}%
\end{pgfscope}%
\begin{pgfscope}%
\pgfpathrectangle{\pgfqpoint{0.800000in}{0.528000in}}{\pgfqpoint{4.960000in}{3.696000in}}%
\pgfusepath{clip}%
\pgfsetbuttcap%
\pgfsetroundjoin%
\definecolor{currentfill}{rgb}{0.894118,0.101961,0.109804}%
\pgfsetfillcolor{currentfill}%
\pgfsetfillopacity{0.600000}%
\pgfsetlinewidth{1.003750pt}%
\definecolor{currentstroke}{rgb}{0.894118,0.101961,0.109804}%
\pgfsetstrokecolor{currentstroke}%
\pgfsetstrokeopacity{0.600000}%
\pgfsetdash{}{0pt}%
\pgfpathmoveto{\pgfqpoint{2.783567in}{2.452786in}}%
\pgfpathcurveto{\pgfqpoint{2.794617in}{2.452786in}}{\pgfqpoint{2.805216in}{2.457176in}}{\pgfqpoint{2.813030in}{2.464990in}}%
\pgfpathcurveto{\pgfqpoint{2.820843in}{2.472803in}}{\pgfqpoint{2.825234in}{2.483402in}}{\pgfqpoint{2.825234in}{2.494452in}}%
\pgfpathcurveto{\pgfqpoint{2.825234in}{2.505503in}}{\pgfqpoint{2.820843in}{2.516102in}}{\pgfqpoint{2.813030in}{2.523915in}}%
\pgfpathcurveto{\pgfqpoint{2.805216in}{2.531729in}}{\pgfqpoint{2.794617in}{2.536119in}}{\pgfqpoint{2.783567in}{2.536119in}}%
\pgfpathcurveto{\pgfqpoint{2.772517in}{2.536119in}}{\pgfqpoint{2.761918in}{2.531729in}}{\pgfqpoint{2.754104in}{2.523915in}}%
\pgfpathcurveto{\pgfqpoint{2.746291in}{2.516102in}}{\pgfqpoint{2.741900in}{2.505503in}}{\pgfqpoint{2.741900in}{2.494452in}}%
\pgfpathcurveto{\pgfqpoint{2.741900in}{2.483402in}}{\pgfqpoint{2.746291in}{2.472803in}}{\pgfqpoint{2.754104in}{2.464990in}}%
\pgfpathcurveto{\pgfqpoint{2.761918in}{2.457176in}}{\pgfqpoint{2.772517in}{2.452786in}}{\pgfqpoint{2.783567in}{2.452786in}}%
\pgfpathlineto{\pgfqpoint{2.783567in}{2.452786in}}%
\pgfpathclose%
\pgfusepath{stroke,fill}%
\end{pgfscope}%
\begin{pgfscope}%
\pgfpathrectangle{\pgfqpoint{0.800000in}{0.528000in}}{\pgfqpoint{4.960000in}{3.696000in}}%
\pgfusepath{clip}%
\pgfsetbuttcap%
\pgfsetroundjoin%
\definecolor{currentfill}{rgb}{0.894118,0.101961,0.109804}%
\pgfsetfillcolor{currentfill}%
\pgfsetfillopacity{0.600000}%
\pgfsetlinewidth{1.003750pt}%
\definecolor{currentstroke}{rgb}{0.894118,0.101961,0.109804}%
\pgfsetstrokecolor{currentstroke}%
\pgfsetstrokeopacity{0.600000}%
\pgfsetdash{}{0pt}%
\pgfpathmoveto{\pgfqpoint{2.327835in}{2.135195in}}%
\pgfpathcurveto{\pgfqpoint{2.338885in}{2.135195in}}{\pgfqpoint{2.349484in}{2.139585in}}{\pgfqpoint{2.357297in}{2.147399in}}%
\pgfpathcurveto{\pgfqpoint{2.365111in}{2.155212in}}{\pgfqpoint{2.369501in}{2.165811in}}{\pgfqpoint{2.369501in}{2.176861in}}%
\pgfpathcurveto{\pgfqpoint{2.369501in}{2.187912in}}{\pgfqpoint{2.365111in}{2.198511in}}{\pgfqpoint{2.357297in}{2.206324in}}%
\pgfpathcurveto{\pgfqpoint{2.349484in}{2.214138in}}{\pgfqpoint{2.338885in}{2.218528in}}{\pgfqpoint{2.327835in}{2.218528in}}%
\pgfpathcurveto{\pgfqpoint{2.316785in}{2.218528in}}{\pgfqpoint{2.306185in}{2.214138in}}{\pgfqpoint{2.298372in}{2.206324in}}%
\pgfpathcurveto{\pgfqpoint{2.290558in}{2.198511in}}{\pgfqpoint{2.286168in}{2.187912in}}{\pgfqpoint{2.286168in}{2.176861in}}%
\pgfpathcurveto{\pgfqpoint{2.286168in}{2.165811in}}{\pgfqpoint{2.290558in}{2.155212in}}{\pgfqpoint{2.298372in}{2.147399in}}%
\pgfpathcurveto{\pgfqpoint{2.306185in}{2.139585in}}{\pgfqpoint{2.316785in}{2.135195in}}{\pgfqpoint{2.327835in}{2.135195in}}%
\pgfpathlineto{\pgfqpoint{2.327835in}{2.135195in}}%
\pgfpathclose%
\pgfusepath{stroke,fill}%
\end{pgfscope}%
\begin{pgfscope}%
\pgfpathrectangle{\pgfqpoint{0.800000in}{0.528000in}}{\pgfqpoint{4.960000in}{3.696000in}}%
\pgfusepath{clip}%
\pgfsetbuttcap%
\pgfsetroundjoin%
\definecolor{currentfill}{rgb}{0.894118,0.101961,0.109804}%
\pgfsetfillcolor{currentfill}%
\pgfsetfillopacity{0.600000}%
\pgfsetlinewidth{1.003750pt}%
\definecolor{currentstroke}{rgb}{0.894118,0.101961,0.109804}%
\pgfsetstrokecolor{currentstroke}%
\pgfsetstrokeopacity{0.600000}%
\pgfsetdash{}{0pt}%
\pgfpathmoveto{\pgfqpoint{2.007017in}{1.896000in}}%
\pgfpathcurveto{\pgfqpoint{2.018067in}{1.896000in}}{\pgfqpoint{2.028666in}{1.900391in}}{\pgfqpoint{2.036480in}{1.908204in}}%
\pgfpathcurveto{\pgfqpoint{2.044293in}{1.916018in}}{\pgfqpoint{2.048684in}{1.926617in}}{\pgfqpoint{2.048684in}{1.937667in}}%
\pgfpathcurveto{\pgfqpoint{2.048684in}{1.948717in}}{\pgfqpoint{2.044293in}{1.959316in}}{\pgfqpoint{2.036480in}{1.967130in}}%
\pgfpathcurveto{\pgfqpoint{2.028666in}{1.974943in}}{\pgfqpoint{2.018067in}{1.979334in}}{\pgfqpoint{2.007017in}{1.979334in}}%
\pgfpathcurveto{\pgfqpoint{1.995967in}{1.979334in}}{\pgfqpoint{1.985368in}{1.974943in}}{\pgfqpoint{1.977554in}{1.967130in}}%
\pgfpathcurveto{\pgfqpoint{1.969741in}{1.959316in}}{\pgfqpoint{1.965350in}{1.948717in}}{\pgfqpoint{1.965350in}{1.937667in}}%
\pgfpathcurveto{\pgfqpoint{1.965350in}{1.926617in}}{\pgfqpoint{1.969741in}{1.916018in}}{\pgfqpoint{1.977554in}{1.908204in}}%
\pgfpathcurveto{\pgfqpoint{1.985368in}{1.900391in}}{\pgfqpoint{1.995967in}{1.896000in}}{\pgfqpoint{2.007017in}{1.896000in}}%
\pgfpathlineto{\pgfqpoint{2.007017in}{1.896000in}}%
\pgfpathclose%
\pgfusepath{stroke,fill}%
\end{pgfscope}%
\begin{pgfscope}%
\pgfpathrectangle{\pgfqpoint{0.800000in}{0.528000in}}{\pgfqpoint{4.960000in}{3.696000in}}%
\pgfusepath{clip}%
\pgfsetbuttcap%
\pgfsetroundjoin%
\definecolor{currentfill}{rgb}{0.894118,0.101961,0.109804}%
\pgfsetfillcolor{currentfill}%
\pgfsetfillopacity{0.600000}%
\pgfsetlinewidth{1.003750pt}%
\definecolor{currentstroke}{rgb}{0.894118,0.101961,0.109804}%
\pgfsetstrokecolor{currentstroke}%
\pgfsetstrokeopacity{0.600000}%
\pgfsetdash{}{0pt}%
\pgfpathmoveto{\pgfqpoint{1.984082in}{1.879244in}}%
\pgfpathcurveto{\pgfqpoint{1.995133in}{1.879244in}}{\pgfqpoint{2.005732in}{1.883634in}}{\pgfqpoint{2.013545in}{1.891447in}}%
\pgfpathcurveto{\pgfqpoint{2.021359in}{1.899261in}}{\pgfqpoint{2.025749in}{1.909860in}}{\pgfqpoint{2.025749in}{1.920910in}}%
\pgfpathcurveto{\pgfqpoint{2.025749in}{1.931960in}}{\pgfqpoint{2.021359in}{1.942559in}}{\pgfqpoint{2.013545in}{1.950373in}}%
\pgfpathcurveto{\pgfqpoint{2.005732in}{1.958187in}}{\pgfqpoint{1.995133in}{1.962577in}}{\pgfqpoint{1.984082in}{1.962577in}}%
\pgfpathcurveto{\pgfqpoint{1.973032in}{1.962577in}}{\pgfqpoint{1.962433in}{1.958187in}}{\pgfqpoint{1.954620in}{1.950373in}}%
\pgfpathcurveto{\pgfqpoint{1.946806in}{1.942559in}}{\pgfqpoint{1.942416in}{1.931960in}}{\pgfqpoint{1.942416in}{1.920910in}}%
\pgfpathcurveto{\pgfqpoint{1.942416in}{1.909860in}}{\pgfqpoint{1.946806in}{1.899261in}}{\pgfqpoint{1.954620in}{1.891447in}}%
\pgfpathcurveto{\pgfqpoint{1.962433in}{1.883634in}}{\pgfqpoint{1.973032in}{1.879244in}}{\pgfqpoint{1.984082in}{1.879244in}}%
\pgfpathlineto{\pgfqpoint{1.984082in}{1.879244in}}%
\pgfpathclose%
\pgfusepath{stroke,fill}%
\end{pgfscope}%
\begin{pgfscope}%
\pgfpathrectangle{\pgfqpoint{0.800000in}{0.528000in}}{\pgfqpoint{4.960000in}{3.696000in}}%
\pgfusepath{clip}%
\pgfsetbuttcap%
\pgfsetroundjoin%
\definecolor{currentfill}{rgb}{0.894118,0.101961,0.109804}%
\pgfsetfillcolor{currentfill}%
\pgfsetfillopacity{0.600000}%
\pgfsetlinewidth{1.003750pt}%
\definecolor{currentstroke}{rgb}{0.894118,0.101961,0.109804}%
\pgfsetstrokecolor{currentstroke}%
\pgfsetstrokeopacity{0.600000}%
\pgfsetdash{}{0pt}%
\pgfpathmoveto{\pgfqpoint{2.821629in}{2.443473in}}%
\pgfpathcurveto{\pgfqpoint{2.832679in}{2.443473in}}{\pgfqpoint{2.843278in}{2.447864in}}{\pgfqpoint{2.851092in}{2.455677in}}%
\pgfpathcurveto{\pgfqpoint{2.858905in}{2.463491in}}{\pgfqpoint{2.863296in}{2.474090in}}{\pgfqpoint{2.863296in}{2.485140in}}%
\pgfpathcurveto{\pgfqpoint{2.863296in}{2.496190in}}{\pgfqpoint{2.858905in}{2.506789in}}{\pgfqpoint{2.851092in}{2.514603in}}%
\pgfpathcurveto{\pgfqpoint{2.843278in}{2.522417in}}{\pgfqpoint{2.832679in}{2.526807in}}{\pgfqpoint{2.821629in}{2.526807in}}%
\pgfpathcurveto{\pgfqpoint{2.810579in}{2.526807in}}{\pgfqpoint{2.799980in}{2.522417in}}{\pgfqpoint{2.792166in}{2.514603in}}%
\pgfpathcurveto{\pgfqpoint{2.784353in}{2.506789in}}{\pgfqpoint{2.779962in}{2.496190in}}{\pgfqpoint{2.779962in}{2.485140in}}%
\pgfpathcurveto{\pgfqpoint{2.779962in}{2.474090in}}{\pgfqpoint{2.784353in}{2.463491in}}{\pgfqpoint{2.792166in}{2.455677in}}%
\pgfpathcurveto{\pgfqpoint{2.799980in}{2.447864in}}{\pgfqpoint{2.810579in}{2.443473in}}{\pgfqpoint{2.821629in}{2.443473in}}%
\pgfpathlineto{\pgfqpoint{2.821629in}{2.443473in}}%
\pgfpathclose%
\pgfusepath{stroke,fill}%
\end{pgfscope}%
\begin{pgfscope}%
\pgfpathrectangle{\pgfqpoint{0.800000in}{0.528000in}}{\pgfqpoint{4.960000in}{3.696000in}}%
\pgfusepath{clip}%
\pgfsetbuttcap%
\pgfsetroundjoin%
\definecolor{currentfill}{rgb}{0.894118,0.101961,0.109804}%
\pgfsetfillcolor{currentfill}%
\pgfsetfillopacity{0.600000}%
\pgfsetlinewidth{1.003750pt}%
\definecolor{currentstroke}{rgb}{0.894118,0.101961,0.109804}%
\pgfsetstrokecolor{currentstroke}%
\pgfsetstrokeopacity{0.600000}%
\pgfsetdash{}{0pt}%
\pgfpathmoveto{\pgfqpoint{3.111071in}{2.513338in}}%
\pgfpathcurveto{\pgfqpoint{3.122121in}{2.513338in}}{\pgfqpoint{3.132720in}{2.517729in}}{\pgfqpoint{3.140533in}{2.525542in}}%
\pgfpathcurveto{\pgfqpoint{3.148347in}{2.533356in}}{\pgfqpoint{3.152737in}{2.543955in}}{\pgfqpoint{3.152737in}{2.555005in}}%
\pgfpathcurveto{\pgfqpoint{3.152737in}{2.566055in}}{\pgfqpoint{3.148347in}{2.576654in}}{\pgfqpoint{3.140533in}{2.584468in}}%
\pgfpathcurveto{\pgfqpoint{3.132720in}{2.592281in}}{\pgfqpoint{3.122121in}{2.596672in}}{\pgfqpoint{3.111071in}{2.596672in}}%
\pgfpathcurveto{\pgfqpoint{3.100020in}{2.596672in}}{\pgfqpoint{3.089421in}{2.592281in}}{\pgfqpoint{3.081608in}{2.584468in}}%
\pgfpathcurveto{\pgfqpoint{3.073794in}{2.576654in}}{\pgfqpoint{3.069404in}{2.566055in}}{\pgfqpoint{3.069404in}{2.555005in}}%
\pgfpathcurveto{\pgfqpoint{3.069404in}{2.543955in}}{\pgfqpoint{3.073794in}{2.533356in}}{\pgfqpoint{3.081608in}{2.525542in}}%
\pgfpathcurveto{\pgfqpoint{3.089421in}{2.517729in}}{\pgfqpoint{3.100020in}{2.513338in}}{\pgfqpoint{3.111071in}{2.513338in}}%
\pgfpathlineto{\pgfqpoint{3.111071in}{2.513338in}}%
\pgfpathclose%
\pgfusepath{stroke,fill}%
\end{pgfscope}%
\begin{pgfscope}%
\pgfpathrectangle{\pgfqpoint{0.800000in}{0.528000in}}{\pgfqpoint{4.960000in}{3.696000in}}%
\pgfusepath{clip}%
\pgfsetbuttcap%
\pgfsetroundjoin%
\definecolor{currentfill}{rgb}{0.894118,0.101961,0.109804}%
\pgfsetfillcolor{currentfill}%
\pgfsetfillopacity{0.600000}%
\pgfsetlinewidth{1.003750pt}%
\definecolor{currentstroke}{rgb}{0.894118,0.101961,0.109804}%
\pgfsetstrokecolor{currentstroke}%
\pgfsetstrokeopacity{0.600000}%
\pgfsetdash{}{0pt}%
\pgfpathmoveto{\pgfqpoint{2.180581in}{1.849242in}}%
\pgfpathcurveto{\pgfqpoint{2.191631in}{1.849242in}}{\pgfqpoint{2.202230in}{1.853632in}}{\pgfqpoint{2.210043in}{1.861446in}}%
\pgfpathcurveto{\pgfqpoint{2.217857in}{1.869259in}}{\pgfqpoint{2.222247in}{1.879858in}}{\pgfqpoint{2.222247in}{1.890909in}}%
\pgfpathcurveto{\pgfqpoint{2.222247in}{1.901959in}}{\pgfqpoint{2.217857in}{1.912558in}}{\pgfqpoint{2.210043in}{1.920371in}}%
\pgfpathcurveto{\pgfqpoint{2.202230in}{1.928185in}}{\pgfqpoint{2.191631in}{1.932575in}}{\pgfqpoint{2.180581in}{1.932575in}}%
\pgfpathcurveto{\pgfqpoint{2.169530in}{1.932575in}}{\pgfqpoint{2.158931in}{1.928185in}}{\pgfqpoint{2.151118in}{1.920371in}}%
\pgfpathcurveto{\pgfqpoint{2.143304in}{1.912558in}}{\pgfqpoint{2.138914in}{1.901959in}}{\pgfqpoint{2.138914in}{1.890909in}}%
\pgfpathcurveto{\pgfqpoint{2.138914in}{1.879858in}}{\pgfqpoint{2.143304in}{1.869259in}}{\pgfqpoint{2.151118in}{1.861446in}}%
\pgfpathcurveto{\pgfqpoint{2.158931in}{1.853632in}}{\pgfqpoint{2.169530in}{1.849242in}}{\pgfqpoint{2.180581in}{1.849242in}}%
\pgfpathlineto{\pgfqpoint{2.180581in}{1.849242in}}%
\pgfpathclose%
\pgfusepath{stroke,fill}%
\end{pgfscope}%
\begin{pgfscope}%
\pgfpathrectangle{\pgfqpoint{0.800000in}{0.528000in}}{\pgfqpoint{4.960000in}{3.696000in}}%
\pgfusepath{clip}%
\pgfsetbuttcap%
\pgfsetroundjoin%
\definecolor{currentfill}{rgb}{0.894118,0.101961,0.109804}%
\pgfsetfillcolor{currentfill}%
\pgfsetfillopacity{0.600000}%
\pgfsetlinewidth{1.003750pt}%
\definecolor{currentstroke}{rgb}{0.894118,0.101961,0.109804}%
\pgfsetstrokecolor{currentstroke}%
\pgfsetstrokeopacity{0.600000}%
\pgfsetdash{}{0pt}%
\pgfpathmoveto{\pgfqpoint{2.651942in}{2.147282in}}%
\pgfpathcurveto{\pgfqpoint{2.662992in}{2.147282in}}{\pgfqpoint{2.673591in}{2.151672in}}{\pgfqpoint{2.681405in}{2.159486in}}%
\pgfpathcurveto{\pgfqpoint{2.689218in}{2.167299in}}{\pgfqpoint{2.693609in}{2.177898in}}{\pgfqpoint{2.693609in}{2.188949in}}%
\pgfpathcurveto{\pgfqpoint{2.693609in}{2.199999in}}{\pgfqpoint{2.689218in}{2.210598in}}{\pgfqpoint{2.681405in}{2.218411in}}%
\pgfpathcurveto{\pgfqpoint{2.673591in}{2.226225in}}{\pgfqpoint{2.662992in}{2.230615in}}{\pgfqpoint{2.651942in}{2.230615in}}%
\pgfpathcurveto{\pgfqpoint{2.640892in}{2.230615in}}{\pgfqpoint{2.630293in}{2.226225in}}{\pgfqpoint{2.622479in}{2.218411in}}%
\pgfpathcurveto{\pgfqpoint{2.614666in}{2.210598in}}{\pgfqpoint{2.610275in}{2.199999in}}{\pgfqpoint{2.610275in}{2.188949in}}%
\pgfpathcurveto{\pgfqpoint{2.610275in}{2.177898in}}{\pgfqpoint{2.614666in}{2.167299in}}{\pgfqpoint{2.622479in}{2.159486in}}%
\pgfpathcurveto{\pgfqpoint{2.630293in}{2.151672in}}{\pgfqpoint{2.640892in}{2.147282in}}{\pgfqpoint{2.651942in}{2.147282in}}%
\pgfpathlineto{\pgfqpoint{2.651942in}{2.147282in}}%
\pgfpathclose%
\pgfusepath{stroke,fill}%
\end{pgfscope}%
\begin{pgfscope}%
\pgfpathrectangle{\pgfqpoint{0.800000in}{0.528000in}}{\pgfqpoint{4.960000in}{3.696000in}}%
\pgfusepath{clip}%
\pgfsetbuttcap%
\pgfsetroundjoin%
\definecolor{currentfill}{rgb}{0.894118,0.101961,0.109804}%
\pgfsetfillcolor{currentfill}%
\pgfsetfillopacity{0.600000}%
\pgfsetlinewidth{1.003750pt}%
\definecolor{currentstroke}{rgb}{0.894118,0.101961,0.109804}%
\pgfsetstrokecolor{currentstroke}%
\pgfsetstrokeopacity{0.600000}%
\pgfsetdash{}{0pt}%
\pgfpathmoveto{\pgfqpoint{2.798786in}{2.206087in}}%
\pgfpathcurveto{\pgfqpoint{2.809836in}{2.206087in}}{\pgfqpoint{2.820435in}{2.210477in}}{\pgfqpoint{2.828249in}{2.218290in}}%
\pgfpathcurveto{\pgfqpoint{2.836063in}{2.226104in}}{\pgfqpoint{2.840453in}{2.236703in}}{\pgfqpoint{2.840453in}{2.247753in}}%
\pgfpathcurveto{\pgfqpoint{2.840453in}{2.258803in}}{\pgfqpoint{2.836063in}{2.269402in}}{\pgfqpoint{2.828249in}{2.277216in}}%
\pgfpathcurveto{\pgfqpoint{2.820435in}{2.285030in}}{\pgfqpoint{2.809836in}{2.289420in}}{\pgfqpoint{2.798786in}{2.289420in}}%
\pgfpathcurveto{\pgfqpoint{2.787736in}{2.289420in}}{\pgfqpoint{2.777137in}{2.285030in}}{\pgfqpoint{2.769324in}{2.277216in}}%
\pgfpathcurveto{\pgfqpoint{2.761510in}{2.269402in}}{\pgfqpoint{2.757120in}{2.258803in}}{\pgfqpoint{2.757120in}{2.247753in}}%
\pgfpathcurveto{\pgfqpoint{2.757120in}{2.236703in}}{\pgfqpoint{2.761510in}{2.226104in}}{\pgfqpoint{2.769324in}{2.218290in}}%
\pgfpathcurveto{\pgfqpoint{2.777137in}{2.210477in}}{\pgfqpoint{2.787736in}{2.206087in}}{\pgfqpoint{2.798786in}{2.206087in}}%
\pgfpathlineto{\pgfqpoint{2.798786in}{2.206087in}}%
\pgfpathclose%
\pgfusepath{stroke,fill}%
\end{pgfscope}%
\begin{pgfscope}%
\pgfpathrectangle{\pgfqpoint{0.800000in}{0.528000in}}{\pgfqpoint{4.960000in}{3.696000in}}%
\pgfusepath{clip}%
\pgfsetbuttcap%
\pgfsetroundjoin%
\definecolor{currentfill}{rgb}{0.894118,0.101961,0.109804}%
\pgfsetfillcolor{currentfill}%
\pgfsetfillopacity{0.600000}%
\pgfsetlinewidth{1.003750pt}%
\definecolor{currentstroke}{rgb}{0.894118,0.101961,0.109804}%
\pgfsetstrokecolor{currentstroke}%
\pgfsetstrokeopacity{0.600000}%
\pgfsetdash{}{0pt}%
\pgfpathmoveto{\pgfqpoint{2.300392in}{1.808887in}}%
\pgfpathcurveto{\pgfqpoint{2.311442in}{1.808887in}}{\pgfqpoint{2.322041in}{1.813278in}}{\pgfqpoint{2.329855in}{1.821091in}}%
\pgfpathcurveto{\pgfqpoint{2.337668in}{1.828905in}}{\pgfqpoint{2.342058in}{1.839504in}}{\pgfqpoint{2.342058in}{1.850554in}}%
\pgfpathcurveto{\pgfqpoint{2.342058in}{1.861604in}}{\pgfqpoint{2.337668in}{1.872203in}}{\pgfqpoint{2.329855in}{1.880017in}}%
\pgfpathcurveto{\pgfqpoint{2.322041in}{1.887831in}}{\pgfqpoint{2.311442in}{1.892221in}}{\pgfqpoint{2.300392in}{1.892221in}}%
\pgfpathcurveto{\pgfqpoint{2.289342in}{1.892221in}}{\pgfqpoint{2.278743in}{1.887831in}}{\pgfqpoint{2.270929in}{1.880017in}}%
\pgfpathcurveto{\pgfqpoint{2.263115in}{1.872203in}}{\pgfqpoint{2.258725in}{1.861604in}}{\pgfqpoint{2.258725in}{1.850554in}}%
\pgfpathcurveto{\pgfqpoint{2.258725in}{1.839504in}}{\pgfqpoint{2.263115in}{1.828905in}}{\pgfqpoint{2.270929in}{1.821091in}}%
\pgfpathcurveto{\pgfqpoint{2.278743in}{1.813278in}}{\pgfqpoint{2.289342in}{1.808887in}}{\pgfqpoint{2.300392in}{1.808887in}}%
\pgfpathlineto{\pgfqpoint{2.300392in}{1.808887in}}%
\pgfpathclose%
\pgfusepath{stroke,fill}%
\end{pgfscope}%
\begin{pgfscope}%
\pgfpathrectangle{\pgfqpoint{0.800000in}{0.528000in}}{\pgfqpoint{4.960000in}{3.696000in}}%
\pgfusepath{clip}%
\pgfsetbuttcap%
\pgfsetroundjoin%
\definecolor{currentfill}{rgb}{0.894118,0.101961,0.109804}%
\pgfsetfillcolor{currentfill}%
\pgfsetfillopacity{0.600000}%
\pgfsetlinewidth{1.003750pt}%
\definecolor{currentstroke}{rgb}{0.894118,0.101961,0.109804}%
\pgfsetstrokecolor{currentstroke}%
\pgfsetstrokeopacity{0.600000}%
\pgfsetdash{}{0pt}%
\pgfpathmoveto{\pgfqpoint{2.739933in}{2.061239in}}%
\pgfpathcurveto{\pgfqpoint{2.750984in}{2.061239in}}{\pgfqpoint{2.761583in}{2.065629in}}{\pgfqpoint{2.769396in}{2.073443in}}%
\pgfpathcurveto{\pgfqpoint{2.777210in}{2.081257in}}{\pgfqpoint{2.781600in}{2.091856in}}{\pgfqpoint{2.781600in}{2.102906in}}%
\pgfpathcurveto{\pgfqpoint{2.781600in}{2.113956in}}{\pgfqpoint{2.777210in}{2.124555in}}{\pgfqpoint{2.769396in}{2.132369in}}%
\pgfpathcurveto{\pgfqpoint{2.761583in}{2.140182in}}{\pgfqpoint{2.750984in}{2.144573in}}{\pgfqpoint{2.739933in}{2.144573in}}%
\pgfpathcurveto{\pgfqpoint{2.728883in}{2.144573in}}{\pgfqpoint{2.718284in}{2.140182in}}{\pgfqpoint{2.710471in}{2.132369in}}%
\pgfpathcurveto{\pgfqpoint{2.702657in}{2.124555in}}{\pgfqpoint{2.698267in}{2.113956in}}{\pgfqpoint{2.698267in}{2.102906in}}%
\pgfpathcurveto{\pgfqpoint{2.698267in}{2.091856in}}{\pgfqpoint{2.702657in}{2.081257in}}{\pgfqpoint{2.710471in}{2.073443in}}%
\pgfpathcurveto{\pgfqpoint{2.718284in}{2.065629in}}{\pgfqpoint{2.728883in}{2.061239in}}{\pgfqpoint{2.739933in}{2.061239in}}%
\pgfpathlineto{\pgfqpoint{2.739933in}{2.061239in}}%
\pgfpathclose%
\pgfusepath{stroke,fill}%
\end{pgfscope}%
\begin{pgfscope}%
\pgfpathrectangle{\pgfqpoint{0.800000in}{0.528000in}}{\pgfqpoint{4.960000in}{3.696000in}}%
\pgfusepath{clip}%
\pgfsetbuttcap%
\pgfsetroundjoin%
\definecolor{currentfill}{rgb}{0.894118,0.101961,0.109804}%
\pgfsetfillcolor{currentfill}%
\pgfsetfillopacity{0.600000}%
\pgfsetlinewidth{1.003750pt}%
\definecolor{currentstroke}{rgb}{0.894118,0.101961,0.109804}%
\pgfsetstrokecolor{currentstroke}%
\pgfsetstrokeopacity{0.600000}%
\pgfsetdash{}{0pt}%
\pgfpathmoveto{\pgfqpoint{2.573079in}{1.928671in}}%
\pgfpathcurveto{\pgfqpoint{2.584130in}{1.928671in}}{\pgfqpoint{2.594729in}{1.933061in}}{\pgfqpoint{2.602542in}{1.940874in}}%
\pgfpathcurveto{\pgfqpoint{2.610356in}{1.948688in}}{\pgfqpoint{2.614746in}{1.959287in}}{\pgfqpoint{2.614746in}{1.970337in}}%
\pgfpathcurveto{\pgfqpoint{2.614746in}{1.981387in}}{\pgfqpoint{2.610356in}{1.991986in}}{\pgfqpoint{2.602542in}{1.999800in}}%
\pgfpathcurveto{\pgfqpoint{2.594729in}{2.007614in}}{\pgfqpoint{2.584130in}{2.012004in}}{\pgfqpoint{2.573079in}{2.012004in}}%
\pgfpathcurveto{\pgfqpoint{2.562029in}{2.012004in}}{\pgfqpoint{2.551430in}{2.007614in}}{\pgfqpoint{2.543617in}{1.999800in}}%
\pgfpathcurveto{\pgfqpoint{2.535803in}{1.991986in}}{\pgfqpoint{2.531413in}{1.981387in}}{\pgfqpoint{2.531413in}{1.970337in}}%
\pgfpathcurveto{\pgfqpoint{2.531413in}{1.959287in}}{\pgfqpoint{2.535803in}{1.948688in}}{\pgfqpoint{2.543617in}{1.940874in}}%
\pgfpathcurveto{\pgfqpoint{2.551430in}{1.933061in}}{\pgfqpoint{2.562029in}{1.928671in}}{\pgfqpoint{2.573079in}{1.928671in}}%
\pgfpathlineto{\pgfqpoint{2.573079in}{1.928671in}}%
\pgfpathclose%
\pgfusepath{stroke,fill}%
\end{pgfscope}%
\begin{pgfscope}%
\pgfpathrectangle{\pgfqpoint{0.800000in}{0.528000in}}{\pgfqpoint{4.960000in}{3.696000in}}%
\pgfusepath{clip}%
\pgfsetbuttcap%
\pgfsetroundjoin%
\definecolor{currentfill}{rgb}{0.894118,0.101961,0.109804}%
\pgfsetfillcolor{currentfill}%
\pgfsetfillopacity{0.600000}%
\pgfsetlinewidth{1.003750pt}%
\definecolor{currentstroke}{rgb}{0.894118,0.101961,0.109804}%
\pgfsetstrokecolor{currentstroke}%
\pgfsetstrokeopacity{0.600000}%
\pgfsetdash{}{0pt}%
\pgfpathmoveto{\pgfqpoint{2.416835in}{1.810771in}}%
\pgfpathcurveto{\pgfqpoint{2.427885in}{1.810771in}}{\pgfqpoint{2.438484in}{1.815162in}}{\pgfqpoint{2.446297in}{1.822975in}}%
\pgfpathcurveto{\pgfqpoint{2.454111in}{1.830789in}}{\pgfqpoint{2.458501in}{1.841388in}}{\pgfqpoint{2.458501in}{1.852438in}}%
\pgfpathcurveto{\pgfqpoint{2.458501in}{1.863488in}}{\pgfqpoint{2.454111in}{1.874087in}}{\pgfqpoint{2.446297in}{1.881901in}}%
\pgfpathcurveto{\pgfqpoint{2.438484in}{1.889715in}}{\pgfqpoint{2.427885in}{1.894105in}}{\pgfqpoint{2.416835in}{1.894105in}}%
\pgfpathcurveto{\pgfqpoint{2.405785in}{1.894105in}}{\pgfqpoint{2.395185in}{1.889715in}}{\pgfqpoint{2.387372in}{1.881901in}}%
\pgfpathcurveto{\pgfqpoint{2.379558in}{1.874087in}}{\pgfqpoint{2.375168in}{1.863488in}}{\pgfqpoint{2.375168in}{1.852438in}}%
\pgfpathcurveto{\pgfqpoint{2.375168in}{1.841388in}}{\pgfqpoint{2.379558in}{1.830789in}}{\pgfqpoint{2.387372in}{1.822975in}}%
\pgfpathcurveto{\pgfqpoint{2.395185in}{1.815162in}}{\pgfqpoint{2.405785in}{1.810771in}}{\pgfqpoint{2.416835in}{1.810771in}}%
\pgfpathlineto{\pgfqpoint{2.416835in}{1.810771in}}%
\pgfpathclose%
\pgfusepath{stroke,fill}%
\end{pgfscope}%
\begin{pgfscope}%
\pgfpathrectangle{\pgfqpoint{0.800000in}{0.528000in}}{\pgfqpoint{4.960000in}{3.696000in}}%
\pgfusepath{clip}%
\pgfsetbuttcap%
\pgfsetroundjoin%
\definecolor{currentfill}{rgb}{0.894118,0.101961,0.109804}%
\pgfsetfillcolor{currentfill}%
\pgfsetfillopacity{0.600000}%
\pgfsetlinewidth{1.003750pt}%
\definecolor{currentstroke}{rgb}{0.894118,0.101961,0.109804}%
\pgfsetstrokecolor{currentstroke}%
\pgfsetstrokeopacity{0.600000}%
\pgfsetdash{}{0pt}%
\pgfpathmoveto{\pgfqpoint{3.084992in}{2.224441in}}%
\pgfpathcurveto{\pgfqpoint{3.096042in}{2.224441in}}{\pgfqpoint{3.106641in}{2.228832in}}{\pgfqpoint{3.114455in}{2.236645in}}%
\pgfpathcurveto{\pgfqpoint{3.122269in}{2.244459in}}{\pgfqpoint{3.126659in}{2.255058in}}{\pgfqpoint{3.126659in}{2.266108in}}%
\pgfpathcurveto{\pgfqpoint{3.126659in}{2.277158in}}{\pgfqpoint{3.122269in}{2.287757in}}{\pgfqpoint{3.114455in}{2.295571in}}%
\pgfpathcurveto{\pgfqpoint{3.106641in}{2.303384in}}{\pgfqpoint{3.096042in}{2.307775in}}{\pgfqpoint{3.084992in}{2.307775in}}%
\pgfpathcurveto{\pgfqpoint{3.073942in}{2.307775in}}{\pgfqpoint{3.063343in}{2.303384in}}{\pgfqpoint{3.055529in}{2.295571in}}%
\pgfpathcurveto{\pgfqpoint{3.047716in}{2.287757in}}{\pgfqpoint{3.043326in}{2.277158in}}{\pgfqpoint{3.043326in}{2.266108in}}%
\pgfpathcurveto{\pgfqpoint{3.043326in}{2.255058in}}{\pgfqpoint{3.047716in}{2.244459in}}{\pgfqpoint{3.055529in}{2.236645in}}%
\pgfpathcurveto{\pgfqpoint{3.063343in}{2.228832in}}{\pgfqpoint{3.073942in}{2.224441in}}{\pgfqpoint{3.084992in}{2.224441in}}%
\pgfpathlineto{\pgfqpoint{3.084992in}{2.224441in}}%
\pgfpathclose%
\pgfusepath{stroke,fill}%
\end{pgfscope}%
\begin{pgfscope}%
\pgfpathrectangle{\pgfqpoint{0.800000in}{0.528000in}}{\pgfqpoint{4.960000in}{3.696000in}}%
\pgfusepath{clip}%
\pgfsetbuttcap%
\pgfsetroundjoin%
\definecolor{currentfill}{rgb}{0.894118,0.101961,0.109804}%
\pgfsetfillcolor{currentfill}%
\pgfsetfillopacity{0.600000}%
\pgfsetlinewidth{1.003750pt}%
\definecolor{currentstroke}{rgb}{0.894118,0.101961,0.109804}%
\pgfsetstrokecolor{currentstroke}%
\pgfsetstrokeopacity{0.600000}%
\pgfsetdash{}{0pt}%
\pgfpathmoveto{\pgfqpoint{2.034679in}{1.493594in}}%
\pgfpathcurveto{\pgfqpoint{2.045729in}{1.493594in}}{\pgfqpoint{2.056328in}{1.497984in}}{\pgfqpoint{2.064142in}{1.505797in}}%
\pgfpathcurveto{\pgfqpoint{2.071956in}{1.513611in}}{\pgfqpoint{2.076346in}{1.524210in}}{\pgfqpoint{2.076346in}{1.535260in}}%
\pgfpathcurveto{\pgfqpoint{2.076346in}{1.546310in}}{\pgfqpoint{2.071956in}{1.556909in}}{\pgfqpoint{2.064142in}{1.564723in}}%
\pgfpathcurveto{\pgfqpoint{2.056328in}{1.572537in}}{\pgfqpoint{2.045729in}{1.576927in}}{\pgfqpoint{2.034679in}{1.576927in}}%
\pgfpathcurveto{\pgfqpoint{2.023629in}{1.576927in}}{\pgfqpoint{2.013030in}{1.572537in}}{\pgfqpoint{2.005216in}{1.564723in}}%
\pgfpathcurveto{\pgfqpoint{1.997403in}{1.556909in}}{\pgfqpoint{1.993013in}{1.546310in}}{\pgfqpoint{1.993013in}{1.535260in}}%
\pgfpathcurveto{\pgfqpoint{1.993013in}{1.524210in}}{\pgfqpoint{1.997403in}{1.513611in}}{\pgfqpoint{2.005216in}{1.505797in}}%
\pgfpathcurveto{\pgfqpoint{2.013030in}{1.497984in}}{\pgfqpoint{2.023629in}{1.493594in}}{\pgfqpoint{2.034679in}{1.493594in}}%
\pgfpathlineto{\pgfqpoint{2.034679in}{1.493594in}}%
\pgfpathclose%
\pgfusepath{stroke,fill}%
\end{pgfscope}%
\begin{pgfscope}%
\pgfpathrectangle{\pgfqpoint{0.800000in}{0.528000in}}{\pgfqpoint{4.960000in}{3.696000in}}%
\pgfusepath{clip}%
\pgfsetbuttcap%
\pgfsetroundjoin%
\definecolor{currentfill}{rgb}{0.894118,0.101961,0.109804}%
\pgfsetfillcolor{currentfill}%
\pgfsetfillopacity{0.600000}%
\pgfsetlinewidth{1.003750pt}%
\definecolor{currentstroke}{rgb}{0.894118,0.101961,0.109804}%
\pgfsetstrokecolor{currentstroke}%
\pgfsetstrokeopacity{0.600000}%
\pgfsetdash{}{0pt}%
\pgfpathmoveto{\pgfqpoint{3.128122in}{2.157959in}}%
\pgfpathcurveto{\pgfqpoint{3.139172in}{2.157959in}}{\pgfqpoint{3.149771in}{2.162350in}}{\pgfqpoint{3.157585in}{2.170163in}}%
\pgfpathcurveto{\pgfqpoint{3.165399in}{2.177977in}}{\pgfqpoint{3.169789in}{2.188576in}}{\pgfqpoint{3.169789in}{2.199626in}}%
\pgfpathcurveto{\pgfqpoint{3.169789in}{2.210676in}}{\pgfqpoint{3.165399in}{2.221275in}}{\pgfqpoint{3.157585in}{2.229089in}}%
\pgfpathcurveto{\pgfqpoint{3.149771in}{2.236903in}}{\pgfqpoint{3.139172in}{2.241293in}}{\pgfqpoint{3.128122in}{2.241293in}}%
\pgfpathcurveto{\pgfqpoint{3.117072in}{2.241293in}}{\pgfqpoint{3.106473in}{2.236903in}}{\pgfqpoint{3.098659in}{2.229089in}}%
\pgfpathcurveto{\pgfqpoint{3.090846in}{2.221275in}}{\pgfqpoint{3.086455in}{2.210676in}}{\pgfqpoint{3.086455in}{2.199626in}}%
\pgfpathcurveto{\pgfqpoint{3.086455in}{2.188576in}}{\pgfqpoint{3.090846in}{2.177977in}}{\pgfqpoint{3.098659in}{2.170163in}}%
\pgfpathcurveto{\pgfqpoint{3.106473in}{2.162350in}}{\pgfqpoint{3.117072in}{2.157959in}}{\pgfqpoint{3.128122in}{2.157959in}}%
\pgfpathlineto{\pgfqpoint{3.128122in}{2.157959in}}%
\pgfpathclose%
\pgfusepath{stroke,fill}%
\end{pgfscope}%
\begin{pgfscope}%
\pgfpathrectangle{\pgfqpoint{0.800000in}{0.528000in}}{\pgfqpoint{4.960000in}{3.696000in}}%
\pgfusepath{clip}%
\pgfsetbuttcap%
\pgfsetroundjoin%
\definecolor{currentfill}{rgb}{0.894118,0.101961,0.109804}%
\pgfsetfillcolor{currentfill}%
\pgfsetfillopacity{0.600000}%
\pgfsetlinewidth{1.003750pt}%
\definecolor{currentstroke}{rgb}{0.894118,0.101961,0.109804}%
\pgfsetstrokecolor{currentstroke}%
\pgfsetstrokeopacity{0.600000}%
\pgfsetdash{}{0pt}%
\pgfpathmoveto{\pgfqpoint{3.263666in}{2.225380in}}%
\pgfpathcurveto{\pgfqpoint{3.274716in}{2.225380in}}{\pgfqpoint{3.285315in}{2.229770in}}{\pgfqpoint{3.293128in}{2.237583in}}%
\pgfpathcurveto{\pgfqpoint{3.300942in}{2.245397in}}{\pgfqpoint{3.305332in}{2.255996in}}{\pgfqpoint{3.305332in}{2.267046in}}%
\pgfpathcurveto{\pgfqpoint{3.305332in}{2.278096in}}{\pgfqpoint{3.300942in}{2.288695in}}{\pgfqpoint{3.293128in}{2.296509in}}%
\pgfpathcurveto{\pgfqpoint{3.285315in}{2.304323in}}{\pgfqpoint{3.274716in}{2.308713in}}{\pgfqpoint{3.263666in}{2.308713in}}%
\pgfpathcurveto{\pgfqpoint{3.252615in}{2.308713in}}{\pgfqpoint{3.242016in}{2.304323in}}{\pgfqpoint{3.234203in}{2.296509in}}%
\pgfpathcurveto{\pgfqpoint{3.226389in}{2.288695in}}{\pgfqpoint{3.221999in}{2.278096in}}{\pgfqpoint{3.221999in}{2.267046in}}%
\pgfpathcurveto{\pgfqpoint{3.221999in}{2.255996in}}{\pgfqpoint{3.226389in}{2.245397in}}{\pgfqpoint{3.234203in}{2.237583in}}%
\pgfpathcurveto{\pgfqpoint{3.242016in}{2.229770in}}{\pgfqpoint{3.252615in}{2.225380in}}{\pgfqpoint{3.263666in}{2.225380in}}%
\pgfpathlineto{\pgfqpoint{3.263666in}{2.225380in}}%
\pgfpathclose%
\pgfusepath{stroke,fill}%
\end{pgfscope}%
\begin{pgfscope}%
\pgfsetbuttcap%
\pgfsetroundjoin%
\definecolor{currentfill}{rgb}{0.000000,0.000000,0.000000}%
\pgfsetfillcolor{currentfill}%
\pgfsetlinewidth{0.803000pt}%
\definecolor{currentstroke}{rgb}{0.000000,0.000000,0.000000}%
\pgfsetstrokecolor{currentstroke}%
\pgfsetdash{}{0pt}%
\pgfsys@defobject{currentmarker}{\pgfqpoint{0.000000in}{-0.048611in}}{\pgfqpoint{0.000000in}{0.000000in}}{%
\pgfpathmoveto{\pgfqpoint{0.000000in}{0.000000in}}%
\pgfpathlineto{\pgfqpoint{0.000000in}{-0.048611in}}%
\pgfusepath{stroke,fill}%
}%
\begin{pgfscope}%
\pgfsys@transformshift{1.153507in}{0.528000in}%
\pgfsys@useobject{currentmarker}{}%
\end{pgfscope}%
\end{pgfscope}%
\begin{pgfscope}%
\definecolor{textcolor}{rgb}{0.000000,0.000000,0.000000}%
\pgfsetstrokecolor{textcolor}%
\pgfsetfillcolor{textcolor}%
\pgftext[x=1.153507in,y=0.430778in,,top]{\color{textcolor}{\rmfamily\fontsize{10.000000}{12.000000}\selectfont\catcode`\^=\active\def^{\ifmmode\sp\else\^{}\fi}\catcode`\%=\active\def%{\%}$\mathdefault{0.05}$}}%
\end{pgfscope}%
\begin{pgfscope}%
\pgfsetbuttcap%
\pgfsetroundjoin%
\definecolor{currentfill}{rgb}{0.000000,0.000000,0.000000}%
\pgfsetfillcolor{currentfill}%
\pgfsetlinewidth{0.803000pt}%
\definecolor{currentstroke}{rgb}{0.000000,0.000000,0.000000}%
\pgfsetstrokecolor{currentstroke}%
\pgfsetdash{}{0pt}%
\pgfsys@defobject{currentmarker}{\pgfqpoint{0.000000in}{-0.048611in}}{\pgfqpoint{0.000000in}{0.000000in}}{%
\pgfpathmoveto{\pgfqpoint{0.000000in}{0.000000in}}%
\pgfpathlineto{\pgfqpoint{0.000000in}{-0.048611in}}%
\pgfusepath{stroke,fill}%
}%
\begin{pgfscope}%
\pgfsys@transformshift{2.060517in}{0.528000in}%
\pgfsys@useobject{currentmarker}{}%
\end{pgfscope}%
\end{pgfscope}%
\begin{pgfscope}%
\definecolor{textcolor}{rgb}{0.000000,0.000000,0.000000}%
\pgfsetstrokecolor{textcolor}%
\pgfsetfillcolor{textcolor}%
\pgftext[x=2.060517in,y=0.430778in,,top]{\color{textcolor}{\rmfamily\fontsize{10.000000}{12.000000}\selectfont\catcode`\^=\active\def^{\ifmmode\sp\else\^{}\fi}\catcode`\%=\active\def%{\%}$\mathdefault{0.10}$}}%
\end{pgfscope}%
\begin{pgfscope}%
\pgfsetbuttcap%
\pgfsetroundjoin%
\definecolor{currentfill}{rgb}{0.000000,0.000000,0.000000}%
\pgfsetfillcolor{currentfill}%
\pgfsetlinewidth{0.803000pt}%
\definecolor{currentstroke}{rgb}{0.000000,0.000000,0.000000}%
\pgfsetstrokecolor{currentstroke}%
\pgfsetdash{}{0pt}%
\pgfsys@defobject{currentmarker}{\pgfqpoint{0.000000in}{-0.048611in}}{\pgfqpoint{0.000000in}{0.000000in}}{%
\pgfpathmoveto{\pgfqpoint{0.000000in}{0.000000in}}%
\pgfpathlineto{\pgfqpoint{0.000000in}{-0.048611in}}%
\pgfusepath{stroke,fill}%
}%
\begin{pgfscope}%
\pgfsys@transformshift{2.967528in}{0.528000in}%
\pgfsys@useobject{currentmarker}{}%
\end{pgfscope}%
\end{pgfscope}%
\begin{pgfscope}%
\definecolor{textcolor}{rgb}{0.000000,0.000000,0.000000}%
\pgfsetstrokecolor{textcolor}%
\pgfsetfillcolor{textcolor}%
\pgftext[x=2.967528in,y=0.430778in,,top]{\color{textcolor}{\rmfamily\fontsize{10.000000}{12.000000}\selectfont\catcode`\^=\active\def^{\ifmmode\sp\else\^{}\fi}\catcode`\%=\active\def%{\%}$\mathdefault{0.15}$}}%
\end{pgfscope}%
\begin{pgfscope}%
\pgfsetbuttcap%
\pgfsetroundjoin%
\definecolor{currentfill}{rgb}{0.000000,0.000000,0.000000}%
\pgfsetfillcolor{currentfill}%
\pgfsetlinewidth{0.803000pt}%
\definecolor{currentstroke}{rgb}{0.000000,0.000000,0.000000}%
\pgfsetstrokecolor{currentstroke}%
\pgfsetdash{}{0pt}%
\pgfsys@defobject{currentmarker}{\pgfqpoint{0.000000in}{-0.048611in}}{\pgfqpoint{0.000000in}{0.000000in}}{%
\pgfpathmoveto{\pgfqpoint{0.000000in}{0.000000in}}%
\pgfpathlineto{\pgfqpoint{0.000000in}{-0.048611in}}%
\pgfusepath{stroke,fill}%
}%
\begin{pgfscope}%
\pgfsys@transformshift{3.874538in}{0.528000in}%
\pgfsys@useobject{currentmarker}{}%
\end{pgfscope}%
\end{pgfscope}%
\begin{pgfscope}%
\definecolor{textcolor}{rgb}{0.000000,0.000000,0.000000}%
\pgfsetstrokecolor{textcolor}%
\pgfsetfillcolor{textcolor}%
\pgftext[x=3.874538in,y=0.430778in,,top]{\color{textcolor}{\rmfamily\fontsize{10.000000}{12.000000}\selectfont\catcode`\^=\active\def^{\ifmmode\sp\else\^{}\fi}\catcode`\%=\active\def%{\%}$\mathdefault{0.20}$}}%
\end{pgfscope}%
\begin{pgfscope}%
\pgfsetbuttcap%
\pgfsetroundjoin%
\definecolor{currentfill}{rgb}{0.000000,0.000000,0.000000}%
\pgfsetfillcolor{currentfill}%
\pgfsetlinewidth{0.803000pt}%
\definecolor{currentstroke}{rgb}{0.000000,0.000000,0.000000}%
\pgfsetstrokecolor{currentstroke}%
\pgfsetdash{}{0pt}%
\pgfsys@defobject{currentmarker}{\pgfqpoint{0.000000in}{-0.048611in}}{\pgfqpoint{0.000000in}{0.000000in}}{%
\pgfpathmoveto{\pgfqpoint{0.000000in}{0.000000in}}%
\pgfpathlineto{\pgfqpoint{0.000000in}{-0.048611in}}%
\pgfusepath{stroke,fill}%
}%
\begin{pgfscope}%
\pgfsys@transformshift{4.781549in}{0.528000in}%
\pgfsys@useobject{currentmarker}{}%
\end{pgfscope}%
\end{pgfscope}%
\begin{pgfscope}%
\definecolor{textcolor}{rgb}{0.000000,0.000000,0.000000}%
\pgfsetstrokecolor{textcolor}%
\pgfsetfillcolor{textcolor}%
\pgftext[x=4.781549in,y=0.430778in,,top]{\color{textcolor}{\rmfamily\fontsize{10.000000}{12.000000}\selectfont\catcode`\^=\active\def^{\ifmmode\sp\else\^{}\fi}\catcode`\%=\active\def%{\%}$\mathdefault{0.25}$}}%
\end{pgfscope}%
\begin{pgfscope}%
\pgfsetbuttcap%
\pgfsetroundjoin%
\definecolor{currentfill}{rgb}{0.000000,0.000000,0.000000}%
\pgfsetfillcolor{currentfill}%
\pgfsetlinewidth{0.803000pt}%
\definecolor{currentstroke}{rgb}{0.000000,0.000000,0.000000}%
\pgfsetstrokecolor{currentstroke}%
\pgfsetdash{}{0pt}%
\pgfsys@defobject{currentmarker}{\pgfqpoint{0.000000in}{-0.048611in}}{\pgfqpoint{0.000000in}{0.000000in}}{%
\pgfpathmoveto{\pgfqpoint{0.000000in}{0.000000in}}%
\pgfpathlineto{\pgfqpoint{0.000000in}{-0.048611in}}%
\pgfusepath{stroke,fill}%
}%
\begin{pgfscope}%
\pgfsys@transformshift{5.688559in}{0.528000in}%
\pgfsys@useobject{currentmarker}{}%
\end{pgfscope}%
\end{pgfscope}%
\begin{pgfscope}%
\definecolor{textcolor}{rgb}{0.000000,0.000000,0.000000}%
\pgfsetstrokecolor{textcolor}%
\pgfsetfillcolor{textcolor}%
\pgftext[x=5.688559in,y=0.430778in,,top]{\color{textcolor}{\rmfamily\fontsize{10.000000}{12.000000}\selectfont\catcode`\^=\active\def^{\ifmmode\sp\else\^{}\fi}\catcode`\%=\active\def%{\%}$\mathdefault{0.30}$}}%
\end{pgfscope}%
\begin{pgfscope}%
\definecolor{textcolor}{rgb}{0.000000,0.000000,0.000000}%
\pgfsetstrokecolor{textcolor}%
\pgfsetfillcolor{textcolor}%
\pgftext[x=3.280000in,y=0.240809in,,top]{\color{textcolor}{\rmfamily\fontsize{16.000000}{19.200000}\selectfont\catcode`\^=\active\def^{\ifmmode\sp\else\^{}\fi}\catcode`\%=\active\def%{\%}Birth}}%
\end{pgfscope}%
\begin{pgfscope}%
\pgfsetbuttcap%
\pgfsetroundjoin%
\definecolor{currentfill}{rgb}{0.000000,0.000000,0.000000}%
\pgfsetfillcolor{currentfill}%
\pgfsetlinewidth{0.803000pt}%
\definecolor{currentstroke}{rgb}{0.000000,0.000000,0.000000}%
\pgfsetstrokecolor{currentstroke}%
\pgfsetdash{}{0pt}%
\pgfsys@defobject{currentmarker}{\pgfqpoint{-0.048611in}{0.000000in}}{\pgfqpoint{-0.000000in}{0.000000in}}{%
\pgfpathmoveto{\pgfqpoint{-0.000000in}{0.000000in}}%
\pgfpathlineto{\pgfqpoint{-0.048611in}{0.000000in}}%
\pgfusepath{stroke,fill}%
}%
\begin{pgfscope}%
\pgfsys@transformshift{0.800000in}{0.770346in}%
\pgfsys@useobject{currentmarker}{}%
\end{pgfscope}%
\end{pgfscope}%
\begin{pgfscope}%
\definecolor{textcolor}{rgb}{0.000000,0.000000,0.000000}%
\pgfsetstrokecolor{textcolor}%
\pgfsetfillcolor{textcolor}%
\pgftext[x=0.305168in, y=0.717584in, left, base]{\color{textcolor}{\rmfamily\fontsize{10.000000}{12.000000}\selectfont\catcode`\^=\active\def^{\ifmmode\sp\else\^{}\fi}\catcode`\%=\active\def%{\%}0.050}}%
\end{pgfscope}%
\begin{pgfscope}%
\pgfsetbuttcap%
\pgfsetroundjoin%
\definecolor{currentfill}{rgb}{0.000000,0.000000,0.000000}%
\pgfsetfillcolor{currentfill}%
\pgfsetlinewidth{0.803000pt}%
\definecolor{currentstroke}{rgb}{0.000000,0.000000,0.000000}%
\pgfsetstrokecolor{currentstroke}%
\pgfsetdash{}{0pt}%
\pgfsys@defobject{currentmarker}{\pgfqpoint{-0.048611in}{0.000000in}}{\pgfqpoint{-0.000000in}{0.000000in}}{%
\pgfpathmoveto{\pgfqpoint{-0.000000in}{0.000000in}}%
\pgfpathlineto{\pgfqpoint{-0.048611in}{0.000000in}}%
\pgfusepath{stroke,fill}%
}%
\begin{pgfscope}%
\pgfsys@transformshift{0.800000in}{1.392146in}%
\pgfsys@useobject{currentmarker}{}%
\end{pgfscope}%
\end{pgfscope}%
\begin{pgfscope}%
\definecolor{textcolor}{rgb}{0.000000,0.000000,0.000000}%
\pgfsetstrokecolor{textcolor}%
\pgfsetfillcolor{textcolor}%
\pgftext[x=0.305168in, y=1.339384in, left, base]{\color{textcolor}{\rmfamily\fontsize{10.000000}{12.000000}\selectfont\catcode`\^=\active\def^{\ifmmode\sp\else\^{}\fi}\catcode`\%=\active\def%{\%}0.100}}%
\end{pgfscope}%
\begin{pgfscope}%
\pgfsetbuttcap%
\pgfsetroundjoin%
\definecolor{currentfill}{rgb}{0.000000,0.000000,0.000000}%
\pgfsetfillcolor{currentfill}%
\pgfsetlinewidth{0.803000pt}%
\definecolor{currentstroke}{rgb}{0.000000,0.000000,0.000000}%
\pgfsetstrokecolor{currentstroke}%
\pgfsetdash{}{0pt}%
\pgfsys@defobject{currentmarker}{\pgfqpoint{-0.048611in}{0.000000in}}{\pgfqpoint{-0.000000in}{0.000000in}}{%
\pgfpathmoveto{\pgfqpoint{-0.000000in}{0.000000in}}%
\pgfpathlineto{\pgfqpoint{-0.048611in}{0.000000in}}%
\pgfusepath{stroke,fill}%
}%
\begin{pgfscope}%
\pgfsys@transformshift{0.800000in}{2.013945in}%
\pgfsys@useobject{currentmarker}{}%
\end{pgfscope}%
\end{pgfscope}%
\begin{pgfscope}%
\definecolor{textcolor}{rgb}{0.000000,0.000000,0.000000}%
\pgfsetstrokecolor{textcolor}%
\pgfsetfillcolor{textcolor}%
\pgftext[x=0.305168in, y=1.961184in, left, base]{\color{textcolor}{\rmfamily\fontsize{10.000000}{12.000000}\selectfont\catcode`\^=\active\def^{\ifmmode\sp\else\^{}\fi}\catcode`\%=\active\def%{\%}0.150}}%
\end{pgfscope}%
\begin{pgfscope}%
\pgfsetbuttcap%
\pgfsetroundjoin%
\definecolor{currentfill}{rgb}{0.000000,0.000000,0.000000}%
\pgfsetfillcolor{currentfill}%
\pgfsetlinewidth{0.803000pt}%
\definecolor{currentstroke}{rgb}{0.000000,0.000000,0.000000}%
\pgfsetstrokecolor{currentstroke}%
\pgfsetdash{}{0pt}%
\pgfsys@defobject{currentmarker}{\pgfqpoint{-0.048611in}{0.000000in}}{\pgfqpoint{-0.000000in}{0.000000in}}{%
\pgfpathmoveto{\pgfqpoint{-0.000000in}{0.000000in}}%
\pgfpathlineto{\pgfqpoint{-0.048611in}{0.000000in}}%
\pgfusepath{stroke,fill}%
}%
\begin{pgfscope}%
\pgfsys@transformshift{0.800000in}{2.635745in}%
\pgfsys@useobject{currentmarker}{}%
\end{pgfscope}%
\end{pgfscope}%
\begin{pgfscope}%
\definecolor{textcolor}{rgb}{0.000000,0.000000,0.000000}%
\pgfsetstrokecolor{textcolor}%
\pgfsetfillcolor{textcolor}%
\pgftext[x=0.305168in, y=2.582983in, left, base]{\color{textcolor}{\rmfamily\fontsize{10.000000}{12.000000}\selectfont\catcode`\^=\active\def^{\ifmmode\sp\else\^{}\fi}\catcode`\%=\active\def%{\%}0.200}}%
\end{pgfscope}%
\begin{pgfscope}%
\pgfsetbuttcap%
\pgfsetroundjoin%
\definecolor{currentfill}{rgb}{0.000000,0.000000,0.000000}%
\pgfsetfillcolor{currentfill}%
\pgfsetlinewidth{0.803000pt}%
\definecolor{currentstroke}{rgb}{0.000000,0.000000,0.000000}%
\pgfsetstrokecolor{currentstroke}%
\pgfsetdash{}{0pt}%
\pgfsys@defobject{currentmarker}{\pgfqpoint{-0.048611in}{0.000000in}}{\pgfqpoint{-0.000000in}{0.000000in}}{%
\pgfpathmoveto{\pgfqpoint{-0.000000in}{0.000000in}}%
\pgfpathlineto{\pgfqpoint{-0.048611in}{0.000000in}}%
\pgfusepath{stroke,fill}%
}%
\begin{pgfscope}%
\pgfsys@transformshift{0.800000in}{3.257544in}%
\pgfsys@useobject{currentmarker}{}%
\end{pgfscope}%
\end{pgfscope}%
\begin{pgfscope}%
\definecolor{textcolor}{rgb}{0.000000,0.000000,0.000000}%
\pgfsetstrokecolor{textcolor}%
\pgfsetfillcolor{textcolor}%
\pgftext[x=0.305168in, y=3.204783in, left, base]{\color{textcolor}{\rmfamily\fontsize{10.000000}{12.000000}\selectfont\catcode`\^=\active\def^{\ifmmode\sp\else\^{}\fi}\catcode`\%=\active\def%{\%}0.250}}%
\end{pgfscope}%
\begin{pgfscope}%
\pgfsetbuttcap%
\pgfsetroundjoin%
\definecolor{currentfill}{rgb}{0.000000,0.000000,0.000000}%
\pgfsetfillcolor{currentfill}%
\pgfsetlinewidth{0.803000pt}%
\definecolor{currentstroke}{rgb}{0.000000,0.000000,0.000000}%
\pgfsetstrokecolor{currentstroke}%
\pgfsetdash{}{0pt}%
\pgfsys@defobject{currentmarker}{\pgfqpoint{-0.048611in}{0.000000in}}{\pgfqpoint{-0.000000in}{0.000000in}}{%
\pgfpathmoveto{\pgfqpoint{-0.000000in}{0.000000in}}%
\pgfpathlineto{\pgfqpoint{-0.048611in}{0.000000in}}%
\pgfusepath{stroke,fill}%
}%
\begin{pgfscope}%
\pgfsys@transformshift{0.800000in}{3.879344in}%
\pgfsys@useobject{currentmarker}{}%
\end{pgfscope}%
\end{pgfscope}%
\begin{pgfscope}%
\definecolor{textcolor}{rgb}{0.000000,0.000000,0.000000}%
\pgfsetstrokecolor{textcolor}%
\pgfsetfillcolor{textcolor}%
\pgftext[x=0.305168in, y=3.826582in, left, base]{\color{textcolor}{\rmfamily\fontsize{10.000000}{12.000000}\selectfont\catcode`\^=\active\def^{\ifmmode\sp\else\^{}\fi}\catcode`\%=\active\def%{\%}0.300}}%
\end{pgfscope}%
\begin{pgfscope}%
\pgfsetbuttcap%
\pgfsetroundjoin%
\definecolor{currentfill}{rgb}{0.000000,0.000000,0.000000}%
\pgfsetfillcolor{currentfill}%
\pgfsetlinewidth{0.803000pt}%
\definecolor{currentstroke}{rgb}{0.000000,0.000000,0.000000}%
\pgfsetstrokecolor{currentstroke}%
\pgfsetdash{}{0pt}%
\pgfsys@defobject{currentmarker}{\pgfqpoint{-0.048611in}{0.000000in}}{\pgfqpoint{-0.000000in}{0.000000in}}{%
\pgfpathmoveto{\pgfqpoint{-0.000000in}{0.000000in}}%
\pgfpathlineto{\pgfqpoint{-0.048611in}{0.000000in}}%
\pgfusepath{stroke,fill}%
}%
\begin{pgfscope}%
\pgfsys@transformshift{0.800000in}{4.076160in}%
\pgfsys@useobject{currentmarker}{}%
\end{pgfscope}%
\end{pgfscope}%
\begin{pgfscope}%
\definecolor{textcolor}{rgb}{0.000000,0.000000,0.000000}%
\pgfsetstrokecolor{textcolor}%
\pgfsetfillcolor{textcolor}%
\pgftext[x=0.455864in, y=4.023398in, left, base]{\color{textcolor}{\rmfamily\fontsize{10.000000}{12.000000}\selectfont\catcode`\^=\active\def^{\ifmmode\sp\else\^{}\fi}\catcode`\%=\active\def%{\%}$+\infty$}}%
\end{pgfscope}%
\begin{pgfscope}%
\definecolor{textcolor}{rgb}{0.000000,0.000000,0.000000}%
\pgfsetstrokecolor{textcolor}%
\pgfsetfillcolor{textcolor}%
\pgftext[x=0.249612in,y=2.376000in,,bottom,rotate=90.000000]{\color{textcolor}{\rmfamily\fontsize{16.000000}{19.200000}\selectfont\catcode`\^=\active\def^{\ifmmode\sp\else\^{}\fi}\catcode`\%=\active\def%{\%}Death}}%
\end{pgfscope}%
\begin{pgfscope}%
\pgfpathrectangle{\pgfqpoint{0.800000in}{0.528000in}}{\pgfqpoint{4.960000in}{3.696000in}}%
\pgfusepath{clip}%
\pgfsetrectcap%
\pgfsetroundjoin%
\pgfsetlinewidth{1.003750pt}%
\definecolor{currentstroke}{rgb}{0.000000,0.000000,0.000000}%
\pgfsetstrokecolor{currentstroke}%
\pgfsetdash{}{0pt}%
\pgfpathmoveto{\pgfqpoint{0.800000in}{0.528000in}}%
\pgfpathlineto{\pgfqpoint{5.760000in}{3.928320in}}%
\pgfusepath{stroke}%
\end{pgfscope}%
\begin{pgfscope}%
\pgfpathrectangle{\pgfqpoint{0.800000in}{0.528000in}}{\pgfqpoint{4.960000in}{3.696000in}}%
\pgfusepath{clip}%
\pgfsetrectcap%
\pgfsetroundjoin%
\pgfsetlinewidth{1.003750pt}%
\definecolor{currentstroke}{rgb}{0.000000,0.000000,0.000000}%
\pgfsetstrokecolor{currentstroke}%
\pgfsetstrokeopacity{0.600000}%
\pgfsetdash{}{0pt}%
\pgfpathmoveto{\pgfqpoint{0.800000in}{4.076160in}}%
\pgfpathlineto{\pgfqpoint{5.760000in}{4.076160in}}%
\pgfusepath{stroke}%
\end{pgfscope}%
\begin{pgfscope}%
\pgfsetrectcap%
\pgfsetmiterjoin%
\pgfsetlinewidth{0.803000pt}%
\definecolor{currentstroke}{rgb}{0.000000,0.000000,0.000000}%
\pgfsetstrokecolor{currentstroke}%
\pgfsetdash{}{0pt}%
\pgfpathmoveto{\pgfqpoint{0.800000in}{0.528000in}}%
\pgfpathlineto{\pgfqpoint{0.800000in}{4.224000in}}%
\pgfusepath{stroke}%
\end{pgfscope}%
\begin{pgfscope}%
\pgfsetrectcap%
\pgfsetmiterjoin%
\pgfsetlinewidth{0.803000pt}%
\definecolor{currentstroke}{rgb}{0.000000,0.000000,0.000000}%
\pgfsetstrokecolor{currentstroke}%
\pgfsetdash{}{0pt}%
\pgfpathmoveto{\pgfqpoint{5.760000in}{0.528000in}}%
\pgfpathlineto{\pgfqpoint{5.760000in}{4.224000in}}%
\pgfusepath{stroke}%
\end{pgfscope}%
\begin{pgfscope}%
\pgfsetrectcap%
\pgfsetmiterjoin%
\pgfsetlinewidth{0.803000pt}%
\definecolor{currentstroke}{rgb}{0.000000,0.000000,0.000000}%
\pgfsetstrokecolor{currentstroke}%
\pgfsetdash{}{0pt}%
\pgfpathmoveto{\pgfqpoint{0.800000in}{0.528000in}}%
\pgfpathlineto{\pgfqpoint{5.760000in}{0.528000in}}%
\pgfusepath{stroke}%
\end{pgfscope}%
\begin{pgfscope}%
\pgfsetrectcap%
\pgfsetmiterjoin%
\pgfsetlinewidth{0.803000pt}%
\definecolor{currentstroke}{rgb}{0.000000,0.000000,0.000000}%
\pgfsetstrokecolor{currentstroke}%
\pgfsetdash{}{0pt}%
\pgfpathmoveto{\pgfqpoint{0.800000in}{4.224000in}}%
\pgfpathlineto{\pgfqpoint{5.760000in}{4.224000in}}%
\pgfusepath{stroke}%
\end{pgfscope}%
\begin{pgfscope}%
\definecolor{textcolor}{rgb}{0.000000,0.000000,0.000000}%
\pgfsetstrokecolor{textcolor}%
\pgfsetfillcolor{textcolor}%
\pgftext[x=3.280000in,y=4.307333in,,base]{\color{textcolor}{\rmfamily\fontsize{16.000000}{19.200000}\selectfont\catcode`\^=\active\def^{\ifmmode\sp\else\^{}\fi}\catcode`\%=\active\def%{\%}Persistence diagram}}%
\end{pgfscope}%
\begin{pgfscope}%
\pgfsetbuttcap%
\pgfsetmiterjoin%
\definecolor{currentfill}{rgb}{1.000000,1.000000,1.000000}%
\pgfsetfillcolor{currentfill}%
\pgfsetfillopacity{0.800000}%
\pgfsetlinewidth{1.003750pt}%
\definecolor{currentstroke}{rgb}{0.800000,0.800000,0.800000}%
\pgfsetstrokecolor{currentstroke}%
\pgfsetstrokeopacity{0.800000}%
\pgfsetdash{}{0pt}%
\pgfpathmoveto{\pgfqpoint{5.129968in}{0.597444in}}%
\pgfpathlineto{\pgfqpoint{5.662778in}{0.597444in}}%
\pgfpathquadraticcurveto{\pgfqpoint{5.690556in}{0.597444in}}{\pgfqpoint{5.690556in}{0.625222in}}%
\pgfpathlineto{\pgfqpoint{5.690556in}{1.019048in}}%
\pgfpathquadraticcurveto{\pgfqpoint{5.690556in}{1.046826in}}{\pgfqpoint{5.662778in}{1.046826in}}%
\pgfpathlineto{\pgfqpoint{5.129968in}{1.046826in}}%
\pgfpathquadraticcurveto{\pgfqpoint{5.102190in}{1.046826in}}{\pgfqpoint{5.102190in}{1.019048in}}%
\pgfpathlineto{\pgfqpoint{5.102190in}{0.625222in}}%
\pgfpathquadraticcurveto{\pgfqpoint{5.102190in}{0.597444in}}{\pgfqpoint{5.129968in}{0.597444in}}%
\pgfpathlineto{\pgfqpoint{5.129968in}{0.597444in}}%
\pgfpathclose%
\pgfusepath{stroke,fill}%
\end{pgfscope}%
\begin{pgfscope}%
\pgfsetbuttcap%
\pgfsetmiterjoin%
\definecolor{currentfill}{rgb}{0.894118,0.101961,0.109804}%
\pgfsetfillcolor{currentfill}%
\pgfsetlinewidth{1.003750pt}%
\definecolor{currentstroke}{rgb}{0.894118,0.101961,0.109804}%
\pgfsetstrokecolor{currentstroke}%
\pgfsetdash{}{0pt}%
\pgfpathmoveto{\pgfqpoint{5.157746in}{0.885747in}}%
\pgfpathlineto{\pgfqpoint{5.435524in}{0.885747in}}%
\pgfpathlineto{\pgfqpoint{5.435524in}{0.982969in}}%
\pgfpathlineto{\pgfqpoint{5.157746in}{0.982969in}}%
\pgfpathlineto{\pgfqpoint{5.157746in}{0.885747in}}%
\pgfpathclose%
\pgfusepath{stroke,fill}%
\end{pgfscope}%
\begin{pgfscope}%
\definecolor{textcolor}{rgb}{0.000000,0.000000,0.000000}%
\pgfsetstrokecolor{textcolor}%
\pgfsetfillcolor{textcolor}%
\pgftext[x=5.546635in,y=0.885747in,left,base]{\color{textcolor}{\rmfamily\fontsize{10.000000}{12.000000}\selectfont\catcode`\^=\active\def^{\ifmmode\sp\else\^{}\fi}\catcode`\%=\active\def%{\%}0}}%
\end{pgfscope}%
\begin{pgfscope}%
\pgfsetbuttcap%
\pgfsetmiterjoin%
\definecolor{currentfill}{rgb}{0.215686,0.494118,0.721569}%
\pgfsetfillcolor{currentfill}%
\pgfsetlinewidth{1.003750pt}%
\definecolor{currentstroke}{rgb}{0.215686,0.494118,0.721569}%
\pgfsetstrokecolor{currentstroke}%
\pgfsetdash{}{0pt}%
\pgfpathmoveto{\pgfqpoint{5.157746in}{0.681890in}}%
\pgfpathlineto{\pgfqpoint{5.435524in}{0.681890in}}%
\pgfpathlineto{\pgfqpoint{5.435524in}{0.779112in}}%
\pgfpathlineto{\pgfqpoint{5.157746in}{0.779112in}}%
\pgfpathlineto{\pgfqpoint{5.157746in}{0.681890in}}%
\pgfpathclose%
\pgfusepath{stroke,fill}%
\end{pgfscope}%
\begin{pgfscope}%
\definecolor{textcolor}{rgb}{0.000000,0.000000,0.000000}%
\pgfsetstrokecolor{textcolor}%
\pgfsetfillcolor{textcolor}%
\pgftext[x=5.546635in,y=0.681890in,left,base]{\color{textcolor}{\rmfamily\fontsize{10.000000}{12.000000}\selectfont\catcode`\^=\active\def^{\ifmmode\sp\else\^{}\fi}\catcode`\%=\active\def%{\%}1}}%
\end{pgfscope}%
\end{pgfpicture}%
\makeatother%
\endgroup%

        }
        \caption{Nested dataset}
        \label{fig:dowker_ph_nested}
    \end{subfigure}
    \begin{subfigure}{0.49\textwidth}
        \resizebox{\textwidth}{!}{
            %% Creator: Matplotlib, PGF backend
%%
%% To include the figure in your LaTeX document, write
%%   \input{<filename>.pgf}
%%
%% Make sure the required packages are loaded in your preamble
%%   \usepackage{pgf}
%%
%% Also ensure that all the required font packages are loaded; for instance,
%% the lmodern package is sometimes necessary when using math font.
%%   \usepackage{lmodern}
%%
%% Figures using additional raster images can only be included by \input if
%% they are in the same directory as the main LaTeX file. For loading figures
%% from other directories you can use the `import` package
%%   \usepackage{import}
%%
%% and then include the figures with
%%   \import{<path to file>}{<filename>.pgf}
%%
%% Matplotlib used the following preamble
%%   \def\mathdefault#1{#1}
%%   \everymath=\expandafter{\the\everymath\displaystyle}
%%   
%%   \ifdefined\pdftexversion\else  % non-pdftex case.
%%     \usepackage{fontspec}
%%     \setmainfont{DejaVuSerif.ttf}[Path=\detokenize{/home/snek/repos/homology-decision-bondaries-clean/venv/lib/python3.9/site-packages/matplotlib/mpl-data/fonts/ttf/}]
%%     \setsansfont{DejaVuSans.ttf}[Path=\detokenize{/home/snek/repos/homology-decision-bondaries-clean/venv/lib/python3.9/site-packages/matplotlib/mpl-data/fonts/ttf/}]
%%     \setmonofont{DejaVuSansMono.ttf}[Path=\detokenize{/home/snek/repos/homology-decision-bondaries-clean/venv/lib/python3.9/site-packages/matplotlib/mpl-data/fonts/ttf/}]
%%   \fi
%%   \makeatletter\@ifpackageloaded{underscore}{}{\usepackage[strings]{underscore}}\makeatother
%%
\begingroup%
\makeatletter%
\begin{pgfpicture}%
\pgfpathrectangle{\pgfpointorigin}{\pgfqpoint{6.400000in}{4.800000in}}%
\pgfusepath{use as bounding box, clip}%
\begin{pgfscope}%
\pgfsetbuttcap%
\pgfsetmiterjoin%
\definecolor{currentfill}{rgb}{1.000000,1.000000,1.000000}%
\pgfsetfillcolor{currentfill}%
\pgfsetlinewidth{0.000000pt}%
\definecolor{currentstroke}{rgb}{1.000000,1.000000,1.000000}%
\pgfsetstrokecolor{currentstroke}%
\pgfsetdash{}{0pt}%
\pgfpathmoveto{\pgfqpoint{0.000000in}{0.000000in}}%
\pgfpathlineto{\pgfqpoint{6.400000in}{0.000000in}}%
\pgfpathlineto{\pgfqpoint{6.400000in}{4.800000in}}%
\pgfpathlineto{\pgfqpoint{0.000000in}{4.800000in}}%
\pgfpathlineto{\pgfqpoint{0.000000in}{0.000000in}}%
\pgfpathclose%
\pgfusepath{fill}%
\end{pgfscope}%
\begin{pgfscope}%
\pgfsetbuttcap%
\pgfsetmiterjoin%
\definecolor{currentfill}{rgb}{1.000000,1.000000,1.000000}%
\pgfsetfillcolor{currentfill}%
\pgfsetlinewidth{0.000000pt}%
\definecolor{currentstroke}{rgb}{0.000000,0.000000,0.000000}%
\pgfsetstrokecolor{currentstroke}%
\pgfsetstrokeopacity{0.000000}%
\pgfsetdash{}{0pt}%
\pgfpathmoveto{\pgfqpoint{0.800000in}{0.528000in}}%
\pgfpathlineto{\pgfqpoint{5.760000in}{0.528000in}}%
\pgfpathlineto{\pgfqpoint{5.760000in}{4.224000in}}%
\pgfpathlineto{\pgfqpoint{0.800000in}{4.224000in}}%
\pgfpathlineto{\pgfqpoint{0.800000in}{0.528000in}}%
\pgfpathclose%
\pgfusepath{fill}%
\end{pgfscope}%
\begin{pgfscope}%
\pgfpathrectangle{\pgfqpoint{0.800000in}{0.528000in}}{\pgfqpoint{4.960000in}{3.696000in}}%
\pgfusepath{clip}%
\pgfsetbuttcap%
\pgfsetmiterjoin%
\definecolor{currentfill}{rgb}{0.827451,0.827451,0.827451}%
\pgfsetfillcolor{currentfill}%
\pgfsetlinewidth{1.003750pt}%
\definecolor{currentstroke}{rgb}{0.827451,0.827451,0.827451}%
\pgfsetstrokecolor{currentstroke}%
\pgfsetdash{}{0pt}%
\pgfpathmoveto{\pgfqpoint{0.800000in}{0.528000in}}%
\pgfpathlineto{\pgfqpoint{5.760000in}{0.528000in}}%
\pgfpathlineto{\pgfqpoint{5.760000in}{3.928320in}}%
\pgfpathlineto{\pgfqpoint{0.800000in}{0.528000in}}%
\pgfpathclose%
\pgfusepath{stroke,fill}%
\end{pgfscope}%
\begin{pgfscope}%
\pgfpathrectangle{\pgfqpoint{0.800000in}{0.528000in}}{\pgfqpoint{4.960000in}{3.696000in}}%
\pgfusepath{clip}%
\pgfsetbuttcap%
\pgfsetroundjoin%
\definecolor{currentfill}{rgb}{0.215686,0.494118,0.721569}%
\pgfsetfillcolor{currentfill}%
\pgfsetfillopacity{0.600000}%
\pgfsetlinewidth{1.003750pt}%
\definecolor{currentstroke}{rgb}{0.215686,0.494118,0.721569}%
\pgfsetstrokecolor{currentstroke}%
\pgfsetstrokeopacity{0.600000}%
\pgfsetdash{}{0pt}%
\pgfpathmoveto{\pgfqpoint{3.664323in}{3.401777in}}%
\pgfpathcurveto{\pgfqpoint{3.675373in}{3.401777in}}{\pgfqpoint{3.685972in}{3.406167in}}{\pgfqpoint{3.693786in}{3.413981in}}%
\pgfpathcurveto{\pgfqpoint{3.701600in}{3.421794in}}{\pgfqpoint{3.705990in}{3.432393in}}{\pgfqpoint{3.705990in}{3.443443in}}%
\pgfpathcurveto{\pgfqpoint{3.705990in}{3.454494in}}{\pgfqpoint{3.701600in}{3.465093in}}{\pgfqpoint{3.693786in}{3.472906in}}%
\pgfpathcurveto{\pgfqpoint{3.685972in}{3.480720in}}{\pgfqpoint{3.675373in}{3.485110in}}{\pgfqpoint{3.664323in}{3.485110in}}%
\pgfpathcurveto{\pgfqpoint{3.653273in}{3.485110in}}{\pgfqpoint{3.642674in}{3.480720in}}{\pgfqpoint{3.634861in}{3.472906in}}%
\pgfpathcurveto{\pgfqpoint{3.627047in}{3.465093in}}{\pgfqpoint{3.622657in}{3.454494in}}{\pgfqpoint{3.622657in}{3.443443in}}%
\pgfpathcurveto{\pgfqpoint{3.622657in}{3.432393in}}{\pgfqpoint{3.627047in}{3.421794in}}{\pgfqpoint{3.634861in}{3.413981in}}%
\pgfpathcurveto{\pgfqpoint{3.642674in}{3.406167in}}{\pgfqpoint{3.653273in}{3.401777in}}{\pgfqpoint{3.664323in}{3.401777in}}%
\pgfpathlineto{\pgfqpoint{3.664323in}{3.401777in}}%
\pgfpathclose%
\pgfusepath{stroke,fill}%
\end{pgfscope}%
\begin{pgfscope}%
\pgfpathrectangle{\pgfqpoint{0.800000in}{0.528000in}}{\pgfqpoint{4.960000in}{3.696000in}}%
\pgfusepath{clip}%
\pgfsetbuttcap%
\pgfsetroundjoin%
\definecolor{currentfill}{rgb}{0.215686,0.494118,0.721569}%
\pgfsetfillcolor{currentfill}%
\pgfsetfillopacity{0.600000}%
\pgfsetlinewidth{1.003750pt}%
\definecolor{currentstroke}{rgb}{0.215686,0.494118,0.721569}%
\pgfsetstrokecolor{currentstroke}%
\pgfsetstrokeopacity{0.600000}%
\pgfsetdash{}{0pt}%
\pgfpathmoveto{\pgfqpoint{4.326793in}{3.738813in}}%
\pgfpathcurveto{\pgfqpoint{4.337843in}{3.738813in}}{\pgfqpoint{4.348442in}{3.743204in}}{\pgfqpoint{4.356256in}{3.751017in}}%
\pgfpathcurveto{\pgfqpoint{4.364069in}{3.758831in}}{\pgfqpoint{4.368460in}{3.769430in}}{\pgfqpoint{4.368460in}{3.780480in}}%
\pgfpathcurveto{\pgfqpoint{4.368460in}{3.791530in}}{\pgfqpoint{4.364069in}{3.802129in}}{\pgfqpoint{4.356256in}{3.809943in}}%
\pgfpathcurveto{\pgfqpoint{4.348442in}{3.817756in}}{\pgfqpoint{4.337843in}{3.822147in}}{\pgfqpoint{4.326793in}{3.822147in}}%
\pgfpathcurveto{\pgfqpoint{4.315743in}{3.822147in}}{\pgfqpoint{4.305144in}{3.817756in}}{\pgfqpoint{4.297330in}{3.809943in}}%
\pgfpathcurveto{\pgfqpoint{4.289517in}{3.802129in}}{\pgfqpoint{4.285126in}{3.791530in}}{\pgfqpoint{4.285126in}{3.780480in}}%
\pgfpathcurveto{\pgfqpoint{4.285126in}{3.769430in}}{\pgfqpoint{4.289517in}{3.758831in}}{\pgfqpoint{4.297330in}{3.751017in}}%
\pgfpathcurveto{\pgfqpoint{4.305144in}{3.743204in}}{\pgfqpoint{4.315743in}{3.738813in}}{\pgfqpoint{4.326793in}{3.738813in}}%
\pgfpathlineto{\pgfqpoint{4.326793in}{3.738813in}}%
\pgfpathclose%
\pgfusepath{stroke,fill}%
\end{pgfscope}%
\begin{pgfscope}%
\pgfpathrectangle{\pgfqpoint{0.800000in}{0.528000in}}{\pgfqpoint{4.960000in}{3.696000in}}%
\pgfusepath{clip}%
\pgfsetbuttcap%
\pgfsetroundjoin%
\definecolor{currentfill}{rgb}{0.215686,0.494118,0.721569}%
\pgfsetfillcolor{currentfill}%
\pgfsetfillopacity{0.600000}%
\pgfsetlinewidth{1.003750pt}%
\definecolor{currentstroke}{rgb}{0.215686,0.494118,0.721569}%
\pgfsetstrokecolor{currentstroke}%
\pgfsetstrokeopacity{0.600000}%
\pgfsetdash{}{0pt}%
\pgfpathmoveto{\pgfqpoint{3.492471in}{3.114775in}}%
\pgfpathcurveto{\pgfqpoint{3.503521in}{3.114775in}}{\pgfqpoint{3.514120in}{3.119165in}}{\pgfqpoint{3.521934in}{3.126979in}}%
\pgfpathcurveto{\pgfqpoint{3.529748in}{3.134792in}}{\pgfqpoint{3.534138in}{3.145391in}}{\pgfqpoint{3.534138in}{3.156441in}}%
\pgfpathcurveto{\pgfqpoint{3.534138in}{3.167492in}}{\pgfqpoint{3.529748in}{3.178091in}}{\pgfqpoint{3.521934in}{3.185904in}}%
\pgfpathcurveto{\pgfqpoint{3.514120in}{3.193718in}}{\pgfqpoint{3.503521in}{3.198108in}}{\pgfqpoint{3.492471in}{3.198108in}}%
\pgfpathcurveto{\pgfqpoint{3.481421in}{3.198108in}}{\pgfqpoint{3.470822in}{3.193718in}}{\pgfqpoint{3.463009in}{3.185904in}}%
\pgfpathcurveto{\pgfqpoint{3.455195in}{3.178091in}}{\pgfqpoint{3.450805in}{3.167492in}}{\pgfqpoint{3.450805in}{3.156441in}}%
\pgfpathcurveto{\pgfqpoint{3.450805in}{3.145391in}}{\pgfqpoint{3.455195in}{3.134792in}}{\pgfqpoint{3.463009in}{3.126979in}}%
\pgfpathcurveto{\pgfqpoint{3.470822in}{3.119165in}}{\pgfqpoint{3.481421in}{3.114775in}}{\pgfqpoint{3.492471in}{3.114775in}}%
\pgfpathlineto{\pgfqpoint{3.492471in}{3.114775in}}%
\pgfpathclose%
\pgfusepath{stroke,fill}%
\end{pgfscope}%
\begin{pgfscope}%
\pgfpathrectangle{\pgfqpoint{0.800000in}{0.528000in}}{\pgfqpoint{4.960000in}{3.696000in}}%
\pgfusepath{clip}%
\pgfsetbuttcap%
\pgfsetroundjoin%
\definecolor{currentfill}{rgb}{0.215686,0.494118,0.721569}%
\pgfsetfillcolor{currentfill}%
\pgfsetfillopacity{0.600000}%
\pgfsetlinewidth{1.003750pt}%
\definecolor{currentstroke}{rgb}{0.215686,0.494118,0.721569}%
\pgfsetstrokecolor{currentstroke}%
\pgfsetstrokeopacity{0.600000}%
\pgfsetdash{}{0pt}%
\pgfpathmoveto{\pgfqpoint{3.734626in}{3.249825in}}%
\pgfpathcurveto{\pgfqpoint{3.745677in}{3.249825in}}{\pgfqpoint{3.756276in}{3.254215in}}{\pgfqpoint{3.764089in}{3.262029in}}%
\pgfpathcurveto{\pgfqpoint{3.771903in}{3.269842in}}{\pgfqpoint{3.776293in}{3.280441in}}{\pgfqpoint{3.776293in}{3.291491in}}%
\pgfpathcurveto{\pgfqpoint{3.776293in}{3.302542in}}{\pgfqpoint{3.771903in}{3.313141in}}{\pgfqpoint{3.764089in}{3.320954in}}%
\pgfpathcurveto{\pgfqpoint{3.756276in}{3.328768in}}{\pgfqpoint{3.745677in}{3.333158in}}{\pgfqpoint{3.734626in}{3.333158in}}%
\pgfpathcurveto{\pgfqpoint{3.723576in}{3.333158in}}{\pgfqpoint{3.712977in}{3.328768in}}{\pgfqpoint{3.705164in}{3.320954in}}%
\pgfpathcurveto{\pgfqpoint{3.697350in}{3.313141in}}{\pgfqpoint{3.692960in}{3.302542in}}{\pgfqpoint{3.692960in}{3.291491in}}%
\pgfpathcurveto{\pgfqpoint{3.692960in}{3.280441in}}{\pgfqpoint{3.697350in}{3.269842in}}{\pgfqpoint{3.705164in}{3.262029in}}%
\pgfpathcurveto{\pgfqpoint{3.712977in}{3.254215in}}{\pgfqpoint{3.723576in}{3.249825in}}{\pgfqpoint{3.734626in}{3.249825in}}%
\pgfpathlineto{\pgfqpoint{3.734626in}{3.249825in}}%
\pgfpathclose%
\pgfusepath{stroke,fill}%
\end{pgfscope}%
\begin{pgfscope}%
\pgfpathrectangle{\pgfqpoint{0.800000in}{0.528000in}}{\pgfqpoint{4.960000in}{3.696000in}}%
\pgfusepath{clip}%
\pgfsetbuttcap%
\pgfsetroundjoin%
\definecolor{currentfill}{rgb}{0.215686,0.494118,0.721569}%
\pgfsetfillcolor{currentfill}%
\pgfsetfillopacity{0.600000}%
\pgfsetlinewidth{1.003750pt}%
\definecolor{currentstroke}{rgb}{0.215686,0.494118,0.721569}%
\pgfsetstrokecolor{currentstroke}%
\pgfsetstrokeopacity{0.600000}%
\pgfsetdash{}{0pt}%
\pgfpathmoveto{\pgfqpoint{3.345896in}{2.844370in}}%
\pgfpathcurveto{\pgfqpoint{3.356946in}{2.844370in}}{\pgfqpoint{3.367545in}{2.848760in}}{\pgfqpoint{3.375359in}{2.856574in}}%
\pgfpathcurveto{\pgfqpoint{3.383173in}{2.864388in}}{\pgfqpoint{3.387563in}{2.874987in}}{\pgfqpoint{3.387563in}{2.886037in}}%
\pgfpathcurveto{\pgfqpoint{3.387563in}{2.897087in}}{\pgfqpoint{3.383173in}{2.907686in}}{\pgfqpoint{3.375359in}{2.915499in}}%
\pgfpathcurveto{\pgfqpoint{3.367545in}{2.923313in}}{\pgfqpoint{3.356946in}{2.927703in}}{\pgfqpoint{3.345896in}{2.927703in}}%
\pgfpathcurveto{\pgfqpoint{3.334846in}{2.927703in}}{\pgfqpoint{3.324247in}{2.923313in}}{\pgfqpoint{3.316433in}{2.915499in}}%
\pgfpathcurveto{\pgfqpoint{3.308620in}{2.907686in}}{\pgfqpoint{3.304230in}{2.897087in}}{\pgfqpoint{3.304230in}{2.886037in}}%
\pgfpathcurveto{\pgfqpoint{3.304230in}{2.874987in}}{\pgfqpoint{3.308620in}{2.864388in}}{\pgfqpoint{3.316433in}{2.856574in}}%
\pgfpathcurveto{\pgfqpoint{3.324247in}{2.848760in}}{\pgfqpoint{3.334846in}{2.844370in}}{\pgfqpoint{3.345896in}{2.844370in}}%
\pgfpathlineto{\pgfqpoint{3.345896in}{2.844370in}}%
\pgfpathclose%
\pgfusepath{stroke,fill}%
\end{pgfscope}%
\begin{pgfscope}%
\pgfpathrectangle{\pgfqpoint{0.800000in}{0.528000in}}{\pgfqpoint{4.960000in}{3.696000in}}%
\pgfusepath{clip}%
\pgfsetbuttcap%
\pgfsetroundjoin%
\definecolor{currentfill}{rgb}{0.215686,0.494118,0.721569}%
\pgfsetfillcolor{currentfill}%
\pgfsetfillopacity{0.600000}%
\pgfsetlinewidth{1.003750pt}%
\definecolor{currentstroke}{rgb}{0.215686,0.494118,0.721569}%
\pgfsetstrokecolor{currentstroke}%
\pgfsetstrokeopacity{0.600000}%
\pgfsetdash{}{0pt}%
\pgfpathmoveto{\pgfqpoint{3.334881in}{2.700578in}}%
\pgfpathcurveto{\pgfqpoint{3.345931in}{2.700578in}}{\pgfqpoint{3.356530in}{2.704968in}}{\pgfqpoint{3.364344in}{2.712782in}}%
\pgfpathcurveto{\pgfqpoint{3.372157in}{2.720596in}}{\pgfqpoint{3.376548in}{2.731195in}}{\pgfqpoint{3.376548in}{2.742245in}}%
\pgfpathcurveto{\pgfqpoint{3.376548in}{2.753295in}}{\pgfqpoint{3.372157in}{2.763894in}}{\pgfqpoint{3.364344in}{2.771708in}}%
\pgfpathcurveto{\pgfqpoint{3.356530in}{2.779521in}}{\pgfqpoint{3.345931in}{2.783912in}}{\pgfqpoint{3.334881in}{2.783912in}}%
\pgfpathcurveto{\pgfqpoint{3.323831in}{2.783912in}}{\pgfqpoint{3.313232in}{2.779521in}}{\pgfqpoint{3.305418in}{2.771708in}}%
\pgfpathcurveto{\pgfqpoint{3.297605in}{2.763894in}}{\pgfqpoint{3.293214in}{2.753295in}}{\pgfqpoint{3.293214in}{2.742245in}}%
\pgfpathcurveto{\pgfqpoint{3.293214in}{2.731195in}}{\pgfqpoint{3.297605in}{2.720596in}}{\pgfqpoint{3.305418in}{2.712782in}}%
\pgfpathcurveto{\pgfqpoint{3.313232in}{2.704968in}}{\pgfqpoint{3.323831in}{2.700578in}}{\pgfqpoint{3.334881in}{2.700578in}}%
\pgfpathlineto{\pgfqpoint{3.334881in}{2.700578in}}%
\pgfpathclose%
\pgfusepath{stroke,fill}%
\end{pgfscope}%
\begin{pgfscope}%
\pgfpathrectangle{\pgfqpoint{0.800000in}{0.528000in}}{\pgfqpoint{4.960000in}{3.696000in}}%
\pgfusepath{clip}%
\pgfsetbuttcap%
\pgfsetroundjoin%
\definecolor{currentfill}{rgb}{0.215686,0.494118,0.721569}%
\pgfsetfillcolor{currentfill}%
\pgfsetfillopacity{0.600000}%
\pgfsetlinewidth{1.003750pt}%
\definecolor{currentstroke}{rgb}{0.215686,0.494118,0.721569}%
\pgfsetstrokecolor{currentstroke}%
\pgfsetstrokeopacity{0.600000}%
\pgfsetdash{}{0pt}%
\pgfpathmoveto{\pgfqpoint{2.875642in}{2.384764in}}%
\pgfpathcurveto{\pgfqpoint{2.886692in}{2.384764in}}{\pgfqpoint{2.897291in}{2.389154in}}{\pgfqpoint{2.905105in}{2.396968in}}%
\pgfpathcurveto{\pgfqpoint{2.912918in}{2.404781in}}{\pgfqpoint{2.917309in}{2.415381in}}{\pgfqpoint{2.917309in}{2.426431in}}%
\pgfpathcurveto{\pgfqpoint{2.917309in}{2.437481in}}{\pgfqpoint{2.912918in}{2.448080in}}{\pgfqpoint{2.905105in}{2.455893in}}%
\pgfpathcurveto{\pgfqpoint{2.897291in}{2.463707in}}{\pgfqpoint{2.886692in}{2.468097in}}{\pgfqpoint{2.875642in}{2.468097in}}%
\pgfpathcurveto{\pgfqpoint{2.864592in}{2.468097in}}{\pgfqpoint{2.853993in}{2.463707in}}{\pgfqpoint{2.846179in}{2.455893in}}%
\pgfpathcurveto{\pgfqpoint{2.838365in}{2.448080in}}{\pgfqpoint{2.833975in}{2.437481in}}{\pgfqpoint{2.833975in}{2.426431in}}%
\pgfpathcurveto{\pgfqpoint{2.833975in}{2.415381in}}{\pgfqpoint{2.838365in}{2.404781in}}{\pgfqpoint{2.846179in}{2.396968in}}%
\pgfpathcurveto{\pgfqpoint{2.853993in}{2.389154in}}{\pgfqpoint{2.864592in}{2.384764in}}{\pgfqpoint{2.875642in}{2.384764in}}%
\pgfpathlineto{\pgfqpoint{2.875642in}{2.384764in}}%
\pgfpathclose%
\pgfusepath{stroke,fill}%
\end{pgfscope}%
\begin{pgfscope}%
\pgfpathrectangle{\pgfqpoint{0.800000in}{0.528000in}}{\pgfqpoint{4.960000in}{3.696000in}}%
\pgfusepath{clip}%
\pgfsetbuttcap%
\pgfsetroundjoin%
\definecolor{currentfill}{rgb}{0.215686,0.494118,0.721569}%
\pgfsetfillcolor{currentfill}%
\pgfsetfillopacity{0.600000}%
\pgfsetlinewidth{1.003750pt}%
\definecolor{currentstroke}{rgb}{0.215686,0.494118,0.721569}%
\pgfsetstrokecolor{currentstroke}%
\pgfsetstrokeopacity{0.600000}%
\pgfsetdash{}{0pt}%
\pgfpathmoveto{\pgfqpoint{3.198056in}{2.555746in}}%
\pgfpathcurveto{\pgfqpoint{3.209106in}{2.555746in}}{\pgfqpoint{3.219705in}{2.560136in}}{\pgfqpoint{3.227519in}{2.567950in}}%
\pgfpathcurveto{\pgfqpoint{3.235333in}{2.575763in}}{\pgfqpoint{3.239723in}{2.586363in}}{\pgfqpoint{3.239723in}{2.597413in}}%
\pgfpathcurveto{\pgfqpoint{3.239723in}{2.608463in}}{\pgfqpoint{3.235333in}{2.619062in}}{\pgfqpoint{3.227519in}{2.626875in}}%
\pgfpathcurveto{\pgfqpoint{3.219705in}{2.634689in}}{\pgfqpoint{3.209106in}{2.639079in}}{\pgfqpoint{3.198056in}{2.639079in}}%
\pgfpathcurveto{\pgfqpoint{3.187006in}{2.639079in}}{\pgfqpoint{3.176407in}{2.634689in}}{\pgfqpoint{3.168593in}{2.626875in}}%
\pgfpathcurveto{\pgfqpoint{3.160780in}{2.619062in}}{\pgfqpoint{3.156389in}{2.608463in}}{\pgfqpoint{3.156389in}{2.597413in}}%
\pgfpathcurveto{\pgfqpoint{3.156389in}{2.586363in}}{\pgfqpoint{3.160780in}{2.575763in}}{\pgfqpoint{3.168593in}{2.567950in}}%
\pgfpathcurveto{\pgfqpoint{3.176407in}{2.560136in}}{\pgfqpoint{3.187006in}{2.555746in}}{\pgfqpoint{3.198056in}{2.555746in}}%
\pgfpathlineto{\pgfqpoint{3.198056in}{2.555746in}}%
\pgfpathclose%
\pgfusepath{stroke,fill}%
\end{pgfscope}%
\begin{pgfscope}%
\pgfpathrectangle{\pgfqpoint{0.800000in}{0.528000in}}{\pgfqpoint{4.960000in}{3.696000in}}%
\pgfusepath{clip}%
\pgfsetbuttcap%
\pgfsetroundjoin%
\definecolor{currentfill}{rgb}{0.215686,0.494118,0.721569}%
\pgfsetfillcolor{currentfill}%
\pgfsetfillopacity{0.600000}%
\pgfsetlinewidth{1.003750pt}%
\definecolor{currentstroke}{rgb}{0.215686,0.494118,0.721569}%
\pgfsetstrokecolor{currentstroke}%
\pgfsetstrokeopacity{0.600000}%
\pgfsetdash{}{0pt}%
\pgfpathmoveto{\pgfqpoint{4.165770in}{3.105143in}}%
\pgfpathcurveto{\pgfqpoint{4.176820in}{3.105143in}}{\pgfqpoint{4.187419in}{3.109534in}}{\pgfqpoint{4.195233in}{3.117347in}}%
\pgfpathcurveto{\pgfqpoint{4.203046in}{3.125161in}}{\pgfqpoint{4.207437in}{3.135760in}}{\pgfqpoint{4.207437in}{3.146810in}}%
\pgfpathcurveto{\pgfqpoint{4.207437in}{3.157860in}}{\pgfqpoint{4.203046in}{3.168459in}}{\pgfqpoint{4.195233in}{3.176273in}}%
\pgfpathcurveto{\pgfqpoint{4.187419in}{3.184086in}}{\pgfqpoint{4.176820in}{3.188477in}}{\pgfqpoint{4.165770in}{3.188477in}}%
\pgfpathcurveto{\pgfqpoint{4.154720in}{3.188477in}}{\pgfqpoint{4.144121in}{3.184086in}}{\pgfqpoint{4.136307in}{3.176273in}}%
\pgfpathcurveto{\pgfqpoint{4.128493in}{3.168459in}}{\pgfqpoint{4.124103in}{3.157860in}}{\pgfqpoint{4.124103in}{3.146810in}}%
\pgfpathcurveto{\pgfqpoint{4.124103in}{3.135760in}}{\pgfqpoint{4.128493in}{3.125161in}}{\pgfqpoint{4.136307in}{3.117347in}}%
\pgfpathcurveto{\pgfqpoint{4.144121in}{3.109534in}}{\pgfqpoint{4.154720in}{3.105143in}}{\pgfqpoint{4.165770in}{3.105143in}}%
\pgfpathlineto{\pgfqpoint{4.165770in}{3.105143in}}%
\pgfpathclose%
\pgfusepath{stroke,fill}%
\end{pgfscope}%
\begin{pgfscope}%
\pgfpathrectangle{\pgfqpoint{0.800000in}{0.528000in}}{\pgfqpoint{4.960000in}{3.696000in}}%
\pgfusepath{clip}%
\pgfsetbuttcap%
\pgfsetroundjoin%
\definecolor{currentfill}{rgb}{0.215686,0.494118,0.721569}%
\pgfsetfillcolor{currentfill}%
\pgfsetfillopacity{0.600000}%
\pgfsetlinewidth{1.003750pt}%
\definecolor{currentstroke}{rgb}{0.215686,0.494118,0.721569}%
\pgfsetstrokecolor{currentstroke}%
\pgfsetstrokeopacity{0.600000}%
\pgfsetdash{}{0pt}%
\pgfpathmoveto{\pgfqpoint{3.284417in}{2.472832in}}%
\pgfpathcurveto{\pgfqpoint{3.295467in}{2.472832in}}{\pgfqpoint{3.306066in}{2.477222in}}{\pgfqpoint{3.313880in}{2.485036in}}%
\pgfpathcurveto{\pgfqpoint{3.321693in}{2.492849in}}{\pgfqpoint{3.326084in}{2.503448in}}{\pgfqpoint{3.326084in}{2.514498in}}%
\pgfpathcurveto{\pgfqpoint{3.326084in}{2.525549in}}{\pgfqpoint{3.321693in}{2.536148in}}{\pgfqpoint{3.313880in}{2.543961in}}%
\pgfpathcurveto{\pgfqpoint{3.306066in}{2.551775in}}{\pgfqpoint{3.295467in}{2.556165in}}{\pgfqpoint{3.284417in}{2.556165in}}%
\pgfpathcurveto{\pgfqpoint{3.273367in}{2.556165in}}{\pgfqpoint{3.262768in}{2.551775in}}{\pgfqpoint{3.254954in}{2.543961in}}%
\pgfpathcurveto{\pgfqpoint{3.247141in}{2.536148in}}{\pgfqpoint{3.242750in}{2.525549in}}{\pgfqpoint{3.242750in}{2.514498in}}%
\pgfpathcurveto{\pgfqpoint{3.242750in}{2.503448in}}{\pgfqpoint{3.247141in}{2.492849in}}{\pgfqpoint{3.254954in}{2.485036in}}%
\pgfpathcurveto{\pgfqpoint{3.262768in}{2.477222in}}{\pgfqpoint{3.273367in}{2.472832in}}{\pgfqpoint{3.284417in}{2.472832in}}%
\pgfpathlineto{\pgfqpoint{3.284417in}{2.472832in}}%
\pgfpathclose%
\pgfusepath{stroke,fill}%
\end{pgfscope}%
\begin{pgfscope}%
\pgfpathrectangle{\pgfqpoint{0.800000in}{0.528000in}}{\pgfqpoint{4.960000in}{3.696000in}}%
\pgfusepath{clip}%
\pgfsetbuttcap%
\pgfsetroundjoin%
\definecolor{currentfill}{rgb}{0.215686,0.494118,0.721569}%
\pgfsetfillcolor{currentfill}%
\pgfsetfillopacity{0.600000}%
\pgfsetlinewidth{1.003750pt}%
\definecolor{currentstroke}{rgb}{0.215686,0.494118,0.721569}%
\pgfsetstrokecolor{currentstroke}%
\pgfsetstrokeopacity{0.600000}%
\pgfsetdash{}{0pt}%
\pgfpathmoveto{\pgfqpoint{3.313098in}{2.452236in}}%
\pgfpathcurveto{\pgfqpoint{3.324148in}{2.452236in}}{\pgfqpoint{3.334747in}{2.456626in}}{\pgfqpoint{3.342561in}{2.464440in}}%
\pgfpathcurveto{\pgfqpoint{3.350375in}{2.472254in}}{\pgfqpoint{3.354765in}{2.482853in}}{\pgfqpoint{3.354765in}{2.493903in}}%
\pgfpathcurveto{\pgfqpoint{3.354765in}{2.504953in}}{\pgfqpoint{3.350375in}{2.515552in}}{\pgfqpoint{3.342561in}{2.523366in}}%
\pgfpathcurveto{\pgfqpoint{3.334747in}{2.531179in}}{\pgfqpoint{3.324148in}{2.535569in}}{\pgfqpoint{3.313098in}{2.535569in}}%
\pgfpathcurveto{\pgfqpoint{3.302048in}{2.535569in}}{\pgfqpoint{3.291449in}{2.531179in}}{\pgfqpoint{3.283635in}{2.523366in}}%
\pgfpathcurveto{\pgfqpoint{3.275822in}{2.515552in}}{\pgfqpoint{3.271431in}{2.504953in}}{\pgfqpoint{3.271431in}{2.493903in}}%
\pgfpathcurveto{\pgfqpoint{3.271431in}{2.482853in}}{\pgfqpoint{3.275822in}{2.472254in}}{\pgfqpoint{3.283635in}{2.464440in}}%
\pgfpathcurveto{\pgfqpoint{3.291449in}{2.456626in}}{\pgfqpoint{3.302048in}{2.452236in}}{\pgfqpoint{3.313098in}{2.452236in}}%
\pgfpathlineto{\pgfqpoint{3.313098in}{2.452236in}}%
\pgfpathclose%
\pgfusepath{stroke,fill}%
\end{pgfscope}%
\begin{pgfscope}%
\pgfpathrectangle{\pgfqpoint{0.800000in}{0.528000in}}{\pgfqpoint{4.960000in}{3.696000in}}%
\pgfusepath{clip}%
\pgfsetbuttcap%
\pgfsetroundjoin%
\definecolor{currentfill}{rgb}{0.215686,0.494118,0.721569}%
\pgfsetfillcolor{currentfill}%
\pgfsetfillopacity{0.600000}%
\pgfsetlinewidth{1.003750pt}%
\definecolor{currentstroke}{rgb}{0.215686,0.494118,0.721569}%
\pgfsetstrokecolor{currentstroke}%
\pgfsetstrokeopacity{0.600000}%
\pgfsetdash{}{0pt}%
\pgfpathmoveto{\pgfqpoint{3.189424in}{2.334149in}}%
\pgfpathcurveto{\pgfqpoint{3.200474in}{2.334149in}}{\pgfqpoint{3.211073in}{2.338539in}}{\pgfqpoint{3.218887in}{2.346353in}}%
\pgfpathcurveto{\pgfqpoint{3.226700in}{2.354166in}}{\pgfqpoint{3.231090in}{2.364765in}}{\pgfqpoint{3.231090in}{2.375815in}}%
\pgfpathcurveto{\pgfqpoint{3.231090in}{2.386866in}}{\pgfqpoint{3.226700in}{2.397465in}}{\pgfqpoint{3.218887in}{2.405278in}}%
\pgfpathcurveto{\pgfqpoint{3.211073in}{2.413092in}}{\pgfqpoint{3.200474in}{2.417482in}}{\pgfqpoint{3.189424in}{2.417482in}}%
\pgfpathcurveto{\pgfqpoint{3.178374in}{2.417482in}}{\pgfqpoint{3.167775in}{2.413092in}}{\pgfqpoint{3.159961in}{2.405278in}}%
\pgfpathcurveto{\pgfqpoint{3.152147in}{2.397465in}}{\pgfqpoint{3.147757in}{2.386866in}}{\pgfqpoint{3.147757in}{2.375815in}}%
\pgfpathcurveto{\pgfqpoint{3.147757in}{2.364765in}}{\pgfqpoint{3.152147in}{2.354166in}}{\pgfqpoint{3.159961in}{2.346353in}}%
\pgfpathcurveto{\pgfqpoint{3.167775in}{2.338539in}}{\pgfqpoint{3.178374in}{2.334149in}}{\pgfqpoint{3.189424in}{2.334149in}}%
\pgfpathlineto{\pgfqpoint{3.189424in}{2.334149in}}%
\pgfpathclose%
\pgfusepath{stroke,fill}%
\end{pgfscope}%
\begin{pgfscope}%
\pgfpathrectangle{\pgfqpoint{0.800000in}{0.528000in}}{\pgfqpoint{4.960000in}{3.696000in}}%
\pgfusepath{clip}%
\pgfsetbuttcap%
\pgfsetroundjoin%
\definecolor{currentfill}{rgb}{0.215686,0.494118,0.721569}%
\pgfsetfillcolor{currentfill}%
\pgfsetfillopacity{0.600000}%
\pgfsetlinewidth{1.003750pt}%
\definecolor{currentstroke}{rgb}{0.215686,0.494118,0.721569}%
\pgfsetstrokecolor{currentstroke}%
\pgfsetstrokeopacity{0.600000}%
\pgfsetdash{}{0pt}%
\pgfpathmoveto{\pgfqpoint{3.943066in}{2.829124in}}%
\pgfpathcurveto{\pgfqpoint{3.954117in}{2.829124in}}{\pgfqpoint{3.964716in}{2.833515in}}{\pgfqpoint{3.972529in}{2.841328in}}%
\pgfpathcurveto{\pgfqpoint{3.980343in}{2.849142in}}{\pgfqpoint{3.984733in}{2.859741in}}{\pgfqpoint{3.984733in}{2.870791in}}%
\pgfpathcurveto{\pgfqpoint{3.984733in}{2.881841in}}{\pgfqpoint{3.980343in}{2.892440in}}{\pgfqpoint{3.972529in}{2.900254in}}%
\pgfpathcurveto{\pgfqpoint{3.964716in}{2.908068in}}{\pgfqpoint{3.954117in}{2.912458in}}{\pgfqpoint{3.943066in}{2.912458in}}%
\pgfpathcurveto{\pgfqpoint{3.932016in}{2.912458in}}{\pgfqpoint{3.921417in}{2.908068in}}{\pgfqpoint{3.913604in}{2.900254in}}%
\pgfpathcurveto{\pgfqpoint{3.905790in}{2.892440in}}{\pgfqpoint{3.901400in}{2.881841in}}{\pgfqpoint{3.901400in}{2.870791in}}%
\pgfpathcurveto{\pgfqpoint{3.901400in}{2.859741in}}{\pgfqpoint{3.905790in}{2.849142in}}{\pgfqpoint{3.913604in}{2.841328in}}%
\pgfpathcurveto{\pgfqpoint{3.921417in}{2.833515in}}{\pgfqpoint{3.932016in}{2.829124in}}{\pgfqpoint{3.943066in}{2.829124in}}%
\pgfpathlineto{\pgfqpoint{3.943066in}{2.829124in}}%
\pgfpathclose%
\pgfusepath{stroke,fill}%
\end{pgfscope}%
\begin{pgfscope}%
\pgfpathrectangle{\pgfqpoint{0.800000in}{0.528000in}}{\pgfqpoint{4.960000in}{3.696000in}}%
\pgfusepath{clip}%
\pgfsetbuttcap%
\pgfsetroundjoin%
\definecolor{currentfill}{rgb}{0.215686,0.494118,0.721569}%
\pgfsetfillcolor{currentfill}%
\pgfsetfillopacity{0.600000}%
\pgfsetlinewidth{1.003750pt}%
\definecolor{currentstroke}{rgb}{0.215686,0.494118,0.721569}%
\pgfsetstrokecolor{currentstroke}%
\pgfsetstrokeopacity{0.600000}%
\pgfsetdash{}{0pt}%
\pgfpathmoveto{\pgfqpoint{4.275554in}{3.044986in}}%
\pgfpathcurveto{\pgfqpoint{4.286604in}{3.044986in}}{\pgfqpoint{4.297203in}{3.049377in}}{\pgfqpoint{4.305016in}{3.057190in}}%
\pgfpathcurveto{\pgfqpoint{4.312830in}{3.065004in}}{\pgfqpoint{4.317220in}{3.075603in}}{\pgfqpoint{4.317220in}{3.086653in}}%
\pgfpathcurveto{\pgfqpoint{4.317220in}{3.097703in}}{\pgfqpoint{4.312830in}{3.108302in}}{\pgfqpoint{4.305016in}{3.116116in}}%
\pgfpathcurveto{\pgfqpoint{4.297203in}{3.123929in}}{\pgfqpoint{4.286604in}{3.128320in}}{\pgfqpoint{4.275554in}{3.128320in}}%
\pgfpathcurveto{\pgfqpoint{4.264504in}{3.128320in}}{\pgfqpoint{4.253905in}{3.123929in}}{\pgfqpoint{4.246091in}{3.116116in}}%
\pgfpathcurveto{\pgfqpoint{4.238277in}{3.108302in}}{\pgfqpoint{4.233887in}{3.097703in}}{\pgfqpoint{4.233887in}{3.086653in}}%
\pgfpathcurveto{\pgfqpoint{4.233887in}{3.075603in}}{\pgfqpoint{4.238277in}{3.065004in}}{\pgfqpoint{4.246091in}{3.057190in}}%
\pgfpathcurveto{\pgfqpoint{4.253905in}{3.049377in}}{\pgfqpoint{4.264504in}{3.044986in}}{\pgfqpoint{4.275554in}{3.044986in}}%
\pgfpathlineto{\pgfqpoint{4.275554in}{3.044986in}}%
\pgfpathclose%
\pgfusepath{stroke,fill}%
\end{pgfscope}%
\begin{pgfscope}%
\pgfpathrectangle{\pgfqpoint{0.800000in}{0.528000in}}{\pgfqpoint{4.960000in}{3.696000in}}%
\pgfusepath{clip}%
\pgfsetbuttcap%
\pgfsetroundjoin%
\definecolor{currentfill}{rgb}{0.215686,0.494118,0.721569}%
\pgfsetfillcolor{currentfill}%
\pgfsetfillopacity{0.600000}%
\pgfsetlinewidth{1.003750pt}%
\definecolor{currentstroke}{rgb}{0.215686,0.494118,0.721569}%
\pgfsetstrokecolor{currentstroke}%
\pgfsetstrokeopacity{0.600000}%
\pgfsetdash{}{0pt}%
\pgfpathmoveto{\pgfqpoint{3.812781in}{2.699141in}}%
\pgfpathcurveto{\pgfqpoint{3.823831in}{2.699141in}}{\pgfqpoint{3.834430in}{2.703532in}}{\pgfqpoint{3.842244in}{2.711345in}}%
\pgfpathcurveto{\pgfqpoint{3.850058in}{2.719159in}}{\pgfqpoint{3.854448in}{2.729758in}}{\pgfqpoint{3.854448in}{2.740808in}}%
\pgfpathcurveto{\pgfqpoint{3.854448in}{2.751858in}}{\pgfqpoint{3.850058in}{2.762457in}}{\pgfqpoint{3.842244in}{2.770271in}}%
\pgfpathcurveto{\pgfqpoint{3.834430in}{2.778085in}}{\pgfqpoint{3.823831in}{2.782475in}}{\pgfqpoint{3.812781in}{2.782475in}}%
\pgfpathcurveto{\pgfqpoint{3.801731in}{2.782475in}}{\pgfqpoint{3.791132in}{2.778085in}}{\pgfqpoint{3.783318in}{2.770271in}}%
\pgfpathcurveto{\pgfqpoint{3.775505in}{2.762457in}}{\pgfqpoint{3.771115in}{2.751858in}}{\pgfqpoint{3.771115in}{2.740808in}}%
\pgfpathcurveto{\pgfqpoint{3.771115in}{2.729758in}}{\pgfqpoint{3.775505in}{2.719159in}}{\pgfqpoint{3.783318in}{2.711345in}}%
\pgfpathcurveto{\pgfqpoint{3.791132in}{2.703532in}}{\pgfqpoint{3.801731in}{2.699141in}}{\pgfqpoint{3.812781in}{2.699141in}}%
\pgfpathlineto{\pgfqpoint{3.812781in}{2.699141in}}%
\pgfpathclose%
\pgfusepath{stroke,fill}%
\end{pgfscope}%
\begin{pgfscope}%
\pgfpathrectangle{\pgfqpoint{0.800000in}{0.528000in}}{\pgfqpoint{4.960000in}{3.696000in}}%
\pgfusepath{clip}%
\pgfsetbuttcap%
\pgfsetroundjoin%
\definecolor{currentfill}{rgb}{0.215686,0.494118,0.721569}%
\pgfsetfillcolor{currentfill}%
\pgfsetfillopacity{0.600000}%
\pgfsetlinewidth{1.003750pt}%
\definecolor{currentstroke}{rgb}{0.215686,0.494118,0.721569}%
\pgfsetstrokecolor{currentstroke}%
\pgfsetstrokeopacity{0.600000}%
\pgfsetdash{}{0pt}%
\pgfpathmoveto{\pgfqpoint{3.612179in}{2.528393in}}%
\pgfpathcurveto{\pgfqpoint{3.623229in}{2.528393in}}{\pgfqpoint{3.633828in}{2.532784in}}{\pgfqpoint{3.641642in}{2.540597in}}%
\pgfpathcurveto{\pgfqpoint{3.649455in}{2.548411in}}{\pgfqpoint{3.653846in}{2.559010in}}{\pgfqpoint{3.653846in}{2.570060in}}%
\pgfpathcurveto{\pgfqpoint{3.653846in}{2.581110in}}{\pgfqpoint{3.649455in}{2.591709in}}{\pgfqpoint{3.641642in}{2.599523in}}%
\pgfpathcurveto{\pgfqpoint{3.633828in}{2.607336in}}{\pgfqpoint{3.623229in}{2.611727in}}{\pgfqpoint{3.612179in}{2.611727in}}%
\pgfpathcurveto{\pgfqpoint{3.601129in}{2.611727in}}{\pgfqpoint{3.590530in}{2.607336in}}{\pgfqpoint{3.582716in}{2.599523in}}%
\pgfpathcurveto{\pgfqpoint{3.574902in}{2.591709in}}{\pgfqpoint{3.570512in}{2.581110in}}{\pgfqpoint{3.570512in}{2.570060in}}%
\pgfpathcurveto{\pgfqpoint{3.570512in}{2.559010in}}{\pgfqpoint{3.574902in}{2.548411in}}{\pgfqpoint{3.582716in}{2.540597in}}%
\pgfpathcurveto{\pgfqpoint{3.590530in}{2.532784in}}{\pgfqpoint{3.601129in}{2.528393in}}{\pgfqpoint{3.612179in}{2.528393in}}%
\pgfpathlineto{\pgfqpoint{3.612179in}{2.528393in}}%
\pgfpathclose%
\pgfusepath{stroke,fill}%
\end{pgfscope}%
\begin{pgfscope}%
\pgfpathrectangle{\pgfqpoint{0.800000in}{0.528000in}}{\pgfqpoint{4.960000in}{3.696000in}}%
\pgfusepath{clip}%
\pgfsetbuttcap%
\pgfsetroundjoin%
\definecolor{currentfill}{rgb}{0.215686,0.494118,0.721569}%
\pgfsetfillcolor{currentfill}%
\pgfsetfillopacity{0.600000}%
\pgfsetlinewidth{1.003750pt}%
\definecolor{currentstroke}{rgb}{0.215686,0.494118,0.721569}%
\pgfsetstrokecolor{currentstroke}%
\pgfsetstrokeopacity{0.600000}%
\pgfsetdash{}{0pt}%
\pgfpathmoveto{\pgfqpoint{3.024581in}{2.121984in}}%
\pgfpathcurveto{\pgfqpoint{3.035631in}{2.121984in}}{\pgfqpoint{3.046230in}{2.126374in}}{\pgfqpoint{3.054044in}{2.134188in}}%
\pgfpathcurveto{\pgfqpoint{3.061857in}{2.142002in}}{\pgfqpoint{3.066248in}{2.152601in}}{\pgfqpoint{3.066248in}{2.163651in}}%
\pgfpathcurveto{\pgfqpoint{3.066248in}{2.174701in}}{\pgfqpoint{3.061857in}{2.185300in}}{\pgfqpoint{3.054044in}{2.193114in}}%
\pgfpathcurveto{\pgfqpoint{3.046230in}{2.200927in}}{\pgfqpoint{3.035631in}{2.205317in}}{\pgfqpoint{3.024581in}{2.205317in}}%
\pgfpathcurveto{\pgfqpoint{3.013531in}{2.205317in}}{\pgfqpoint{3.002932in}{2.200927in}}{\pgfqpoint{2.995118in}{2.193114in}}%
\pgfpathcurveto{\pgfqpoint{2.987305in}{2.185300in}}{\pgfqpoint{2.982914in}{2.174701in}}{\pgfqpoint{2.982914in}{2.163651in}}%
\pgfpathcurveto{\pgfqpoint{2.982914in}{2.152601in}}{\pgfqpoint{2.987305in}{2.142002in}}{\pgfqpoint{2.995118in}{2.134188in}}%
\pgfpathcurveto{\pgfqpoint{3.002932in}{2.126374in}}{\pgfqpoint{3.013531in}{2.121984in}}{\pgfqpoint{3.024581in}{2.121984in}}%
\pgfpathlineto{\pgfqpoint{3.024581in}{2.121984in}}%
\pgfpathclose%
\pgfusepath{stroke,fill}%
\end{pgfscope}%
\begin{pgfscope}%
\pgfpathrectangle{\pgfqpoint{0.800000in}{0.528000in}}{\pgfqpoint{4.960000in}{3.696000in}}%
\pgfusepath{clip}%
\pgfsetbuttcap%
\pgfsetroundjoin%
\definecolor{currentfill}{rgb}{0.215686,0.494118,0.721569}%
\pgfsetfillcolor{currentfill}%
\pgfsetfillopacity{0.600000}%
\pgfsetlinewidth{1.003750pt}%
\definecolor{currentstroke}{rgb}{0.215686,0.494118,0.721569}%
\pgfsetstrokecolor{currentstroke}%
\pgfsetstrokeopacity{0.600000}%
\pgfsetdash{}{0pt}%
\pgfpathmoveto{\pgfqpoint{3.389741in}{2.346091in}}%
\pgfpathcurveto{\pgfqpoint{3.400792in}{2.346091in}}{\pgfqpoint{3.411391in}{2.350481in}}{\pgfqpoint{3.419204in}{2.358294in}}%
\pgfpathcurveto{\pgfqpoint{3.427018in}{2.366108in}}{\pgfqpoint{3.431408in}{2.376707in}}{\pgfqpoint{3.431408in}{2.387757in}}%
\pgfpathcurveto{\pgfqpoint{3.431408in}{2.398807in}}{\pgfqpoint{3.427018in}{2.409406in}}{\pgfqpoint{3.419204in}{2.417220in}}%
\pgfpathcurveto{\pgfqpoint{3.411391in}{2.425034in}}{\pgfqpoint{3.400792in}{2.429424in}}{\pgfqpoint{3.389741in}{2.429424in}}%
\pgfpathcurveto{\pgfqpoint{3.378691in}{2.429424in}}{\pgfqpoint{3.368092in}{2.425034in}}{\pgfqpoint{3.360279in}{2.417220in}}%
\pgfpathcurveto{\pgfqpoint{3.352465in}{2.409406in}}{\pgfqpoint{3.348075in}{2.398807in}}{\pgfqpoint{3.348075in}{2.387757in}}%
\pgfpathcurveto{\pgfqpoint{3.348075in}{2.376707in}}{\pgfqpoint{3.352465in}{2.366108in}}{\pgfqpoint{3.360279in}{2.358294in}}%
\pgfpathcurveto{\pgfqpoint{3.368092in}{2.350481in}}{\pgfqpoint{3.378691in}{2.346091in}}{\pgfqpoint{3.389741in}{2.346091in}}%
\pgfpathlineto{\pgfqpoint{3.389741in}{2.346091in}}%
\pgfpathclose%
\pgfusepath{stroke,fill}%
\end{pgfscope}%
\begin{pgfscope}%
\pgfpathrectangle{\pgfqpoint{0.800000in}{0.528000in}}{\pgfqpoint{4.960000in}{3.696000in}}%
\pgfusepath{clip}%
\pgfsetbuttcap%
\pgfsetroundjoin%
\definecolor{currentfill}{rgb}{0.215686,0.494118,0.721569}%
\pgfsetfillcolor{currentfill}%
\pgfsetfillopacity{0.600000}%
\pgfsetlinewidth{1.003750pt}%
\definecolor{currentstroke}{rgb}{0.215686,0.494118,0.721569}%
\pgfsetstrokecolor{currentstroke}%
\pgfsetstrokeopacity{0.600000}%
\pgfsetdash{}{0pt}%
\pgfpathmoveto{\pgfqpoint{3.637172in}{2.489555in}}%
\pgfpathcurveto{\pgfqpoint{3.648222in}{2.489555in}}{\pgfqpoint{3.658822in}{2.493946in}}{\pgfqpoint{3.666635in}{2.501759in}}%
\pgfpathcurveto{\pgfqpoint{3.674449in}{2.509573in}}{\pgfqpoint{3.678839in}{2.520172in}}{\pgfqpoint{3.678839in}{2.531222in}}%
\pgfpathcurveto{\pgfqpoint{3.678839in}{2.542272in}}{\pgfqpoint{3.674449in}{2.552871in}}{\pgfqpoint{3.666635in}{2.560685in}}%
\pgfpathcurveto{\pgfqpoint{3.658822in}{2.568498in}}{\pgfqpoint{3.648222in}{2.572889in}}{\pgfqpoint{3.637172in}{2.572889in}}%
\pgfpathcurveto{\pgfqpoint{3.626122in}{2.572889in}}{\pgfqpoint{3.615523in}{2.568498in}}{\pgfqpoint{3.607710in}{2.560685in}}%
\pgfpathcurveto{\pgfqpoint{3.599896in}{2.552871in}}{\pgfqpoint{3.595506in}{2.542272in}}{\pgfqpoint{3.595506in}{2.531222in}}%
\pgfpathcurveto{\pgfqpoint{3.595506in}{2.520172in}}{\pgfqpoint{3.599896in}{2.509573in}}{\pgfqpoint{3.607710in}{2.501759in}}%
\pgfpathcurveto{\pgfqpoint{3.615523in}{2.493946in}}{\pgfqpoint{3.626122in}{2.489555in}}{\pgfqpoint{3.637172in}{2.489555in}}%
\pgfpathlineto{\pgfqpoint{3.637172in}{2.489555in}}%
\pgfpathclose%
\pgfusepath{stroke,fill}%
\end{pgfscope}%
\begin{pgfscope}%
\pgfpathrectangle{\pgfqpoint{0.800000in}{0.528000in}}{\pgfqpoint{4.960000in}{3.696000in}}%
\pgfusepath{clip}%
\pgfsetbuttcap%
\pgfsetroundjoin%
\definecolor{currentfill}{rgb}{0.215686,0.494118,0.721569}%
\pgfsetfillcolor{currentfill}%
\pgfsetfillopacity{0.600000}%
\pgfsetlinewidth{1.003750pt}%
\definecolor{currentstroke}{rgb}{0.215686,0.494118,0.721569}%
\pgfsetstrokecolor{currentstroke}%
\pgfsetstrokeopacity{0.600000}%
\pgfsetdash{}{0pt}%
\pgfpathmoveto{\pgfqpoint{3.812699in}{2.604656in}}%
\pgfpathcurveto{\pgfqpoint{3.823749in}{2.604656in}}{\pgfqpoint{3.834348in}{2.609047in}}{\pgfqpoint{3.842162in}{2.616860in}}%
\pgfpathcurveto{\pgfqpoint{3.849975in}{2.624674in}}{\pgfqpoint{3.854366in}{2.635273in}}{\pgfqpoint{3.854366in}{2.646323in}}%
\pgfpathcurveto{\pgfqpoint{3.854366in}{2.657373in}}{\pgfqpoint{3.849975in}{2.667972in}}{\pgfqpoint{3.842162in}{2.675786in}}%
\pgfpathcurveto{\pgfqpoint{3.834348in}{2.683599in}}{\pgfqpoint{3.823749in}{2.687990in}}{\pgfqpoint{3.812699in}{2.687990in}}%
\pgfpathcurveto{\pgfqpoint{3.801649in}{2.687990in}}{\pgfqpoint{3.791050in}{2.683599in}}{\pgfqpoint{3.783236in}{2.675786in}}%
\pgfpathcurveto{\pgfqpoint{3.775423in}{2.667972in}}{\pgfqpoint{3.771032in}{2.657373in}}{\pgfqpoint{3.771032in}{2.646323in}}%
\pgfpathcurveto{\pgfqpoint{3.771032in}{2.635273in}}{\pgfqpoint{3.775423in}{2.624674in}}{\pgfqpoint{3.783236in}{2.616860in}}%
\pgfpathcurveto{\pgfqpoint{3.791050in}{2.609047in}}{\pgfqpoint{3.801649in}{2.604656in}}{\pgfqpoint{3.812699in}{2.604656in}}%
\pgfpathlineto{\pgfqpoint{3.812699in}{2.604656in}}%
\pgfpathclose%
\pgfusepath{stroke,fill}%
\end{pgfscope}%
\begin{pgfscope}%
\pgfpathrectangle{\pgfqpoint{0.800000in}{0.528000in}}{\pgfqpoint{4.960000in}{3.696000in}}%
\pgfusepath{clip}%
\pgfsetbuttcap%
\pgfsetroundjoin%
\definecolor{currentfill}{rgb}{0.215686,0.494118,0.721569}%
\pgfsetfillcolor{currentfill}%
\pgfsetfillopacity{0.600000}%
\pgfsetlinewidth{1.003750pt}%
\definecolor{currentstroke}{rgb}{0.215686,0.494118,0.721569}%
\pgfsetstrokecolor{currentstroke}%
\pgfsetstrokeopacity{0.600000}%
\pgfsetdash{}{0pt}%
\pgfpathmoveto{\pgfqpoint{2.762220in}{1.856384in}}%
\pgfpathcurveto{\pgfqpoint{2.773270in}{1.856384in}}{\pgfqpoint{2.783869in}{1.860774in}}{\pgfqpoint{2.791682in}{1.868588in}}%
\pgfpathcurveto{\pgfqpoint{2.799496in}{1.876401in}}{\pgfqpoint{2.803886in}{1.887000in}}{\pgfqpoint{2.803886in}{1.898050in}}%
\pgfpathcurveto{\pgfqpoint{2.803886in}{1.909100in}}{\pgfqpoint{2.799496in}{1.919700in}}{\pgfqpoint{2.791682in}{1.927513in}}%
\pgfpathcurveto{\pgfqpoint{2.783869in}{1.935327in}}{\pgfqpoint{2.773270in}{1.939717in}}{\pgfqpoint{2.762220in}{1.939717in}}%
\pgfpathcurveto{\pgfqpoint{2.751170in}{1.939717in}}{\pgfqpoint{2.740571in}{1.935327in}}{\pgfqpoint{2.732757in}{1.927513in}}%
\pgfpathcurveto{\pgfqpoint{2.724943in}{1.919700in}}{\pgfqpoint{2.720553in}{1.909100in}}{\pgfqpoint{2.720553in}{1.898050in}}%
\pgfpathcurveto{\pgfqpoint{2.720553in}{1.887000in}}{\pgfqpoint{2.724943in}{1.876401in}}{\pgfqpoint{2.732757in}{1.868588in}}%
\pgfpathcurveto{\pgfqpoint{2.740571in}{1.860774in}}{\pgfqpoint{2.751170in}{1.856384in}}{\pgfqpoint{2.762220in}{1.856384in}}%
\pgfpathlineto{\pgfqpoint{2.762220in}{1.856384in}}%
\pgfpathclose%
\pgfusepath{stroke,fill}%
\end{pgfscope}%
\begin{pgfscope}%
\pgfpathrectangle{\pgfqpoint{0.800000in}{0.528000in}}{\pgfqpoint{4.960000in}{3.696000in}}%
\pgfusepath{clip}%
\pgfsetbuttcap%
\pgfsetroundjoin%
\definecolor{currentfill}{rgb}{0.215686,0.494118,0.721569}%
\pgfsetfillcolor{currentfill}%
\pgfsetfillopacity{0.600000}%
\pgfsetlinewidth{1.003750pt}%
\definecolor{currentstroke}{rgb}{0.215686,0.494118,0.721569}%
\pgfsetstrokecolor{currentstroke}%
\pgfsetstrokeopacity{0.600000}%
\pgfsetdash{}{0pt}%
\pgfpathmoveto{\pgfqpoint{3.428277in}{2.289726in}}%
\pgfpathcurveto{\pgfqpoint{3.439327in}{2.289726in}}{\pgfqpoint{3.449926in}{2.294116in}}{\pgfqpoint{3.457740in}{2.301930in}}%
\pgfpathcurveto{\pgfqpoint{3.465553in}{2.309744in}}{\pgfqpoint{3.469944in}{2.320343in}}{\pgfqpoint{3.469944in}{2.331393in}}%
\pgfpathcurveto{\pgfqpoint{3.469944in}{2.342443in}}{\pgfqpoint{3.465553in}{2.353042in}}{\pgfqpoint{3.457740in}{2.360856in}}%
\pgfpathcurveto{\pgfqpoint{3.449926in}{2.368669in}}{\pgfqpoint{3.439327in}{2.373059in}}{\pgfqpoint{3.428277in}{2.373059in}}%
\pgfpathcurveto{\pgfqpoint{3.417227in}{2.373059in}}{\pgfqpoint{3.406628in}{2.368669in}}{\pgfqpoint{3.398814in}{2.360856in}}%
\pgfpathcurveto{\pgfqpoint{3.391000in}{2.353042in}}{\pgfqpoint{3.386610in}{2.342443in}}{\pgfqpoint{3.386610in}{2.331393in}}%
\pgfpathcurveto{\pgfqpoint{3.386610in}{2.320343in}}{\pgfqpoint{3.391000in}{2.309744in}}{\pgfqpoint{3.398814in}{2.301930in}}%
\pgfpathcurveto{\pgfqpoint{3.406628in}{2.294116in}}{\pgfqpoint{3.417227in}{2.289726in}}{\pgfqpoint{3.428277in}{2.289726in}}%
\pgfpathlineto{\pgfqpoint{3.428277in}{2.289726in}}%
\pgfpathclose%
\pgfusepath{stroke,fill}%
\end{pgfscope}%
\begin{pgfscope}%
\pgfpathrectangle{\pgfqpoint{0.800000in}{0.528000in}}{\pgfqpoint{4.960000in}{3.696000in}}%
\pgfusepath{clip}%
\pgfsetbuttcap%
\pgfsetroundjoin%
\definecolor{currentfill}{rgb}{0.894118,0.101961,0.109804}%
\pgfsetfillcolor{currentfill}%
\pgfsetfillopacity{0.600000}%
\pgfsetlinewidth{1.003750pt}%
\definecolor{currentstroke}{rgb}{0.894118,0.101961,0.109804}%
\pgfsetstrokecolor{currentstroke}%
\pgfsetstrokeopacity{0.600000}%
\pgfsetdash{}{0pt}%
\pgfpathmoveto{\pgfqpoint{1.231304in}{4.034493in}}%
\pgfpathcurveto{\pgfqpoint{1.242354in}{4.034493in}}{\pgfqpoint{1.252954in}{4.038884in}}{\pgfqpoint{1.260767in}{4.046697in}}%
\pgfpathcurveto{\pgfqpoint{1.268581in}{4.054511in}}{\pgfqpoint{1.272971in}{4.065110in}}{\pgfqpoint{1.272971in}{4.076160in}}%
\pgfpathcurveto{\pgfqpoint{1.272971in}{4.087210in}}{\pgfqpoint{1.268581in}{4.097809in}}{\pgfqpoint{1.260767in}{4.105623in}}%
\pgfpathcurveto{\pgfqpoint{1.252954in}{4.113436in}}{\pgfqpoint{1.242354in}{4.117827in}}{\pgfqpoint{1.231304in}{4.117827in}}%
\pgfpathcurveto{\pgfqpoint{1.220254in}{4.117827in}}{\pgfqpoint{1.209655in}{4.113436in}}{\pgfqpoint{1.201842in}{4.105623in}}%
\pgfpathcurveto{\pgfqpoint{1.194028in}{4.097809in}}{\pgfqpoint{1.189638in}{4.087210in}}{\pgfqpoint{1.189638in}{4.076160in}}%
\pgfpathcurveto{\pgfqpoint{1.189638in}{4.065110in}}{\pgfqpoint{1.194028in}{4.054511in}}{\pgfqpoint{1.201842in}{4.046697in}}%
\pgfpathcurveto{\pgfqpoint{1.209655in}{4.038884in}}{\pgfqpoint{1.220254in}{4.034493in}}{\pgfqpoint{1.231304in}{4.034493in}}%
\pgfpathlineto{\pgfqpoint{1.231304in}{4.034493in}}%
\pgfpathclose%
\pgfusepath{stroke,fill}%
\end{pgfscope}%
\begin{pgfscope}%
\pgfpathrectangle{\pgfqpoint{0.800000in}{0.528000in}}{\pgfqpoint{4.960000in}{3.696000in}}%
\pgfusepath{clip}%
\pgfsetbuttcap%
\pgfsetroundjoin%
\definecolor{currentfill}{rgb}{0.894118,0.101961,0.109804}%
\pgfsetfillcolor{currentfill}%
\pgfsetfillopacity{0.600000}%
\pgfsetlinewidth{1.003750pt}%
\definecolor{currentstroke}{rgb}{0.894118,0.101961,0.109804}%
\pgfsetstrokecolor{currentstroke}%
\pgfsetstrokeopacity{0.600000}%
\pgfsetdash{}{0pt}%
\pgfpathmoveto{\pgfqpoint{1.335808in}{2.047577in}}%
\pgfpathcurveto{\pgfqpoint{1.346858in}{2.047577in}}{\pgfqpoint{1.357457in}{2.051967in}}{\pgfqpoint{1.365270in}{2.059780in}}%
\pgfpathcurveto{\pgfqpoint{1.373084in}{2.067594in}}{\pgfqpoint{1.377474in}{2.078193in}}{\pgfqpoint{1.377474in}{2.089243in}}%
\pgfpathcurveto{\pgfqpoint{1.377474in}{2.100293in}}{\pgfqpoint{1.373084in}{2.110892in}}{\pgfqpoint{1.365270in}{2.118706in}}%
\pgfpathcurveto{\pgfqpoint{1.357457in}{2.126520in}}{\pgfqpoint{1.346858in}{2.130910in}}{\pgfqpoint{1.335808in}{2.130910in}}%
\pgfpathcurveto{\pgfqpoint{1.324757in}{2.130910in}}{\pgfqpoint{1.314158in}{2.126520in}}{\pgfqpoint{1.306345in}{2.118706in}}%
\pgfpathcurveto{\pgfqpoint{1.298531in}{2.110892in}}{\pgfqpoint{1.294141in}{2.100293in}}{\pgfqpoint{1.294141in}{2.089243in}}%
\pgfpathcurveto{\pgfqpoint{1.294141in}{2.078193in}}{\pgfqpoint{1.298531in}{2.067594in}}{\pgfqpoint{1.306345in}{2.059780in}}%
\pgfpathcurveto{\pgfqpoint{1.314158in}{2.051967in}}{\pgfqpoint{1.324757in}{2.047577in}}{\pgfqpoint{1.335808in}{2.047577in}}%
\pgfpathlineto{\pgfqpoint{1.335808in}{2.047577in}}%
\pgfpathclose%
\pgfusepath{stroke,fill}%
\end{pgfscope}%
\begin{pgfscope}%
\pgfpathrectangle{\pgfqpoint{0.800000in}{0.528000in}}{\pgfqpoint{4.960000in}{3.696000in}}%
\pgfusepath{clip}%
\pgfsetbuttcap%
\pgfsetroundjoin%
\definecolor{currentfill}{rgb}{0.894118,0.101961,0.109804}%
\pgfsetfillcolor{currentfill}%
\pgfsetfillopacity{0.600000}%
\pgfsetlinewidth{1.003750pt}%
\definecolor{currentstroke}{rgb}{0.894118,0.101961,0.109804}%
\pgfsetstrokecolor{currentstroke}%
\pgfsetstrokeopacity{0.600000}%
\pgfsetdash{}{0pt}%
\pgfpathmoveto{\pgfqpoint{1.793819in}{2.283507in}}%
\pgfpathcurveto{\pgfqpoint{1.804869in}{2.283507in}}{\pgfqpoint{1.815468in}{2.287897in}}{\pgfqpoint{1.823281in}{2.295711in}}%
\pgfpathcurveto{\pgfqpoint{1.831095in}{2.303524in}}{\pgfqpoint{1.835485in}{2.314123in}}{\pgfqpoint{1.835485in}{2.325174in}}%
\pgfpathcurveto{\pgfqpoint{1.835485in}{2.336224in}}{\pgfqpoint{1.831095in}{2.346823in}}{\pgfqpoint{1.823281in}{2.354636in}}%
\pgfpathcurveto{\pgfqpoint{1.815468in}{2.362450in}}{\pgfqpoint{1.804869in}{2.366840in}}{\pgfqpoint{1.793819in}{2.366840in}}%
\pgfpathcurveto{\pgfqpoint{1.782768in}{2.366840in}}{\pgfqpoint{1.772169in}{2.362450in}}{\pgfqpoint{1.764356in}{2.354636in}}%
\pgfpathcurveto{\pgfqpoint{1.756542in}{2.346823in}}{\pgfqpoint{1.752152in}{2.336224in}}{\pgfqpoint{1.752152in}{2.325174in}}%
\pgfpathcurveto{\pgfqpoint{1.752152in}{2.314123in}}{\pgfqpoint{1.756542in}{2.303524in}}{\pgfqpoint{1.764356in}{2.295711in}}%
\pgfpathcurveto{\pgfqpoint{1.772169in}{2.287897in}}{\pgfqpoint{1.782768in}{2.283507in}}{\pgfqpoint{1.793819in}{2.283507in}}%
\pgfpathlineto{\pgfqpoint{1.793819in}{2.283507in}}%
\pgfpathclose%
\pgfusepath{stroke,fill}%
\end{pgfscope}%
\begin{pgfscope}%
\pgfpathrectangle{\pgfqpoint{0.800000in}{0.528000in}}{\pgfqpoint{4.960000in}{3.696000in}}%
\pgfusepath{clip}%
\pgfsetbuttcap%
\pgfsetroundjoin%
\definecolor{currentfill}{rgb}{0.894118,0.101961,0.109804}%
\pgfsetfillcolor{currentfill}%
\pgfsetfillopacity{0.600000}%
\pgfsetlinewidth{1.003750pt}%
\definecolor{currentstroke}{rgb}{0.894118,0.101961,0.109804}%
\pgfsetstrokecolor{currentstroke}%
\pgfsetstrokeopacity{0.600000}%
\pgfsetdash{}{0pt}%
\pgfpathmoveto{\pgfqpoint{2.079836in}{2.404103in}}%
\pgfpathcurveto{\pgfqpoint{2.090886in}{2.404103in}}{\pgfqpoint{2.101485in}{2.408493in}}{\pgfqpoint{2.109299in}{2.416307in}}%
\pgfpathcurveto{\pgfqpoint{2.117112in}{2.424120in}}{\pgfqpoint{2.121503in}{2.434719in}}{\pgfqpoint{2.121503in}{2.445769in}}%
\pgfpathcurveto{\pgfqpoint{2.121503in}{2.456820in}}{\pgfqpoint{2.117112in}{2.467419in}}{\pgfqpoint{2.109299in}{2.475232in}}%
\pgfpathcurveto{\pgfqpoint{2.101485in}{2.483046in}}{\pgfqpoint{2.090886in}{2.487436in}}{\pgfqpoint{2.079836in}{2.487436in}}%
\pgfpathcurveto{\pgfqpoint{2.068786in}{2.487436in}}{\pgfqpoint{2.058187in}{2.483046in}}{\pgfqpoint{2.050373in}{2.475232in}}%
\pgfpathcurveto{\pgfqpoint{2.042560in}{2.467419in}}{\pgfqpoint{2.038169in}{2.456820in}}{\pgfqpoint{2.038169in}{2.445769in}}%
\pgfpathcurveto{\pgfqpoint{2.038169in}{2.434719in}}{\pgfqpoint{2.042560in}{2.424120in}}{\pgfqpoint{2.050373in}{2.416307in}}%
\pgfpathcurveto{\pgfqpoint{2.058187in}{2.408493in}}{\pgfqpoint{2.068786in}{2.404103in}}{\pgfqpoint{2.079836in}{2.404103in}}%
\pgfpathlineto{\pgfqpoint{2.079836in}{2.404103in}}%
\pgfpathclose%
\pgfusepath{stroke,fill}%
\end{pgfscope}%
\begin{pgfscope}%
\pgfpathrectangle{\pgfqpoint{0.800000in}{0.528000in}}{\pgfqpoint{4.960000in}{3.696000in}}%
\pgfusepath{clip}%
\pgfsetbuttcap%
\pgfsetroundjoin%
\definecolor{currentfill}{rgb}{0.894118,0.101961,0.109804}%
\pgfsetfillcolor{currentfill}%
\pgfsetfillopacity{0.600000}%
\pgfsetlinewidth{1.003750pt}%
\definecolor{currentstroke}{rgb}{0.894118,0.101961,0.109804}%
\pgfsetstrokecolor{currentstroke}%
\pgfsetstrokeopacity{0.600000}%
\pgfsetdash{}{0pt}%
\pgfpathmoveto{\pgfqpoint{1.726801in}{2.151544in}}%
\pgfpathcurveto{\pgfqpoint{1.737851in}{2.151544in}}{\pgfqpoint{1.748450in}{2.155935in}}{\pgfqpoint{1.756264in}{2.163748in}}%
\pgfpathcurveto{\pgfqpoint{1.764078in}{2.171562in}}{\pgfqpoint{1.768468in}{2.182161in}}{\pgfqpoint{1.768468in}{2.193211in}}%
\pgfpathcurveto{\pgfqpoint{1.768468in}{2.204261in}}{\pgfqpoint{1.764078in}{2.214860in}}{\pgfqpoint{1.756264in}{2.222674in}}%
\pgfpathcurveto{\pgfqpoint{1.748450in}{2.230487in}}{\pgfqpoint{1.737851in}{2.234878in}}{\pgfqpoint{1.726801in}{2.234878in}}%
\pgfpathcurveto{\pgfqpoint{1.715751in}{2.234878in}}{\pgfqpoint{1.705152in}{2.230487in}}{\pgfqpoint{1.697338in}{2.222674in}}%
\pgfpathcurveto{\pgfqpoint{1.689525in}{2.214860in}}{\pgfqpoint{1.685134in}{2.204261in}}{\pgfqpoint{1.685134in}{2.193211in}}%
\pgfpathcurveto{\pgfqpoint{1.685134in}{2.182161in}}{\pgfqpoint{1.689525in}{2.171562in}}{\pgfqpoint{1.697338in}{2.163748in}}%
\pgfpathcurveto{\pgfqpoint{1.705152in}{2.155935in}}{\pgfqpoint{1.715751in}{2.151544in}}{\pgfqpoint{1.726801in}{2.151544in}}%
\pgfpathlineto{\pgfqpoint{1.726801in}{2.151544in}}%
\pgfpathclose%
\pgfusepath{stroke,fill}%
\end{pgfscope}%
\begin{pgfscope}%
\pgfpathrectangle{\pgfqpoint{0.800000in}{0.528000in}}{\pgfqpoint{4.960000in}{3.696000in}}%
\pgfusepath{clip}%
\pgfsetbuttcap%
\pgfsetroundjoin%
\definecolor{currentfill}{rgb}{0.894118,0.101961,0.109804}%
\pgfsetfillcolor{currentfill}%
\pgfsetfillopacity{0.600000}%
\pgfsetlinewidth{1.003750pt}%
\definecolor{currentstroke}{rgb}{0.894118,0.101961,0.109804}%
\pgfsetstrokecolor{currentstroke}%
\pgfsetstrokeopacity{0.600000}%
\pgfsetdash{}{0pt}%
\pgfpathmoveto{\pgfqpoint{1.864465in}{2.212557in}}%
\pgfpathcurveto{\pgfqpoint{1.875515in}{2.212557in}}{\pgfqpoint{1.886114in}{2.216948in}}{\pgfqpoint{1.893927in}{2.224761in}}%
\pgfpathcurveto{\pgfqpoint{1.901741in}{2.232575in}}{\pgfqpoint{1.906131in}{2.243174in}}{\pgfqpoint{1.906131in}{2.254224in}}%
\pgfpathcurveto{\pgfqpoint{1.906131in}{2.265274in}}{\pgfqpoint{1.901741in}{2.275873in}}{\pgfqpoint{1.893927in}{2.283687in}}%
\pgfpathcurveto{\pgfqpoint{1.886114in}{2.291501in}}{\pgfqpoint{1.875515in}{2.295891in}}{\pgfqpoint{1.864465in}{2.295891in}}%
\pgfpathcurveto{\pgfqpoint{1.853414in}{2.295891in}}{\pgfqpoint{1.842815in}{2.291501in}}{\pgfqpoint{1.835002in}{2.283687in}}%
\pgfpathcurveto{\pgfqpoint{1.827188in}{2.275873in}}{\pgfqpoint{1.822798in}{2.265274in}}{\pgfqpoint{1.822798in}{2.254224in}}%
\pgfpathcurveto{\pgfqpoint{1.822798in}{2.243174in}}{\pgfqpoint{1.827188in}{2.232575in}}{\pgfqpoint{1.835002in}{2.224761in}}%
\pgfpathcurveto{\pgfqpoint{1.842815in}{2.216948in}}{\pgfqpoint{1.853414in}{2.212557in}}{\pgfqpoint{1.864465in}{2.212557in}}%
\pgfpathlineto{\pgfqpoint{1.864465in}{2.212557in}}%
\pgfpathclose%
\pgfusepath{stroke,fill}%
\end{pgfscope}%
\begin{pgfscope}%
\pgfpathrectangle{\pgfqpoint{0.800000in}{0.528000in}}{\pgfqpoint{4.960000in}{3.696000in}}%
\pgfusepath{clip}%
\pgfsetbuttcap%
\pgfsetroundjoin%
\definecolor{currentfill}{rgb}{0.894118,0.101961,0.109804}%
\pgfsetfillcolor{currentfill}%
\pgfsetfillopacity{0.600000}%
\pgfsetlinewidth{1.003750pt}%
\definecolor{currentstroke}{rgb}{0.894118,0.101961,0.109804}%
\pgfsetstrokecolor{currentstroke}%
\pgfsetstrokeopacity{0.600000}%
\pgfsetdash{}{0pt}%
\pgfpathmoveto{\pgfqpoint{1.728765in}{2.036844in}}%
\pgfpathcurveto{\pgfqpoint{1.739816in}{2.036844in}}{\pgfqpoint{1.750415in}{2.041235in}}{\pgfqpoint{1.758228in}{2.049048in}}%
\pgfpathcurveto{\pgfqpoint{1.766042in}{2.056862in}}{\pgfqpoint{1.770432in}{2.067461in}}{\pgfqpoint{1.770432in}{2.078511in}}%
\pgfpathcurveto{\pgfqpoint{1.770432in}{2.089561in}}{\pgfqpoint{1.766042in}{2.100160in}}{\pgfqpoint{1.758228in}{2.107974in}}%
\pgfpathcurveto{\pgfqpoint{1.750415in}{2.115787in}}{\pgfqpoint{1.739816in}{2.120178in}}{\pgfqpoint{1.728765in}{2.120178in}}%
\pgfpathcurveto{\pgfqpoint{1.717715in}{2.120178in}}{\pgfqpoint{1.707116in}{2.115787in}}{\pgfqpoint{1.699303in}{2.107974in}}%
\pgfpathcurveto{\pgfqpoint{1.691489in}{2.100160in}}{\pgfqpoint{1.687099in}{2.089561in}}{\pgfqpoint{1.687099in}{2.078511in}}%
\pgfpathcurveto{\pgfqpoint{1.687099in}{2.067461in}}{\pgfqpoint{1.691489in}{2.056862in}}{\pgfqpoint{1.699303in}{2.049048in}}%
\pgfpathcurveto{\pgfqpoint{1.707116in}{2.041235in}}{\pgfqpoint{1.717715in}{2.036844in}}{\pgfqpoint{1.728765in}{2.036844in}}%
\pgfpathlineto{\pgfqpoint{1.728765in}{2.036844in}}%
\pgfpathclose%
\pgfusepath{stroke,fill}%
\end{pgfscope}%
\begin{pgfscope}%
\pgfpathrectangle{\pgfqpoint{0.800000in}{0.528000in}}{\pgfqpoint{4.960000in}{3.696000in}}%
\pgfusepath{clip}%
\pgfsetbuttcap%
\pgfsetroundjoin%
\definecolor{currentfill}{rgb}{0.894118,0.101961,0.109804}%
\pgfsetfillcolor{currentfill}%
\pgfsetfillopacity{0.600000}%
\pgfsetlinewidth{1.003750pt}%
\definecolor{currentstroke}{rgb}{0.894118,0.101961,0.109804}%
\pgfsetstrokecolor{currentstroke}%
\pgfsetstrokeopacity{0.600000}%
\pgfsetdash{}{0pt}%
\pgfpathmoveto{\pgfqpoint{2.108585in}{2.223100in}}%
\pgfpathcurveto{\pgfqpoint{2.119636in}{2.223100in}}{\pgfqpoint{2.130235in}{2.227490in}}{\pgfqpoint{2.138048in}{2.235303in}}%
\pgfpathcurveto{\pgfqpoint{2.145862in}{2.243117in}}{\pgfqpoint{2.150252in}{2.253716in}}{\pgfqpoint{2.150252in}{2.264766in}}%
\pgfpathcurveto{\pgfqpoint{2.150252in}{2.275816in}}{\pgfqpoint{2.145862in}{2.286415in}}{\pgfqpoint{2.138048in}{2.294229in}}%
\pgfpathcurveto{\pgfqpoint{2.130235in}{2.302043in}}{\pgfqpoint{2.119636in}{2.306433in}}{\pgfqpoint{2.108585in}{2.306433in}}%
\pgfpathcurveto{\pgfqpoint{2.097535in}{2.306433in}}{\pgfqpoint{2.086936in}{2.302043in}}{\pgfqpoint{2.079123in}{2.294229in}}%
\pgfpathcurveto{\pgfqpoint{2.071309in}{2.286415in}}{\pgfqpoint{2.066919in}{2.275816in}}{\pgfqpoint{2.066919in}{2.264766in}}%
\pgfpathcurveto{\pgfqpoint{2.066919in}{2.253716in}}{\pgfqpoint{2.071309in}{2.243117in}}{\pgfqpoint{2.079123in}{2.235303in}}%
\pgfpathcurveto{\pgfqpoint{2.086936in}{2.227490in}}{\pgfqpoint{2.097535in}{2.223100in}}{\pgfqpoint{2.108585in}{2.223100in}}%
\pgfpathlineto{\pgfqpoint{2.108585in}{2.223100in}}%
\pgfpathclose%
\pgfusepath{stroke,fill}%
\end{pgfscope}%
\begin{pgfscope}%
\pgfpathrectangle{\pgfqpoint{0.800000in}{0.528000in}}{\pgfqpoint{4.960000in}{3.696000in}}%
\pgfusepath{clip}%
\pgfsetbuttcap%
\pgfsetroundjoin%
\definecolor{currentfill}{rgb}{0.894118,0.101961,0.109804}%
\pgfsetfillcolor{currentfill}%
\pgfsetfillopacity{0.600000}%
\pgfsetlinewidth{1.003750pt}%
\definecolor{currentstroke}{rgb}{0.894118,0.101961,0.109804}%
\pgfsetstrokecolor{currentstroke}%
\pgfsetstrokeopacity{0.600000}%
\pgfsetdash{}{0pt}%
\pgfpathmoveto{\pgfqpoint{1.754370in}{1.946416in}}%
\pgfpathcurveto{\pgfqpoint{1.765420in}{1.946416in}}{\pgfqpoint{1.776019in}{1.950807in}}{\pgfqpoint{1.783833in}{1.958620in}}%
\pgfpathcurveto{\pgfqpoint{1.791647in}{1.966434in}}{\pgfqpoint{1.796037in}{1.977033in}}{\pgfqpoint{1.796037in}{1.988083in}}%
\pgfpathcurveto{\pgfqpoint{1.796037in}{1.999133in}}{\pgfqpoint{1.791647in}{2.009732in}}{\pgfqpoint{1.783833in}{2.017546in}}%
\pgfpathcurveto{\pgfqpoint{1.776019in}{2.025360in}}{\pgfqpoint{1.765420in}{2.029750in}}{\pgfqpoint{1.754370in}{2.029750in}}%
\pgfpathcurveto{\pgfqpoint{1.743320in}{2.029750in}}{\pgfqpoint{1.732721in}{2.025360in}}{\pgfqpoint{1.724908in}{2.017546in}}%
\pgfpathcurveto{\pgfqpoint{1.717094in}{2.009732in}}{\pgfqpoint{1.712704in}{1.999133in}}{\pgfqpoint{1.712704in}{1.988083in}}%
\pgfpathcurveto{\pgfqpoint{1.712704in}{1.977033in}}{\pgfqpoint{1.717094in}{1.966434in}}{\pgfqpoint{1.724908in}{1.958620in}}%
\pgfpathcurveto{\pgfqpoint{1.732721in}{1.950807in}}{\pgfqpoint{1.743320in}{1.946416in}}{\pgfqpoint{1.754370in}{1.946416in}}%
\pgfpathlineto{\pgfqpoint{1.754370in}{1.946416in}}%
\pgfpathclose%
\pgfusepath{stroke,fill}%
\end{pgfscope}%
\begin{pgfscope}%
\pgfpathrectangle{\pgfqpoint{0.800000in}{0.528000in}}{\pgfqpoint{4.960000in}{3.696000in}}%
\pgfusepath{clip}%
\pgfsetbuttcap%
\pgfsetroundjoin%
\definecolor{currentfill}{rgb}{0.894118,0.101961,0.109804}%
\pgfsetfillcolor{currentfill}%
\pgfsetfillopacity{0.600000}%
\pgfsetlinewidth{1.003750pt}%
\definecolor{currentstroke}{rgb}{0.894118,0.101961,0.109804}%
\pgfsetstrokecolor{currentstroke}%
\pgfsetstrokeopacity{0.600000}%
\pgfsetdash{}{0pt}%
\pgfpathmoveto{\pgfqpoint{2.014848in}{2.115083in}}%
\pgfpathcurveto{\pgfqpoint{2.025899in}{2.115083in}}{\pgfqpoint{2.036498in}{2.119473in}}{\pgfqpoint{2.044311in}{2.127287in}}%
\pgfpathcurveto{\pgfqpoint{2.052125in}{2.135100in}}{\pgfqpoint{2.056515in}{2.145699in}}{\pgfqpoint{2.056515in}{2.156749in}}%
\pgfpathcurveto{\pgfqpoint{2.056515in}{2.167799in}}{\pgfqpoint{2.052125in}{2.178399in}}{\pgfqpoint{2.044311in}{2.186212in}}%
\pgfpathcurveto{\pgfqpoint{2.036498in}{2.194026in}}{\pgfqpoint{2.025899in}{2.198416in}}{\pgfqpoint{2.014848in}{2.198416in}}%
\pgfpathcurveto{\pgfqpoint{2.003798in}{2.198416in}}{\pgfqpoint{1.993199in}{2.194026in}}{\pgfqpoint{1.985386in}{2.186212in}}%
\pgfpathcurveto{\pgfqpoint{1.977572in}{2.178399in}}{\pgfqpoint{1.973182in}{2.167799in}}{\pgfqpoint{1.973182in}{2.156749in}}%
\pgfpathcurveto{\pgfqpoint{1.973182in}{2.145699in}}{\pgfqpoint{1.977572in}{2.135100in}}{\pgfqpoint{1.985386in}{2.127287in}}%
\pgfpathcurveto{\pgfqpoint{1.993199in}{2.119473in}}{\pgfqpoint{2.003798in}{2.115083in}}{\pgfqpoint{2.014848in}{2.115083in}}%
\pgfpathlineto{\pgfqpoint{2.014848in}{2.115083in}}%
\pgfpathclose%
\pgfusepath{stroke,fill}%
\end{pgfscope}%
\begin{pgfscope}%
\pgfpathrectangle{\pgfqpoint{0.800000in}{0.528000in}}{\pgfqpoint{4.960000in}{3.696000in}}%
\pgfusepath{clip}%
\pgfsetbuttcap%
\pgfsetroundjoin%
\definecolor{currentfill}{rgb}{0.894118,0.101961,0.109804}%
\pgfsetfillcolor{currentfill}%
\pgfsetfillopacity{0.600000}%
\pgfsetlinewidth{1.003750pt}%
\definecolor{currentstroke}{rgb}{0.894118,0.101961,0.109804}%
\pgfsetstrokecolor{currentstroke}%
\pgfsetstrokeopacity{0.600000}%
\pgfsetdash{}{0pt}%
\pgfpathmoveto{\pgfqpoint{1.954322in}{2.057339in}}%
\pgfpathcurveto{\pgfqpoint{1.965372in}{2.057339in}}{\pgfqpoint{1.975971in}{2.061730in}}{\pgfqpoint{1.983784in}{2.069543in}}%
\pgfpathcurveto{\pgfqpoint{1.991598in}{2.077357in}}{\pgfqpoint{1.995988in}{2.087956in}}{\pgfqpoint{1.995988in}{2.099006in}}%
\pgfpathcurveto{\pgfqpoint{1.995988in}{2.110056in}}{\pgfqpoint{1.991598in}{2.120655in}}{\pgfqpoint{1.983784in}{2.128469in}}%
\pgfpathcurveto{\pgfqpoint{1.975971in}{2.136282in}}{\pgfqpoint{1.965372in}{2.140673in}}{\pgfqpoint{1.954322in}{2.140673in}}%
\pgfpathcurveto{\pgfqpoint{1.943271in}{2.140673in}}{\pgfqpoint{1.932672in}{2.136282in}}{\pgfqpoint{1.924859in}{2.128469in}}%
\pgfpathcurveto{\pgfqpoint{1.917045in}{2.120655in}}{\pgfqpoint{1.912655in}{2.110056in}}{\pgfqpoint{1.912655in}{2.099006in}}%
\pgfpathcurveto{\pgfqpoint{1.912655in}{2.087956in}}{\pgfqpoint{1.917045in}{2.077357in}}{\pgfqpoint{1.924859in}{2.069543in}}%
\pgfpathcurveto{\pgfqpoint{1.932672in}{2.061730in}}{\pgfqpoint{1.943271in}{2.057339in}}{\pgfqpoint{1.954322in}{2.057339in}}%
\pgfpathlineto{\pgfqpoint{1.954322in}{2.057339in}}%
\pgfpathclose%
\pgfusepath{stroke,fill}%
\end{pgfscope}%
\begin{pgfscope}%
\pgfpathrectangle{\pgfqpoint{0.800000in}{0.528000in}}{\pgfqpoint{4.960000in}{3.696000in}}%
\pgfusepath{clip}%
\pgfsetbuttcap%
\pgfsetroundjoin%
\definecolor{currentfill}{rgb}{0.894118,0.101961,0.109804}%
\pgfsetfillcolor{currentfill}%
\pgfsetfillopacity{0.600000}%
\pgfsetlinewidth{1.003750pt}%
\definecolor{currentstroke}{rgb}{0.894118,0.101961,0.109804}%
\pgfsetstrokecolor{currentstroke}%
\pgfsetstrokeopacity{0.600000}%
\pgfsetdash{}{0pt}%
\pgfpathmoveto{\pgfqpoint{2.029242in}{2.000574in}}%
\pgfpathcurveto{\pgfqpoint{2.040292in}{2.000574in}}{\pgfqpoint{2.050891in}{2.004964in}}{\pgfqpoint{2.058705in}{2.012778in}}%
\pgfpathcurveto{\pgfqpoint{2.066518in}{2.020591in}}{\pgfqpoint{2.070909in}{2.031190in}}{\pgfqpoint{2.070909in}{2.042240in}}%
\pgfpathcurveto{\pgfqpoint{2.070909in}{2.053291in}}{\pgfqpoint{2.066518in}{2.063890in}}{\pgfqpoint{2.058705in}{2.071703in}}%
\pgfpathcurveto{\pgfqpoint{2.050891in}{2.079517in}}{\pgfqpoint{2.040292in}{2.083907in}}{\pgfqpoint{2.029242in}{2.083907in}}%
\pgfpathcurveto{\pgfqpoint{2.018192in}{2.083907in}}{\pgfqpoint{2.007593in}{2.079517in}}{\pgfqpoint{1.999779in}{2.071703in}}%
\pgfpathcurveto{\pgfqpoint{1.991966in}{2.063890in}}{\pgfqpoint{1.987575in}{2.053291in}}{\pgfqpoint{1.987575in}{2.042240in}}%
\pgfpathcurveto{\pgfqpoint{1.987575in}{2.031190in}}{\pgfqpoint{1.991966in}{2.020591in}}{\pgfqpoint{1.999779in}{2.012778in}}%
\pgfpathcurveto{\pgfqpoint{2.007593in}{2.004964in}}{\pgfqpoint{2.018192in}{2.000574in}}{\pgfqpoint{2.029242in}{2.000574in}}%
\pgfpathlineto{\pgfqpoint{2.029242in}{2.000574in}}%
\pgfpathclose%
\pgfusepath{stroke,fill}%
\end{pgfscope}%
\begin{pgfscope}%
\pgfpathrectangle{\pgfqpoint{0.800000in}{0.528000in}}{\pgfqpoint{4.960000in}{3.696000in}}%
\pgfusepath{clip}%
\pgfsetbuttcap%
\pgfsetroundjoin%
\definecolor{currentfill}{rgb}{0.894118,0.101961,0.109804}%
\pgfsetfillcolor{currentfill}%
\pgfsetfillopacity{0.600000}%
\pgfsetlinewidth{1.003750pt}%
\definecolor{currentstroke}{rgb}{0.894118,0.101961,0.109804}%
\pgfsetstrokecolor{currentstroke}%
\pgfsetstrokeopacity{0.600000}%
\pgfsetdash{}{0pt}%
\pgfpathmoveto{\pgfqpoint{1.964601in}{1.809159in}}%
\pgfpathcurveto{\pgfqpoint{1.975651in}{1.809159in}}{\pgfqpoint{1.986251in}{1.813549in}}{\pgfqpoint{1.994064in}{1.821363in}}%
\pgfpathcurveto{\pgfqpoint{2.001878in}{1.829176in}}{\pgfqpoint{2.006268in}{1.839775in}}{\pgfqpoint{2.006268in}{1.850826in}}%
\pgfpathcurveto{\pgfqpoint{2.006268in}{1.861876in}}{\pgfqpoint{2.001878in}{1.872475in}}{\pgfqpoint{1.994064in}{1.880288in}}%
\pgfpathcurveto{\pgfqpoint{1.986251in}{1.888102in}}{\pgfqpoint{1.975651in}{1.892492in}}{\pgfqpoint{1.964601in}{1.892492in}}%
\pgfpathcurveto{\pgfqpoint{1.953551in}{1.892492in}}{\pgfqpoint{1.942952in}{1.888102in}}{\pgfqpoint{1.935139in}{1.880288in}}%
\pgfpathcurveto{\pgfqpoint{1.927325in}{1.872475in}}{\pgfqpoint{1.922935in}{1.861876in}}{\pgfqpoint{1.922935in}{1.850826in}}%
\pgfpathcurveto{\pgfqpoint{1.922935in}{1.839775in}}{\pgfqpoint{1.927325in}{1.829176in}}{\pgfqpoint{1.935139in}{1.821363in}}%
\pgfpathcurveto{\pgfqpoint{1.942952in}{1.813549in}}{\pgfqpoint{1.953551in}{1.809159in}}{\pgfqpoint{1.964601in}{1.809159in}}%
\pgfpathlineto{\pgfqpoint{1.964601in}{1.809159in}}%
\pgfpathclose%
\pgfusepath{stroke,fill}%
\end{pgfscope}%
\begin{pgfscope}%
\pgfpathrectangle{\pgfqpoint{0.800000in}{0.528000in}}{\pgfqpoint{4.960000in}{3.696000in}}%
\pgfusepath{clip}%
\pgfsetbuttcap%
\pgfsetroundjoin%
\definecolor{currentfill}{rgb}{0.894118,0.101961,0.109804}%
\pgfsetfillcolor{currentfill}%
\pgfsetfillopacity{0.600000}%
\pgfsetlinewidth{1.003750pt}%
\definecolor{currentstroke}{rgb}{0.894118,0.101961,0.109804}%
\pgfsetstrokecolor{currentstroke}%
\pgfsetstrokeopacity{0.600000}%
\pgfsetdash{}{0pt}%
\pgfpathmoveto{\pgfqpoint{2.096816in}{1.838458in}}%
\pgfpathcurveto{\pgfqpoint{2.107866in}{1.838458in}}{\pgfqpoint{2.118465in}{1.842848in}}{\pgfqpoint{2.126279in}{1.850662in}}%
\pgfpathcurveto{\pgfqpoint{2.134092in}{1.858476in}}{\pgfqpoint{2.138482in}{1.869075in}}{\pgfqpoint{2.138482in}{1.880125in}}%
\pgfpathcurveto{\pgfqpoint{2.138482in}{1.891175in}}{\pgfqpoint{2.134092in}{1.901774in}}{\pgfqpoint{2.126279in}{1.909587in}}%
\pgfpathcurveto{\pgfqpoint{2.118465in}{1.917401in}}{\pgfqpoint{2.107866in}{1.921791in}}{\pgfqpoint{2.096816in}{1.921791in}}%
\pgfpathcurveto{\pgfqpoint{2.085766in}{1.921791in}}{\pgfqpoint{2.075167in}{1.917401in}}{\pgfqpoint{2.067353in}{1.909587in}}%
\pgfpathcurveto{\pgfqpoint{2.059539in}{1.901774in}}{\pgfqpoint{2.055149in}{1.891175in}}{\pgfqpoint{2.055149in}{1.880125in}}%
\pgfpathcurveto{\pgfqpoint{2.055149in}{1.869075in}}{\pgfqpoint{2.059539in}{1.858476in}}{\pgfqpoint{2.067353in}{1.850662in}}%
\pgfpathcurveto{\pgfqpoint{2.075167in}{1.842848in}}{\pgfqpoint{2.085766in}{1.838458in}}{\pgfqpoint{2.096816in}{1.838458in}}%
\pgfpathlineto{\pgfqpoint{2.096816in}{1.838458in}}%
\pgfpathclose%
\pgfusepath{stroke,fill}%
\end{pgfscope}%
\begin{pgfscope}%
\pgfpathrectangle{\pgfqpoint{0.800000in}{0.528000in}}{\pgfqpoint{4.960000in}{3.696000in}}%
\pgfusepath{clip}%
\pgfsetbuttcap%
\pgfsetroundjoin%
\definecolor{currentfill}{rgb}{0.894118,0.101961,0.109804}%
\pgfsetfillcolor{currentfill}%
\pgfsetfillopacity{0.600000}%
\pgfsetlinewidth{1.003750pt}%
\definecolor{currentstroke}{rgb}{0.894118,0.101961,0.109804}%
\pgfsetstrokecolor{currentstroke}%
\pgfsetstrokeopacity{0.600000}%
\pgfsetdash{}{0pt}%
\pgfpathmoveto{\pgfqpoint{2.026008in}{1.662632in}}%
\pgfpathcurveto{\pgfqpoint{2.037058in}{1.662632in}}{\pgfqpoint{2.047657in}{1.667023in}}{\pgfqpoint{2.055470in}{1.674836in}}%
\pgfpathcurveto{\pgfqpoint{2.063284in}{1.682650in}}{\pgfqpoint{2.067674in}{1.693249in}}{\pgfqpoint{2.067674in}{1.704299in}}%
\pgfpathcurveto{\pgfqpoint{2.067674in}{1.715349in}}{\pgfqpoint{2.063284in}{1.725948in}}{\pgfqpoint{2.055470in}{1.733762in}}%
\pgfpathcurveto{\pgfqpoint{2.047657in}{1.741575in}}{\pgfqpoint{2.037058in}{1.745966in}}{\pgfqpoint{2.026008in}{1.745966in}}%
\pgfpathcurveto{\pgfqpoint{2.014958in}{1.745966in}}{\pgfqpoint{2.004359in}{1.741575in}}{\pgfqpoint{1.996545in}{1.733762in}}%
\pgfpathcurveto{\pgfqpoint{1.988731in}{1.725948in}}{\pgfqpoint{1.984341in}{1.715349in}}{\pgfqpoint{1.984341in}{1.704299in}}%
\pgfpathcurveto{\pgfqpoint{1.984341in}{1.693249in}}{\pgfqpoint{1.988731in}{1.682650in}}{\pgfqpoint{1.996545in}{1.674836in}}%
\pgfpathcurveto{\pgfqpoint{2.004359in}{1.667023in}}{\pgfqpoint{2.014958in}{1.662632in}}{\pgfqpoint{2.026008in}{1.662632in}}%
\pgfpathlineto{\pgfqpoint{2.026008in}{1.662632in}}%
\pgfpathclose%
\pgfusepath{stroke,fill}%
\end{pgfscope}%
\begin{pgfscope}%
\pgfpathrectangle{\pgfqpoint{0.800000in}{0.528000in}}{\pgfqpoint{4.960000in}{3.696000in}}%
\pgfusepath{clip}%
\pgfsetbuttcap%
\pgfsetroundjoin%
\definecolor{currentfill}{rgb}{0.894118,0.101961,0.109804}%
\pgfsetfillcolor{currentfill}%
\pgfsetfillopacity{0.600000}%
\pgfsetlinewidth{1.003750pt}%
\definecolor{currentstroke}{rgb}{0.894118,0.101961,0.109804}%
\pgfsetstrokecolor{currentstroke}%
\pgfsetstrokeopacity{0.600000}%
\pgfsetdash{}{0pt}%
\pgfpathmoveto{\pgfqpoint{2.499207in}{1.947749in}}%
\pgfpathcurveto{\pgfqpoint{2.510257in}{1.947749in}}{\pgfqpoint{2.520856in}{1.952140in}}{\pgfqpoint{2.528670in}{1.959953in}}%
\pgfpathcurveto{\pgfqpoint{2.536484in}{1.967767in}}{\pgfqpoint{2.540874in}{1.978366in}}{\pgfqpoint{2.540874in}{1.989416in}}%
\pgfpathcurveto{\pgfqpoint{2.540874in}{2.000466in}}{\pgfqpoint{2.536484in}{2.011065in}}{\pgfqpoint{2.528670in}{2.018879in}}%
\pgfpathcurveto{\pgfqpoint{2.520856in}{2.026692in}}{\pgfqpoint{2.510257in}{2.031083in}}{\pgfqpoint{2.499207in}{2.031083in}}%
\pgfpathcurveto{\pgfqpoint{2.488157in}{2.031083in}}{\pgfqpoint{2.477558in}{2.026692in}}{\pgfqpoint{2.469745in}{2.018879in}}%
\pgfpathcurveto{\pgfqpoint{2.461931in}{2.011065in}}{\pgfqpoint{2.457541in}{2.000466in}}{\pgfqpoint{2.457541in}{1.989416in}}%
\pgfpathcurveto{\pgfqpoint{2.457541in}{1.978366in}}{\pgfqpoint{2.461931in}{1.967767in}}{\pgfqpoint{2.469745in}{1.959953in}}%
\pgfpathcurveto{\pgfqpoint{2.477558in}{1.952140in}}{\pgfqpoint{2.488157in}{1.947749in}}{\pgfqpoint{2.499207in}{1.947749in}}%
\pgfpathlineto{\pgfqpoint{2.499207in}{1.947749in}}%
\pgfpathclose%
\pgfusepath{stroke,fill}%
\end{pgfscope}%
\begin{pgfscope}%
\pgfpathrectangle{\pgfqpoint{0.800000in}{0.528000in}}{\pgfqpoint{4.960000in}{3.696000in}}%
\pgfusepath{clip}%
\pgfsetbuttcap%
\pgfsetroundjoin%
\definecolor{currentfill}{rgb}{0.894118,0.101961,0.109804}%
\pgfsetfillcolor{currentfill}%
\pgfsetfillopacity{0.600000}%
\pgfsetlinewidth{1.003750pt}%
\definecolor{currentstroke}{rgb}{0.894118,0.101961,0.109804}%
\pgfsetstrokecolor{currentstroke}%
\pgfsetstrokeopacity{0.600000}%
\pgfsetdash{}{0pt}%
\pgfpathmoveto{\pgfqpoint{2.089395in}{1.559693in}}%
\pgfpathcurveto{\pgfqpoint{2.100445in}{1.559693in}}{\pgfqpoint{2.111044in}{1.564083in}}{\pgfqpoint{2.118857in}{1.571897in}}%
\pgfpathcurveto{\pgfqpoint{2.126671in}{1.579710in}}{\pgfqpoint{2.131061in}{1.590310in}}{\pgfqpoint{2.131061in}{1.601360in}}%
\pgfpathcurveto{\pgfqpoint{2.131061in}{1.612410in}}{\pgfqpoint{2.126671in}{1.623009in}}{\pgfqpoint{2.118857in}{1.630822in}}%
\pgfpathcurveto{\pgfqpoint{2.111044in}{1.638636in}}{\pgfqpoint{2.100445in}{1.643026in}}{\pgfqpoint{2.089395in}{1.643026in}}%
\pgfpathcurveto{\pgfqpoint{2.078344in}{1.643026in}}{\pgfqpoint{2.067745in}{1.638636in}}{\pgfqpoint{2.059932in}{1.630822in}}%
\pgfpathcurveto{\pgfqpoint{2.052118in}{1.623009in}}{\pgfqpoint{2.047728in}{1.612410in}}{\pgfqpoint{2.047728in}{1.601360in}}%
\pgfpathcurveto{\pgfqpoint{2.047728in}{1.590310in}}{\pgfqpoint{2.052118in}{1.579710in}}{\pgfqpoint{2.059932in}{1.571897in}}%
\pgfpathcurveto{\pgfqpoint{2.067745in}{1.564083in}}{\pgfqpoint{2.078344in}{1.559693in}}{\pgfqpoint{2.089395in}{1.559693in}}%
\pgfpathlineto{\pgfqpoint{2.089395in}{1.559693in}}%
\pgfpathclose%
\pgfusepath{stroke,fill}%
\end{pgfscope}%
\begin{pgfscope}%
\pgfpathrectangle{\pgfqpoint{0.800000in}{0.528000in}}{\pgfqpoint{4.960000in}{3.696000in}}%
\pgfusepath{clip}%
\pgfsetbuttcap%
\pgfsetroundjoin%
\definecolor{currentfill}{rgb}{0.894118,0.101961,0.109804}%
\pgfsetfillcolor{currentfill}%
\pgfsetfillopacity{0.600000}%
\pgfsetlinewidth{1.003750pt}%
\definecolor{currentstroke}{rgb}{0.894118,0.101961,0.109804}%
\pgfsetstrokecolor{currentstroke}%
\pgfsetstrokeopacity{0.600000}%
\pgfsetdash{}{0pt}%
\pgfpathmoveto{\pgfqpoint{3.127107in}{2.267487in}}%
\pgfpathcurveto{\pgfqpoint{3.138157in}{2.267487in}}{\pgfqpoint{3.148756in}{2.271878in}}{\pgfqpoint{3.156570in}{2.279691in}}%
\pgfpathcurveto{\pgfqpoint{3.164383in}{2.287505in}}{\pgfqpoint{3.168773in}{2.298104in}}{\pgfqpoint{3.168773in}{2.309154in}}%
\pgfpathcurveto{\pgfqpoint{3.168773in}{2.320204in}}{\pgfqpoint{3.164383in}{2.330803in}}{\pgfqpoint{3.156570in}{2.338617in}}%
\pgfpathcurveto{\pgfqpoint{3.148756in}{2.346430in}}{\pgfqpoint{3.138157in}{2.350821in}}{\pgfqpoint{3.127107in}{2.350821in}}%
\pgfpathcurveto{\pgfqpoint{3.116057in}{2.350821in}}{\pgfqpoint{3.105458in}{2.346430in}}{\pgfqpoint{3.097644in}{2.338617in}}%
\pgfpathcurveto{\pgfqpoint{3.089830in}{2.330803in}}{\pgfqpoint{3.085440in}{2.320204in}}{\pgfqpoint{3.085440in}{2.309154in}}%
\pgfpathcurveto{\pgfqpoint{3.085440in}{2.298104in}}{\pgfqpoint{3.089830in}{2.287505in}}{\pgfqpoint{3.097644in}{2.279691in}}%
\pgfpathcurveto{\pgfqpoint{3.105458in}{2.271878in}}{\pgfqpoint{3.116057in}{2.267487in}}{\pgfqpoint{3.127107in}{2.267487in}}%
\pgfpathlineto{\pgfqpoint{3.127107in}{2.267487in}}%
\pgfpathclose%
\pgfusepath{stroke,fill}%
\end{pgfscope}%
\begin{pgfscope}%
\pgfpathrectangle{\pgfqpoint{0.800000in}{0.528000in}}{\pgfqpoint{4.960000in}{3.696000in}}%
\pgfusepath{clip}%
\pgfsetbuttcap%
\pgfsetroundjoin%
\definecolor{currentfill}{rgb}{0.894118,0.101961,0.109804}%
\pgfsetfillcolor{currentfill}%
\pgfsetfillopacity{0.600000}%
\pgfsetlinewidth{1.003750pt}%
\definecolor{currentstroke}{rgb}{0.894118,0.101961,0.109804}%
\pgfsetstrokecolor{currentstroke}%
\pgfsetstrokeopacity{0.600000}%
\pgfsetdash{}{0pt}%
\pgfpathmoveto{\pgfqpoint{2.216791in}{1.637269in}}%
\pgfpathcurveto{\pgfqpoint{2.227842in}{1.637269in}}{\pgfqpoint{2.238441in}{1.641660in}}{\pgfqpoint{2.246254in}{1.649473in}}%
\pgfpathcurveto{\pgfqpoint{2.254068in}{1.657287in}}{\pgfqpoint{2.258458in}{1.667886in}}{\pgfqpoint{2.258458in}{1.678936in}}%
\pgfpathcurveto{\pgfqpoint{2.258458in}{1.689986in}}{\pgfqpoint{2.254068in}{1.700585in}}{\pgfqpoint{2.246254in}{1.708399in}}%
\pgfpathcurveto{\pgfqpoint{2.238441in}{1.716212in}}{\pgfqpoint{2.227842in}{1.720603in}}{\pgfqpoint{2.216791in}{1.720603in}}%
\pgfpathcurveto{\pgfqpoint{2.205741in}{1.720603in}}{\pgfqpoint{2.195142in}{1.716212in}}{\pgfqpoint{2.187329in}{1.708399in}}%
\pgfpathcurveto{\pgfqpoint{2.179515in}{1.700585in}}{\pgfqpoint{2.175125in}{1.689986in}}{\pgfqpoint{2.175125in}{1.678936in}}%
\pgfpathcurveto{\pgfqpoint{2.175125in}{1.667886in}}{\pgfqpoint{2.179515in}{1.657287in}}{\pgfqpoint{2.187329in}{1.649473in}}%
\pgfpathcurveto{\pgfqpoint{2.195142in}{1.641660in}}{\pgfqpoint{2.205741in}{1.637269in}}{\pgfqpoint{2.216791in}{1.637269in}}%
\pgfpathlineto{\pgfqpoint{2.216791in}{1.637269in}}%
\pgfpathclose%
\pgfusepath{stroke,fill}%
\end{pgfscope}%
\begin{pgfscope}%
\pgfpathrectangle{\pgfqpoint{0.800000in}{0.528000in}}{\pgfqpoint{4.960000in}{3.696000in}}%
\pgfusepath{clip}%
\pgfsetbuttcap%
\pgfsetroundjoin%
\definecolor{currentfill}{rgb}{0.894118,0.101961,0.109804}%
\pgfsetfillcolor{currentfill}%
\pgfsetfillopacity{0.600000}%
\pgfsetlinewidth{1.003750pt}%
\definecolor{currentstroke}{rgb}{0.894118,0.101961,0.109804}%
\pgfsetstrokecolor{currentstroke}%
\pgfsetstrokeopacity{0.600000}%
\pgfsetdash{}{0pt}%
\pgfpathmoveto{\pgfqpoint{2.554304in}{1.836097in}}%
\pgfpathcurveto{\pgfqpoint{2.565354in}{1.836097in}}{\pgfqpoint{2.575953in}{1.840487in}}{\pgfqpoint{2.583767in}{1.848300in}}%
\pgfpathcurveto{\pgfqpoint{2.591580in}{1.856114in}}{\pgfqpoint{2.595971in}{1.866713in}}{\pgfqpoint{2.595971in}{1.877763in}}%
\pgfpathcurveto{\pgfqpoint{2.595971in}{1.888813in}}{\pgfqpoint{2.591580in}{1.899412in}}{\pgfqpoint{2.583767in}{1.907226in}}%
\pgfpathcurveto{\pgfqpoint{2.575953in}{1.915040in}}{\pgfqpoint{2.565354in}{1.919430in}}{\pgfqpoint{2.554304in}{1.919430in}}%
\pgfpathcurveto{\pgfqpoint{2.543254in}{1.919430in}}{\pgfqpoint{2.532655in}{1.915040in}}{\pgfqpoint{2.524841in}{1.907226in}}%
\pgfpathcurveto{\pgfqpoint{2.517028in}{1.899412in}}{\pgfqpoint{2.512637in}{1.888813in}}{\pgfqpoint{2.512637in}{1.877763in}}%
\pgfpathcurveto{\pgfqpoint{2.512637in}{1.866713in}}{\pgfqpoint{2.517028in}{1.856114in}}{\pgfqpoint{2.524841in}{1.848300in}}%
\pgfpathcurveto{\pgfqpoint{2.532655in}{1.840487in}}{\pgfqpoint{2.543254in}{1.836097in}}{\pgfqpoint{2.554304in}{1.836097in}}%
\pgfpathlineto{\pgfqpoint{2.554304in}{1.836097in}}%
\pgfpathclose%
\pgfusepath{stroke,fill}%
\end{pgfscope}%
\begin{pgfscope}%
\pgfpathrectangle{\pgfqpoint{0.800000in}{0.528000in}}{\pgfqpoint{4.960000in}{3.696000in}}%
\pgfusepath{clip}%
\pgfsetbuttcap%
\pgfsetroundjoin%
\definecolor{currentfill}{rgb}{0.894118,0.101961,0.109804}%
\pgfsetfillcolor{currentfill}%
\pgfsetfillopacity{0.600000}%
\pgfsetlinewidth{1.003750pt}%
\definecolor{currentstroke}{rgb}{0.894118,0.101961,0.109804}%
\pgfsetstrokecolor{currentstroke}%
\pgfsetstrokeopacity{0.600000}%
\pgfsetdash{}{0pt}%
\pgfpathmoveto{\pgfqpoint{2.474950in}{1.750214in}}%
\pgfpathcurveto{\pgfqpoint{2.486000in}{1.750214in}}{\pgfqpoint{2.496599in}{1.754605in}}{\pgfqpoint{2.504412in}{1.762418in}}%
\pgfpathcurveto{\pgfqpoint{2.512226in}{1.770232in}}{\pgfqpoint{2.516616in}{1.780831in}}{\pgfqpoint{2.516616in}{1.791881in}}%
\pgfpathcurveto{\pgfqpoint{2.516616in}{1.802931in}}{\pgfqpoint{2.512226in}{1.813530in}}{\pgfqpoint{2.504412in}{1.821344in}}%
\pgfpathcurveto{\pgfqpoint{2.496599in}{1.829157in}}{\pgfqpoint{2.486000in}{1.833548in}}{\pgfqpoint{2.474950in}{1.833548in}}%
\pgfpathcurveto{\pgfqpoint{2.463899in}{1.833548in}}{\pgfqpoint{2.453300in}{1.829157in}}{\pgfqpoint{2.445487in}{1.821344in}}%
\pgfpathcurveto{\pgfqpoint{2.437673in}{1.813530in}}{\pgfqpoint{2.433283in}{1.802931in}}{\pgfqpoint{2.433283in}{1.791881in}}%
\pgfpathcurveto{\pgfqpoint{2.433283in}{1.780831in}}{\pgfqpoint{2.437673in}{1.770232in}}{\pgfqpoint{2.445487in}{1.762418in}}%
\pgfpathcurveto{\pgfqpoint{2.453300in}{1.754605in}}{\pgfqpoint{2.463899in}{1.750214in}}{\pgfqpoint{2.474950in}{1.750214in}}%
\pgfpathlineto{\pgfqpoint{2.474950in}{1.750214in}}%
\pgfpathclose%
\pgfusepath{stroke,fill}%
\end{pgfscope}%
\begin{pgfscope}%
\pgfpathrectangle{\pgfqpoint{0.800000in}{0.528000in}}{\pgfqpoint{4.960000in}{3.696000in}}%
\pgfusepath{clip}%
\pgfsetbuttcap%
\pgfsetroundjoin%
\definecolor{currentfill}{rgb}{0.894118,0.101961,0.109804}%
\pgfsetfillcolor{currentfill}%
\pgfsetfillopacity{0.600000}%
\pgfsetlinewidth{1.003750pt}%
\definecolor{currentstroke}{rgb}{0.894118,0.101961,0.109804}%
\pgfsetstrokecolor{currentstroke}%
\pgfsetstrokeopacity{0.600000}%
\pgfsetdash{}{0pt}%
\pgfpathmoveto{\pgfqpoint{2.586970in}{1.801668in}}%
\pgfpathcurveto{\pgfqpoint{2.598020in}{1.801668in}}{\pgfqpoint{2.608619in}{1.806059in}}{\pgfqpoint{2.616432in}{1.813872in}}%
\pgfpathcurveto{\pgfqpoint{2.624246in}{1.821686in}}{\pgfqpoint{2.628636in}{1.832285in}}{\pgfqpoint{2.628636in}{1.843335in}}%
\pgfpathcurveto{\pgfqpoint{2.628636in}{1.854385in}}{\pgfqpoint{2.624246in}{1.864984in}}{\pgfqpoint{2.616432in}{1.872798in}}%
\pgfpathcurveto{\pgfqpoint{2.608619in}{1.880611in}}{\pgfqpoint{2.598020in}{1.885002in}}{\pgfqpoint{2.586970in}{1.885002in}}%
\pgfpathcurveto{\pgfqpoint{2.575919in}{1.885002in}}{\pgfqpoint{2.565320in}{1.880611in}}{\pgfqpoint{2.557507in}{1.872798in}}%
\pgfpathcurveto{\pgfqpoint{2.549693in}{1.864984in}}{\pgfqpoint{2.545303in}{1.854385in}}{\pgfqpoint{2.545303in}{1.843335in}}%
\pgfpathcurveto{\pgfqpoint{2.545303in}{1.832285in}}{\pgfqpoint{2.549693in}{1.821686in}}{\pgfqpoint{2.557507in}{1.813872in}}%
\pgfpathcurveto{\pgfqpoint{2.565320in}{1.806059in}}{\pgfqpoint{2.575919in}{1.801668in}}{\pgfqpoint{2.586970in}{1.801668in}}%
\pgfpathlineto{\pgfqpoint{2.586970in}{1.801668in}}%
\pgfpathclose%
\pgfusepath{stroke,fill}%
\end{pgfscope}%
\begin{pgfscope}%
\pgfpathrectangle{\pgfqpoint{0.800000in}{0.528000in}}{\pgfqpoint{4.960000in}{3.696000in}}%
\pgfusepath{clip}%
\pgfsetbuttcap%
\pgfsetroundjoin%
\definecolor{currentfill}{rgb}{0.894118,0.101961,0.109804}%
\pgfsetfillcolor{currentfill}%
\pgfsetfillopacity{0.600000}%
\pgfsetlinewidth{1.003750pt}%
\definecolor{currentstroke}{rgb}{0.894118,0.101961,0.109804}%
\pgfsetstrokecolor{currentstroke}%
\pgfsetstrokeopacity{0.600000}%
\pgfsetdash{}{0pt}%
\pgfpathmoveto{\pgfqpoint{2.832971in}{1.952978in}}%
\pgfpathcurveto{\pgfqpoint{2.844021in}{1.952978in}}{\pgfqpoint{2.854620in}{1.957368in}}{\pgfqpoint{2.862433in}{1.965182in}}%
\pgfpathcurveto{\pgfqpoint{2.870247in}{1.972996in}}{\pgfqpoint{2.874637in}{1.983595in}}{\pgfqpoint{2.874637in}{1.994645in}}%
\pgfpathcurveto{\pgfqpoint{2.874637in}{2.005695in}}{\pgfqpoint{2.870247in}{2.016294in}}{\pgfqpoint{2.862433in}{2.024107in}}%
\pgfpathcurveto{\pgfqpoint{2.854620in}{2.031921in}}{\pgfqpoint{2.844021in}{2.036311in}}{\pgfqpoint{2.832971in}{2.036311in}}%
\pgfpathcurveto{\pgfqpoint{2.821920in}{2.036311in}}{\pgfqpoint{2.811321in}{2.031921in}}{\pgfqpoint{2.803508in}{2.024107in}}%
\pgfpathcurveto{\pgfqpoint{2.795694in}{2.016294in}}{\pgfqpoint{2.791304in}{2.005695in}}{\pgfqpoint{2.791304in}{1.994645in}}%
\pgfpathcurveto{\pgfqpoint{2.791304in}{1.983595in}}{\pgfqpoint{2.795694in}{1.972996in}}{\pgfqpoint{2.803508in}{1.965182in}}%
\pgfpathcurveto{\pgfqpoint{2.811321in}{1.957368in}}{\pgfqpoint{2.821920in}{1.952978in}}{\pgfqpoint{2.832971in}{1.952978in}}%
\pgfpathlineto{\pgfqpoint{2.832971in}{1.952978in}}%
\pgfpathclose%
\pgfusepath{stroke,fill}%
\end{pgfscope}%
\begin{pgfscope}%
\pgfpathrectangle{\pgfqpoint{0.800000in}{0.528000in}}{\pgfqpoint{4.960000in}{3.696000in}}%
\pgfusepath{clip}%
\pgfsetbuttcap%
\pgfsetroundjoin%
\definecolor{currentfill}{rgb}{0.894118,0.101961,0.109804}%
\pgfsetfillcolor{currentfill}%
\pgfsetfillopacity{0.600000}%
\pgfsetlinewidth{1.003750pt}%
\definecolor{currentstroke}{rgb}{0.894118,0.101961,0.109804}%
\pgfsetstrokecolor{currentstroke}%
\pgfsetstrokeopacity{0.600000}%
\pgfsetdash{}{0pt}%
\pgfpathmoveto{\pgfqpoint{2.133860in}{1.451560in}}%
\pgfpathcurveto{\pgfqpoint{2.144910in}{1.451560in}}{\pgfqpoint{2.155509in}{1.455950in}}{\pgfqpoint{2.163322in}{1.463764in}}%
\pgfpathcurveto{\pgfqpoint{2.171136in}{1.471577in}}{\pgfqpoint{2.175526in}{1.482177in}}{\pgfqpoint{2.175526in}{1.493227in}}%
\pgfpathcurveto{\pgfqpoint{2.175526in}{1.504277in}}{\pgfqpoint{2.171136in}{1.514876in}}{\pgfqpoint{2.163322in}{1.522689in}}%
\pgfpathcurveto{\pgfqpoint{2.155509in}{1.530503in}}{\pgfqpoint{2.144910in}{1.534893in}}{\pgfqpoint{2.133860in}{1.534893in}}%
\pgfpathcurveto{\pgfqpoint{2.122810in}{1.534893in}}{\pgfqpoint{2.112210in}{1.530503in}}{\pgfqpoint{2.104397in}{1.522689in}}%
\pgfpathcurveto{\pgfqpoint{2.096583in}{1.514876in}}{\pgfqpoint{2.092193in}{1.504277in}}{\pgfqpoint{2.092193in}{1.493227in}}%
\pgfpathcurveto{\pgfqpoint{2.092193in}{1.482177in}}{\pgfqpoint{2.096583in}{1.471577in}}{\pgfqpoint{2.104397in}{1.463764in}}%
\pgfpathcurveto{\pgfqpoint{2.112210in}{1.455950in}}{\pgfqpoint{2.122810in}{1.451560in}}{\pgfqpoint{2.133860in}{1.451560in}}%
\pgfpathlineto{\pgfqpoint{2.133860in}{1.451560in}}%
\pgfpathclose%
\pgfusepath{stroke,fill}%
\end{pgfscope}%
\begin{pgfscope}%
\pgfpathrectangle{\pgfqpoint{0.800000in}{0.528000in}}{\pgfqpoint{4.960000in}{3.696000in}}%
\pgfusepath{clip}%
\pgfsetbuttcap%
\pgfsetroundjoin%
\definecolor{currentfill}{rgb}{0.894118,0.101961,0.109804}%
\pgfsetfillcolor{currentfill}%
\pgfsetfillopacity{0.600000}%
\pgfsetlinewidth{1.003750pt}%
\definecolor{currentstroke}{rgb}{0.894118,0.101961,0.109804}%
\pgfsetstrokecolor{currentstroke}%
\pgfsetstrokeopacity{0.600000}%
\pgfsetdash{}{0pt}%
\pgfpathmoveto{\pgfqpoint{2.545539in}{1.713070in}}%
\pgfpathcurveto{\pgfqpoint{2.556589in}{1.713070in}}{\pgfqpoint{2.567188in}{1.717460in}}{\pgfqpoint{2.575001in}{1.725274in}}%
\pgfpathcurveto{\pgfqpoint{2.582815in}{1.733087in}}{\pgfqpoint{2.587205in}{1.743686in}}{\pgfqpoint{2.587205in}{1.754737in}}%
\pgfpathcurveto{\pgfqpoint{2.587205in}{1.765787in}}{\pgfqpoint{2.582815in}{1.776386in}}{\pgfqpoint{2.575001in}{1.784199in}}%
\pgfpathcurveto{\pgfqpoint{2.567188in}{1.792013in}}{\pgfqpoint{2.556589in}{1.796403in}}{\pgfqpoint{2.545539in}{1.796403in}}%
\pgfpathcurveto{\pgfqpoint{2.534488in}{1.796403in}}{\pgfqpoint{2.523889in}{1.792013in}}{\pgfqpoint{2.516076in}{1.784199in}}%
\pgfpathcurveto{\pgfqpoint{2.508262in}{1.776386in}}{\pgfqpoint{2.503872in}{1.765787in}}{\pgfqpoint{2.503872in}{1.754737in}}%
\pgfpathcurveto{\pgfqpoint{2.503872in}{1.743686in}}{\pgfqpoint{2.508262in}{1.733087in}}{\pgfqpoint{2.516076in}{1.725274in}}%
\pgfpathcurveto{\pgfqpoint{2.523889in}{1.717460in}}{\pgfqpoint{2.534488in}{1.713070in}}{\pgfqpoint{2.545539in}{1.713070in}}%
\pgfpathlineto{\pgfqpoint{2.545539in}{1.713070in}}%
\pgfpathclose%
\pgfusepath{stroke,fill}%
\end{pgfscope}%
\begin{pgfscope}%
\pgfsetbuttcap%
\pgfsetroundjoin%
\definecolor{currentfill}{rgb}{0.000000,0.000000,0.000000}%
\pgfsetfillcolor{currentfill}%
\pgfsetlinewidth{0.803000pt}%
\definecolor{currentstroke}{rgb}{0.000000,0.000000,0.000000}%
\pgfsetstrokecolor{currentstroke}%
\pgfsetdash{}{0pt}%
\pgfsys@defobject{currentmarker}{\pgfqpoint{0.000000in}{-0.048611in}}{\pgfqpoint{0.000000in}{0.000000in}}{%
\pgfpathmoveto{\pgfqpoint{0.000000in}{0.000000in}}%
\pgfpathlineto{\pgfqpoint{0.000000in}{-0.048611in}}%
\pgfusepath{stroke,fill}%
}%
\begin{pgfscope}%
\pgfsys@transformshift{0.819493in}{0.528000in}%
\pgfsys@useobject{currentmarker}{}%
\end{pgfscope}%
\end{pgfscope}%
\begin{pgfscope}%
\definecolor{textcolor}{rgb}{0.000000,0.000000,0.000000}%
\pgfsetstrokecolor{textcolor}%
\pgfsetfillcolor{textcolor}%
\pgftext[x=0.819493in,y=0.430778in,,top]{\color{textcolor}{\rmfamily\fontsize{10.000000}{12.000000}\selectfont\catcode`\^=\active\def^{\ifmmode\sp\else\^{}\fi}\catcode`\%=\active\def%{\%}$\mathdefault{0.0}$}}%
\end{pgfscope}%
\begin{pgfscope}%
\pgfsetbuttcap%
\pgfsetroundjoin%
\definecolor{currentfill}{rgb}{0.000000,0.000000,0.000000}%
\pgfsetfillcolor{currentfill}%
\pgfsetlinewidth{0.803000pt}%
\definecolor{currentstroke}{rgb}{0.000000,0.000000,0.000000}%
\pgfsetstrokecolor{currentstroke}%
\pgfsetdash{}{0pt}%
\pgfsys@defobject{currentmarker}{\pgfqpoint{0.000000in}{-0.048611in}}{\pgfqpoint{0.000000in}{0.000000in}}{%
\pgfpathmoveto{\pgfqpoint{0.000000in}{0.000000in}}%
\pgfpathlineto{\pgfqpoint{0.000000in}{-0.048611in}}%
\pgfusepath{stroke,fill}%
}%
\begin{pgfscope}%
\pgfsys@transformshift{1.890217in}{0.528000in}%
\pgfsys@useobject{currentmarker}{}%
\end{pgfscope}%
\end{pgfscope}%
\begin{pgfscope}%
\definecolor{textcolor}{rgb}{0.000000,0.000000,0.000000}%
\pgfsetstrokecolor{textcolor}%
\pgfsetfillcolor{textcolor}%
\pgftext[x=1.890217in,y=0.430778in,,top]{\color{textcolor}{\rmfamily\fontsize{10.000000}{12.000000}\selectfont\catcode`\^=\active\def^{\ifmmode\sp\else\^{}\fi}\catcode`\%=\active\def%{\%}$\mathdefault{0.1}$}}%
\end{pgfscope}%
\begin{pgfscope}%
\pgfsetbuttcap%
\pgfsetroundjoin%
\definecolor{currentfill}{rgb}{0.000000,0.000000,0.000000}%
\pgfsetfillcolor{currentfill}%
\pgfsetlinewidth{0.803000pt}%
\definecolor{currentstroke}{rgb}{0.000000,0.000000,0.000000}%
\pgfsetstrokecolor{currentstroke}%
\pgfsetdash{}{0pt}%
\pgfsys@defobject{currentmarker}{\pgfqpoint{0.000000in}{-0.048611in}}{\pgfqpoint{0.000000in}{0.000000in}}{%
\pgfpathmoveto{\pgfqpoint{0.000000in}{0.000000in}}%
\pgfpathlineto{\pgfqpoint{0.000000in}{-0.048611in}}%
\pgfusepath{stroke,fill}%
}%
\begin{pgfscope}%
\pgfsys@transformshift{2.960940in}{0.528000in}%
\pgfsys@useobject{currentmarker}{}%
\end{pgfscope}%
\end{pgfscope}%
\begin{pgfscope}%
\definecolor{textcolor}{rgb}{0.000000,0.000000,0.000000}%
\pgfsetstrokecolor{textcolor}%
\pgfsetfillcolor{textcolor}%
\pgftext[x=2.960940in,y=0.430778in,,top]{\color{textcolor}{\rmfamily\fontsize{10.000000}{12.000000}\selectfont\catcode`\^=\active\def^{\ifmmode\sp\else\^{}\fi}\catcode`\%=\active\def%{\%}$\mathdefault{0.2}$}}%
\end{pgfscope}%
\begin{pgfscope}%
\pgfsetbuttcap%
\pgfsetroundjoin%
\definecolor{currentfill}{rgb}{0.000000,0.000000,0.000000}%
\pgfsetfillcolor{currentfill}%
\pgfsetlinewidth{0.803000pt}%
\definecolor{currentstroke}{rgb}{0.000000,0.000000,0.000000}%
\pgfsetstrokecolor{currentstroke}%
\pgfsetdash{}{0pt}%
\pgfsys@defobject{currentmarker}{\pgfqpoint{0.000000in}{-0.048611in}}{\pgfqpoint{0.000000in}{0.000000in}}{%
\pgfpathmoveto{\pgfqpoint{0.000000in}{0.000000in}}%
\pgfpathlineto{\pgfqpoint{0.000000in}{-0.048611in}}%
\pgfusepath{stroke,fill}%
}%
\begin{pgfscope}%
\pgfsys@transformshift{4.031664in}{0.528000in}%
\pgfsys@useobject{currentmarker}{}%
\end{pgfscope}%
\end{pgfscope}%
\begin{pgfscope}%
\definecolor{textcolor}{rgb}{0.000000,0.000000,0.000000}%
\pgfsetstrokecolor{textcolor}%
\pgfsetfillcolor{textcolor}%
\pgftext[x=4.031664in,y=0.430778in,,top]{\color{textcolor}{\rmfamily\fontsize{10.000000}{12.000000}\selectfont\catcode`\^=\active\def^{\ifmmode\sp\else\^{}\fi}\catcode`\%=\active\def%{\%}$\mathdefault{0.3}$}}%
\end{pgfscope}%
\begin{pgfscope}%
\pgfsetbuttcap%
\pgfsetroundjoin%
\definecolor{currentfill}{rgb}{0.000000,0.000000,0.000000}%
\pgfsetfillcolor{currentfill}%
\pgfsetlinewidth{0.803000pt}%
\definecolor{currentstroke}{rgb}{0.000000,0.000000,0.000000}%
\pgfsetstrokecolor{currentstroke}%
\pgfsetdash{}{0pt}%
\pgfsys@defobject{currentmarker}{\pgfqpoint{0.000000in}{-0.048611in}}{\pgfqpoint{0.000000in}{0.000000in}}{%
\pgfpathmoveto{\pgfqpoint{0.000000in}{0.000000in}}%
\pgfpathlineto{\pgfqpoint{0.000000in}{-0.048611in}}%
\pgfusepath{stroke,fill}%
}%
\begin{pgfscope}%
\pgfsys@transformshift{5.102387in}{0.528000in}%
\pgfsys@useobject{currentmarker}{}%
\end{pgfscope}%
\end{pgfscope}%
\begin{pgfscope}%
\definecolor{textcolor}{rgb}{0.000000,0.000000,0.000000}%
\pgfsetstrokecolor{textcolor}%
\pgfsetfillcolor{textcolor}%
\pgftext[x=5.102387in,y=0.430778in,,top]{\color{textcolor}{\rmfamily\fontsize{10.000000}{12.000000}\selectfont\catcode`\^=\active\def^{\ifmmode\sp\else\^{}\fi}\catcode`\%=\active\def%{\%}$\mathdefault{0.4}$}}%
\end{pgfscope}%
\begin{pgfscope}%
\definecolor{textcolor}{rgb}{0.000000,0.000000,0.000000}%
\pgfsetstrokecolor{textcolor}%
\pgfsetfillcolor{textcolor}%
\pgftext[x=3.280000in,y=0.240809in,,top]{\color{textcolor}{\rmfamily\fontsize{16.000000}{19.200000}\selectfont\catcode`\^=\active\def^{\ifmmode\sp\else\^{}\fi}\catcode`\%=\active\def%{\%}Birth}}%
\end{pgfscope}%
\begin{pgfscope}%
\pgfsetbuttcap%
\pgfsetroundjoin%
\definecolor{currentfill}{rgb}{0.000000,0.000000,0.000000}%
\pgfsetfillcolor{currentfill}%
\pgfsetlinewidth{0.803000pt}%
\definecolor{currentstroke}{rgb}{0.000000,0.000000,0.000000}%
\pgfsetstrokecolor{currentstroke}%
\pgfsetdash{}{0pt}%
\pgfsys@defobject{currentmarker}{\pgfqpoint{-0.048611in}{0.000000in}}{\pgfqpoint{-0.000000in}{0.000000in}}{%
\pgfpathmoveto{\pgfqpoint{-0.000000in}{0.000000in}}%
\pgfpathlineto{\pgfqpoint{-0.048611in}{0.000000in}}%
\pgfusepath{stroke,fill}%
}%
\begin{pgfscope}%
\pgfsys@transformshift{0.800000in}{0.541364in}%
\pgfsys@useobject{currentmarker}{}%
\end{pgfscope}%
\end{pgfscope}%
\begin{pgfscope}%
\definecolor{textcolor}{rgb}{0.000000,0.000000,0.000000}%
\pgfsetstrokecolor{textcolor}%
\pgfsetfillcolor{textcolor}%
\pgftext[x=0.305168in, y=0.488602in, left, base]{\color{textcolor}{\rmfamily\fontsize{10.000000}{12.000000}\selectfont\catcode`\^=\active\def^{\ifmmode\sp\else\^{}\fi}\catcode`\%=\active\def%{\%}0.000}}%
\end{pgfscope}%
\begin{pgfscope}%
\pgfsetbuttcap%
\pgfsetroundjoin%
\definecolor{currentfill}{rgb}{0.000000,0.000000,0.000000}%
\pgfsetfillcolor{currentfill}%
\pgfsetlinewidth{0.803000pt}%
\definecolor{currentstroke}{rgb}{0.000000,0.000000,0.000000}%
\pgfsetstrokecolor{currentstroke}%
\pgfsetdash{}{0pt}%
\pgfsys@defobject{currentmarker}{\pgfqpoint{-0.048611in}{0.000000in}}{\pgfqpoint{-0.000000in}{0.000000in}}{%
\pgfpathmoveto{\pgfqpoint{-0.000000in}{0.000000in}}%
\pgfpathlineto{\pgfqpoint{-0.048611in}{0.000000in}}%
\pgfusepath{stroke,fill}%
}%
\begin{pgfscope}%
\pgfsys@transformshift{0.800000in}{1.275396in}%
\pgfsys@useobject{currentmarker}{}%
\end{pgfscope}%
\end{pgfscope}%
\begin{pgfscope}%
\definecolor{textcolor}{rgb}{0.000000,0.000000,0.000000}%
\pgfsetstrokecolor{textcolor}%
\pgfsetfillcolor{textcolor}%
\pgftext[x=0.305168in, y=1.222635in, left, base]{\color{textcolor}{\rmfamily\fontsize{10.000000}{12.000000}\selectfont\catcode`\^=\active\def^{\ifmmode\sp\else\^{}\fi}\catcode`\%=\active\def%{\%}0.100}}%
\end{pgfscope}%
\begin{pgfscope}%
\pgfsetbuttcap%
\pgfsetroundjoin%
\definecolor{currentfill}{rgb}{0.000000,0.000000,0.000000}%
\pgfsetfillcolor{currentfill}%
\pgfsetlinewidth{0.803000pt}%
\definecolor{currentstroke}{rgb}{0.000000,0.000000,0.000000}%
\pgfsetstrokecolor{currentstroke}%
\pgfsetdash{}{0pt}%
\pgfsys@defobject{currentmarker}{\pgfqpoint{-0.048611in}{0.000000in}}{\pgfqpoint{-0.000000in}{0.000000in}}{%
\pgfpathmoveto{\pgfqpoint{-0.000000in}{0.000000in}}%
\pgfpathlineto{\pgfqpoint{-0.048611in}{0.000000in}}%
\pgfusepath{stroke,fill}%
}%
\begin{pgfscope}%
\pgfsys@transformshift{0.800000in}{2.009429in}%
\pgfsys@useobject{currentmarker}{}%
\end{pgfscope}%
\end{pgfscope}%
\begin{pgfscope}%
\definecolor{textcolor}{rgb}{0.000000,0.000000,0.000000}%
\pgfsetstrokecolor{textcolor}%
\pgfsetfillcolor{textcolor}%
\pgftext[x=0.305168in, y=1.956668in, left, base]{\color{textcolor}{\rmfamily\fontsize{10.000000}{12.000000}\selectfont\catcode`\^=\active\def^{\ifmmode\sp\else\^{}\fi}\catcode`\%=\active\def%{\%}0.200}}%
\end{pgfscope}%
\begin{pgfscope}%
\pgfsetbuttcap%
\pgfsetroundjoin%
\definecolor{currentfill}{rgb}{0.000000,0.000000,0.000000}%
\pgfsetfillcolor{currentfill}%
\pgfsetlinewidth{0.803000pt}%
\definecolor{currentstroke}{rgb}{0.000000,0.000000,0.000000}%
\pgfsetstrokecolor{currentstroke}%
\pgfsetdash{}{0pt}%
\pgfsys@defobject{currentmarker}{\pgfqpoint{-0.048611in}{0.000000in}}{\pgfqpoint{-0.000000in}{0.000000in}}{%
\pgfpathmoveto{\pgfqpoint{-0.000000in}{0.000000in}}%
\pgfpathlineto{\pgfqpoint{-0.048611in}{0.000000in}}%
\pgfusepath{stroke,fill}%
}%
\begin{pgfscope}%
\pgfsys@transformshift{0.800000in}{2.743462in}%
\pgfsys@useobject{currentmarker}{}%
\end{pgfscope}%
\end{pgfscope}%
\begin{pgfscope}%
\definecolor{textcolor}{rgb}{0.000000,0.000000,0.000000}%
\pgfsetstrokecolor{textcolor}%
\pgfsetfillcolor{textcolor}%
\pgftext[x=0.305168in, y=2.690700in, left, base]{\color{textcolor}{\rmfamily\fontsize{10.000000}{12.000000}\selectfont\catcode`\^=\active\def^{\ifmmode\sp\else\^{}\fi}\catcode`\%=\active\def%{\%}0.300}}%
\end{pgfscope}%
\begin{pgfscope}%
\pgfsetbuttcap%
\pgfsetroundjoin%
\definecolor{currentfill}{rgb}{0.000000,0.000000,0.000000}%
\pgfsetfillcolor{currentfill}%
\pgfsetlinewidth{0.803000pt}%
\definecolor{currentstroke}{rgb}{0.000000,0.000000,0.000000}%
\pgfsetstrokecolor{currentstroke}%
\pgfsetdash{}{0pt}%
\pgfsys@defobject{currentmarker}{\pgfqpoint{-0.048611in}{0.000000in}}{\pgfqpoint{-0.000000in}{0.000000in}}{%
\pgfpathmoveto{\pgfqpoint{-0.000000in}{0.000000in}}%
\pgfpathlineto{\pgfqpoint{-0.048611in}{0.000000in}}%
\pgfusepath{stroke,fill}%
}%
\begin{pgfscope}%
\pgfsys@transformshift{0.800000in}{3.477494in}%
\pgfsys@useobject{currentmarker}{}%
\end{pgfscope}%
\end{pgfscope}%
\begin{pgfscope}%
\definecolor{textcolor}{rgb}{0.000000,0.000000,0.000000}%
\pgfsetstrokecolor{textcolor}%
\pgfsetfillcolor{textcolor}%
\pgftext[x=0.305168in, y=3.424733in, left, base]{\color{textcolor}{\rmfamily\fontsize{10.000000}{12.000000}\selectfont\catcode`\^=\active\def^{\ifmmode\sp\else\^{}\fi}\catcode`\%=\active\def%{\%}0.400}}%
\end{pgfscope}%
\begin{pgfscope}%
\pgfsetbuttcap%
\pgfsetroundjoin%
\definecolor{currentfill}{rgb}{0.000000,0.000000,0.000000}%
\pgfsetfillcolor{currentfill}%
\pgfsetlinewidth{0.803000pt}%
\definecolor{currentstroke}{rgb}{0.000000,0.000000,0.000000}%
\pgfsetstrokecolor{currentstroke}%
\pgfsetdash{}{0pt}%
\pgfsys@defobject{currentmarker}{\pgfqpoint{-0.048611in}{0.000000in}}{\pgfqpoint{-0.000000in}{0.000000in}}{%
\pgfpathmoveto{\pgfqpoint{-0.000000in}{0.000000in}}%
\pgfpathlineto{\pgfqpoint{-0.048611in}{0.000000in}}%
\pgfusepath{stroke,fill}%
}%
\begin{pgfscope}%
\pgfsys@transformshift{0.800000in}{4.076160in}%
\pgfsys@useobject{currentmarker}{}%
\end{pgfscope}%
\end{pgfscope}%
\begin{pgfscope}%
\definecolor{textcolor}{rgb}{0.000000,0.000000,0.000000}%
\pgfsetstrokecolor{textcolor}%
\pgfsetfillcolor{textcolor}%
\pgftext[x=0.455864in, y=4.023398in, left, base]{\color{textcolor}{\rmfamily\fontsize{10.000000}{12.000000}\selectfont\catcode`\^=\active\def^{\ifmmode\sp\else\^{}\fi}\catcode`\%=\active\def%{\%}$+\infty$}}%
\end{pgfscope}%
\begin{pgfscope}%
\definecolor{textcolor}{rgb}{0.000000,0.000000,0.000000}%
\pgfsetstrokecolor{textcolor}%
\pgfsetfillcolor{textcolor}%
\pgftext[x=0.249612in,y=2.376000in,,bottom,rotate=90.000000]{\color{textcolor}{\rmfamily\fontsize{16.000000}{19.200000}\selectfont\catcode`\^=\active\def^{\ifmmode\sp\else\^{}\fi}\catcode`\%=\active\def%{\%}Death}}%
\end{pgfscope}%
\begin{pgfscope}%
\pgfpathrectangle{\pgfqpoint{0.800000in}{0.528000in}}{\pgfqpoint{4.960000in}{3.696000in}}%
\pgfusepath{clip}%
\pgfsetrectcap%
\pgfsetroundjoin%
\pgfsetlinewidth{1.003750pt}%
\definecolor{currentstroke}{rgb}{0.000000,0.000000,0.000000}%
\pgfsetstrokecolor{currentstroke}%
\pgfsetdash{}{0pt}%
\pgfpathmoveto{\pgfqpoint{0.800000in}{0.528000in}}%
\pgfpathlineto{\pgfqpoint{5.760000in}{3.928320in}}%
\pgfusepath{stroke}%
\end{pgfscope}%
\begin{pgfscope}%
\pgfpathrectangle{\pgfqpoint{0.800000in}{0.528000in}}{\pgfqpoint{4.960000in}{3.696000in}}%
\pgfusepath{clip}%
\pgfsetrectcap%
\pgfsetroundjoin%
\pgfsetlinewidth{1.003750pt}%
\definecolor{currentstroke}{rgb}{0.000000,0.000000,0.000000}%
\pgfsetstrokecolor{currentstroke}%
\pgfsetstrokeopacity{0.600000}%
\pgfsetdash{}{0pt}%
\pgfpathmoveto{\pgfqpoint{0.800000in}{4.076160in}}%
\pgfpathlineto{\pgfqpoint{5.760000in}{4.076160in}}%
\pgfusepath{stroke}%
\end{pgfscope}%
\begin{pgfscope}%
\pgfsetrectcap%
\pgfsetmiterjoin%
\pgfsetlinewidth{0.803000pt}%
\definecolor{currentstroke}{rgb}{0.000000,0.000000,0.000000}%
\pgfsetstrokecolor{currentstroke}%
\pgfsetdash{}{0pt}%
\pgfpathmoveto{\pgfqpoint{0.800000in}{0.528000in}}%
\pgfpathlineto{\pgfqpoint{0.800000in}{4.224000in}}%
\pgfusepath{stroke}%
\end{pgfscope}%
\begin{pgfscope}%
\pgfsetrectcap%
\pgfsetmiterjoin%
\pgfsetlinewidth{0.803000pt}%
\definecolor{currentstroke}{rgb}{0.000000,0.000000,0.000000}%
\pgfsetstrokecolor{currentstroke}%
\pgfsetdash{}{0pt}%
\pgfpathmoveto{\pgfqpoint{5.760000in}{0.528000in}}%
\pgfpathlineto{\pgfqpoint{5.760000in}{4.224000in}}%
\pgfusepath{stroke}%
\end{pgfscope}%
\begin{pgfscope}%
\pgfsetrectcap%
\pgfsetmiterjoin%
\pgfsetlinewidth{0.803000pt}%
\definecolor{currentstroke}{rgb}{0.000000,0.000000,0.000000}%
\pgfsetstrokecolor{currentstroke}%
\pgfsetdash{}{0pt}%
\pgfpathmoveto{\pgfqpoint{0.800000in}{0.528000in}}%
\pgfpathlineto{\pgfqpoint{5.760000in}{0.528000in}}%
\pgfusepath{stroke}%
\end{pgfscope}%
\begin{pgfscope}%
\pgfsetrectcap%
\pgfsetmiterjoin%
\pgfsetlinewidth{0.803000pt}%
\definecolor{currentstroke}{rgb}{0.000000,0.000000,0.000000}%
\pgfsetstrokecolor{currentstroke}%
\pgfsetdash{}{0pt}%
\pgfpathmoveto{\pgfqpoint{0.800000in}{4.224000in}}%
\pgfpathlineto{\pgfqpoint{5.760000in}{4.224000in}}%
\pgfusepath{stroke}%
\end{pgfscope}%
\begin{pgfscope}%
\definecolor{textcolor}{rgb}{0.000000,0.000000,0.000000}%
\pgfsetstrokecolor{textcolor}%
\pgfsetfillcolor{textcolor}%
\pgftext[x=3.280000in,y=4.307333in,,base]{\color{textcolor}{\rmfamily\fontsize{16.000000}{19.200000}\selectfont\catcode`\^=\active\def^{\ifmmode\sp\else\^{}\fi}\catcode`\%=\active\def%{\%}Persistence diagram}}%
\end{pgfscope}%
\begin{pgfscope}%
\pgfsetbuttcap%
\pgfsetmiterjoin%
\definecolor{currentfill}{rgb}{1.000000,1.000000,1.000000}%
\pgfsetfillcolor{currentfill}%
\pgfsetfillopacity{0.800000}%
\pgfsetlinewidth{1.003750pt}%
\definecolor{currentstroke}{rgb}{0.800000,0.800000,0.800000}%
\pgfsetstrokecolor{currentstroke}%
\pgfsetstrokeopacity{0.800000}%
\pgfsetdash{}{0pt}%
\pgfpathmoveto{\pgfqpoint{5.129968in}{0.597444in}}%
\pgfpathlineto{\pgfqpoint{5.662778in}{0.597444in}}%
\pgfpathquadraticcurveto{\pgfqpoint{5.690556in}{0.597444in}}{\pgfqpoint{5.690556in}{0.625222in}}%
\pgfpathlineto{\pgfqpoint{5.690556in}{1.019048in}}%
\pgfpathquadraticcurveto{\pgfqpoint{5.690556in}{1.046826in}}{\pgfqpoint{5.662778in}{1.046826in}}%
\pgfpathlineto{\pgfqpoint{5.129968in}{1.046826in}}%
\pgfpathquadraticcurveto{\pgfqpoint{5.102190in}{1.046826in}}{\pgfqpoint{5.102190in}{1.019048in}}%
\pgfpathlineto{\pgfqpoint{5.102190in}{0.625222in}}%
\pgfpathquadraticcurveto{\pgfqpoint{5.102190in}{0.597444in}}{\pgfqpoint{5.129968in}{0.597444in}}%
\pgfpathlineto{\pgfqpoint{5.129968in}{0.597444in}}%
\pgfpathclose%
\pgfusepath{stroke,fill}%
\end{pgfscope}%
\begin{pgfscope}%
\pgfsetbuttcap%
\pgfsetmiterjoin%
\definecolor{currentfill}{rgb}{0.894118,0.101961,0.109804}%
\pgfsetfillcolor{currentfill}%
\pgfsetlinewidth{1.003750pt}%
\definecolor{currentstroke}{rgb}{0.894118,0.101961,0.109804}%
\pgfsetstrokecolor{currentstroke}%
\pgfsetdash{}{0pt}%
\pgfpathmoveto{\pgfqpoint{5.157746in}{0.885747in}}%
\pgfpathlineto{\pgfqpoint{5.435524in}{0.885747in}}%
\pgfpathlineto{\pgfqpoint{5.435524in}{0.982969in}}%
\pgfpathlineto{\pgfqpoint{5.157746in}{0.982969in}}%
\pgfpathlineto{\pgfqpoint{5.157746in}{0.885747in}}%
\pgfpathclose%
\pgfusepath{stroke,fill}%
\end{pgfscope}%
\begin{pgfscope}%
\definecolor{textcolor}{rgb}{0.000000,0.000000,0.000000}%
\pgfsetstrokecolor{textcolor}%
\pgfsetfillcolor{textcolor}%
\pgftext[x=5.546635in,y=0.885747in,left,base]{\color{textcolor}{\rmfamily\fontsize{10.000000}{12.000000}\selectfont\catcode`\^=\active\def^{\ifmmode\sp\else\^{}\fi}\catcode`\%=\active\def%{\%}0}}%
\end{pgfscope}%
\begin{pgfscope}%
\pgfsetbuttcap%
\pgfsetmiterjoin%
\definecolor{currentfill}{rgb}{0.215686,0.494118,0.721569}%
\pgfsetfillcolor{currentfill}%
\pgfsetlinewidth{1.003750pt}%
\definecolor{currentstroke}{rgb}{0.215686,0.494118,0.721569}%
\pgfsetstrokecolor{currentstroke}%
\pgfsetdash{}{0pt}%
\pgfpathmoveto{\pgfqpoint{5.157746in}{0.681890in}}%
\pgfpathlineto{\pgfqpoint{5.435524in}{0.681890in}}%
\pgfpathlineto{\pgfqpoint{5.435524in}{0.779112in}}%
\pgfpathlineto{\pgfqpoint{5.157746in}{0.779112in}}%
\pgfpathlineto{\pgfqpoint{5.157746in}{0.681890in}}%
\pgfpathclose%
\pgfusepath{stroke,fill}%
\end{pgfscope}%
\begin{pgfscope}%
\definecolor{textcolor}{rgb}{0.000000,0.000000,0.000000}%
\pgfsetstrokecolor{textcolor}%
\pgfsetfillcolor{textcolor}%
\pgftext[x=5.546635in,y=0.681890in,left,base]{\color{textcolor}{\rmfamily\fontsize{10.000000}{12.000000}\selectfont\catcode`\^=\active\def^{\ifmmode\sp\else\^{}\fi}\catcode`\%=\active\def%{\%}1}}%
\end{pgfscope}%
\end{pgfpicture}%
\makeatother%
\endgroup%

        }
        \caption{Manyholes dataset}
        \label{fig:dowker_ph_manyholes}
    \end{subfigure}
    \caption{Persistence diagrams of the Dowker complex on the nested and manyholes datasets.}
    \label{fig:dowker_ph}
\end{figure}
